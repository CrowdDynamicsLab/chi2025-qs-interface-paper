\section{Introduction}
States utilize referendums to aggregate constituents' opinions and form policy decisions; organizations such as Pew Research Center in the United States deploy surveys to identify public perspectives on societal challenges; and city council meetings provide public forums where community members can voice their concerns and foster consensus. To facilitate effective collaborative decision-making, robust tools are required to accurately capture individual responses, attitudes, and preferences. In the domains of Computer-Supported Collaborative Work (CSCW) and Human-Computer Interaction (HCI), researchers extensively explored surveying methods that involved rating and ranking of attitudes to gather data and understand collaborative behaviors which informs decisions through empirical studies~\cite{}. Recent advances in CSCW research have particularly emphasized the efficacy of Quadratic Voting (QV) over traditional Likert-scale surveys for societal resource allocation, as it more accurately reflects individual preferences~\cite{chengCanShowWhat2021} in forms of survey. In fact, the use of QV in the in the public~\cite{rogersColoradoTriedNew2019, teamTaiwanDigitalMinister} and private sectors~\cite{Gov4gitDecentralizedPlatform2023} had begun using such mechanism to inform collaborative decision making. However, there exists no research on designing interfaces for this attitude-capturing tool. 

States use referendums to gather constituents' opinions and form policy decisions; organizations like the Pew Research Center use surveys to identify public perspectives on societal challenges; and city council meetings offer forums for community members to voice concerns and build consensus. To facilitate effective collaborative decision-making, effective tools are needed to accurately capture individual responses, attitudes, and preferences. In the domains of Computer-Supported Collaborative Work (CSCW) and Human-Computer Interaction (HCI), researchers extensively explored surveying methods that rate and rank attitudes to gather data and understand collaborative behaviors which informs decisions through empirical studies~\cite{}. Recent CSCW research highlights the efficacy of Quadratic Voting (QV) for societal resource allocation, as it reflects individual preferences more accurately~\cite{chengCanShowWhat2021} in forms of survey. The public and private sectors have begun using QV to inform collaborative decision making~\cite{rogersColoradoTriedNew2019, teamTaiwanDigitalMinister, Gov4gitDecentralizedPlatform2023}. However, no research has explored designing interfaces for this attitude-capturing tool.

We introduce Quadratic Surveys (QS) to describe surveys embedding QV, as explored by~\textcite{quarfoot2017quadratic} and~\textcite{chengCanShowWhat2021}. This study aims to design interactive interfaces for QS that focus on societal causes. 

The design of attitude-capturing tools is critical as it significantly affects how individuals express their attitudes~\cite{engstrom2020politics, weijtersEffectRatingScale2010, kierujVariationsResponseStyle2010, toepoelSmileysStarsHearts2019, farzandAestheticsEvaluatingResponse2024, xiaoTellMeYourself2020, pielotDidYouMisclick2024}. Better digital interfaces yield higher quality responses, leading to more accurate and reliable collective decision-making. Identifying challenges faced by survey respondents is crucial for improving effectiveness. Since QS elicits higher quality responses, the interface is as important as the mechanism. Therefore, we ask:~\textit{How can we design interactive interfaces to support participants in completing Quadratic Surveys?}

There are two reasons why addressing this design question is important: First, QS requires respondents to construct preferences, a mentally demanding task which raises cognitive load. Second, QS respondents must express preferences numerically, making trade-offs under a budget. Since no prior literature explores QV interface design or user needs, we turn to preference construction literature to inform our design. As~\textcite{lichtensteinConstructionPreference2006} note, individuals construct preferences in situ when preferences are undefined, necessitating trade-offs or quantifying opinions. \textcite{svensonDifferentiationConsolidationTheory1992} theorized individuals differentiate among options before consolidating a result. We also consulted survey response format literature\cite{} to inform our design iterations, including drag-and-drop components. We explored ranking, selection, and organization approaches over three design iterations to facilitate preference construction. During our iteration, we learn that survey respondents need not construct a full set of preferences among all options to allocate their credits when completing QS.

Finally, we introduce a novel interactive QS interface (Figure~\ref{fig:interactiveInterface}) that nudges QS survey respondents to take a two-step `organize-then-vote' approach to preference construction and elicitation. The final interface aimed to scaffold the `differentiation and construction' decision-making process from decision making research, supporting the preference construction journey through a two-step approach: first organizing, then voting. Participants are showed all options on the survey one-at-a-time, and categorize them into a three-tiered preference category. Participants can organize their thoughts within the bins as they wish. These categories then inform the position of the voting interface, where survey respondents finally expresses their preferences.

In addition that QS increases cognitive load due to its mechanism, and a long list of options can cause cognitive overload. Real-world QV implementations, such as the Colorado House of Representatives, considered 107 bills in one QV event~\cite{QuadraticVotingColorado}. Too many choices can lead to challenges like choice overload and overchoice~\cite{iyengarWhenChoiceDemotivating2000, gourvilleOverchoiceAssortmentType2005}. Increased choices raise cognitive load, leading to more survey response errors and ‘good enough’ answers instead of optimal ones~\cite{lenznerCognitiveBurdenSurvey2010, blessAskingDifficultQuestions1992}. Reducing the number of options can be impractical if the goal is to elicit fine-grained individual preferences for decision makers. Thus, this research examines the influence of interactive interfaces on QS with varying option lengths. We seek to answer the following research questions:

\begin{itemize}
    \item RQ1. How does the number of options in QS impact respondents' cognitive load?
    \item RQ2a. How does the two-phase interactive interface impact respondents' cognitive load compared to a text interface?
    \item RQ2b. What are the similarities and differences in sources of cognitive load across the two interfaces?
    \item RQ3. What are the differences in QS respondents' behaviors when coping with long lists of options across the two-phase interactive interface and the text interface?
\end{itemize}

To answer these research questions, we designed a 2x2 between-subject in-lab study. Participants completed a QS with either a short or long list of options using our designed text or interactive interface. We measured participants' cognitive load using NASA-TLX and conducted interviews focusing on its different elements. We collected clickstream data as participants completed the QS. We recruited 41 participants from a local community and surveyed their attitudes on various societal issues. We analyzed study results, transcribed, coded, and thematically analyzed interviews.

\paragraph{Contributions}
We contributed to CSCW by proposing the first interactive interface specifically designed for QS and QV-like applications. No prior research has investigated interfaces for QS, especially long ones that lead to cognitive overload. Our two-stage organize-then-vote interface facilitates critical decision making and limits satisficing behaviors. This design promotes incremental updates and deeper engagement, enhancing understanding and decision quality. Second, we conducted the first in-depth qualitative analysis identifying key factors contributing to cognitive load among survey respondents. Our qualitative interviews identified design challenges for QS, driving further research directions.

The remainder of this paper is structured as follows: related works are covered in section~\ref{sec:relatedWorks}, followed by design decisions for the interactive QS interface~\ref{sec:interfaceDesign}. Section~\ref{sec:experiment} details the experiment design. Study findings are presented in sections~\ref{sec:cog_result}, section~\ref{sec:behave_result}. We discuss our findings and future work in section~\ref{sec:discussion}.

% Some additional comments re questions
% -- does this version cut down too much?

% maybe remove stat if we move to desriptive stat. Highlight first in-depth interview unserstand challegnes and explore design solutions for QM apps. 
% However, is QV requires computer supported interfaces to facilitate given it's complexity to perform using pen-and-paper. 
 % we learn from QV implementations  that very often there can be a lot of options for one to express their views.
 
% old tex lives here
% This study introduces the first interactive interface specifically designed for Quadratic Surveys and Quadratic Voting-like applications, addressing a significant gap in existing research tools. Additionally, we conduct the first in-depth qualitative analysis which identifyed key factors contributing to cognitive load among survey respondents, leading to practical design recommendations. Although our findings did not show statistically significant differences in cognitive load as measured by NASA-TLX, our two-step interactive interface proved effective in facilitating more holistic and reflective decision-making, as evidenced by qualitative interview results. This scaffolding design promotes incremental updates and fosters deeper engagement, ultimately enhancing understanding and decision quality.
% Effectively capturing individuals' responses, attitudes, and preferences is the cornerstone of forming consensus within computer-supported collaborative work (CSCW) and is crucial for studying human subjects in the CSCW community. Computer-supported tools, such as digital surveys, facilitate the expression of community preferences and opinions to provide decision-makers with insights that influence final outcomes influencing the community. Recently,
% ~\textcite{chengCanShowWhat2021} showed that QV, when used as a survey, not only allowed for capturing both rankings and ratings across multiple options, but also provide better aligned results than traditional Likert scale surveys.

% these concurrent research that explores similar techniques to better survey individual preferences~\cite{quarfoot2017quadratic, chengCanShowWhat2021}.

% \begin{figure}[ht]
%     \centering
%     \includegraphics[width=\textwidth]{content/image/header.pdf}
%     \caption{A novel two-phase interactive interface for Quadratic Surveys detailed in Sec.~\ref{sec:finalInterfaceDesign}}
%     \label{fig:header}
% \end{figure}



% This challenge emphasizes the critical importance of designing and developing suitable interfaces for Quadratic Surveys to elicit truthful and in-depth preference information from respondents. Good design is essential; without it, the quality of collected data can suffer significantly. 
% These reasons strongly motivate  Addressing this question fills a important gap in the literature and enhances the practical utility of QS in capturing high-quality data across various applications.

% Highlight this is also the first work to investigate where people felt challenging when completing QS.
% At the same time, reducing cognitive load and making survey less challenging . Effective design may mitigate these overload challenges, ensuring that the Quadratic Survey mechanism fulfills its potential to capture detailed and accurate preferences. Thus, more concretely, this study focused on answering the following three research questions:

% 1. we design
% 2. we test across condition
% 3. understand what is difficult/hard through qualitative and quant responses
% --> design and usage recommendations

% Cited without details
% U.S. Colorado state government~\cite{rogersColoradoTriedNew2019}, the Democratic
%  Caucus of the House of Representatives~\cite{QuadraticVotingColorado} in the U.S., government-sponsored hackathons~\cite{teamTaiwanDigitalMinister}, and the recent Gov4git~\cite{Gov4gitDecentralizedPlatform2023}
% Despite the advocacy of Quadratic Voting by Posner and Weyl~\cite{posner2018radical}
% Political scientists have demonstrated that ballot designs alone can sway voter decisions~\cite{engstrom2020politics}, marketing and psychology researchers have examined how the presentation of questions influences responses~\cite{weijtersEffectRatingScale2010, kierujVariationsResponseStyle2010, toepoelSmileysStarsHearts2019}, and Human-Computer Interaction researchers have focused on evaluating and understanding web surveys and smart interfaces for surveys~\cite{farzandAestheticsEvaluatingResponse2024, xiaoTellMeYourself2020, pielotDidYouMisclick2024}. These studies highlight the importance of studying the interface and design for survey mechanisms.
% The Quadratic Mechanism is undeniably more complex than other voting and surveying mechanisms like the Likert scale survey~\cite{likertTechniqueMeasurementAttitudes1932}, where individuals select from a few responses, and Approval Voting~\cite{bramsApprovalVoting1978}, where participants mark as many options as they approve without constraints.

% Move this to related works.
% The effectiveness of eliciting these responses hinges upon the study protocol, survey mechanism, and design of the tool at hand~\cite{olsonWaysKnowingHCI2014, couperWebSurveyDesign2001, jackoHumancomputerInteractionHandbook2012}. While much research has explored the influence of the former two aspects, this research focuses on the design of a specific survey -- Quadratic Surveys. 

% Move this to discussion + contribution
% Thus, the primary goal of this study is to present a novel interactive interface designed for quadratic surveys, which could presumably extend to other applications that utilize the quadratic mechanism. 

% TODO, for a cleaner thesis statement in the first par? mention challenges and interface here before diving deeper.
% The importance of design in surveying tools, the growing usage of applications on the quadratic mechanism, and the lack of research on the design regarding quadratic mechanisms that one could apply, motivated our main research question: \textit{How can we design interactive interfaces to support participants in completing Quadratic Surveys?}

% Quadratic Surveys and other quadratic mechanism powered applications allow individuals 
% \begin{itemize}
% \item RQ1. How does the number of options on QS impact respondents' cognitive load?
% \item RQ2a. How does the interactive interface involving grouping and direct manipulation interface influence QS respondents' cognitive load compared to text-based interface?
% \item RQ2b. Across the two interfaces, what are the sources of cognitive load from?
% \item RQ3. What are differences in QS respondents' behaviors when coping with long lists of options?
% \end{itemize}

% Before answering these research questions, we iteratively designed and built an interactive interface informed by prior literature in the questionnaire and survey response format. Then, 