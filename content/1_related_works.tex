\section{Related works}
\label{sec:relatedWorks}
This research is situated between three core areas -- quadratic mechanisms, the importance of survey and voting interface design, and cognitive psychology.  In this section, we present essential background information about each of these areas .

\subsection{Quadratic Survey and the Quadratic Mechanism}
A Quadratic Survey (QS) is a surveying technique that presents a Quadratic Voting item for surveying an individual's attitude across a series of options. Both tools share the same quadratic mechanism used to inform collective decision-making. This mechanism allows respondents to express their preference intensity by casting multiple votes at a quadratic cost. Made popular by \textcite{posner2018radical, lalley2018quadratic}, it aims to mitigate the tyranny of the majority inherent in traditional one-person-one-vote systems. QV is not subject to Arrow's impossibility theorem as it does not require individuals to aggregate rankings of preferences. \yhc{what is arrow's impossibility theorem, does everyone in CSCW know this? Citation maybe?} Quadratic Surveys adapt this mechanism for survey contexts, allowing participants to vote for or against an option, presenting two distinct choices in the same survey. This adaptation was utilized by \textcite{quarfoot2017quadratic} and implemented as an open-source platform by \textcite{bassettiCivicbaseOpensourcePlatform2023}. While these studies did not explicitly label this as a 'quadratic survey', we use this term to differentiate it from the voting mechanism.

To formally define QV, in a scenario where $S$ participants are involved, each participant is allocated a fixed quantity of voice credits, denoted as $B$. These credits can be distributed among various options. Importantly, each individual can cast multiple votes, either in favor of or against each option. However, this voting system incorporates a quadratic cost for voting: casting $n_k$ votes for a particular option $k$ incurs a cost $c(n_k)$, which is proportional to $n_k^2$. Consequently, the aggregate cost in voice credits for all options chosen by a participant must not exceed their allocated budget $B$. This necessitates that the sum of the squares of votes cast for each option ($\sum_k n_k^2$) remains within the limit of $B$, where $n_k$ represents the number of votes allocated to option $k$. QV results determine the winner by summing up the total votes cast by all participants for each option. This design allows the marginal cost to cast one additional vote to linearly increase with the number of votes already cast on that option, inducing rational participants to vote proportionally to how much they care about an issue~\cite{posner2018radical}. Quadratic Surveys extend the same mechanism but allow participants to denote positive (upvotes) or negative values (downvotes) on each option. The survey administrators compile and analyze the results by summing up the total votes and allowing cancellation between upvotes and downvotes.

Empirical studies and applications of the quadratic mechanism and quadratic voting have increased in the past few years. Several studies have explored the empirical use cases for QV, including \textcite{quarfoot2017quadratic}'s study on 4,500 participants' attitudes across ten public policies, highlighting differences between QV and Likert scale survey results. \textcite{chengCanShowWhat2021} applied quadratic surveys in Human-Computer Interaction (HCI) and subsequently showed QV's effectiveness in reflecting true preferences in monetary decision tasks. \textcite{naylor2017first} used QV in educational research to gauge student opinions on factors affecting university success, and \textcite{cavailleWhoCaresMeasuring} examined QV in polarized choice scenarios.

Another form of research focuses on the transformation and application of the quadratic mechanism into different tools. Recent work by \textcite{southPluralManagement2024} applies the quadratic mechanism as part of the management framework to support networked authority, which was later applied to Gov4git~\cite{Gov4gitDecentralizedPlatform2023}. Quadratic Funding focuses on the redistribution of funds following outcomes from consensus made using the quadratic mechanism~\cite{buterinFlexibleDesignFunding2019a, freitasQuadraticFundingIncomplete2024}. Despite the breadth of applications, there is little attention investigating the user experience and interface design supporting individuals to express attitudes. .

\subsection{Survey, Voting, and QV Design}
% maybe add in this section: survey and voting at the end of the day is an interface for people to express attitudes, just they aim for different outcomes.
The relative lack of research in QS interface design is concerning, as prior research in survey and voting interfaces has demonstrated substantial impact even seemingly minor design decisions can have on respondent behaviors and outcomes. The notorious butterfly ballot~\cite{wandButterflyDidIt2001} is one example of this -- \textcite{wandButterflyDidIt2001} argue that an atypical ballot design may have caused enough accidental votes to swing the 2000 U.S. Presidential Election. Researchers like \textcite{engstrom2020politics}, \textcite{chisnellDemocracyDesignProblem2016}, and organizations like the Center for Civic Design, which publishes reports like "Designing Usable Ballots"~\cite{DesigningUsableBallots2015}, stress that democracy is a design problem. We group this literature into three main categories: designs that shifted voter decisions, designs that influenced human errors, and designs that incorporated technologies to improve usability.

\paragraph{Designs that shifted voter decisions: } For example, states without the option for straight-party ticket voting (the option to circle an option that votes for all the candidates in the same party) exhibited higher rates of split-ticket voting~\cite{engstrom2020politics}. Another example from the Australian ballot with an office block and no party box (having a box that clearly segments the position that the candidates are competing for) has been shown to enhance incumbency advantages.
\paragraph{Designs that influenced errors: } Butterfly ballots increased voter errors because voters could not correctly identify the punch hole on the ballot. Splitting contestants across columns increases the chance for voters to overvote~\cite{quesenberyOpinionGoodDesign2020}. On the other hand, \textcite{everettElectronicVotingMachines2008} showed the use of incorporating physical voting behaviors, like lever voting, into GUI interfaces.
\paragraph{Designs that incorporated technologies: } Other projects like the Caltech-MIT Voting Technology Project have sparked research to address accessibility challenges, resulting in innovations like EZ Ballot~\cite{leeUniversalDesignBallot2016}, Anywhere Ballot~\cite{summers2014making}, and Prime III~\cite{dawkinsPrimeIIIInnovative2009}. In addition, \textcite{gilbertAnomalyDetectionElectronic2013} investigated optimal touchpoints on voting interfaces, and \textcite{conradElectronicVotingEliminates2009} examined zoomable voting interfaces.

These findings underscore the profound impact of design and how it influences elicited individual attitudes. Research in the marketing and research community studying survey and questionnaire design, usability, and interaction finds similar trends. The term `Response Format' is often used to describe the style and presentation of a question presented on a survey. Various studies have shown that different designs of response formats can influence outcomes. For example, \textcite{weijtersExtremityHorizontalVertical2021} demonstrated that horizontal distances between options are more influential than vertical distances, with the latter recommended for reduced bias. Slider bars, which operate on a drag-and-drop principle, show lower mean scores and higher nonresponse rates compared to buttons, indicating they are more prone to bias and difficult to use. In contrast, visual analogue scales that operate on a point-and-click principle perform better~\cite{toepoelSlidersVisualAnalogue2018}. 

While the design of voting systems and question response format markedly influence voter behavior and decision accuracy, these interface elements also directly impact the cognitive load on users. An effective design would enhance usability and reduce cognitive challenges faced by survey respondents, especially in complex response mechanisms like QS.

\subsection{Cognitive Challenges and Choice Overload}
Despite the deep insight prior research learned about voting and surveying techniques and the robust mechanism demonstrated in theoretical applications of quadratic mechanisms, the inherent challenge that survey respondents are required to make many difficult decisions poses a unique cognitive challenge that no prior literature has tackled.~\textcite{lichtensteinConstructionPreference2006} laid out the three key elements that make decisions difficult. They include people making decisions within an unfamiliar context, people forced to make tradeoffs due to conflicts among choices, and people quantifying values for their opinions. QS fits into the description of all three elements, as participants can face options placed by the decision maker which they have never seen before. Participants are bounded by budgets that force them to make tradeoffs, and the final votes are presented in values. Hence, we believe that QS introduces a high cognitive load.

Previous studies have demonstrated that cognitive overload can adversely affect performance, for instance, causing individuals to rely more on heuristics rather than engaging in deliberate and logical decision-making~\cite{daniel2017thinking}. In addition, some researchers believe that preferences are constructed in situ just as memories are. Thus, when too much information is presented to an individual, they can `satisfice' their decisions~\cite{simonBehavioralModelRational1955, payneAdaptiveStrategySelection1988, tverskyJudgmentsRepresentativeness}. This overload can happen because of the presence of too many options. Subsequently, too many options can lead to individuals feeling overloaded, leading to decision paralysis, demotivation, and dissatisfaction~\cite{iyengarWhenChoiceDemotivating2000}. 

Additionally, \textcite{alwinMeasurementValuesSurveys1985} highlighted that the use of ranking techniques in surveys can be time-consuming and potentially more costly to administer. These challenges are compounded when there are numerous items to rank, requiring substantial cognitive sophistication and concentration from survey respondents \cite{featherMeasurementValuesEffects1973}.

However, in several notable applications of Quadratic Voting in society, there can be hundreds of options within a single QV question. For instance, the 2019 Colorado House of Representatives considered 107 bills \cite{NewWayVoting}, and the 2019 Taiwan Presidential Hackathon featured 136 proposals \cite{QuadraticVotingFrontend2022}. These psychological and behavioral research highlighted the importance of understanding how individuals navigate and can potentially benefit from interfaces under long-list QS conditions.



 %As \textcite{chengCanShowWhat2021} noted, it is essential to better understand how the number of options influences the usability of QS and to design interfaces that effectively support survey respondents.
 % there is limited research on interfaces for Constant Sum surveys~\cite{hauserIntensityMeasuresConsumer1980a}, a mechanism similar to QS that aims to elicit both ranking and rating preferences from individuals.
 %The closest work discussing interfaces for QV is an arXiv paper~\cite{} that transformed the knapsack voting platform developed by \textcite{goelKnapsackVotingVoting}
%  While both fields have deep insights into understanding design's influence on attitude elicitation, QS's unique capability of supporting both ranking and rating~\cite{chengCanShowWhat2021} makes designing an interface important and challenging. Subsequently, this research aims to understand how this interface influences an individual's QS response behavior. Requiring the distribution of budgets following the quadratic mechanism introduces new and complex decisions. 