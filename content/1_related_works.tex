\section{Related Work}
\label{sec:relatedWorks}
This research lies at the intersection of three core areas: quadratic surveys, survey and voting interface design, and choice overload along with cognitive challenges. In this section, we review the related works in each of these areas.

\subsection{Quadratic Survey and the Quadratic Mechanism}
We introduce the term \textbf{Quadratic Survey (QS)} to describe surveys that utilize the quadratic mechanism to collect individual attitudes. The~\textbf{quadratic mechanism} is a theoretical framework designed to encourage the truthful revelation of individual preferences through a quadratic cost function~\cite{grovesOptimalAllocationPublic1977}. This framework gained popularity through~\textbf{Quadratic Voting (QV)}, also known as plural voting, which uses a quadratic cost function in a voting framework to facilitate collective decision-making~\cite{lalley2016quadratic}.%~\textcite{quarfoot2017quadratic} demonstrated that QV effectively gauges public opinion and mitigates the tyranny of the majority in traditional voting systems. Furthermore, QV is not subject to Arrow's impossibility theorem, which states that no voting system can perfectly aggregate individual preferences without trade-offs~\cite{morreau2014arrow}, because it does not require aggregating rankings.  

To illustrate how QS works, we formally define the mechanism: each survey respondent is allocated a fixed budget, denoted by $B$, to distribute among various options. Participants can cast $n$ votes for or against option~$k$. The cost~$c_k$ for each option $k$ is derived as:

\[c_k = n_k^2 \quad \text{where}\quad n_k \in \mathbb{Z}\]

The total cost of all votes must not exceed the participant's budget:

\[\sum_k c_k \leq B\]

Survey results are determined by summing the total votes for each option:

\[ \text{Total Votes for Option } k = \sum_{i=1}^{S} n_{i,k} \]

where $S$ represents the total number of participants, and~$n_{i,k}$ is the number of votes cast by participant~$i$ for option~$k$. Each additional vote for each option increases the marginal cost linearly, encouraging participants to vote proportionally to their level of concern for an issue~\cite{posner2018radical}.

QS adapts these strengths of the quadratic mechanism in \textit{voting} to encourage truthful expression of preferences in \textit{surveys}. Unlike traditional surveys that elicit either rankings~\textit{or} ratings, QS allows for~\textit{both}, enabling participants to cast multiple votes for or against options, incurring a quadratic cost.~\textcite{chengCanShowWhat2021} showed that this mechanism aligns individual preferences with behaviors more accurately than Likert Scale surveys, particularly in resource-constrained scenarios like prioritizing user feedback on user experiences.

In recent years, empirical studies on QV have expanded into various domains~\cite{naylor2017first, cavailleWhoCaresMeasuring}. Applications based on the quadratic mechanism have also grown, including Quadratic Funding, which redistributes funds based on outcomes from consensus made using the quadratic mechanism~\cite{buterinFlexibleDesignFunding2019a, freitasQuadraticFundingIncomplete2024}. Recent work by \textcite{southPluralManagement2024} applies the quadratic mechanism to networked authority management, later used in Gov4git~\cite{Gov4gitDecentralizedPlatform2023}. Despite the increasing breadth and depth of applications utilizing the quadratic mechanism, little attention has been paid to user experience and interface design, which support individuals in expressing their preference intensity. Our work aims to address this by designing interfaces supporting quadratic mechanisms.

\afterpage{
\clearpage
\begin{figure}[p]
    \centering
    \begin{subfigure}[b]{0.47\textwidth}
        \centering
        \includegraphics[width=0.67\textwidth]{content/image/curr_interface/radical_market_wedesign.png}
        \caption{Software by WeDesign, used in the first empirical QV research~\cite{quarfoot2017quadratic}. Little information is available about the software, except for an image from~\textcite{posner2018radical}. In the image, each prompt has thumbs up and down icons to update the vote in the center. The remaining budget appears as a progress bar at the top.}
        \label{fig:wedesignInterface}
    \end{subfigure}
    \hspace{0.4cm}
    \begin{subfigure}[b]{0.47\textwidth}
        \centering
        \includegraphics[width=0.72\textwidth]{content/image/curr_interface/rxc_interface.png}
        \caption{An open-sourced QV interface~\cite{RadicalxChangeQuadraticvoting2024} forked from GitCoin~\cite{ReadWhitepaperGitcoin}, used by the RadicalxChange community~\cite{RxC}. This interface presents total credits as small blocks. Votes are updated using plus and minus buttons, with numerical counts shown under each option and surface area as costs.}
        \label{fig:rxcvotingInterface}
    \end{subfigure}
    
    \vspace{0.12cm}
    
    \begin{subfigure}[b]{0.47\textwidth}
        \centering
        \includegraphics[width=0.72\textwidth]{content/image/curr_interface/geek.sg_interface.png}
        \caption{An open-source QV interface~\cite{yehjxraymondYehjxraymondQvapp2024} offers a publicly available service. Options show only the current number of votes, with credits displayed in the top right corner. This interface does not show the costs of votes but supports sorting options.}
        \label{fig:yehInterface}
    \end{subfigure}
    \hspace{0.4cm}
    \begin{subfigure}[b]{0.47\textwidth}
        \centering
        \includegraphics[width=0.72\textwidth]{content/image/curr_interface/appvote.png}
        \caption{The interface designed for gov4git~\cite{Gov4gitDecentralizedPlatform2023} updates votes using arrows under each option, with the associated cost shown as a percentage bar to the right. A search bar exists for searching specific pull requests or issues.}
        \label{fig:gov4gitInterface}
    \end{subfigure}
    
    \vspace{0.15cm}
    
    \begin{subfigure}[b]{0.7\textwidth}
        \centering
        \includegraphics[width=0.50\textwidth]{content/image/curr_interface/cheng_qv.png}
        \caption{The interface used in the research by~\textcite{chengCanShowWhat2021} employs the most visual components. Icons depict the current number of votes, with progress bars signifying the current spending.}
        \label{fig:chengInterface}
    \end{subfigure}
    
    \caption{Recent interface for applications using the quadratic mechanism.}
    \label{fig:qv_interface_external}
\end{figure}
\clearpage
}
\subsection{Design Implications from Surveys and Voting Interfaces}
We began by examining existing QV applications~(Fig.~\ref{fig:qv_interface_external}), which share the same mechanism as QS, and identifying shared interface components. All QV interfaces generally include:

\begin{itemize}
    \item Option list: A list of options for voting.
    \item Vote controls: Buttons to increase or decrease votes for each option.
    \item Individual vote tally: A display of the votes cast per option.
    \item Summary: An auto-generated summary of costs and the remaining budget.
\end{itemize}

These components present options, calculate costs, and allow vote adjustments. However, these interfaces focus purely on mechanics without little understanding of voter's usability needs or offering cognitive support to help them complete the task effectively. In addition, the HCI community conducted few research~\cite{nobarany2012design, van2007design} on survey and questionnaire interfaces components, with more work focusing more on alternative input modalities like bots, voice, and virtual reality~\cite{voiceWei2022, khullar2021, kimComparingDataChatbot2019, feick2020virtual}. Thus, we turn to marketing and research literature for insights into how digital survey interface elements can impact user behavior and usability.

Research in the marketing and research communities focusing on survey and questionnaire design, usability, and interactions examines the influence of presentation styles and `response format.'~\textcite{weijtersExtremityHorizontalVertical2021} demonstrated that horizontal distances between options are more influential than vertical distances, with the latter recommended for reduced bias. Slider bars, which operate on a drag-and-drop principle, show lower mean scores and higher nonresponse rates compared to buttons, indicating they are more prone to bias and difficult to use. In contrast, visual analog scales that operate on a point-and-click principle perform better~\cite{toepoelSlidersVisualAnalogue2018}. These studies show how even small design changes can have a large impact on usability, highlighting the importance of designing interfaces that prioritize human-centered interaction rather than focusing solely on functionality.

Voting interfaces are a specialized type of survey interface that not only elicit individual choices but often have consequential impacts. For example, the butterfly ballot, an atypical design, may have influenced the outcome of the 2000 U.S. Presidential Election~\cite{wandButterflyDidIt2001}. Research has shown that ballot interfaces can significantly influence democratic processes~\cite{engstrom2020politics, chisnellDemocracyDesignProblem2016, DesigningUsableBallots2015}. Several studies also highlighted how voting interface designs shift voter decisions~\cite{engstrom2020politics}, reduce usability errors~\cite{quesenberyOpinionGoodDesign2020, everettElectronicVotingMachines2008}, or improve interaction~\cite{leeUniversalDesignBallot2016, summers2014making, dawkinsPrimeIIIInnovative2009, gilbertAnomalyDetectionElectronic2013, conradElectronicVotingEliminates2009}. We provide more details to these voting interfaces in the Appendix~\ref{apdx:relatedVoting}.

From the QV implementations, response format literature, and voting interfaces, we identified how interfaces significantly influence respondent behavior, decision accuracy, and cognitive load. While these systems are functional, they lack the human-centered design needed to reduce cognitive load and make them truly usable, rather than simply operable. These burdens are especially problematic for complex systems like QS, where high cognitive demands may deter researchers and users alike. Developing effective, human-centered interfaces for QS could enhance usability, reduce cognitive overload, and increase adoption in both research and practical applications.

\subsection{Cognitive Challenges and Choice Overload}
The challenge of respondents making difficult decisions using quadratic mechanisms remains unexplored in the literature.~\textcite{lichtensteinConstructionPreference2006} identified three key elements that make decisions difficult. These elements include making decisions in unfamiliar contexts, being forced to make trade-offs due to conflicting choices, and quantifying the value of one's opinions. QS fits all three elements: participants may encounter unfamiliar options set by the decision-maker, are constrained by budgets that require trade-offs, and cast final votes as numerical values. Thus, we believe QS introduces high cognitive load.

According to cognitive load theory, cognitive load refers to the demands placed on a user's working memory during the interaction process, which significantly influences the usability of the system~\cite{cooper1998research, paas2003cognitive}. Cognitive overload can adversely affect performance~\cite{drommi2001interface}, leading individuals to rely on heuristics rather than deliberate, logical decision-making~\cite{daniel2017thinking}. When presented with excessive information, such as too many options, individuals 'satisfice', settling for a 'good enough' solution rather than an optimal one~\cite{simonBehavioralModelRational1955, payneAdaptiveStrategySelection1988, tverskyJudgmentsRepresentativeness}. Subsequently, too many options can overwhelm individuals, resulting in decision paralysis, demotivation, and dissatisfaction~\cite{iyengarWhenChoiceDemotivating2000}.

Additionally,~\textcite{alwinMeasurementValuesSurveys1985} highlighted that the use of ranking techniques in surveys can be time-consuming and potentially more costly to administer. These challenges are compounded when ranking numerous items, requiring substantial cognitive sophistication and concentration from survey respondents \cite{featherMeasurementValuesEffects1973}.

Notable applications of Quadratic Voting include the $2019$ Colorado House, which considered $107$ bills~\cite{coyNewWayVoting2019}, and the $2019$ Taiwan Presidential Hackathon, which featured $136$ proposals~\cite{QuadraticVotingFrontend2022}; both used a single QV question with hundreds of options. Psychological and behavioral research highlights the importance of understanding how individuals navigate and benefit from new interfaces under long-list QS conditions. These empirical applications of QV suggest QS's potential to elicit individual preferences, emphasizing the need to study cognitive load and interface design.

%  Removed text and citations
% These influences can even increase dropouts due to additional burdens on survey respondents~\textcite{galesicDropoutsWebEffects2006} which reduces the urge for researchers to use QS.
%As \textcite{chengCanShowWhat2021} noted, it is essential to better understand how the number of options influences the usability of QS and to design interfaces that effectively support survey respondents.
% there is limited research on interfaces for Constant Sum surveys~\cite{hauserIntensityMeasuresConsumer1980a}, a mechanism similar to QS that aims to elicit both ranking and rating preferences from individuals.
%The closest work discussing interfaces for QV is an arXiv paper~\cite{} that transformed the knapsack voting platform developed by \textcite{goelKnapsackVotingVoting}
% While both fields have deep insights into understanding design's influence on attitude elicitation, QS's unique capability of supporting both ranking and rating~\cite{chengCanShowWhat2021} makes designing an interface important and challenging. Subsequently, this research aims to understand how this interface influences an individual's QS response behavior. Requiring the distribution of budgets following the quadratic mechanism introduces new and complex decisions. 
% Empirical studies and applications of the quadratic mechanism and quadratic voting have increased in the past few years. Several studies have explored the empirical use cases for QV, including \textcite{quarfoot2017quadratic}'s study on 4,500 participants' attitudes across ten public policies, highlighting differences between QV and Likert scale survey results. \textcite{chengCanShowWhat2021} applied quadratic surveys in Human-Computer Interaction (HCI) and subsequently showed QV's effectiveness in reflecting true preferences in monetary decision tasks. \textcite{naylor2017first} used QV in educational research to gauge student opinions on factors affecting university success, and \textcite{cavailleWhoCaresMeasuring} examined QV in polarized choice scenarios.