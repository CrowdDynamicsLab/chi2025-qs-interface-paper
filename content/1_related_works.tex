\section{Related works}
\label{sec:relatedWorks}
This research sits at the intersection of three core areas: quadratic surveys, survey and voting interface design, and choice overload and its cognitive challenges. In this section, we review related works in each of these areas.

\subsection{Quadratic Survey and the Quadratic Mechanism}
We introduce the term \textbf{Quadratic Survey (QS)} to describe surveys that use the Quadratic Mechanism to collect individual attitudes. The~\textbf{Quadratic Mechanism} is a theoretical framework designed to encourage truthful revelation of individual preferences through a quadratic cost function~\cite{grovesOptimalAllocationPublic1977}. This framework gained popularity through~\textbf{Quadratic Voting (QV)}, also known as plural voting, which uses a quadratic cost function in a voting framework to facilitate collective decision-making~\cite{lalley2016quadratic}. QV is not subject to Arrow's impossibility theorem, which states that no voting system can perfectly aggregate individual preferences without trade-offs~\cite{morreau2014arrow}, because it does not require aggregating rankings. ~\textcite{quarfoot2017quadratic} demonstrated that QV effectively gauges public opinions and mitigates the tyranny of the majority in traditional voting systems. Unlike traditional surveys that elicit either rankings~\textit{or} ratings, QS allows detailed presentations of~\textit{both} by casting multiple votes for or against options, incurring a quadratic cost.~\textcite{chengCanShowWhat2021} showed that this mechanism aligns individual preferences more accurately with their behaviors than Likert Scale surveys, especially in resource-constrained scenarios.

The concept of QS is rooted in While QV , QS adapts these strengths to encourage truthful preference expression in surveys.

To illustrate how QS works, we formally define the mechanism as follows: Each survey respondent is allocated a fixed budget, denoted as $B$, to distribute among various options. Participants can cast $n$ votes for or against each option~$k$. The cost~$c_k$ for each option $k$ is derived as:

\[c_k = n_k^2 \quad \text{where}\quad n_k \in \mathbb{Z}\]

The total cost of all votes must not exceed the participant's budget:

\[\sum_k c_k \leq B\]

Survey results are determined by summing the total votes for each option:

\[ \text{Total Votes for Option } k = \sum_{i=1}^{S} n_{i,k} \]

where $S$ is the total number of participants, and~$n_{i,k}$ is the number of votes cast by participant~$i$ for option~$k$. Each additional vote for each option increases the marginal cost linearly, encouraging participants to vote proportionally to their level of concern for an issue~\cite{posner2018radical}.

In recent years, empirical studies on QV have expanded to different domains~\cite{naylor2017first, cavailleWhoCaresMeasuring}. Applications based on the quadratic mechanism have also grown, such as Quadratic Funding, which redistributes funds based on outcomes from consensus made using the quadratic mechanism~\cite{buterinFlexibleDesignFunding2019a, freitasQuadraticFundingIncomplete2024}. Recent work by \textcite{southPluralManagement2024} applies the quadratic mechanism to networked authority management, later used in Gov4git~\cite{Gov4gitDecentralizedPlatform2023}. However, despite the growth in depth and breath of applications using the quadratic mechanism, little attention has been given to the user experience and interface design that support individuals in expressing their preference intensity.

\subsection{Survey, Questionnaire, and Voting Design}
The relative lack of research in quadratic mechanism and QS interface design is concerning, as prior research in survey and questionnaire interfaces demonstrated substantial impact on the response and individual's experience on even seemingly minor design decisions. 

Research in the marketing and research community studying survey and questionnaire design, usability, and interactions focuses on understanding the infuence of styles and question presentation, or `Response Format,' of a survey or questionnaire. ~\textcite{weijtersExtremityHorizontalVertical2021} demonstrated that horizontal distances between options are more influential than vertical distances, with the latter recommended for reduced bias. Slider bars, which operate on a drag-and-drop principle, show lower mean scores and higher nonresponse rates compared to buttons, indicating they are more prone to bias and difficult to use. In contrast, visual analogue scales that operate on a point-and-click principle perform better~\cite{toepoelSlidersVisualAnalogue2018} These research highlighted outcomes are influenced by the different designs.

Given that voting, similiar to surveys and questionnaires are designed to elicit individual choices, we turn to voting interface literature. Voting interfaces can have an even more substantial influence on behaviors and outcomes.The notorious butterfly ballot~\cite{wandButterflyDidIt2001} is one example of this -- \textcite{wandButterflyDidIt2001} argue that an atypical ballot design may have caused enough accidental votes to swing the 2000 U.S. Presidential Election. Researchers like~\textcite{engstrom2020politics},~\textcite{chisnellDemocracyDesignProblem2016}, and organizations like the Center for Civic Design, which publishes reports like "Designing Usable Ballots"~\cite{DesigningUsableBallots2015}, stress the importance of interface design and how it can influence democratic processes. We group this literature into three main categories: designs that shifted voter decisions, designs that influenced human errors, and designs that incorporated technologies to improve usability.

\paragraph{Designs that shifted voter decisions: } For example, states without the option for straight-party ticket voting (the option to circle an option that votes for all the candidates in the same party) exhibited higher rates of split-ticket voting~\cite{engstrom2020politics}. Another example from the Australian ballot with an office block and no party box (having a box that clearly segments the position that the candidates are competing for) has been shown to enhance incumbency advantages.
\paragraph{Designs that influenced errors: } Butterfly ballots increased voter errors because voters could not correctly identify the punch hole on the ballot. Splitting contestants across columns increases the chance for voters to overvote~\cite{quesenberyOpinionGoodDesign2020}. On the other hand, \textcite{everettElectronicVotingMachines2008} showed the use of incorporating physical voting behaviors, like lever voting, into GUI interfaces.
% TODO: maybe state how this influences error rates, Since what we want to get across is "design of voting interfaces matters," maybe state the different impacts of each of these (if you can do so succinctly)
\paragraph{Designs that incorporated technologies: } Other projects like the Caltech-MIT Voting Technology Project have sparked research to address accessibility challenges, resulting in innovations like EZ Ballot~\cite{leeUniversalDesignBallot2016}, Anywhere Ballot~\cite{summers2014making}, and Prime III~\cite{dawkinsPrimeIIIInnovative2009}. In addition, \textcite{gilbertAnomalyDetectionElectronic2013} investigated optimal touchpoints on voting interfaces, and \textcite{conradElectronicVotingEliminates2009} examined zoomable voting interfaces.

While the design of voting systems and question response format markedly influence voter behavior and decision accuracy, these interface elements also directly impact the cognitive load on users. An effective design would enhance usability and reduce cognitive challenges faced by survey respondents, especially in complex response mechanisms like QS.
% TODO above paragraph: vkoshy2: do we have any citations for this? Or examples of higher/lower cognitive load interfaces? You might also want to state why cognitive load is a useful metric for surveyors to think about in the first place rather than error rate or something else. Like its clear why a survey participant would care (low cognitive load = easier), but why would a designer care?

\subsection{Cognitive Challenges and Choice Overload}
Despite the deep insight prior research learned about voting and surveying techniques and the robust mechanism demonstrated in theoretical applications of quadratic mechanisms, the inherent challenge that survey respondents are required to make many difficult decisions poses a unique cognitive challenge that no prior literature has tackled.~\textcite{lichtensteinConstructionPreference2006} laid out the three key elements that make decisions difficult. They include people making decisions within an unfamiliar context, people forced to make tradeoffs due to conflicts among choices, and people quantifying values for their opinions. QS fits into the description of all three elements, as participants can face options placed by the decision maker which they have never seen before. Participants are bounded by budgets that force them to make tradeoffs, and the final votes are presented in values. Hence, we believe that QS introduces a high cognitive load.

\textcite{daniel2017thinking} demonstrated that cognitive overload can adversely affect performance, for instance, causing individuals to rely more on heuristics rather than engaging in deliberate and logical decision-making. In addition, some researchers believe that preferences are constructed in situ just as memories are. Thus, when too much information is presented to an individual, they can `satisfice' their decisions~\cite{simonBehavioralModelRational1955, payneAdaptiveStrategySelection1988, tverskyJudgmentsRepresentativeness}. This behavior refers to when an individual settles on a `good enough' solution rather than `optimal' response. This overload can happen because of the presence of too many options. Subsequently, too many options can lead to individuals feeling overloaded, leading to decision paralysis, demotivation, and dissatisfaction~\cite{iyengarWhenChoiceDemotivating2000}.

Additionally, \textcite{alwinMeasurementValuesSurveys1985} highlighted that the use of ranking techniques in surveys can be time-consuming and potentially more costly to administer. These challenges are compounded when there are numerous items to rank, requiring substantial cognitive sophistication and concentration from survey respondents \cite{featherMeasurementValuesEffects1973}.

However, in several notable applications of Quadratic Voting in society, there can be hundreds of options within a single QV question. For instance, the 2019 Colorado House of Representatives considered 107 bills \cite{coyNewWayVoting2019}, and the 2019 Taiwan Presidential Hackathon featured 136 proposals \cite{QuadraticVotingFrontend2022}. These psychological and behavioral research highlighted the importance of understanding how individuals navigate and can potentially benefit from interfaces under long-list QS conditions.

% TODO: hari sundaram: You need a concluding para: what is the takeaway message from this section? How does the prior work  support the need for work in this paper?

%As \textcite{chengCanShowWhat2021} noted, it is essential to better understand how the number of options influences the usability of QS and to design interfaces that effectively support survey respondents.
% there is limited research on interfaces for Constant Sum surveys~\cite{hauserIntensityMeasuresConsumer1980a}, a mechanism similar to QS that aims to elicit both ranking and rating preferences from individuals.
%The closest work discussing interfaces for QV is an arXiv paper~\cite{} that transformed the knapsack voting platform developed by \textcite{goelKnapsackVotingVoting}
% While both fields have deep insights into understanding design's influence on attitude elicitation, QS's unique capability of supporting both ranking and rating~\cite{chengCanShowWhat2021} makes designing an interface important and challenging. Subsequently, this research aims to understand how this interface influences an individual's QS response behavior. Requiring the distribution of budgets following the quadratic mechanism introduces new and complex decisions. 
% Empirical studies and applications of the quadratic mechanism and quadratic voting have increased in the past few years. Several studies have explored the empirical use cases for QV, including \textcite{quarfoot2017quadratic}'s study on 4,500 participants' attitudes across ten public policies, highlighting differences between QV and Likert scale survey results. \textcite{chengCanShowWhat2021} applied quadratic surveys in Human-Computer Interaction (HCI) and subsequently showed QV's effectiveness in reflecting true preferences in monetary decision tasks. \textcite{naylor2017first} used QV in educational research to gauge student opinions on factors affecting university success, and \textcite{cavailleWhoCaresMeasuring} examined QV in polarized choice scenarios.