\section{Related Work}
\label{sec:relatedWorks}
This research lies at the intersection of three core areas: quadratic surveys, survey and voting interface design, and choice overload along with cognitive challenges. In this section, we review the related works in each of these areas.

\subsection{Quadratic Survey and the Quadratic Mechanism}
We introduce the term \textbf{Quadratic Survey (QS)} to describe surveys that utilize the quadratic mechanism to collect individual attitudes. The~\textbf{quadratic mechanism} is a theoretical framework designed to encourage the truthful revelation of individual preferences through a quadratic cost function~\cite{grovesOptimalAllocationPublic1977}. This framework gained popularity through~\textbf{Quadratic Voting (QV)}, also known as plural voting, which uses a quadratic cost function in a voting framework to facilitate collective decision-making~\cite{lalley2016quadratic}.%~\textcite{quarfoot2017quadratic} demonstrated that QV effectively gauges public opinion and mitigates the tyranny of the majority in traditional voting systems. Furthermore, QV is not subject to Arrow's impossibility theorem, which states that no voting system can perfectly aggregate individual preferences without trade-offs~\cite{morreau2014arrow}, because it does not require aggregating rankings.  

To illustrate how QS works, we formally define the mechanism: each survey respondent is allocated a fixed budget, denoted by $B$, to distribute among various options. Participants can cast $n$ votes for or against option~$k$. The cost~$c_k$ for each option $k$ is derived as:

\[c_k = n_k^2 \quad \text{where}\quad n_k \in \mathbb{Z}\]

The total cost of all votes must not exceed the participant's budget:

\[\sum_k c_k \leq B\]

Survey results are determined by summing the total votes for each option:

\[ \text{Total Votes for Option } k = \sum_{i=1}^{S} n_{i,k} \]

where $S$ represents the total number of participants, and~$n_{i,k}$ is the number of votes cast by participant~$i$ for option~$k$. Each additional vote for each option increases the marginal cost linearly, encouraging participants to vote proportionally to their level of concern for an issue~\cite{posner2018radical}.

QS adapts these strengths of the quadratic mechanism in \textit{voting} to encourage truthful expression of preferences in \textit{surveys}. Unlike traditional surveys that elicit either rankings~\textit{or} ratings, QS allows for~\textit{both}, enabling participants to cast multiple votes for or against options, incurring a quadratic cost.~\textcite{chengCanShowWhat2021} showed that this mechanism aligns individual preferences with behaviors more accurately than Likert Scale surveys, particularly in resource-constrained scenarios like prioritizing user feedback on user experiences.

In recent years, empirical studies on QV have expanded into various domains~\cite{naylor2017first, cavailleWhoCaresMeasuring}. Applications based on the quadratic mechanism have also grown, including Quadratic Funding, which redistributes funds based on outcomes from consensus made using the quadratic mechanism~\cite{buterinFlexibleDesignFunding2019a, freitasQuadraticFundingIncomplete2024}. Recent work by \textcite{southPluralManagement2024} applies the quadratic mechanism to networked authority management, later used in Gov4git~\cite{Gov4gitDecentralizedPlatform2023}. Despite the increasing breadth and depth of applications utilizing the quadratic mechanism, little attention has been paid to user experience and interface design, which support individuals in expressing their preference intensity. Our work aims to address this by designing interfaces supporting quadratic mechanisms.

\subsection{Design Implications for Surveys, Questionnaires, and Voting Systems}
The relative lack of research in quadratic mechanism and QS interface design is concerning, as prior work on survey and questionnaire interfaces has demonstrated substantial impacts on responses and user experiences, even with seemingly minor design decisions.

Research in the marketing and research communities focusing on survey and questionnaire design, usability, and interactions examines the influence of presentation styles and `response format.'~\textcite{weijtersExtremityHorizontalVertical2021} demonstrated that horizontal distances between options are more influential than vertical distances, with the latter recommended for reduced bias. Slider bars, which operate on a drag-and-drop principle, show lower mean scores and higher nonresponse rates compared to buttons, indicating they are more prone to bias and difficult to use. In contrast, visual analog scales that operate on a point-and-click principle perform better~\cite{toepoelSlidersVisualAnalogue2018}. These prior works highlighted outcomes that are influenced by the different designs.

Voting interfaces, like surveys and questionnaires, elicit individual choices, but often with consequential impacts. A well-known example is the butterfly ballot, whose atypical ballot design may have influenced the outcome of the 2000 U.S. Presidential Election.~\cite{wandButterflyDidIt2001} Researchers like~\textcite{engstrom2020politics},~\textcite{chisnellDemocracyDesignProblem2016}, and organizations like the Center for Civic Design, which publishes reports like ``Designing Usable Ballots''~\cite{DesigningUsableBallots2015}, emphasize how ballot design influences democratic processes.

Existing research surfaced how various voting interface designs shifted voter decisions, influenced human errors, or improved usability through technologies. For instance, states without straight-party voting exhibited higher rates of split-ticket voting~\cite{engstrom2020politics}, and Australian ballots, which list candidates by office without party labels, often give incumbents an advantage. Poor designs, like the butterfly ballot, have led to voter errors due to confusion over punch holes, and splitting contestants across columns increases the likelihood of overvoting~\cite{quesenberyOpinionGoodDesign2020}. \textcite{everettElectronicVotingMachines2008} further explored how digital voting interfaces improve usability over physical voting behaviors, such as lever voting. Other projects like the Caltech-MIT Voting Technology Project have sparked research to address accessibility challenges, resulting in innovations like EZ Ballot~\cite{leeUniversalDesignBallot2016}, Anywhere Ballot~\cite{summers2014making}, and Prime III~\cite{dawkinsPrimeIIIInnovative2009}. In addition, \textcite{gilbertAnomalyDetectionElectronic2013} investigated optimal touchpoints on voting interfaces, and \textcite{conradElectronicVotingEliminates2009} examined zoomable voting interfaces for improved user interactability.

The design of Voting systems and response formats significantly influence respondent behavior, decision accuracy, and cognitive load. Research like~\textcite{galesicDropoutsWebEffects2006} showed that the burden on survey respondents increases dropouts. An effective design would enhance usability and reduce cognitive challenges faced by survey respondents, especially in complex response mechanisms like QS.

\subsection{Cognitive Challenges and Choice Overload}
Despite insights from studies on quadratic mechanisms, voting, and surveying techniques, the challenge of respondents making difficult decisions using quadratic mechanisms remains unexplored in the literature.~\textcite{lichtensteinConstructionPreference2006} identified three key elements that make decisions difficult. These elements include making decisions in unfamiliar contexts, being forced to make tradeoffs due to conflicting choices, and quantifying the value of one's opinions. QS fits all three elements: participants may encounter unfamiliar options set by the decision maker, are constrained by budgets that require tradeoffs, and cast final votes as numerical values. Thus, we believe QS introduces high cognitive load.

Cognitive overload can adversely affect performance, leading individuals to rely on heuristics rather than deliberate, logical decision-making~\cite{daniel2017thinking}. When presented with excessive information, such as too many options, individuals 'satisfice', settling for a 'good enough' solution rather than an optimal one~\cite{simonBehavioralModelRational1955, payneAdaptiveStrategySelection1988, tverskyJudgmentsRepresentativeness}. Subsequently, too many options can overwhelm individuals, resulting in decision paralysis, demotivation, and dissatisfaction~\cite{iyengarWhenChoiceDemotivating2000}.

Additionally,~\textcite{alwinMeasurementValuesSurveys1985} highlighted that the use of ranking techniques in surveys can be time-consuming and potentially more costly to administer. These challenges are compounded when ranking numerous items, requiring substantial cognitive sophistication and concentration from survey respondents \cite{featherMeasurementValuesEffects1973}.

Notable applications of Quadratic Voting include the $2019$ Colorado House, which considered $107$ bills~\cite{coyNewWayVoting2019}, and the $2019$ Taiwan Presidential Hackathon, which featured $136$ proposals~\cite{QuadraticVotingFrontend2022}; both used a single QV question with hundreds of options. Psychological and behavioral research highlights the importance of understanding how individuals navigate and benefit from new interfaces under long-list QS conditions. These empirical applications of QV suggest QS's potential to elicit individual preferences, emphasizing the need to study cognitive load and interface design.

%As \textcite{chengCanShowWhat2021} noted, it is essential to better understand how the number of options influences the usability of QS and to design interfaces that effectively support survey respondents.
% there is limited research on interfaces for Constant Sum surveys~\cite{hauserIntensityMeasuresConsumer1980a}, a mechanism similar to QS that aims to elicit both ranking and rating preferences from individuals.
%The closest work discussing interfaces for QV is an arXiv paper~\cite{} that transformed the knapsack voting platform developed by \textcite{goelKnapsackVotingVoting}
% While both fields have deep insights into understanding design's influence on attitude elicitation, QS's unique capability of supporting both ranking and rating~\cite{chengCanShowWhat2021} makes designing an interface important and challenging. Subsequently, this research aims to understand how this interface influences an individual's QS response behavior. Requiring the distribution of budgets following the quadratic mechanism introduces new and complex decisions. 
% Empirical studies and applications of the quadratic mechanism and quadratic voting have increased in the past few years. Several studies have explored the empirical use cases for QV, including \textcite{quarfoot2017quadratic}'s study on 4,500 participants' attitudes across ten public policies, highlighting differences between QV and Likert scale survey results. \textcite{chengCanShowWhat2021} applied quadratic surveys in Human-Computer Interaction (HCI) and subsequently showed QV's effectiveness in reflecting true preferences in monetary decision tasks. \textcite{naylor2017first} used QV in educational research to gauge student opinions on factors affecting university success, and \textcite{cavailleWhoCaresMeasuring} examined QV in polarized choice scenarios.