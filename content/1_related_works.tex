\section{Related works}
\subsection{Quadratic Voting}
Quadratic Survey is a surveying technique derived from the quadratic voting mechanism. Quadratic voting (QV) is a method for collective decision-making where individuals can express their preferences' intensity by casting multiple votes at a quadratic cost. This approach, developed by \textcite{posner2018radical, lalley2018quadratic}, aims to mitigate the tyranny of the majority inherent in traditional one-person-one-vote systems. QV is not subject to Arrow's impossibility theorem as it does not require individuals to aggregate rankings of preferences. Quadratic Surveys adapt this mechanism for survey contexts, with a modification allowing participants to vote for or against an option, effectively presenting two distinct choices in the same survey. This adaptation was utilized by \textcite{quarfoot2017quadratic} and implemented as an open-source platform by \textcite{bassettiCivicbaseOpensourcePlatform2023}. While these studies did not explicitly label this as 'quadratic survey', we will use this term to differentiate it from the voting mechanism.

Here we again formally define QV. In a scenario where $S$ participants are involved, each participant is allocated a fixed quantity of voice credits, denoted as $B$. These credits can be distributed among various options. Importantly, each individual has the ability to cast multiple votes, either in favor or against each option. However, this voting system incorporates a quadratic cost for voting: casting $n_k$ votes for a particular option $k$ incurs a cost $c(n_k)$, which is proportional to $n_k^2$. Consequently, the aggregate cost in voice credits for all options chosen by a participant must not exceed their allocated budget $B$. This necessitates that the sum of the squares of votes cast for each option ($\sum_k n_k^2$) remains within the limit of $B$, where $n_k$ represents the number of votes allocated to option $k$. Finally, the survey administrators compile and analyze the results by summing up the total votes cast by all participants for each option. This design allos the marginal cost to cast $1$ additional vote to linearly increase with the number of votes already cast on that option, inducing rational participants to vote proportionally to how much they care about an issue~\cite{posner2018radical}. This design is why, unlike many traditional voting methods, each QV vote comes with a quadratic cost.

Although empirical studies on QS are increasing, we are unaware of any peer-reviewed research focusing on its user experience and interface design. Several studies have explored the empirical use of QV in various contexts. \textcite{quarfoot2017quadratic} surveyed 4500 participants on ten public policies, finding differences between QV and Likert scale survey results. \textcite{chengCanShowWhat2021} applied quadratic surveys in Human-Computer Interaction (HCI), showing QV's effectiveness in reflecting true preferences in monetary decision tasks. \textcite{naylor2017first} used QV in educational research to gauge student opinions on factors affecting university success, and \textcite{cavailleWhoCaresMeasuring} examined QV in polarized choice scenarios. As highlighted by \textcite{quarfoot2017quadratic}, further investigation is needed to enhance user experience in QV and explore its potential.

\subsection{Survey, Voting, and QV Design}
To emphasize the significance of interface design and understanding user experience, we reviewed prior literature on the influence of systems in surveying and voting. \textcite{engstrom2020politics} discussed how ballot design significantly affects election outcomes. For instance, states without the option for straight party ticket voting exhibited higher rates of split-ticket voting. Additionally, the Australian ballot with an office block and no party box has been shown to enhance incumbency advantages. Such examples demonstrate the impact of nudging techniques on voter decisions. Poor ballot design also leads to increased error rates, as evidenced by the notorious butterfly ballots~\cite{wandButterflyDidIt2001}. Subsequently, projects like the Caltech-MIT Voting Technology Project have sparked research to address accessibility challenges, resulting in innovations like EZ Ballot~\cite{leeUniversalDesignBallot2016}, Anywhere Ballot \cite{summers2014making}, and Prime III\cite{dawkinsPrimeIIIInnovative2009}. Other research focuses on digital challenges, such as Electronic Voting interface Challenges~\cite{herrnsonEVALUATIONMARYLANDNEW2003} and error reduction~\cite{everettElectronicVotingMachines2008}. For example, \textcite{everettElectronicVotingMachines2008} explored how Graphical User Interfaces (GUI) can reduce errors by incorporating physical voting behaviors, like lever voting, into GUI interfaces. \textcite{gilbertAnomalyDetectionElectronic2013} investigated optimal touchpoints on voting interfaces, and \textcite{conradElectronicVotingEliminates2009} examined zoomable voting interfaces. This research underscores the importance and multifaceted challenges that voting interfaces can have on voting behavior. However, little of this research directly translates to the Quadratic Voting (QV) mechanism, which requires complex decisions involving both ranking and rating within a budget constraint. Thus, we turn to related works from the survey and questionnaire community.

The marketing and research community studying survey and questionnaire design, usability, and interaction often refers to `response format' as the format in which responses are gathered. Various studies have shown that different designs of response formats can influence outcomes. For example, \textcite{weijtersExtremityHorizontalVertical2021} demonstrated that horizontal distances between options are more influential than vertical distances, with the latter recommended for reduced bias. Slider bars, which operate on a drag-and-drop principle, show lower mean scores and higher nonresponse rates compared to buttons, indicating they are more prone to bias and difficult to use. In contrast, visual analogue scales that operate on a point-and-click principle perform better~\cite{toepoelSlidersVisualAnalogue2018}. While these insights are valuable for enhancing QS design, there is limited research on interfaces for Constant Sum surveys~\cite{hauserIntensityMeasuresConsumer1980a}, a mechanism similar to QS that aims to elicit both ranking and rating preferences from individuals.

Although there are several empirical studies on QV, to our knowledge, only a recent arXiv paper has examined the user experience of QV respondents based on the knapsack voting platform developed by \textcite{goelKnapsackVotingVoting} and tested several modifications. Given the critical influence of interface and response format on survey respondents' behavior, it is imperative to adopt a critical lens for the design of QS and to provide adequate scaffolding for respondents.

\subsection{Cognitive Challenges and Choice Overload}
Since Quadratic Surveys (QS) involve survey respondents expressing relative preferences across a set of options, the primary challenge they face is a high cognitive load. Previous studies have demonstrated that cognitive overload can adversely affect performance. Moreover, a high cognitive burden may lead individuals to rely more on heuristics rather than engaging in deliberate and logical decision-making \cite{daniel2017thinking}. \textcite{iyengarWhenChoiceDemotivating2000} found that when participants are presented with multiple options, they often experience choice overload, leading to decision paralysis, demotivation, and dissatisfaction.

Additionally, \textcite{alwinMeasurementValuesSurveys1985} highlighted that the use of ranking techniques in surveys can be time-consuming and potentially more costly to administer. These challenges are compounded when there are numerous items to rank, requiring substantial cognitive sophistication and concentration from survey respondents \cite{featherMeasurementValuesEffects1973}.

However, in several notable applications of Quadratic Voting in society, there can be hundreds of options within a single QV question. For instance, the 2019 Colorado House of Representatives considered 107 bills \cite{NewWayVoting}, and the 2019 Taiwan Presidential Hackathon featured 136 proposals \cite{QuadraticVotingFrontend2022}. As \textcite{chengCanShowWhat2021} noted, it is essential to better understand how the number of options influences the usability of QS and to design interfaces that effectively support survey respondents.