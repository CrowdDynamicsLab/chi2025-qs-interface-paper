\begin{figure}[ht]
    \centering
    \begin{subfigure}[b]{0.3\textwidth}
        \centering
        \includegraphics[width=\textwidth]{content/image/curr_interface/radical_market_wedesign.png}
        \caption{Software designed by WeDesign used in~\cite{quarfoot2017quadratic}. Image taken from~\cite{posner2018radical}.}
        \label{fig:wedesignInterface}
    \end{subfigure}
    \hfill
    \begin{subfigure}[b]{0.3\textwidth}
        \centering
        \includegraphics[width=\textwidth]{content/image/curr_interface/geek.sg_interface.png}
        \caption{An open-source QV interface~\cite{yehjxraymondYehjxraymondQvapp2024} with a publicly available service.}
        \label{fig:yehInterface}
    \end{subfigure}
    \hfill
    \begin{subfigure}[b]{0.3\textwidth}
        \centering
        \includegraphics[width=\textwidth]{content/image/curr_interface/rxc_interface.png}
        \caption{An open-source QV interface~\cite{RadicalxChangeQuadraticvoting2024} forked from GitCoin~\cite{ReadWhitepaperGitcoin} used by the RadicalxChange community~\cite{RxC}.}
        \label{fig:rxcvotingInterface}
    \end{subfigure}
    \vskip\baselineskip
    \begin{subfigure}[b]{0.3\textwidth}
        \centering
        \includegraphics[width=\textwidth]{content/image/curr_interface/appvote.png}
        \caption{The interface designed for gov4git~\cite{Gov4gitDecentralizedPlatform2023}.}
        \label{fig:gov4gitInterface}
    \end{subfigure}
    \begin{subfigure}[b]{0.3\textwidth}
        \centering
        \includegraphics[width=\textwidth]{content/image/curr_interface/cheng_qv.png}
        \caption{The interface used in the research by~\textcite{chengCanShowWhat2021}.}
        \label{fig:chengInterface}
    \end{subfigure}
    \caption{Recent implementations of interfaces applying the quadratic mechanism.}
    \label{fig:qv_interface_external}
\end{figure}

\section{Interface Design}
\label{sec:interfaceDesign}
In this study, we developed an interactive interface for QS based on prior literature for the experiment condition. Since there were no studies and standard interfaces associated with quadratic survey-based tools, we designed and constructed two versions of the interface to study how interactive components influenced participants' cognitive load and behaviors. In the following subsections, we describe and justify these interface design decisions.

\subsection{Text-based Interface}
First, we surveyed the current implementation of QV interfaces to understand the development of such tools. We presented a selection in Figure~\ref{fig:qv_interface_external}. All five interfaces retained and presented the following components:
\begin{itemize}
    \item Option list: A list of options contesting for votes.
    \item Vote Controls: Two buttons to increase and decrease votes associated with each option.
    \item Individual vote tally: A representation of votes associated with an option.
    \item Summary: A summary that automatically calculated the cost across options and the remaining budget.
\end{itemize}

We constructed a text-based interface that included all five components but removed the use of emojis (i.e., thumbs up and thumbs down present in Figure~\ref{fig:wedesignInterface}), progress bars, and other visualizations in the summary section (i.e., progress bars in Figure~\ref{fig:wedesignInterface} and~\ref{fig:chengInterface} or blocks presented in Figure~\ref{fig:rxcvotingInterface}), and the visual cues for individual vote counts (i.e., the colored counts and icons present in Figure~\ref{fig:gov4gitInterface} and~\ref{fig:chengInterface}).

Prior literature suggested that the use of emojis might influence the interpretations of surveys~\cite{herringGenderAgeInfluences2020} and decrease user satisfaction~\cite{toepoelSmileysStarsHearts2019}. Prior literature also noted that not all data visualization elements reduce cognitive demand~\cite{huangMeasuringEffectivenessGraph2009a}. Even though effective visualization can aid decision making, it remains an open question that this study does not aim to address, thus we also removed all visualization elements such as blocks, progress bars, and percentage indicators. Lastly, different from all these interfaces, we decided to present all the options on the same screen. Prior research emphasized the importance of placing all the options on the same digital ballot screen to avoid losing votes (missing citations). This echoes the proverb "out of sight, out of mind," where individuals might be biased toward options that are shown to them, and additional effort is required for individuals to retrieve specific information if options are hidden (citation needed).

These design decisions led to the interface shown in Figure~\ref{fig:textInterface}. The interface contained the question prompt at the top of the screen. The options were presented in the list underneath the prompt. Survey respondents could update the votes by selecting from a dropdown that provided all possible voting options and cost given the number of credits. A small summary box to the right of the interface showed the current total cost and the remaining credits for the respondent. Options were always randomly presented on the interface to avoid ordering bias~\cite{ferberOrderBiasMail1952, couperWebSurveyDesign2001}.

\begin{figure}[H]
    \centering
    \includegraphics[width=0.6\textwidth]{content/image/text_interface.png}
    \caption{The text-based interface}
    \label{fig:textInterface}
\end{figure}

\subsection{Interactive Interface}
The design objective for the interactive interface was to facilitate preference construction and reduce cognitive load. The interactive interface, shown in Figure~\ref{fig:interactiveInterface}, built additional interactive elements on top of the text interface to maintain consistency that allowed comparison of the direct manipulation of the designed interactive elements. We designed two additional components: An additional organization step prior to voting and a drag-and-drop interface throughout the QS responding session informed through prior literature.

\paragraph{A two phase approach}

If preferences are constructed, by nature, they consist of a series of constructed decision-making processes~\cite{lichtensteinConstructionPreference2006}. Two major decision-making theories informed the design decision of a two-step interaction interface design:~\textcite{montgomeryDecisionRulesSearch1983}'s Search for a Dominance Structure Theory (Dominance Theory) and~\textcite{svensonDifferentiationConsolidationTheory1992}'s Differentiation and Consolidation Theory (Diff-Con Theory). The former suggested that decision-makers prioritize creating dominant choices to minimize cognitive effort by focusing on evidently superior options~\cite{montgomeryDecisionRulesSearch1983}. The latter described a two-phase process where decisions are formed by initially ~\textit{differentiating} among alternatives and then~\textit{consolidating} these distinctions to form a stable preference~\cite{svensonDifferentiationConsolidationTheory1992}. Both theories guided the design decision in building the interactive experience to reduce initial decision dimensions and the mental procedures involved in emphasizing relatively important options and forming decisions.

Hence, the two-phase design -- organize then vote -- aimed to facilitate this cognitive journey explicitly. The first phase focused on differentiating and identifying dominant options, enabling survey respondents to preliminarily categorize and prioritize their choices. The second phase presented these categorized options in a comparable manner, with drag-and drop functionality, enhancing one's ability to consolidate preferences. This structured approach aimed to construct a clear decision-making procedure that reduced cognitive load and enhanced clarity and confidence in the decisions made.

\paragraph{Phase 1: Organization Phase}
The goal of the organization phase was to support participants in identifying dominating options or partitioning options into differentiable groups. In this section, we first describe how the interaction worked, then we detail reasons for the different design decisions implemented.

The organizing interface, depicted on the left side of Figure~\ref{fig:interactiveInterface}, sequentially presented each survey option. Participants selected a response among three ordinal categories -- lean positive, lean negative, or lean neutral. Once selected, the system moved that option to the respective category. Participants could skip the option if they did not want to indicate a preference. Options within the groups were draggable and rearrangeable to other groups should the participants wish.

\textcite{strackThinkingJudgingCommunicating1987}'s research showed that upon understanding a survey question, respondents either recalled a prior judgment or constructed a new one when completing an attitude survey. In addition, revealing one option at a time gated the amount of information presented to the survey respondent and thereby reduced the extraneous load~\cite{swellerCognitiveLoadTheory2011}. This process allowed participants to form or express opinions on individual options incrementally.

The three possible options, positive, neutral, and negative, aimed to scaffold participants in constructing their own choice architecture~\cite{munscherReviewTaxonomyChoice2016, thalerNudgeImprovingDecisions2008a}, which strategically segmented options into diverse and alternative choice presentations while avoiding the biases from defaults. We believed that these three categories were sufficient for participants to segment the options. However, we chose not to limit the number of options one could place into a category to prevent restricting user agency, a core user interface design principle~\cite{norman2013design}.

Immediate feedback displaying the placement of options and allowing participants to rearrange them via drag-and-drop adhered to key interface design principles~\cite{norman2013design}. At the same time, it allowed finer grain control for individuals to surface dominating options and create differentiating groups of options.

This design underwent paper prototypes and various iterations, which all maintained the combination of these theoretical bases aimed at reducing cognitive load and scaffolding the decision-making process. We describe these iterations and the design process in Appendix A.
\begin{figure}[ht]
    \centering
    \includegraphics[width=1\textwidth]{content/image/interface.png}
    \caption{The interactive interface}
    \label{fig:interactiveInterface}
\end{figure}

\paragraph{Phase 2: Interactive Voting Phase}

The objective of the voting phase was to facilitate the consolidation of differentiated options through interactive elements while reinforcing the differentiation across options constructed by participants from the previous phase. This facilitation was achieved by retaining the drag-and-drop functionality for direct manipulation of position and enabling sorting within each category.

Options were displayed as they were categorized within each category from the previous step and in the following section orders -- lean positive, lean neutral, lean negative, and skipped or undecided as detailed on the right-hand side of Figure~\ref{fig:interactiveInterface}. The Skipped or Undecided category contained options left in the organization queue, possibly because survey respondents had a pre-existing preference or chose not to organize their thoughts further. The original order within these categories was preserved to maintain and reinforce the differentiated options. Respondents had the flexibility to return to the organization interface at any point during the survey to revise their choices.

In the interactive interface, options remained draggable, enabling participants to modify or reinforce their preference decisions as needed. Each category featured a sort-by-vote function that enabled reordering within the same category. Although these interactions did not influence the final voting outcome, they were designed to support consolidation and positional proximity in information organization. This design aimed to automate the grouping of similar options while providing an intuitive drag-and-drop mechanism, thereby facilitating decision-making by placing similar options near each other. This echoed the principles of the proximity compatibility principle, particularly emphasizing spatial proximity and mental compatibility~\cite{wickens1990proximity}. The interface design anticipated that participants would find it easier to consolidate their choices when similar options were positioned close together, thereby reducing cognitive load.

While multiple interaction mechanisms exist, drag-and-drop has been extensively explored in rank-based surveys. For instance,~\textcite{krosnick2018measurement} demonstrated that replacing drag-and-drop with traditional number-filling rank-based questions improved participants' satisfaction with little trade-off in their time. Similarly,~\textcite{timbrook2013comparison} found that integrating drag-and-drop into the ranking process, despite potentially reducing outcome stability, was justified by the increased satisfaction and ease of use reported by respondents. The trade-off was deemed worthwhile as QS did not use the final position of options as part of the outcome if it significantly enhanced user satisfaction and usability~\cite{rintoulVisualAnimatedResponse}.

Together, these design decisions led to our belief that a two-step interactive interface with direct interface manipulation could reduce the cognitive load for survey respondents to form preference decisions when completing QS.