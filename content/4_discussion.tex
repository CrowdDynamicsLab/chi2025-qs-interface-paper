\section{Discussion and Future Works}
\label{sec:discussion}

In this section, we interpret the results related to cognitive load and survey respondent behaviors, emphasizing why the interactive interface did not uniformly reduce cognitive load in the long text interface while providing practical recommendations for practitioners deploying QS.

Our discussion centers on three key topics: elements of the two-phase interface that support preference construction, design recommendations for practitioners, and future challenges. Ultimately, we conclude that the two-phase interface has differential effects on the short and long surveys. While trends suggest a reduction in cognitive load with the two-phase interface compared to the text interface, we observe evidence of deeper engagement with options and enhanced preference construction, particularly in the long survey condition.

\subsection{Result Interpretation}
\subsubsection{Deeper engagements through preference construction in two-phase interfaces}

Our main findings indicate that the survey results, qualitative data, and observed behavioral differences reveal shifts in the types of cognitive load experienced by participants, especially for those completing the long survey. Cognitive load theory~\cite{swellerCognitiveLoadTheory2011}, when applied to the context of QS, identifies the three components of cognitive load: intrinsic load (the cognitive demand required to understand questions and response options), germane load (associated with deeper processing and evaluation of preferences), and extraneous load (stemming from navigating and operating the survey interface).

Participants are randomly assigned to experimental conditions, with both survey lengths containing options randomly drawn from a common pool to control intrinsic load within the same group.  

In the short survey condition, participants engage with all options simultaneously. The two-phase interface reduces some extraneous load associated with navigating the interface during voting, though it requires participants to complete the grouping phase. Despite this additional task, participants across both interface types report minimal or no physical demand. The two-phase interface likely facilitates easier engagement with preference construction due to its lower-trended cognitive load, as reflected in the increased likelihood of perceived lower cognitive load.  

In the long survey condition, participants cannot engage with all options simultaneously, resulting in a higher intrinsic load at the start of the survey. The organization phase in the two-phase interface shapes participant behavior during the voting phase. While it streamlines the process of locating options, as exemplified by the reduction of edit distance, this benefit may be offset by the additional physical effort required to complete the grouping phase, as reflected by the slightly increased physical demand.  

However, qualitative data regarding the voting task suggest that participants maintain their ability to invoke deeper engagement with options. Quantitative data reveal that participants make no fewer overall edits, with a bimodal distribution suggesting continued revision even at low budgets. Additionally, participants strategically consider broader options as they deliberate on nearby ones. These findings indicate a cognitive shift toward germane load, particularly during the voting phase.

In contrast, participants in the long text interface experience higher extraneous load, evident in shallower reflection and shorter overall voting times, despite exhibiting a greater overall edit distance. While some might argue that the additional grouping phase offers participants more opportunities to familiarize themselves with the options, the long edit distance suggests that participants in the text interface traverse the list frequently, providing ample opportunity to adjust their preferences. Qualitative data indicate that 70\% of long text participants (N=7) scan the list while voting, with edit distance data reflecting multiple passes across the list.

The deliberate one-option-at-a-time presentation during the voting task in the two-phase interface reduces reliance on defaults and encourages deeper reflection. This is best-illustrated by~\texttt{S013}, who emphasizes how the organization phase supports their preference construction:  
\begin{displayquote}  
\bracketellipsis it (organization phase) gives you time to just focus on that single thing and rank it based on how you feel at that moment. \hfill\quoteby{S013 (SI)}  
\end{displayquote}  

Thus, based on this evidence, we argue that a text-based interface is not an optimal solution for long QS where deeper engagement and preference construction are desired. A two-phase interface enables participants to effectively exercise germane load, fostering deeper engagement with the content.


% Conversely, participants using the text interface, like \texttt{S003}, described how default placements influenced their decisions:

% \begin{displayquote}
%     Honestly, if medical research~\bracketellipsis was the first one I saw, I think it would automatically give it a lot more. \hfill\quoteby{S003 (ST)}
% \end{displayquote}

% possiblity from the overall reduction of extraneous load. This additional benifit from added time and slightly higher cognitive load, at the minimum, prevented cognitive overload, , fostering reflection and deliberation which is relatively important when completing surveys.

% Based on the current evidence,  The two-phase interface, by supporting reflection and minimizing reliance on defaults, demonstrates the potential for more effective design strategies in similar contexts.

\subsubsection{Plausible satisficing behaviors in long QS}
In addition, the observed lower overall cognitive load in the long text interface may partly reflect~\textit{satisficing behaviors}. Satisficing refers to participants settling for \textit{good enough} rather than \textit{optimal} decisions~\cite{gigerenzerReasoningFastFrugal1996} when unable to process all available information. Interviews reveal that 40\% of participants (N=4) in this condition describe using satisficing strategies, while none from the long two-phase interface report such behaviors.These strategies are exemplified by participants prioritizing minimal effort over thorough evaluation, as illustrated by:

\begin{displayquote}
    ~\bracketellipsis you thought of enough things, you know, and so it wasn't the most effort I could put in because again, that would have been diminishing returns. I tried to think of enough things~\bracketellipsis and then move on.~\bracketellipsis 
    
    I felt like that (the response) was satisfied, but not perfect. Cause perfect is not a reality. \hfill\quoteby{S036 (ST)}
\end{displayquote}

This quote illustrates satisficing decision-making, where participants settle for suboptimal choices. Additional participants describe similar strategies when deciding on votes:  

\begin{displayquote}
    ~\bracketellipsis Because that was what was left. [Laughter] I probably wouldn't use that on <optionA> instead of <optionB>.~\bracketellipsis \hfill\quoteby{S015 (LT)}

    I tried to use them~\bracketellipsis it went negative, and then I just settled for just \$6 remaining. ~\bracketellipsis I don't think it's perfect. But I think I'm satisfied. Yeah, I'm satisfied.  \hfill\quoteby{S033 (LT)}

    ~\bracketellipsis when I had first started like looking at the first few, I was just doing it kinda like willy nilly, I'm not really paying that much attention to necessarily how many credits I had, or how many categories there were. \hfill\quoteby{S041 (LT)}
\end{displayquote}

These quotes highlight how participants in the long text interface adjust to external constraints rather than carefully weighing internal preferences. This behavior suggests that cognitive overload may lead participants to adopt less effortful strategies. However, further research is needed to fully understand the prevalence and impact of satisficing in long QS surveys.  

\textbf{In summary}, the two-phase interface likely reduces extraneous load, particularly in the long survey condition, facilitating a cognitive \textit{shift} toward deeper reflection and more deliberate decision-making. While the extent to which long QS surveys induce cognitive overload or satisficing remains unclear, the interactive interface shows promise in promoting deeper engagement with options and supporting comprehensive preference construction. The following section explores the specific elements that guide participants toward these outcomes.

% ============================== %
\subsection{Construction of Preference on Quadratic Survey}

Completing QS involves a series of difficult decision tasks~\textcite{lichtensteinConstructionPreference2006}. ~\textcite{svensonDifferentiationConsolidationTheory1992}'s differentiation and consolidation theory help explain how participants process these decisions. The decision process begins with differentiation, where participants identify differences and eliminate less favorable options, followed by consolidation, which strengthens their commitment to selected choices. This theory aligns with how the two-phase interface helps participants decompose options into categories, effectively reducing decision complexity.

Participants start by constructing preferences in situ, especially regarding options they have not previously considered:
\begin{displayquote}
    \bracketellipsis`Oh, there are other aspects that I never care about.' And actually~\ldots some people care <an option>. Sure. Why? Why (should) I spend money on that? \\\hfill\quoteby{S037 (LI)}
\end{displayquote}

Those using the text interface, lacking the interactive tools, find it challenging to facilitate differentiation, as~\texttt{S025} notes:

\begin{displayquote}
    I would like to be able to like, click and drag the categories themselves so I could maybe reorder them to like my priorities.~\bracketellipsis make myself categories and subcategories out of this list~\ldots If I could organize it. \hfill\quoteby{S025 (LT)}
\end{displayquote}

In contrast, the two-phase interface allows participants to express at least one dimension of differentiation more easily. The drag-and-drop feature helps blend this differentiation into the consolidation phase. Not only do participants drag-and-drop options post-voting to reflect and assure a correct vote allocation, but it also enables participants, like~\texttt{S039}, to make fine-grain comparisons between options:  

\begin{displayquote}  
    I think the system was actually really helpful because I could just drag them.~\bracketellipsis I can really compare them, I can drag this one up here, and then compare it to the top one~\bracketellipsis \hfill\quoteby{S039 (SI)}  
\end{displayquote}  

The bi-modal behavior observed in the long interactive interface participants aligns with the differentiation and consolidation framework, as described in the results. Participants in the two-phase interface begin differentiating options earlier, allowing them to later adjust fine-grain votes. The faster and smaller vote updates indicate participants are consolidating. The less prominent bi-modal behavior from the long text interface participants implies that the interface guides this decision framework, as participant~\texttt{037} explains:

\begin{displayquote}
    I only start from the positive one~\bracketellipsis I finish everything~\ldots and then I move to the second part (the neutral box).~\bracketellipsis I want to focus on these and make sure that resources are at least they get the attention they want. And if there's surplus and they can move to the second part. \hfill\quoteby{S037 (LI)}
\end{displayquote}

In addition, the three key elements of the organization phase—presenting options one at a time, grouping them into categories, and enabling drag-and-drop—work together to structure participant preferences. These elements align with cognitive strategies like~\textit{\smash{problem decomposition}}~\cite{simonSciencesArtificial1996} and~\textit{\smash{dimension reduction}}, which reduce cognitive overload. Bounded rationality highlights how cognitive limitations lead to sub-optimal decision-making due to the inability to process all available information~\cite{simonBehavioralModelRational1955}. It illuminates the importance of decision-making support interfaces rather than serving as a critique of human behaviors. One participant explains how the organization phase breaks down complex decisions into manageable steps:  

\begin{displayquote}  
\bracketellipsis being able to have a preliminary categorization of all the topics. First, it introduced me to all the topics,~\bracketellipsis to think about and process~\bracketellipsis being able to digest all the information prior to actually allocating the budget or completing the quadratic survey. \hfill\quoteby{S009 (LI)}  
\end{displayquote}  

Participants using the two-phase interface, especially in the long version, organize options along dimensions such as topics (e.g., health vs. humanitarian) and preferences (positive vs. negative) before voting. Others express that the upfront introduction of all options and the ability to rank and group them help manage their cognitive load effectively. In contrast, almost half of the participants using the long text interface, like~\texttt{S028}, express a desire for features that could help reduce the decision space when responding to the QS, further supporting the importance of these organizational design elements:

\begin{displayquote}  
Because with this many (options), especially when I'm thinking \ldots\ Ok, where was (the option) \ldots\ Where was (the option) you know? Oh, that's right. Maybe I could give another upvote to the, you know, whatever~\bracketellipsis \hfill\quoteby{S028 (LT)}  
\end{displayquote}  

This quote reflects participants' need to manually track and revisit options, which occupies extraneous load, without a more structured interface.  

These evidence explain how the organization phase and the drag-and-drop features support differentiation and consolidation, and scaffold a decision-making framework that enables deeper engagement.  

\textbf{In summary}, participants construct their preferences as they complete QS. We observe behaviors and qualitative insights that align with the differentiation and consolidation theory in decision-making. Our interface scaffolds many of the differentiation stages through pre-voting organization and some consolidation phases through drag-and-drop, explaining how the two-phase approach supports preference construction to yield deeper engagement with QS options.  

% ========================= %
\subsection{Future Work: Opportunities for better budget management}
Budget management is a recurring theme in participant interviews. While they appreciated the automatic calculation feature in modern QV interfaces, we identified three challenges for future QS interfaces: ~\textit{cognitive load},~\textit{the cold-start problem}, and~\textit{navigating between budget, votes, and outcome}.

\subsubsection{Automatic calculation is critical}
Over one-third of participants ($N=14$) from all four experiment conditions emphasized the importance of automated calculation for deriving costs and tracking expenditures. For example:

\begin{displayquote}
I thought I have \bracketellipsis (to) do all the numbers or calculations myself \bracketellipsis The credit summary section was really wonderful in doing all the calculations on that end. \hfill\quoteby{S005 (LT)}
\end{displayquote}

The quotes marked the importance that QS must be facilitated by computer-supported interfaces.

\subsubsection{The coldstart problem}
We notice from the study that one of the biggest challenges for participants is deciding 'how many votes' to start with. This challenge pertains to the initial vote, not the relative vote. Some participants began by equally distributing their credits to all options and then made adjustments. Others established $1$, $2$, and $3$ votes as starting points. A small handful surprisingly used the tutorial's example of 4 upvotes as their anchor.

This arbitrary voting decision echoes discussions in prior literature about the existence of an absolute value for individuals. Coherent arbitrariness~\cite{arielyCoherentArbitrarinessStable2003}, similar to the anchoring effect in marketing, refers to participants' willingness to allocate votes, which can be influenced by an arbitrary value. However, the ordinal utility remains intact among the set of preferences.

\subsubsection{Navigating Between Budget, Votes, and Actual Impact}
The third challenge is participants' confusion between budget, votes, and outcomes, despite understanding their definitions. One participant stated:

\begin{displayquote}

~\bracketellipsis get rid of the upvote column or just get rid of the word upvote and just really focus on the money column. Listen. You're an organization or your participant. You have X amount of dollars you need to. You can only distribute X amount of dollars to these causes. So you have to figure out which ones get the most, which ones don't get as much.~\bracketellipsis 

Interviewer: ~\bracketellipsis Do you feel that the more votes you're giving to a cause you're actually spending more on it?

Yeah. \hfill\quoteby{S003 (ST)}
\end{displayquote}

Participants like \texttt{S003} bypassed the quadratic formulation, directly translating votes to resource allocation. While this does not invalidate the power of the quadratic mechanism, it causes frustration and friction for participants to construct a clear picture of how to make voting decisions. Future interfaces should better communicate these relationships to facilitate respondents' trade-offs.

\textbf{In summary}, while the interface supports budget management through automated cost calculation, participants still face cognitive load from managing the budget. The cold-start problem and the confusion between budget, votes, and actual impact are open questions for future research. These challenges highlight the need for better budget management support to complete the QS interface.

\subsection{Quadratic Survey Usage, Design Recommendations and Future Work}
With a deeper understanding of how survey respondents interact with QS and the sources of cognitive load, we recognize that while this current interface may not significantly reduce cognitive load, it represents a crucial step toward constructing better interfaces to support individuals responding to QS. In this subsection, we outline usage and design recommendations applicable to all applications using the quadratic mechanism and highlight directions for future work.

\subsubsection{Usage Recommendation: QS for Critical Evaluations}
Our study highlighted the complex cognitive challenges and in-depth consideration required when ranking and rating options using QS, even in a short survey. Similar to survey respondents needing to make trade-offs across options, researchers and agencies seeking additional insights and alignment with respondent preferences must ensure that survey respondents have the cognitive capacity to complete such surveys rigorously. QS should be designed for critical evaluations, such as investment decisions, or situations where participants have ample time to think and process the survey. Pactioners should also caution the use of long QS. If long QS is not avoidable, considering allowing participants to deliberate on each option prior to deploying QS without the organizing phase. For instance, revealing the options ahead of time can aid in preference construction.

\subsubsection{Design Recommendations}
\paragraph{Use Organization Phases for Quadratic Mechanism Applications}
Our study demonstrated that preference construction can shift from operational to strategic and higher-level causes. An additional organizational phase with direct manipulation capability allows survey respondents to engage in higher-level critical thinking. We believe this approach should extend beyond QS to other ranking-based surveying tools, such as rank-choice voting and constant sum surveys. Further research should examine how implementing such functionality alters survey respondents' mental models.

\paragraph{Facilitate Differentiation through Categorization, Not Ranking}
Participants in our study were less inclined to provide a full rank unless necessary. The final 'rank' of option preferences often emerged as a byproduct of their vote allocation, constructed in situ. Therefore, for survey designs to be effective in constructing preferences, it is more important to facilitate differentiation than to focus on direct manipulation solely for fine-tuning. Emphasizing categorization can better support participants in articulating their preferences.

\subsubsection{Future Work: Support for Absolute Credit Decision}
Deciding the absolute amount of credits in QS is highly demanding. Designing interfaces and interactions that address the cold start challenge and help participants decide the absolute vote value while considering ways to limit direct influences remains an open question. Future research should explore innovative solutions to support participants in making these complex decisions effectively.

By implementing these recommendations and pursuing future research directions, we can improve the usability and effectiveness of QS and other quadratic mechanism-powered applications, ultimately aiding respondents in making more informed and accurate decisions.

\section{Limitations}
\label{sec:limitations}
Evaluating the QS interface is challenging due to its novelty. During the study, we identified several limitations that require further research.

\paragraph{Understanding results influence on decision-makers}
Further research is required to understand how the QS interface impacts decision-makers and broader societal resource distributions. Since QS is still in its early stages, we prioritize its widespread adoption and usage before attempting a comprehensive assessment of its influence on decision-making. Future studies will examine how decision-makers interpret and use QS data, as well as its broader implications for societal decisions.

\paragraph{Individual differences in cognitive capacity}
Variations in individual cognitive capacity influenced participants' cognitive scores. For example, participants with more experience in decision-making might be able to manage multiple options more effectively. A within-subject study could clarify cognitive load shifts, but deconstructing established preferences and altering options further complicates this. Thus, we opted for this in-depth, between-subject study, although the small sample size may introduce noise that distorts the actual cognitive load. Future research should quantify the impact of different QS interfaces. In addition, participants completed this study in a controlled lab environment with options displayed on a large screen. Future work should also explore how individuals respond to QS on smaller devices in a less controlled environment.

\paragraph{Limited experience with QS}
Participants had no prior experience with the QS interface. Following a tutorial and quiz, participants proceeded to complete tasks using the QS interface. While participants understood the QS mechanics, familiarity with the interface still influences strategies and cognitive load. As quadratic mechanisms become more prevalent, future research can compare novices and experts.

\paragraph{Duration between clicks to represent decision-making}
Click duration may include time spent considering other options, so it must be treated as an approximate measure of decision-making time. For instance, deciding between two options may take longer for the first option and less time for the second. Despite its limitations, this approach provides valuable insights into decision-making within our experimental constraints.

% 

% Recall that this survey aims to assist community organizers in distributing resources to a societal cause. This participant decided to `skip' over the quadratic formulation and the concept that their votes are governed by the quadratic formulation, drawing a direct translation between votes and the resources to which community organizers ought to contribute. 
% \begin{displayquote} I guess to see what my ranking looks like~\ldots and see if I could give more money or not. \hfill\quoteby{S021 (LI)} \end{displayquote}
% \begin{displayquote}
% If I had to choose a number like that in the beginning. That would have been really bad, but positive, neutral, negative. That was good enough. \hfill\quoteby{S026 (LI)}
% \end{displayquote}

% \begin{displayquote}
% I think \ldots\ ranking at the beginning one's impression towards these issues helps to like determine how many votes should be put towards them.  \hfill\quoteby{S002 (SI)}
% \end{displayquote}

% \begin{displayquote}
% If anything, I think I would like to be able to like, click and drag the categories themselves so I could maybe reorder them to like my priorities. \hfill\quoteby{S025 (LT)}
% \end{displayquote}

%  of well-organized interfaces in managing cognitive load.

% Participants ($N=4$, $2$ using the long two-phase interface) mentioned that organization support helped them to allot the intensity of votes by helping them focus and prioritize options through ranking. This exercise allows them to follow a clear decision-making process that avoids confusion.
% It is important to note that bounded rationality does not critique or exploit biases, but emphasizes the importance of designing interfaces that prevent decisions which diverge from one's true preferences. For example,~\underline{\smash{problem decomposition}}~\cite{simonSciencesArtificial1996} and~\underline{\smash{dimension reduction}} are strategic approaches to managing cognitive overload. Several participants would create a two-axis grouping, regardless of their experiment group. Participants clustered topics (e.g., health vs. humanitarian) and preferences (positive vs. negative). The difference between conditions was whether these groupings were representable on the interface.
% In addition, results indicate long text interface participants were satisfied due to cognitive overload from having too many options. They have to read more text, allocate more credits, and consider more options. Section~\ref{sec:cog_result} and Section~\ref{sec:behave_result} show how counterintuitive that this group had fewer participants experiencing high cognitive load compared to the short text interface. This group also experienced the least temporal demand (Sec.~\ref{sec:temporal}) while showing no difference in time spent per option compared to the text interface (Figure~\ref{fig:vote_time}). Participants in the long text interface also expressed the least frustration with operational tasks (Sec~\ref{sec:frustration}). 
% % These participants engaged with higher-level strategic challenges, in contrast to the more operational tasks emphasized in the text interface. 
% \paragraph{Familarity to the options}
% 1. primed on the local community, 
% 2. limited experience with qs
% We also acknowledge the possibility that the elicited values are pure noise and do not reflect the actual cognitive load. This could be due to the small sample size, the nature of the task, or the participants' understanding of the cognitive load scale. While this true for small sample sizes, we believe that the qualitative insights from the interviews provide a more nuanced understanding of the cognitive load sources. We detail limitations in Section~\ref{sec:limitations}.
% Maybe large scale AB testing and within subject testing in periodic collective decsion making enviornments.
% 3. time associated with the option.

% \subsection{The Quadratic Mechanism is Challenging}
% % We know QV is accurate and that QM allows specific expression of preferences
% % However, QM is diffucult to manage, internally construction of preference is diffuclt but so is the QM.

% % we tried to scaffold the construction of preference in interface design, for which we did help participants get to the exact values faster, but identifying and managing the construction is not something organization interfaces can fully support
% Most challenges participants faced come from the task itself: deciding the number of votes/credits to allot. I created the following hierarchical theme
% CI_3: deciding number of votes and credits (N=9/40, v1:1; v2:1; v3:4; v4:3)
% We can see participants in the long version group struggles more with this challenge. (2/20 vs. 7/20). So what exactly contributes to this decision process? We broke it down to the following themes:

% % Challenge lies in the mechanism itself
% CI_1: working with the QS mechanism (N=6, v1:5; v3:1)
% distinguishing between credits and votes (v3 participant)
% quadratic mechanism (all the rest)
% The first finding is that non of the interactive interface groups (v2 and v4) expressed feeling challenged due to the QS mechanism. The second finding is that the majority of this challenge comes from the short-list group. I think an explanation to this is clearer when we put up the second theme:

% CI_2: use up the remaining credits (N=4, v1:3, v2:1)
% The participants is struggling to express specific level of preference with limited credits. 
% Revisit one of the quotes from CI_1:
% “I wish I could just put the $2 towards the museum, or something like that.” (S036,v1)
% “It would be nice if I can use that one credit if there is an option, because the way it is done is in quadratic...I don't know why that is there...but if there is an option to not have it, and just [inaudible], that would be awesome.” (S012, v1)
% In other words, the expressiveness is constraint by the limited credit, amplified by the quadratic nature, forcing participants to forgo unused credits. This is also likey tied to prospect theory, that we will discuss later.
% The interface in the second group could have eliminated this because some options were eliminated, or that some ranking were established, prior to the voting process.


% \subsection{Construction of Preference}
%  \subsection{Design Implications}
% % Your content for the subsection on design implications goes here

% IN_T1: Dropdown (N=6/40)
% This is a common issue that participants dislike, across all versions.
% 3 from v1, the rest of the version each has 1

% IP_T3: Seeing all options – a sign in making decisions
% Participants like the ability to see all options on one screen (N=8/40)
% Comparing Long (N=5) and Short (N=3)
% The interactive interface requires a stronger need to see all options, as I hypothesis that this is because the need to interact and see the hierarchical groupings (Text: 2; Interactive: 6)

%% on positioning shift and the power of priming
%  use the performance quotes to highlight how participants are thinking in the shoes of decision-maker
% look for literature
% survey designed for decison makers to aid decisions

% Participants either felt positive or no issues using all four interfaces (N=33/40).


% \subsection{Limitations and Future Work}
% % Your content for the subsection on limitations and future work goes here


% We first show that participants constructed their preferences in situ. While some participants had existing preferences (e.g., environmental issues are important), they needed to reconsider aspects of the options or map them to their beliefs.

% \begin{displayquote}

% ~\bracketellipsis the other part of the mental demanding was probably trying to associate with (what) I'm concerned in soci(ety)~\bracketellipsis is that question able to deal with my social concerns like, for example, climate change~\bracketellipsis How does that fit in?

% \noindent \hfill -- S006, long interactive interface
% \end{displayquote}


% Behavior analysis in section~\ref{res:act} of participants using the long text and interactive interfaces revealed that they made small adjustments on the votes, clustered toward budget depiction with lesser time spent. These fine-grain adjustments indicated that participants are making less ad-hoc decisions; rather, they are deciding how to better utilize the remainder of the budget when the budget runs low. We identified a bi-modal interaction pattern.

% indicated that participants are making less ad-hoc decisions; rather, they are deciding how to better utilize the remainder of the budget when the budget runs low. We identified a bi-modal interaction pattern.

% \

% Conversely, in the text interface, one participant proactively mentioned a request to add click-and-drag functionalities to the interface. The participant described such function to group by topic categorization and also priority placement through direct manipulation.


% Throughout the preference construction journey, we confirm that the two-stage interactive interface and the direct manipulation through drag-and-drop facilitated participants in constructing and reflecting on their preferences, adhering to preference construction theory.

% Additionally, several participants mentioned how the direct manipulation functionality, allowing individuals to drag and drop options for repositioning, supports their reflective thinking during preference construction. One participant noted:
% \begin{displayquote}
% So I tried to make a ranking \bracketellipsis and by creating this ranking, by dragging the related issues \ldots\ I don’t know \ldots\ that helped me organize my ideas.
% \noindent \hfill -- S021, long interactive interface.
% \end{displayquote}

% into these categories, making completing the entire QS a series of difficult decisions.
% Literature from~\textcite{lichtensteinConstructionPreference2006} identifies three types of difficult decision-making scenarios: when one's preferences are not clearly defined, necessitating trade-offs, or quantifying opinions.  
% Since the interface supported some participants in managing their limited cognitive ability to make decisions, as shown in the previous subsection, we argue that the interactive interface \textit{shifted} the cognitive focus onto contributing to more in-depth preference construction and fine-tuning, even if it did not significantly reduce the cognitive load. Here we provide more evidence.

% Literature from~\textcite{lichtensteinConstructionPreference2006} identifies three types of difficult decision-making scenarios: when one's preferences are not clearly defined, necessitating trade-offs, or quantifying opinions. 


% Two participants highlighted the importance of automated calculation regarding the cost for each vote.
% Twelve participants highlighted the summarization box and the automated summation of the current credit spent, allowing them to focus on managing their next voting decision and expressing their preferences.

% \begin{displayquote}
% I like that I don't have to make the calculation of the dollars that it does it automatically. So if I had to do it myself it would be more tedious. And so I think that that effort and frustration and mental demand would be much higher. So I appreciate that that calculation occurs automatically and very easily.
% \noindent \hfill -- S017, short interactive interface.
% \end{displayquote}

% This is less significant in the short QS likely due to the reduced complexity~\footnote{We show in Appendix~\ref{sec:appendix_short_breakdown} that short interfaces exhibits the same bimodal behaviors but less obvious.}.