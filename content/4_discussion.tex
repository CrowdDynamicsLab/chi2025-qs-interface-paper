\section{Discussion and Future Works}
\label{sec:discussion}

In this section, we interpret the findings on cognitive load and respondent behavior in QS. We focus on the rationale and elements supporting the two-phase interface for preference construction and its potential to mitigate satisficing behavior. Additionally, we offer usage and design recommendations for practitioners and outline future directions for improving QS interfaces.

\subsection{Two-phase interface: a worthwhile trade-off}
Decision makers who deploy surveys aim to elicit thoughtful responses from participants. This means the interface should balance survey usability, respondent satisfaction, and the effort individuals invest in their responses. Results from the study lead us to conclude that the two-phase interface encouraged deeper participant engagement with the options and reduced satisficing behaviors, despite its increased time per option and higher cognitive load for long QS.

\subsubsection{Analysis through the lens of cognitive load theory}
Cognitive load theory~\cite{swellerCognitiveLoadTheory2011}, when applied to QS, identifies three components of cognitive load: intrinsic load (the cognitive demand required to understand questions and response options), germane load (associated with deeper processing and preference evaluation), and extraneous load (stemming from navigating and operating the survey interface).

Participants were randomly assigned to experimental conditions, with survey lengths containing options randomly drawn from a common pool to control intrinsic load within the same group. 

When QS is short, participants can engage with all options simultaneously. Participants using the two-phase interface traded a slightly longer survey response time for a potential reduction in cognitive load and edit distance. We interpret this as participants freeing up cognitive demand from extraneous load for germane load, prompting them to better construct and express their preferences.

When QS is long, participants face more options, resulting in a higher intrinsic load at the start of the survey. We believe the two-phase interface traded a longer survey response time and a potential increase in cognitive load for deeper engagement with the survey.Quantitatively, these participants made shorter traversals while spending more time on each option. Qualitatively, they reported experiencing demand more from strategic considerations (germane load) than from operational actions (extraneous load), which text interface participants experienced more frequently.

While some might argue that the additional organizing phase offers participants more opportunities to familiarize themselves with the options compared to text interface participants, the greater overall edit distance and high variance in edit distance per option suggest that text interface participants traversed the list frequently. This finding is further supported by qualitative data, where 70\% of long-text participants (N=7) reported scanning the list while voting. This behavior suggests that while long-text participants had opportunities to familiarize themselves with the options, the explicit organization phase encouraged deeper reflection on their preferences.

The effect of the two-phase interface shows nuanced differences influencing cognitive load outcomes; however, both analyses suggest that the two-phase interface \textit{shifted} participants' cognitive focus when completing QS.

\subsubsection{Potential in limiting Satisficing}
Qualitative findings (Section~\ref{sec:satisficing}) on potential satisficing behavior highlight the importance of careful consideration when deploying long QS. However, the two-phase interface appeared to limit satisficing behaviors, as evidenced by fewer observations compared to the long text interface for long QS and none for short QS. We believe the potential reasons lie in the design of the two-phase interface, which scaffolds the preference construction process.

The deliberate one-option-at-a-time presentation during the voting task in the two-phase interface reduced reliance on defaults and encouraged deeper reflection using cognitive strategies such as \textit{\smash{problem decomposition}}~\cite{simonSciencesArtificial1996} and \textit{\smash{dimension reduction}}, both of which are known to reduce cognitive overload.

When asked about their experience with the interface, four participants highlighted how the organization phase supported their preference construction.\texttt{S013} illustrated how the one-option-at-a-time approach reduced the dimensions of decision-making:

\begin{displayquote}  
\bracketellipsis it (organization phase) gives you time to just focus on that single thing and rank it based on how you feel at that moment. \hfill\quoteby{S013 (S2P)}  
\end{displayquote}  

This focused mode enabled deeper reflection. When considering relative preferences among QS,\texttt{S013} described how it structurally decomposed the problem:

\begin{displayquote}  
\bracketellipsis to have a preliminary categorization of all the topics ~\bracketellipsis (allowed me) to think about and process~\bracketellipsis digest all the information prior to actually allocating the budget~\bracketellipsis \hfill\quoteby{S009 (L2P)}  
\end{displayquote}  

This quote highlighted how participants' deliberation occurred during the organization phase, enabling them to focus on constructing preferences without worrying about budget management—both of which are cited sources of cognitive load. Although direct measurement of satisficing behavior reduction is challenging, qualitative data and participant feedback suggest that the two-phase interface has the potential to limit such behaviors. Based on this evidence, we advise against using long QS unless paired with a two-phase interface and ample time for participants to complete. We suggest future research investigate the mental processes underlying satisficing behaviors in long QS. 

\textbf{In summary,} we argue that the trade-off of a longer completion time and potentially higher cognitive load in the two-phase interface is justified. Drawing on cognitive load theory, we propose that the interface fosters deeper engagement with the options. Additionally, our qualitative findings and participant feedback suggest that the interface may reduce satisficing, aligning with decision-makers' goals of obtaining thoughtful and deliberate responses from participants.

% ============================== %
\subsection{Preference Construction guided by Organize, Then Vote}
Completing QS involves a series of in-situ difficult decision tasks~\textcite{lichtensteinConstructionPreference2006}. As one participant reflected when completing the survey with options they had never considered before:

\begin{displayquote}
Oh, there are other aspects that I never care about.~\bracketellipsis Why (should) I spend money on that? \\\hfill\quoteby{S037 (L2P)}
\end{displayquote}

When processing these unfamiliar options, we believe the two-phase interface supported participants' preference construction process.

First, 40\% of long-text participants (N=3) found it challenging to facilitate differentiation without organization tools that would allow grouping or drag-and-drop, as ~\texttt{S025} said:

\begin{displayquote}
    I would like to be able to like, click and drag the categories themselves so I could maybe reorder them to like my priorities.~\bracketellipsis make myself categories and subcategories out of this list~\ldots If I could organize it. \hfill\quoteby{S025 (LT)}
\end{displayquote}

In contrast, 60\% (N=6) of long two-phase participants appreciated the upfront introduction of all options, which enabled them to organize and use drag-and-drop features to facilitate completing QS. Not only did participants use drag-and-drop options post-voting to reflect and ensure correct vote allocation, but it also enabled participants, like~\texttt{S039}, to make fine-grained comparisons between options:

\begin{displayquote}  
    I think the system was actually really helpful because I could just drag them.~\bracketellipsis I can really compare them, I can drag this one up here, and then compare it to the top one~\bracketellipsis \hfill\quoteby{S039 (S2P)}  
\end{displayquote}  

This supports our intention of applying~\textcite{svensonDifferentiationConsolidationTheory1992}'s differentiation and consolidation theory, where participants attempt to identify differences and eliminate less favorable options. The organization phase and the drag-and-drop supported some degree of differentiation process.

\begin{displayquote}
    ~\bracketellipsis the hardest part deciding in which category of place (prefernce bin) each issue is. \hfill\quoteby{S021 (L2P)}
\end{displayquote}

This quote by~\texttt{S021} best represents the potential of the organization phase in separating part of the difficult decisions one needs to make when differentiating their preferences during preference construction. With the selected options, the shorter edit distance of long two-phase interface participants suggested that they were consolidating their identified preferences through votes.


% ========================= %

\subsection{What We Learned: Quadratic Survey Usage and Design Recommendations}
This study represents a crucial step toward developing better interfaces to support individuals responding to QS, by providing a deeper understanding of how survey respondents interact with QS and the sources of cognitive load. In this subsection, we outline usage and design recommendations applicable to all applications of the quadratic mechanism.

\subsubsection{QS should have Limited Options or for critical evaluations}
We recommend that QS, even with our two-phase interface, be deployed with limited options or used for critical evaluations, such as eliciting stakeholder preferences before making investment decisions, as our findings reveal complex cognitive challenges and increased time requirements when the number of options grows. Even though our two-phase interface scaffolds the decision-making process, we suggest that practitioners allow ample time for survey participants to deliberate on the options and complete their responses. When the two-phase QS interface is not available, survey designers should present the options ahead of time, allowing participants to familiarize themselves with the choices and deliberate before completing the QS.

\subsubsection{Facilitate Quadratic Mechanism Applications through Categorization, Not Ranking}
We suggest that future quadratic mechanism interface designs focus on categorization rather than ranking. With or without the organization phase, participants did not exhibit a ranking process. The final 'rank' of option preferences often emerged as a byproduct of vote allocation, constructed in situ. Therefore, for survey designs to effectively construct preferences, it is more important to facilitate differentiation than to focus on direct manipulation for fine-tuning.

We demonstrated this through the organization phase, where participants exhibited deeper engagement with survey options and potentially completed the survey more effectively.We believe this approach should extend beyond QS to other ranking-based survey tools, such as ranked-choice voting and constant-sum surveys. Further research should examine how implementing such functionality influences survey respondents' mental models.

\subsection{Future work: Opportunities for Better Budget Management}
Budget management emerged as one of the most prominent issues in our study, which the two-phase interface did not address. 35\% of participants ($N=14$) emphasized the ability of current quadratic mechanism applications to perform automated calculations, but noted that this is not sufficient.  We identified three key challenges for future work:

First, participants struggled to decide on an initial vote allocation. Some distributed credits equally across options, while others used $1$, $2$, or $3$ votes as starting points. A few anchored their decisions to the tutorial's example of four upvotes. This suggests a need to better understand whether individuals have absolute value preferences among options. Second, 12.5\% of participants ($N=5$) expressed confusion about the relationship between budget, votes, and outcomes, despite understanding their definitions. They struggled to make trade-offs between votes and budget, leading to frustration and hampered decision-making. Third, determining the absolute amount of credits in QS is highly demanding. Designing interfaces and interactions to address the cold start challenge and help participants decide on the absolute vote value, while also considering ways to limit direct influences, remains an open question.

We believe that, with the power of computing and a better understanding of how individuals calculate trade-offs can provide innovative solutions to help participants more easily express their preferences using QS.

\section{Limitations}
\label{sec:limitations}
Evaluating the QS interface is challenging because of its novelty. During the study, we identified several limitations that warrant further research.

\paragraph{Individual differences in cognitive capacity}
Variations in individual cognitive capacity influenced participants' performance and cognitive scores. For example, participants with greater experience in decision-making may be better able to manage multiple options.  A within-subject study could clarify shifts in cognitive load, but deconstructing established preferences and altering options introduces additional complexity. Therefore, we opted for this in-depth, between-subject study, although the small sample size may introduce noise, potentially distorting the measurement of cognitive load. Future research should aim to quantify the impact of different QS interfaces on cognitive load at a larger scale. Furthermore, participants completed this study in a controlled laboratory environment, with options displayed on a large screen. Future work should also investigate how individuals respond to QS on smaller devices and in less controlled environments.

\paragraph{Limited experience with QS}
Participants lacked prior experience with the QS interface. After completing a tutorial and quiz, participants proceeded to perform tasks using the QS interface. While participants understood the mechanics of QS, their familiarity with the interface likely influenced their strategies and cognitive load. As quadratic mechanisms become more prevalent, future research could compare the performance of novices and experts.

\paragraph{Duration between clicks and edit distance to represent decision-making}
While time and distance are common metrics for quantifying decision-making, it is likely that participants considered other options simultaneously. We acknowledge that these metrics are approximate indicators of decision-making effort. Despite these limitations, this approach provides valuable insights into decision-making within our experimental constraints.




% \paragraph{}
% Participants in our study were less inclined to provide a full rank unless necessary.   Emphasizing categorization can better support participants in articulating their preferences.


% \paragraph{Understanding results influence on decision-makers}
% Further research is required to understand how the QS interface impacts decision-makers and broader societal resource distributions. Since QS is still in its early stages, we prioritize its widespread adoption and usage before attempting a comprehensive assessment of its influence on decision-making. Future studies will examine how decision-makers interpret and use QS data, as well as its broader implications for societal decisions.

% \begin{displayquote}  
% Because with this many (options), especially when I'm thinking \ldots\ Ok, where was (the option) \ldots\ Where was (the option) you know? Oh, that's right. Maybe I could give another upvote to the, you know, whatever~\bracketellipsis \hfill\quoteby{S028 (LT)}  
% \end{displayquote}  

% exemplified by the reduction of edit distance, this benefit may be offset by the additional physical effort required to complete the grouping phase, as reflected by the slightly increased physical demand.  



% Participants using the two-phase interface, especially in the long version, organize options along dimensions such as topics (e.g., health vs. humanitarian) and preferences (positive vs. negative) before voting. 
% These evidence explain how the organization phase and the drag-and-drop features support differentiation and consolidation, and scaffold a decision-making framework that enables deeper engagement.  
% The bi-modal behavior observed in the long interactive interface participants aligns with the differentiation and consolidation framework, as described in the results. Participants in the two-phase interface begin differentiating options earlier, allowing them to later adjust fine-grain votes. The faster and smaller vote updates indicate participants are consolidating. The less prominent bi-modal behavior from the long text interface participants implies that the interface guides this decision framework, as participant~\texttt{037} explains:
% \begin{displayquote}
%     I only start from the positive one~\bracketellipsis I finish everything~\ldots and then I move to the second part (the neutral box).~\bracketellipsis I want to focus on these and make sure that resources are at least they get the attention they want. And if there's surplus and they can move to the second part. \hfill\quoteby{S037 (L2P)}
% \end{displayquote}


% Recall that this survey aims to assist community organizers in distributing resources to a societal cause. This participant decided to `skip' over the quadratic formulation and the concept that their votes are governed by the quadratic formulation, drawing a direct translation between votes and the resources to which community organizers ought to contribute. 
% \begin{displayquote} I guess to see what my ranking looks like~\ldots and see if I could give more money or not. \hfill\quoteby{S021 (L2P)} \end{displayquote}
% \begin{displayquote}
% If I had to choose a number like that in the beginning. That would have been really bad, but positive, neutral, negative. That was good enough. \hfill\quoteby{S026 (L2P)}
% \end{displayquote}

% \begin{displayquote}
% I think \ldots\ ranking at the beginning one's impression towards these issues helps to like determine how many votes should be put towards them.  \hfill\quoteby{S002 (S2P)}
% \end{displayquote}

% \begin{displayquote}
% If anything, I think I would like to be able to like, click and drag the categories themselves so I could maybe reorder them to like my priorities. \hfill\quoteby{S025 (LT)}
% \end{displayquote}

%  of well-organized interfaces in managing cognitive load.

% Participants ($N=4$, $2$ using the long two-phase interface) mentioned that organization support helped them to allot the intensity of votes by helping them focus and prioritize options through ranking. This exercise allows them to follow a clear decision-making process that avoids confusion.
% It is important to note that bounded rationality does not critique or exploit biases, but emphasizes the importance of designing interfaces that prevent decisions which diverge from one's true preferences. For example,~\underline{\smash{problem decomposition}}~\cite{simonSciencesArtificial1996} and~\underline{\smash{dimension reduction}} are strategic approaches to managing cognitive overload. Several participants would create a two-axis grouping, regardless of their experiment group. Participants clustered topics (e.g., health vs. humanitarian) and preferences (positive vs. negative). The difference between conditions was whether these groupings were representable on the interface.
% In addition, results indicate long text interface participants were satisfied due to cognitive overload from having too many options. They have to read more text, allocate more credits, and consider more options. Section~\ref{sec:cog_result} and Section~\ref{sec:behave_result} show how counterintuitive that this group had fewer participants experiencing high cognitive load compared to the short text interface. This group also experienced the least temporal demand (Sec.~\ref{sec:temporal}) while showing no difference in time spent per option compared to the text interface (Figure~\ref{fig:vote_time}). Participants in the long text interface also expressed the least frustration with operational tasks (Sec~\ref{sec:frustration}). 
% % These participants engaged with higher-level strategic challenges, in contrast to the more operational tasks emphasized in the text interface. 
% \paragraph{Familarity to the options}
% 1. primed on the local community, 
% 2. limited experience with qs
% We also acknowledge the possibility that the elicited values are pure noise and do not reflect the actual cognitive load. This could be due to the small sample size, the nature of the task, or the participants' understanding of the cognitive load scale. While this true for small sample sizes, we believe that the qualitative insights from the interviews provide a more nuanced understanding of the cognitive load sources. We detail limitations in Section~\ref{sec:limitations}.
% Maybe large scale AB testing and within subject testing in periodic collective decsion making enviornments.
% 3. time associated with the option.

% \subsection{The Quadratic Mechanism is Challenging}
% % We know QV is accurate and that QM allows specific expression of preferences
% % However, QM is diffucult to manage, internally construction of preference is diffuclt but so is the QM.

% % we tried to scaffold the construction of preference in interface design, for which we did help participants get to the exact values faster, but identifying and managing the construction is not something organization interfaces can fully support
% Most challenges participants faced come from the task itself: deciding the number of votes/credits to allot. I created the following hierarchical theme
% CI_3: deciding number of votes and credits (N=9/40, v1:1; v2:1; v3:4; v4:3)
% We can see participants in the long version group struggles more with this challenge. (2/20 vs. 7/20). So what exactly contributes to this decision process? We broke it down to the following themes:

% % Challenge lies in the mechanism itself
% CI_1: working with the QS mechanism (N=6, v1:5; v3:1)
% distinguishing between credits and votes (v3 participant)
% quadratic mechanism (all the rest)
% The first finding is that non of the interactive interface groups (v2 and v4) expressed feeling challenged due to the QS mechanism. The second finding is that the majority of this challenge comes from the short-list group. I think an explanation to this is clearer when we put up the second theme:

% CI_2: use up the remaining credits (N=4, v1:3, v2:1)
% The participants is struggling to express specific level of preference with limited credits. 
% Revisit one of the quotes from CI_1:
% “I wish I could just put the $2 towards the museum, or something like that.” (S036,v1)
% “It would be nice if I can use that one credit if there is an option, because the way it is done is in quadratic...I don't know why that is there...but if there is an option to not have it, and just [inaudible], that would be awesome.” (S012, v1)
% In other words, the expressiveness is constraint by the limited credit, amplified by the quadratic nature, forcing participants to forgo unused credits. This is also likey tied to prospect theory, that we will discuss later.
% The interface in the second group could have eliminated this because some options were eliminated, or that some ranking were established, prior to the voting process.


% \subsection{Construction of Preference}
%  \subsection{Design Implications}
% % Your content for the subsection on design implications goes here

% IN_T1: Dropdown (N=6/40)
% This is a common issue that participants dislike, across all versions.
% 3 from v1, the rest of the version each has 1

% IP_T3: Seeing all options – a sign in making decisions
% Participants like the ability to see all options on one screen (N=8/40)
% Comparing Long (N=5) and Short (N=3)
% The interactive interface requires a stronger need to see all options, as I hypothesis that this is because the need to interact and see the hierarchical groupings (Text: 2; Interactive: 6)

%% on positioning shift and the power of priming
%  use the performance quotes to highlight how participants are thinking in the shoes of decision-maker
% look for literature
% survey designed for decison makers to aid decisions

% Participants either felt positive or no issues using all four interfaces (N=33/40).


% \subsection{Limitations and Future Work}
% % Your content for the subsection on limitations and future work goes here


% We first show that participants constructed their preferences in situ. While some participants had existing preferences (e.g., environmental issues are important), they needed to reconsider aspects of the options or map them to their beliefs.

% \begin{displayquote}

% ~\bracketellipsis the other part of the mental demanding was probably trying to associate with (what) I'm concerned in soci(ety)~\bracketellipsis is that question able to deal with my social concerns like, for example, climate change~\bracketellipsis How does that fit in?

% \noindent \hfill -- S006, long interactive interface
% \end{displayquote}


% Behavior analysis in section~\ref{res:act} of participants using the long text and interactive interfaces revealed that they made small adjustments on the votes, clustered toward budget depiction with lesser time spent. These fine-grain adjustments indicated that participants are making less ad-hoc decisions; rather, they are deciding how to better utilize the remainder of the budget when the budget runs low. We identified a bi-modal interaction pattern.

% indicated that participants are making less ad-hoc decisions; rather, they are deciding how to better utilize the remainder of the budget when the budget runs low. We identified a bi-modal interaction pattern.

% \

% Conversely, in the text interface, one participant proactively mentioned a request to add click-and-drag functionalities to the interface. The participant described such function to group by topic categorization and also priority placement through direct manipulation.


% Throughout the preference construction journey, we confirm that the two-stage interactive interface and the direct manipulation through drag-and-drop facilitated participants in constructing and reflecting on their preferences, adhering to preference construction theory.

% Additionally, several participants mentioned how the direct manipulation functionality, allowing individuals to drag and drop options for repositioning, supports their reflective thinking during preference construction. One participant noted:
% \begin{displayquote}
% So I tried to make a ranking \bracketellipsis and by creating this ranking, by dragging the related issues \ldots\ I don’t know \ldots\ that helped me organize my ideas.
% \noindent \hfill -- S021, long interactive interface.
% \end{displayquote}

% into these categories, making completing the entire QS a series of difficult decisions.
% Literature from~\textcite{lichtensteinConstructionPreference2006} identifies three types of difficult decision-making scenarios: when one's preferences are not clearly defined, necessitating trade-offs, or quantifying opinions.  
% Since the interface supported some participants in managing their limited cognitive ability to make decisions, as shown in the previous subsection, we argue that the interactive interface \textit{shifted} the cognitive focus onto contributing to more in-depth preference construction and fine-tuning, even if it did not significantly reduce the cognitive load. Here we provide more evidence.

% Literature from~\textcite{lichtensteinConstructionPreference2006} identifies three types of difficult decision-making scenarios: when one's preferences are not clearly defined, necessitating trade-offs, or quantifying opinions. 


% Two participants highlighted the importance of automated calculation regarding the cost for each vote.
% Twelve participants highlighted the summarization box and the automated summation of the current credit spent, allowing them to focus on managing their next voting decision and expressing their preferences.

% \begin{displayquote}
% I like that I don't have to make the calculation of the dollars that it does it automatically. So if I had to do it myself it would be more tedious. And so I think that that effort and frustration and mental demand would be much higher. So I appreciate that that calculation occurs automatically and very easily.
% \noindent \hfill -- S017, short interactive interface.
% \end{displayquote}

% This is less significant in the short QS likely due to the reduced complexity~\footnote{We show in Appendix~\ref{sec:appendix_short_breakdown} that short interfaces exhibits the same bimodal behaviors but less obvious.}.