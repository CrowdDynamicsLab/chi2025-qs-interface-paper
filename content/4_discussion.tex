\section{Discussion and Future Works}
\label{sec:discussion}

\subsection{Bounded Rationality and interface design}
One core repeated theme that emerged throughout participants' responses during the interview relates to Bounded Rationality. In earilier sections, we highlighted the challenge of multiple options presented on the quadratic survey. Now, we also consider the budget avaliable to survey respondents, which credits become a broad space of possible voting options, adds additional layers and numbers of decision to make. This additional set of decisions are highlighed by several participants when they expressed apprication of the dop down menu showing all possible options with their costs precalculated.

However, the dropdown does not mitigate the bigger challenge of bounded rationality. Bounded rationality~\cite{simonBehavioralModelRational1955} highlights individuals' cognitive limitations to process and utilize information and therefore formulate and solve complex problems. This sea of decision making requires participants to recall and scramble many information at once which is extremely difficult.

\begin{displayquote}
So I did say, Okay, you know, you thought of enough things, you know, and so it wasn't the most effort I could put in because again, that would have been diminishing returns. I tried to think of enough things that I could make, make a meaningful decision and then move on.

\noindent \hfill -- S036, short text interface.
\end{displayquote}

The byproduct of bounded rationality often translate to individuals satisficing behaviors~\cite{gigerenzerReasoningFastFrugal1996}, creating Heuristics~\cite{tverskyJudgmentUncertaintyHeuristics1974}, overreliance on defaults~\cite{thalerNudgeImprovingDecisions2008a}, and problem decompositionn~\cite{simonSciencesArtificial1996}. 

Satisficing is the most common behavior observed among the participants, which refers to survey respondents making decisions that are not optimal but rather complete a `good enough' decision. The same participant~\texttt{036} when asked about demand from performance then continues to describe:

\begin{displayquote}
I think that that's just not a realistic expectation (to be perfect), but I felt satisfied.~\bracketellipsis I felt like that (the response) was satisfied, but not perfect. Cause perfect is not a reality

\noindent \hfill -- S036, short text interface.
\end{displayquote}

Problem decomposition and dimension reduction is the other common behavior that we observed. Severl participant would create a duo-dimension grouping, dispite the group they are in. Participants would have categories that cluster similiar topics (i.e., all the topics related to health vs. humanitarian), and categories that depict the positivity of their preference (i.e., positive v.s. negative). The goal of highlighting bounded rationality is not to critice or exploit the possible biases that this mechanism might introduce, but highlight the importance of designing interface interventions to prevent survey respondents from enacting decisions that differs from their true preferences.

For example, the design of showing one option at a time in the interactive interface lowers the possibility for participants to be influenced by the default positions of options. One participant from the short text interface said,
\begin{displayquote}
Honestly, if medical research~\bracketellipsis I think if it was the first option, the first thing I saw, I probably would have given it more~\bracketellipsis because medical research~\bracketellipsis to me this seems like the most important, but I think if if it was the first one I saw, I think it would automatically gave it a lot more.
    
\noindent \hfill -- S003, short text interface.
\end{displayquote}

Another example comes from another participant from the long text interface.
\begin{displayquote}
I think the categories were kind of in the same location. The environment stuff is at the bottom. Education policy is like in the top half. So I think I just looked and determined (my votes) that way.
    
\noindent \hfill -- S035, long text interface.
\end{displayquote}
Recall that the options presented on the survey are randomly generated; even though there are some options related to the environment and education at the relevant location, participants were inferring the options to these topics. Active management of the options forced participants to think about their rough preference for each option at minimal cognitive requirements and the repositioning of options allowed participants to focus on subsets of the options during their decison making process. These are reflected in the positve responses from the interface comments on organization and direct manipulation. 

\subsection{Quadratic mechanism is challenging}
On the other hand, the interface did not include elements that help participants kick off their voting process. One of the most difficult challenges for participants is for them to decide `how many votes' to begin with. This challenge does not refer to the relative vote, but the starting vote. Some participants would begin by first equally distributing their credits to all options and then make adjustments (find quote), some participants established 1, 2, and 3 votes as three 'tiers' of votes as starting points, and a small handful of participants, out of our surprise, used the number of votes in the tutorial (which showed an example with 4 upvotes as the highest value), as their anchor. 

This seemingly arbitrary voting decisions echos prior literature's discussion on whether an absolute value exists for an individual. Coherent arbitrariness~\cite{arielyCoherentArbitrarinessStable2003}, similar to the anchoring effect in marketing, refers to participants' willingness to pay can be influenced by an arbitrary value. However, the ordinal utility remains intact among the set of preferences. 

Participants are also required to navigate between three elements: budget, credit, votes, and thinking about how the results would impact the 'shared resource.' This is not straightforward. 

\begin{displayquote}

~\bracketellipsis get rid of the Upvote column or just get rid of the word upvote and just really focus on the money column. Listen. You're an organization or your participant. You have X amount of dollars you need to. You can only distribute X amount of dollars to these these causes. So you have to figure out which ones get the most, which ones don't get as much.~\bracketellipsis 

Interviewer: So when you're operating this interface. Do you feel that the more votes you're giving to a cause you're actually spending more on it?

Yeah.
       
\noindent \hfill -- S003, short text interface.
\end{displayquote}
Recall that this survey aims to assist community organizers in distributing resources to a societal cause. This participant decided to 'skip' over the quadratic formulation and the concept that their votes are governed by the quadratic formulation, drawing a direct translation between votes and the resources to which community organizers ought to contribute. While this does not invalid the power of quadratic mechanism, it builds frustration and friction for participants to construct a clean picture of how to make voting decisions. 

Budget related sources draw across mental demand, temporal demand, preference demand, and fustration. These span from making sure to keep within budget to recovering from overbudgeting. While prior literature 

Adds the final layer to operating the quadratic survey. While prior scarcity literature~\cite{Shah2015a} believes that values and careful decisions are derived from limited resources, prospect theory~\cite{kahnemanProspectTheoryAnalysis1979} also highlight a higher negative value of~\textit{precieved} utility for individuals when cuts ought to be made.

These three major challenges do not threaten the integrity of Quadratic Survey and relavent tools using this mechanism, but as we demonstrated in the results section, across all experiment conditions, the NASA-TLX scales show medium to high cognitive load even for the short interactive interface. In other words, we believe that the improvement of the Quadratic Survey's ability to elicit more accurate preferences, yet, it comes at the cost of higher cognitive load.


\subsection{Construction of Preference on QS}

We believe that even if the interface did not significantly reduce the cognitive load from participants, the interface~\textit{shifted} the cognitive focus onto contributing upon more in-depth preference construction and more fine-tuning.

We show that participants constructed their preferences in situ. While some participants came in with some existing preference (i.e., environmental issues are important), participants need to reconsider aspects of the options they had not expected, or they need to map options on the survey to their beliefs.

\begin{displayquote}

~\bracketellipsis the other part of the mental demanding was probably trying to associate with (what) I'm concerned in soci(ety)~\bracketellipsis is that question able to deal with my social concerns like, for example, climate change~\bracketellipsis How does that fit in?

\noindent \hfill -- S006, long interactive interface
\end{displayquote}

\begin{displayquote}

I mean, it's not necessarily a challenge, but it's interesting to see: `Oh, there are other aspects that I never care about.' And actually~\ldots some people care <an option>. Sure. Why? Why (should) I spend money on that? That's the first thought that comes to mind.

\noindent \hfill -- S037, long interactive interface
\end{displayquote}

Both quotes highlighted the cause of construction of preference, but also highlighted the individuals reflecting on their personal preferences. As stated in prior works~\cite{chengCanShowWhat2021, naylor2017first}, QS, by listing a list of options bounded by a common credit, forced participants to consider within options. This is supported by the qualitative analysis as one of the main causes of mental demand --~\textit{preference construction}.

Next, the long text and interactive interface participant behavior analysis surfaced participants, despite sharing a similar number of actions, small adjustments on the votes are clustered toward budget depiction with lesser time spent. These fine grain adjustments represented participants are not making ad-hoc decisions as they complete QS, rather they are deciding how they ought to better utilize the remainder of the budget. This is not as obvious in the short survey likely because of the limited options (hence budget), there are less decision space and adjustments that individual participants can make. We were still able to identify the bi-modal interaction pattern but it is less clear if there are differences in the clusters.

We believe that the bimodal behavior observed in the voting actions across groups is the realization of the Differentiation and Consolidation Theory presented by ~\textcite{svensonDifferentiationConsolidationTheory1992}. The theory segmented the decision-making process into two steps: differentiation and consolidation. The former supports individuals on focusing on the differences and eliminating less favorable alternatives. The latter is a process where individuals strengthen their commitment to the chosen option, even mentally. The bimodality reflected participants differentiating some options through change of votes in the beginning of the survey and then consolidating with smaller adjustments toward the end. The two-stage decision making is stronger with the interactive interface because part of the differentiation is completed in the organization phase. As one participant mentioned:

\begin{displayquote}
I only start from the positive one~\bracketellipsis I finish everything~\ldots and then I move to the second part (the netural box).~\bracketellipsis I want to focus on these and make sure that resources are at least they get the attention they want. And if there's surplus and they can move to the second part

\noindent \hfill -- S037, long interactive interface
\end{displayquote}

\subsection{QS Usage and Design Recommendation and Future Works}
This study proposed an interface that supports thought organization and preference construction. While this interface was not able to show a decrease in cognitive load significantly, this study identified additional challenges and insights into how survey respondents complete QS. We also identify open directions to support individual decision making for collective outcomes.

\subsubsection{QS for critical evaluations}
This study highlighted the complex cognitive challenges and in-depth consideration when ranking and rating across options using QS, even in a short survey. Similar to survey respondents needing to make trade-offs across options, researchers and agencies that wanted the additional insights and alignment to respondent preferences need to make trade-offs in assuring survey respondents have the cognitive capacity to complete such surveys rigorously. We recommend that QS should be designed for specific use cases that require critical evaluations, i.e., investment decisions or settings where participants have enough time to think and process the survey, i.e., revealing the options ahead of time for preference construction.

\subsubsection{Using organization processes}
This study demonstrated differences among the source of various demands shifted from operational causes to strategic and higher-level causes. This shift in preference constructions highlighted how an additional organizational phase with direct manipulation capability allowed survey respondents to access higher-level critical thinking. We believe that such behavior shift should not only relate to QS, but extends to other ranking-based surveying tools such as rank-choice voting and constant sum surveys. Further work should examine if implementing such functionality altered survey respondent's mental model.

\subsection{Support for absolute credit decision}
Deciding the~\textit{absolute} amount of credits in QS is very demanding. Designing interfaces and interactions to support the cold start challenge as well as helping deciding the absolute vote value yet considering limiting direct influences is an open question.

% \section{Limitations}
% 1. primed on the local community, 



% \subsection{The Quadratic Mechanism is Challenging}
% % We know QV is accurate and that QM allows specific expression of preferences
% % However, QM is diffucult to manage, internally construction of preference is diffuclt but so is the QM.

% % we tried to scaffold the construction of preference in interface design, for which we did help participants get to the exact values faster, but identifying and managing the construction is not something organization interfaces can fully support
% Most challenges participants faced come from the task itself: deciding the number of votes/credits to allot. I created the following hierarchical theme
% CI_3: deciding number of votes and credits (N=9/40, v1:1; v2:1; v3:4; v4:3)
% We can see participants in the long version group struggles more with this challenge. (2/20 vs. 7/20). So what exactly contributes to this decision process? We broke it down to the following themes:



% % Challenge lies in the mechanism itself
% CI_1: working with the QS mechanism (N=6, v1:5; v3:1)
% distinguishing between credits and votes (v3 participant)
% quadratic mechanism (all the rest)
% The first finding is that non of the interactive interface groups (v2 and v4) expressed feeling challenged due to the QS mechanism. The second finding is that the majority of this challenge comes from the short-list group. I think an explanation to this is clearer when we put up the second theme:

% CI_2: use up the remaining credits (N=4, v1:3, v2:1)
% The participants is struggling to express specific level of preference with limited credits. 
% Revisit one of the quotes from CI_1:
% “I wish I could just put the $2 towards the museum, or something like that.” (S036,v1)
% “It would be nice if I can use that one credit if there is an option, because the way it is done is in quadratic...I don't know why that is there...but if there is an option to not have it, and just [inaudible], that would be awesome.” (S012, v1)
% In other words, the expressiveness is constraint by the limited credit, amplified by the quadratic nature, forcing participants to forgo unused credits. This is also likey tied to prospect theory, that we will discuss later.
% The interface in the second group could have eliminated this because some options were eliminated, or that some ranking were established, prior to the voting process.


% \subsection{Construction of Preference}




% \subsection{Design Implications}
% % Your content for the subsection on design implications goes here

% IN_T1: Dropdown (N=6/40)
% This is a common issue that participants dislike, across all versions.
% 3 from v1, the rest of the version each has 1

% IP_T3: Seeing all options – a sign in making decisions
% Participants like the ability to see all options on one screen (N=8/40)
% Comparing Long (N=5) and Short (N=3)
% The interactive interface requires a stronger need to see all options, as I hypothesis that this is because the need to interact and see the hierarchical groupings (Text: 2; Interactive: 6)

%% on positioning shift and the power of priming
%  use the performance quotes to highlight how participants are thinking in the shoes of decision-maker
% look for literature
% survey designed for decison makers to aid decisions

% Participants either felt positive or no issues using all four interfaces (N=33/40).


% \subsection{Limitations and Future Work}
% % Your content for the subsection on limitations and future work goes here
