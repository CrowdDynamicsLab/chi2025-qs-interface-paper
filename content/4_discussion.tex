\section{Discussion and Future Works}
\label{sec:discussion}

% \subsection{The Quadratic Mechanism is Challenging}
% % We know QV is accurate and that QM allows specific expression of preferences
% % However, QM is diffucult to manage, internally construction of preference is diffuclt but so is the QM.

% % we tried to scaffold the construction of preference in interface design, for which we did help participants get to the exact values faster, but identifying and managing the construction is not something organization interfaces can fully support
% Most challenges participants faced come from the task itself: deciding the number of votes/credits to allot. I created the following hierarchical theme
% CI_3: deciding number of votes and credits (N=9/40, v1:1; v2:1; v3:4; v4:3)
% We can see participants in the long version group struggles more with this challenge. (2/20 vs. 7/20). So what exactly contributes to this decision process? We broke it down to the following themes:



% % Challenge lies in the mechanism itself
% CI_1: working with the QS mechanism (N=6, v1:5; v3:1)
% distinguishing between credits and votes (v3 participant)
% quadratic mechanism (all the rest)
% The first finding is that non of the interactive interface groups (v2 and v4) expressed feeling challenged due to the QS mechanism. The second finding is that the majority of this challenge comes from the short-list group. I think an explanation to this is clearer when we put up the second theme:

% CI_2: use up the remaining credits (N=4, v1:3, v2:1)
% The participants is struggling to express specific level of preference with limited credits. 
% Revisit one of the quotes from CI_1:
% “I wish I could just put the $2 towards the museum, or something like that.” (S036,v1)
% “It would be nice if I can use that one credit if there is an option, because the way it is done is in quadratic...I don't know why that is there...but if there is an option to not have it, and just [inaudible], that would be awesome.” (S012, v1)
% In other words, the expressiveness is constraint by the limited credit, amplified by the quadratic nature, forcing participants to forgo unused credits. This is also likey tied to prospect theory, that we will discuss later.
% The interface in the second group could have eliminated this because some options were eliminated, or that some ranking were established, prior to the voting process.


% \subsection{Construction of Preference}
% % Your content for the subsection on construction of preference goes here

% % participant constructing preferences in situ, leave this in discussion
% % Ti-Chung Cheng: You mean any other challenges? [Challenges]
% % S037: Other than the ones that you've mentioned.
% % Ti-Chung Cheng: I mean, it's not necessarily a challenge, but it's interesting to see. Oh, there are other aspects that I never care about. And actually some people care like…Jewish. Sure. Why? Why do I spend money on that? That's the first thought that comes to mind.
% % S037: Okay, so any other challenges?
% % Ti-Chung Cheng: Not really.

% % S037: Okay. So how does a software interface support you or hinder you from expressing your preferences on responding to these societal issues?
% % Ti-Chung Cheng: [inaudible] I don't really know if that kinda shifts my behavioral or perspective when sorting this. Oh, but I do feel that when we were sorting, you know, into different basket. Umm…we only see one flash flashcard. We click, and then it drops. Originally, I was thinking, you know, I have a whole bunch of cards right in front of me, and I got to, you know, mix a match. Yeah, I mean, we can do that afterwards. But I think that definitely. tet me think on its own. You know, when I look at medical research re-research. It really based on my initial impression on this item. Instead of, you know. relatively comparing medical research comparing to education policy, which one is more important which one deserves to go to the important basket. I’m not sure if that deteriorates my judgment? Or that kinda helps? But that definitely affects something.

% CI_3c: translating internal preferences into votes (N=6, v1:2; v2:1; v3:1, v4:2)
% “So I think I have a clear idea what I'm most passionate about, so I don't think that was a challenge for me so it was mostly just [deciding] upvotes … the amount [...]. (S039, v4)
% This theme surrounds quantifying the final value based on personal preferences/rankings

% > The takeaway here between CI3b and 3c is that 3b composes no participants in the interactive interface. this could mean that participants are trying to adhere their final votes to a specific strategy and ‘rule’ they created (i.e., a should be greater than b then c) but have to place these struggles internally, rather then having the chance to populate these differences on the interface, offloading some of these struggles.

% CI_4: forming judgments (N=9, v1:3; v2:1; v3:2, v4:3)
% building judgement to generate an opinion(N=8, v1:2; v2:1; v3:2, v4:3)
% justifying voting decisions (N=3, v1:1; v4:2)
% > Both codes are processes where participants are struggling to develop their preferences either prior or post the voting decision. This surroundings a lack of access to relevant information (see mental model for survey completion) This can only be aided with additional information and is subject to the context.

% CI_6: Too many options/credits to work with (N=7, v1:1; v3:3, v4:1)


% bounded rationality and satisficing %%
% hints from the interview, 
% this draws that QS is not a tool for the general public, but requires deliberate planning and usage
% interface with less option seems the saftest to push forward,dispite less significant statistically. Larger scale experiment need to be done.
%  soruce of support, use temporal and perforamce quotes and numbers.

% \subsection{Design Implications}
% % Your content for the subsection on design implications goes here

% IN_T1: Dropdown (N=6/40)
% This is a common issue that participants dislike, across all versions.
% 3 from v1, the rest of the version each has 1

% IP_T3: Seeing all options – a sign in making decisions
% Participants like the ability to see all options on one screen (N=8/40)
% Comparing Long (N=5) and Short (N=3)
% The interactive interface requires a stronger need to see all options, as I hypothesis that this is because the need to interact and see the hierarchical groupings (Text: 2; Interactive: 6)

%% on positioning shift and the power of priming
%  use the performance quotes to highlight how participants are thinking in the shoes of decision-maker
% look for literature
% survey designed for decison makers to aid decisions

% Participants either felt positive or no issues using all four interfaces (N=33/40).


% \subsection{Limitations and Future Work}
% % Your content for the subsection on limitations and future work goes here
