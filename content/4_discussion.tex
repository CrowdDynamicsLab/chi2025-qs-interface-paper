\section{Discussion and Future Works}
\label{sec:discussion}

\subsection{Bounded Rationality and interface design}
One core repeated theme that emerged throughout participants' responses during the interview relates to Bounded Rationality. In earilier sections, we highlighted the challenge of multiple options presented on the quadratic survey. Now, we also consider the budget avaliable to survey respondents, which credits become a broad space of possible voting options, adds additional layers and numbers of decision to make. This additional set of decisions are highlighed by several participants when they expressed apprication of the dop down menu showing all possible options with their costs precalculated.

However, the dropdown does not mitigate the bigger challenge of bounded rationality. Bounded rationality~\cite{simonBehavioralModelRational1955} highlights individuals' cognitive limitations to process and utilize information and therefore formulate and solve complex problems. This sea of decision making requires participants to recall and scramble many information at once which is extremely difficult.

\begin{displayquote}
So I did say, Okay, you know, you thought of enough things, you know, and so it wasn't the most effort I could put in because again, that would have been diminishing returns. I tried to think of enough things that I could make, make a meaningful decision and then move on.

\noindent \hfill -- S036, short text interface.
\end{displayquote}

The byproduct of bounded rationality often translate to individuals satisficing behaviors~\cite{gigerenzerReasoningFastFrugal1996}, creating Heuristics~\cite{tverskyJudgmentUncertaintyHeuristics1974}, overreliance on defaults~\cite{thalerNudgeImprovingDecisions2008a}, and problem decompositionn~\cite{simonSciencesArtificial1996}. 

Satisficing is the most common behavior observed among the participants, which refers to survey respondents making decisions that are not optimal but rather complete a `good enough' decision. The same participant~\texttt{036} when asked about demand from performance then continues to describe:

\begin{displayquote}
I think that that's just not a realistic expectation (to be perfect), but I felt satisfied.~\bracketellipsis I felt like that (the response) was satisfied, but not perfect. Cause perfect is not a reality

\noindent \hfill -- S036, short text interface.
\end{displayquote}

Problem decomposition and dimension reduction is the other common behavior that we observed. Severl participant would create a duo-dimension grouping, dispite the group they are in. Participants would have categories that cluster similiar topics (i.e., all the topics related to health vs. humanitarian), and categories that depict the positivity of their preference (i.e., positive v.s. negative). The goal of highlighting bounded rationality is not to critice or exploit the possible biases that this mechanism might introduce, but highlight the importance of designing interface interventions to prevent survey respondents from enacting decisions that differs from their true preferences.

For example, the design of showing one option at a time in the interactive interface lowers the possibility for participants to be influenced by the default positions of options. One participant from the short text interface said,
\begin{displayquote}
Honestly, if medical research~\bracketellipsis I think if it was the first option, the first thing I saw, I probably would have given it more~\bracketellipsis because medical research~\bracketellipsis to me this seems like the most important, but I think if if it was the first one I saw, I think it would automatically gave it a lot more.
    
\noindent \hfill -- S003, short text interface.
\end{displayquote}

Another example comes from another participant from the long text interface.
\begin{displayquote}
I think the categories were kind of in the same location. The environment stuff is at the bottom. Education policy is like in the top half. So I think I just looked and determined (my votes) that way.
    
\noindent \hfill -- S035, long text interface.
\end{displayquote}
Recall that the options presented on the survey are randomly generated; even though there are some options related to the environment and education at the relevant location, participants were inferring the options to these topics. Active management of the options forced participants to think about their rough preference for each option at minimal cognitive requirements and the repositioning of options allowed participants to focus on subsets of the options during their decison making process. These are reflected in the positve responses from the interface comments on organization and direct manipulation. 

\subsection{Quadratic mechanism is challenging}
On the other hand, the interface did not include elements that help participants kick off their voting process. One of the most difficult challenges for participants is for them to decide `how many votes' to begin with. This challenge does not refer to the relative vote, but the starting vote. Some participants would begin by first equally distributing their credits to all options and then make adjustments (find quote), some participants established 1, 2, and 3 votes as three 'tiers' of votes as starting points, and a small handful of participants, out of our surprise, used the number of votes in the tutorial (which showed an example with 4 upvotes as the highest value), as their anchor. 

This seemingly arbitrary voting decisions echos prior literature's discussion on whether an absolute value exists for an individual. Coherent arbitrariness~\cite{arielyCoherentArbitrarinessStable2003}, similar to the anchoring effect in marketing, refers to participants' willingness to pay can be influenced by an arbitrary value. However, the ordinal utility remains intact among the set of preferences. 

Participants are also required to navigate between three elements: budget, credit, votes, and thinking about how the results would impact the 'shared resource.' This is not straightforward. 

\begin{displayquote}

~\bracketellipsis get rid of the Upvote column or just get rid of the word upvote and just really focus on the money column. Listen. You're an organization or your participant. You have X amount of dollars you need to. You can only distribute X amount of dollars to these these causes. So you have to figure out which ones get the most, which ones don't get as much.~\bracketellipsis 

Interviewer: So when you're operating this interface. Do you feel that the more votes you're giving to a cause you're actually spending more on it?

Yeah.
       
\noindent \hfill -- S003, short text interface.
\end{displayquote}
Recall that this survey aims to assist community organizers in distributing resources to a societal cause. This participant decided to 'skip' over the quadratic formulation and the concept that their votes are governed by the quadratic formulation, drawing a direct translation between votes and the resources to which community organizers ought to contribute. While this does not invalid the power of quadratic mechanism, it builds frustration and friction for participants to construct a clean picture of how to make voting decisions. 

Budget related sources draw across mental demand, temporal demand, preference demand, and fustration. These span from making sure to keep within budget to recovering from overbudgeting. While prior literature 

Adds the final layer to operating the quadratic survey. While prior scarcity literature~\cite{Shah2015a} believes that values and careful decisions are derived from limited resources, prospect theory~\cite{kahnemanProspectTheoryAnalysis1979} also highlight a higher negative value of~\textit{precieved} utility for individuals when cuts ought to be made.

These three major challenges do not threaten the integrity of Quadratic Survey and relavent tools using this mechanism, but as we demonstrated in the results section, across all experiment conditions, the NASA-TLX scales show medium to high cognitive load even for the short interactive interface. In other words, we believe that the improvement of the Quadratic Survey's ability to elicit more accurate preferences, yet, it comes at the cost of higher cognitive load.


\subsection{Construction of Preference on QS}


% bounded rationality and satisficing %%
% hints from the interview, 
% this draws that QS is not a tool for the general public, but requires deliberate planning and usage
% interface with less option seems the saftest to push forward,dispite less significant statistically. Larger scale experiment need to be done.
%  soruce of support, use temporal and perforamce quotes and numbers.

\subsection{Design Recommendation and Future Works}



\section{Limitations}



% \subsection{The Quadratic Mechanism is Challenging}
% % We know QV is accurate and that QM allows specific expression of preferences
% % However, QM is diffucult to manage, internally construction of preference is diffuclt but so is the QM.

% % we tried to scaffold the construction of preference in interface design, for which we did help participants get to the exact values faster, but identifying and managing the construction is not something organization interfaces can fully support
% Most challenges participants faced come from the task itself: deciding the number of votes/credits to allot. I created the following hierarchical theme
% CI_3: deciding number of votes and credits (N=9/40, v1:1; v2:1; v3:4; v4:3)
% We can see participants in the long version group struggles more with this challenge. (2/20 vs. 7/20). So what exactly contributes to this decision process? We broke it down to the following themes:



% % Challenge lies in the mechanism itself
% CI_1: working with the QS mechanism (N=6, v1:5; v3:1)
% distinguishing between credits and votes (v3 participant)
% quadratic mechanism (all the rest)
% The first finding is that non of the interactive interface groups (v2 and v4) expressed feeling challenged due to the QS mechanism. The second finding is that the majority of this challenge comes from the short-list group. I think an explanation to this is clearer when we put up the second theme:

% CI_2: use up the remaining credits (N=4, v1:3, v2:1)
% The participants is struggling to express specific level of preference with limited credits. 
% Revisit one of the quotes from CI_1:
% “I wish I could just put the $2 towards the museum, or something like that.” (S036,v1)
% “It would be nice if I can use that one credit if there is an option, because the way it is done is in quadratic...I don't know why that is there...but if there is an option to not have it, and just [inaudible], that would be awesome.” (S012, v1)
% In other words, the expressiveness is constraint by the limited credit, amplified by the quadratic nature, forcing participants to forgo unused credits. This is also likey tied to prospect theory, that we will discuss later.
% The interface in the second group could have eliminated this because some options were eliminated, or that some ranking were established, prior to the voting process.


% \subsection{Construction of Preference}
% % Your content for the subsection on construction of preference goes here

% % participant constructing preferences in situ, leave this in discussion
% % Ti-Chung Cheng: You mean any other challenges? [Challenges]
% % S037: Other than the ones that you've mentioned.
% % Ti-Chung Cheng: I mean, it's not necessarily a challenge, but it's interesting to see. Oh, there are other aspects that I never care about. And actually some people care like…Jewish. Sure. Why? Why do I spend money on that? That's the first thought that comes to mind.
% % S037: Okay, so any other challenges?
% % Ti-Chung Cheng: Not really.

% % S037: Okay. So how does a software interface support you or hinder you from expressing your preferences on responding to these societal issues?
% % Ti-Chung Cheng: [inaudible] I don't really know if that kinda shifts my behavioral or perspective when sorting this. Oh, but I do feel that when we were sorting, you know, into different basket. Umm…we only see one flash flashcard. We click, and then it drops. Originally, I was thinking, you know, I have a whole bunch of cards right in front of me, and I got to, you know, mix a match. Yeah, I mean, we can do that afterwards. But I think that definitely. tet me think on its own. You know, when I look at medical research re-research. It really based on my initial impression on this item. Instead of, you know. relatively comparing medical research comparing to education policy, which one is more important which one deserves to go to the important basket. I’m not sure if that deteriorates my judgment? Or that kinda helps? But that definitely affects something.

% CI_3c: translating internal preferences into votes (N=6, v1:2; v2:1; v3:1, v4:2)
% “So I think I have a clear idea what I'm most passionate about, so I don't think that was a challenge for me so it was mostly just [deciding] upvotes … the amount [...]. (S039, v4)
% This theme surrounds quantifying the final value based on personal preferences/rankings

% > The takeaway here between CI3b and 3c is that 3b composes no participants in the interactive interface. this could mean that participants are trying to adhere their final votes to a specific strategy and ‘rule’ they created (i.e., a should be greater than b then c) but have to place these struggles internally, rather then having the chance to populate these differences on the interface, offloading some of these struggles.

% CI_4: forming judgments (N=9, v1:3; v2:1; v3:2, v4:3)
% building judgement to generate an opinion(N=8, v1:2; v2:1; v3:2, v4:3)
% justifying voting decisions (N=3, v1:1; v4:2)
% > Both codes are processes where participants are struggling to develop their preferences either prior or post the voting decision. This surroundings a lack of access to relevant information (see mental model for survey completion) This can only be aided with additional information and is subject to the context.

% CI_6: Too many options/credits to work with (N=7, v1:1; v3:3, v4:1)



% \subsection{Design Implications}
% % Your content for the subsection on design implications goes here

% IN_T1: Dropdown (N=6/40)
% This is a common issue that participants dislike, across all versions.
% 3 from v1, the rest of the version each has 1

% IP_T3: Seeing all options – a sign in making decisions
% Participants like the ability to see all options on one screen (N=8/40)
% Comparing Long (N=5) and Short (N=3)
% The interactive interface requires a stronger need to see all options, as I hypothesis that this is because the need to interact and see the hierarchical groupings (Text: 2; Interactive: 6)

%% on positioning shift and the power of priming
%  use the performance quotes to highlight how participants are thinking in the shoes of decision-maker
% look for literature
% survey designed for decison makers to aid decisions

% Participants either felt positive or no issues using all four interfaces (N=33/40).


% \subsection{Limitations and Future Work}
% % Your content for the subsection on limitations and future work goes here
