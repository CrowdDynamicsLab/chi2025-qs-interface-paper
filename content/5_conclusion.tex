\section{Conclusion}
In this study, we designed and evaluated a novel two-phase ``Organize-then-Vote" interface aimed at guiding Quadratic Survey (QS) respondents in effectively constructing their preferences. Through an in-lab study employing NASA-TLX and interviews, we explored how this two-phase interface influenced individuals' cognitive load and survey response behaviors when engaging with societal issues of varying lengths. The interface’s organization and voting phases, designed to reduce cognitive overload by structuring the decision-making process, allowed respondents to differentiate between options before voting. \change{Results revealed that the two-phase design reduced participant's edit distance between vote adjustments throughout the survey despite spending more time per option. Qualitative insights highlighted two-phase interface  encouraged more iterative and reflective preference construction and it's potential at reducing satisficing behaviors even though it did not clearly reduce overall cognitive load for the longer QS. Nonetheless, this design shift promoted deeper engagement and strategic thinking compared to the text-based interface, by distributing cognitive effort more effectively.} By integrating the organization and drag-and-drop functions, the interface facilitated both preference differentiation and consolidation, making it easier for respondents to refine their decisions. This two-phase interface design supports the development of future software tools that facilitate preference construction and promote the broader adoption of Quadratic Surveys. Future research should explore how to better support individuals in deciding the allocation of budget and design interfaces for smaller devices.
