\section{Conclusion}
Surveys enable decision-makers to aggregate crowd opinions. In this study, we use QS to elicit individual responses in the context of social resource allotment. After multiple design iterations, we propose an interactive interface for QS. We then examined its influence on individuals' cognitive load and behaviors when faced with societal issues of varying lengths. In a 2x2 between-subject study, we had participants experience either a long or short QS using a text-based or interactive interface. NASA-TLX questionnaires and interviews revealed that participants using the interactive interface for a long QS demonstrated a more comprehensive and critical evaluation of societal issues, despite not experiencing a lower cognitive load. Participants using the long text interface experienced cognitive overload, which led to satisficing behaviors or mental shortcuts. Analyzing click-stream data, we identified that participants made fine-grain iterations using the long interactive interface when credits were low. We demonstrate that a two-phase, organize-then-vote interface can scaffold the complex decision-making process, helping individuals express their opinions for collective societal decisions. Through the iterative design process and detailed interviews, we identified future directions and design recommendations for collective decision-making applications using the quadratic mechanism.