\section{Conclusion}
This study introduces and evaluates a two-phase ``Organize-then-Vote" interface to help QS respondents construct their preferences. We examined how the interface affected cognitive load and response behaviors across societal issues of varying lengths through an in-lab study, NASA-TLX, and interviews. The interface's organization and voting phases, designed to reduce cognitive overload by structuring the decision-making process, allowed respondents to differentiate between options before voting. Results revealed that the two-phase design reduced participants' edit distance between vote adjustments throughout the survey despite spending more time per option. Qualitative insights highlighted that the two-phase interface encouraged more iterative and reflective preference construction and its potential for reducing satisficing behaviors even though it did not clearly reduce the overall cognitive load for the longer QS. Nonetheless, this design shift promoted deeper engagement and strategic thinking compared to the text-based interface, by distributing cognitive effort more effectively. By integrating the organization and drag-and-drop functions, the interface facilitated both preference differentiation and consolidation, making it easier for respondents to refine their decisions. This two-phase interface design supports the development of future software tools that facilitate preference construction and promote the broader adoption of QSs. Future research should explore how to better support individuals' budget allocation and design interfaces for smaller devices.
