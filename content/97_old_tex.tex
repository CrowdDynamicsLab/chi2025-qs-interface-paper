\subsection{Cognitive Load Measurement}
The cognitive load theory (CLT) assumes a limited working memory that needs to be allocated to all mental tasks currently performed~\cite{tracy2006measuring}. An efficient interface design needs to respect the limits of the cognitive process and avoid cognitive overload~\cite{tracy2006measuring}. Cognitive load can be divided into intrinsic cognitive load, extraneous cognitive load, and germane cognitive load.\cite{klepsch2017development}\cite{schmutz2009cognitive} The intrinsic cognitive load is induced by the inherent complexity of the learning task.\cite{klepsch2017development} The extraneous cognitive load is imposed by inappropriate design and organization of the materials, which is the mental resources into processes that are not relevant for the main task itself.\cite{klepsch2017development} The germane cognitive load is also influenced by the instructional design. Unlike the extraneous cognitive load, it results in the design of a favorable present way that enhances learning and eases the learning process. \cite{paas2003cognitive}\cite{klepsch2017development} Based on the above description, it is obvious that optimizing interface design would focus on decreasing the extraneous cognitive load and maximizing germane load. \cite{schmutz2009cognitive} 

There are many different methods to measure users’ cognitive load. We discuss three popular measurement approaches here, including dual-task methods, physiological measures, and subjective measures \cite{klepsch2017development, tracy2006measuring, chen2017construct}.

Dual-task methodology is a direct and objective measure of assessing cognitive load.\cite{schmutz2009cognitive} It requires the user to perform a separate secondary task with the main task simultaneously and analyze the performance in the primary task.\cite{klepsch2017development}\cite{schmutz2009cognitive}\cite{chen2017construct} It bases on the assumption that the increase in resource allocation for one task would decrease resources for the other task.\cite{klepsch2017development}\cite{schmutz2009cognitive} However, one of the disadvantages of the dual-task method is the second task would disturb the primary task and that disturbance would vary for different people.\cite{klepsch2017development} Besides, it is hard to adopt it in our experiment since it is hard to design the secondary task. The secondary task requires to use of the same cognitive resources as the primary tasks, and the cognitive resource for filling in the survey would be variable depending on different participants.\cite{moray2013mental}

Physiological measures are observing the body’s functions to approach the cognitive load.\cite{chen2017construct} There is a wide range of physiological parameters have been used for the measurement: heart rate, electroencephalography, pupil dilation, and hormone level.\cite{klepsch2017development} Nevertheless, the interpretation of the physiological process is still unclear and hard to explain.\cite{klepsch2017development} Also, the measurement process is complicated and expensive. Therefore, we don’t adopt the measure of physiological parameters in our research.\cite{klepsch2017development, brunken2002assessment}

Subjective measures use self-report ratings to collect options from participants.\cite{brunken2010measuring} Although there is criticism about its validity and vulnerability, it is the most frequently used measurement because of its low cost and ease of administration. Because of these considerations, our study would focus on subjective measures.\cite{ramkumar2017using} NASA Task Load Index (TLX) is one of the most widely used subjective measures. It is a multidimensional scoring procedure that uses the weighted average of six subscale scores to represent the overall workload. The six subscales include the dimensions related to the demands imposed on the subject (mental demands, physical demands, and temporal demands) and the interaction of a subject with the task (performance, effort, and frustration).\cite{hart1988development, hart2006nasa, cain2007review} NASA-TLX uses a priori workload definition of subjects to weight and average subscale ratings, which asks the subject to evaluate the contribution of each weight to the workload of a specific task.\cite{hart1988development} This process would decrease the between rater variability since it indicates the differences in workload definition between raters within a task and the differences in the sources of workload between tasks.\cite{cain2007review} \cite{dey2010sensitivity} It provides diagnostic information on the nature of the workload associated with the task. In addition, the NASA-TLX rating scales represent comparative judgments concerning extreme values with natural psychological significance.\cite{hart1988development} In addition, its grading scales include 21 gradations, which help NASA-TLX considerably increase its sensitivity and its variability between subjects compared with other subjective measures like SWAT which has only three discrete values, especially for the small workload change. \cite{rubio2004evaluation}\cite{galy2012relationship} NASA-TLX has been tested on a variety of experimental tasks and lab tasks, and the workload scores derived from these tests were found to have significantly less variability of the evaluators than the one-dimensional workload scores. Based on the above reasons, we would use NASA-TLX to measure the cognitive load in our study.