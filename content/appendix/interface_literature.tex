\section{Voting Interface Breakdown}\label{apdx:relatedVoting}
Compared to digital survey interfaces, there exists a rich literature on voting interfaces, which we argue is a special type of survey interface. We categorize these related works into three main categories detailed below:

\paragraph{Designs that shifted voter decisions: } For example, states without straight-party ticket voting~(where voters can select all candidates from one party through a single choice) exhibited higher rates of split-ticket voting~\cite{engstrom2020politics}. Another example from the Australian ballot showing incumbency advantages is where candidates are listed by the office they are running for, with no party labels or boxes.
\paragraph{Designs that influenced errors: } Butterfly ballots increased voter errors because voters could not correctly identify the punch hole on the ballot. Splitting contestants across columns increases the chance for voters to overvote~\cite{quesenberyOpinionGoodDesign2020}. On the other hand, \textcite{everettElectronicVotingMachines2008} showed the use of incorporating physical voting behaviors, like lever voting, into graphical user interfaces.

\paragraph{Designs that incorporated technologies: } Other projects like the Caltech-MIT Voting Technology Project have sparked research to address accessibility challenges, resulting in innovations like EZ Ballot~\cite{leeUniversalDesignBallot2016}, Anywhere Ballot~\cite{summers2014making}, and Prime III~\cite{dawkinsPrimeIIIInnovative2009}. In addition, \textcite{gilbertAnomalyDetectionElectronic2013} investigated optimal touchpoints on voting interfaces, and \textcite{conradElectronicVotingEliminates2009} examined zoomable voting interfaces.

\section{Voting Interfaces and Response Format}

Research in the marketing and research communities focusing on survey and questionnaire design, usability, and interactions examines the influence of presentation styles and `response format.'~\textcite{weijtersExtremityHorizontalVertical2021} demonstrated that horizontal distances between options are more influential than vertical distances, with the latter recommended for reduced bias. Slider bars, which operate on a drag-and-drop principle, show lower mean scores and higher nonresponse rates compared to buttons, indicating they are more prone to bias and difficult to use. In contrast, visual analog scales that operate on a point-and-click principle perform better~\cite{toepoelSlidersVisualAnalogue2018}. These studies show how even small design changes can have a large impact on usability, highlighting the importance of designing interfaces that prioritize human-centered interaction rather than focusing solely on functionality.

Voting interfaces are a specialized type of survey interface that not only elicit individual choices but often have consequential impacts. For example, the butterfly ballot, an atypical design, may have influenced the outcome of the 2000 U.S. Presidential Election~\cite{wandButterflyDidIt2001}. Research has shown that ballot interfaces can significantly influence democratic processes~\cite{engstrom2020politics, chisnellDemocracyDesignProblem2016, DesigningUsableBallots2015}. Several studies also highlighted how voting interface designs shift voter decisions~\cite{engstrom2020politics}, reduce usability errors~\cite{quesenberyOpinionGoodDesign2020, everettElectronicVotingMachines2008}, or improve interaction~\cite{leeUniversalDesignBallot2016, summers2014making, dawkinsPrimeIIIInnovative2009, gilbertAnomalyDetectionElectronic2013, conradElectronicVotingEliminates2009}. We provide more details to these voting interfaces in the Appendix~\ref{apdx:relatedVoting}.

From the QV implementations, response format literature, and voting interfaces, we identified how interfaces significantly influence respondent behavior, decision accuracy, and cognitive load. While these systems are functional, they lack the human-centered design needed to reduce cognitive load and make them truly usable, rather than simply operable. These burdens are especially problematic for complex systems like QS, where high cognitive demands may deter researchers and users alike. Developing effective, human-centered interfaces for QS could enhance usability, reduce cognitive overload, and increase adoption in both research and practical applications.
