\section{Modeling Total Time} \label{sec:apdx:model_time}

\subsubsection{Dependent Variables} The dependent variable is the total time $T_i$ spent on option $i$ measured in seconds. This measure captures both the duration participants took to vote and, where applicable, the time they spent organizing or reordering their options beforehand. We categorize the data into four experimental conditions: Short Text, Short Two-Phase, Long Text, and Long Two-Phase. These conditions are indexed by $k$, fit using separate submodels.

\subsection{Modeling Approach} We modeled the total time for each experimental condition using separate Gamma likelihood models. The Gamma distribution is well-suited for modeling positive continuous data, such as time measurements, which are often skewed and strictly positive. Equation~\ref{eq:time_main} shows the model for the total time. The shape parameter $\alpha_k$ and rate parameter $\beta_k$ were each assigned priors drawn from their own Gamma distributions, as described in Equations~\ref{eq:alpha_prior} and \ref{eq:beta_prior}.

\begin{align}
    T_i &\sim \text{Gamma}(\alpha_k, \beta_k) \label{eq:time_main} \\
    \alpha_k &\sim \text{Gamma}(2.0, 0.5) \label{eq:alpha_prior} \\
    \beta_k &\sim \text{Gamma}(1.0, 1.0) \label{eq:beta_prior}
\end{align}



% \section{Modeling Total Time} \label{sec:apdx:model_time}

% In this section, we outline how we modeled the total time ($T_i$) that participants spent considering each option in our experiment, accounting for both voting activities and, where applicable, the organization phase.

% \subsubsection{Dependent Variables} The dependent variable is the total time $T_i$, recorded for each option under consideration. This measurement includes any time participants devoted to categorizing or reordering options before making their votes.

% \subsubsection{Experimental Conditions} We distinguish four experimental conditions: Short Text, Short Two-Phase, Long Text, and Long Two-Phase. Each condition is indexed by $k$, and we fit a separate submodel for each, reflecting potential differences in the time distributions across conditions.

% \subsection{Modeling Approach} We employed a Gamma likelihood to model total time in each condition, since the Gamma distribution captures strictly positive, right-skewed data commonly observed in time measurements. Specifically, we define: \begin{align} T_i &\sim \text{Gamma}(\alpha_k, \beta_k), \label{eq:time_main} \end{align} where $\alpha_k$ is the shape parameter and $\beta_k$ is the rate parameter of the Gamma distribution for condition $k$. We assign priors to these parameters as follows: \begin{align} \alpha_k &\sim \text{Gamma}(2.0, 0.5), \label{eq:alpha_prior} \ \beta_k &\sim \text{Gamma}(1.0, 1.0). \label{eq:beta_prior} \end{align}

% By modeling each condition independently with its own shape and rate parameters, we allow the time distributions to vary flexibly across conditions, while maintaining a straightforward, interpretable structure.