% \documentclass[format=acmsmall, natbib=false, review=false, authordraft=false, anonymous=true, screen=true]{acmart}
\documentclass[manuscript, review, anonymous, natbib=false]{acmart}

\usepackage{enumitem}
\usepackage{graphicx}  % another package that works for figures
\usepackage{booktabs} % For formal tables
\usepackage{cleveref} % for better references
\usepackage{caption,subcaption}
\usepackage[english]{babel}% Recommended
\usepackage{csquotes}% Recommended
\usepackage{siunitx}
\usepackage{tabularx}
\usepackage{xcolor}
\usepackage{rotating}
\usepackage{pdflscape}
\usepackage{afterpage}
\usepackage{hyperref}
% \usepackage{rerunfilecheck}
% \usepackage{epstopdf}
\usepackage{lscape}
\usepackage{wrapfig}
\usepackage{sidecap}
\usepackage{rotating}
\usepackage{adjustbox} 

\usepackage{mdframed}
\usepackage{fontawesome} % for the icon
\usepackage{amsthm}
\usepackage{tcolorbox}

\newtcolorbox{tldrbox}[1][]{colback=gray!10, colframe=darkgray, left=5pt, right=5pt, top=3pt, bottom=3pt, arc=0pt, outer arc=0pt, toprule=0pt, bottomrule=0pt, leftrule=1.5pt, rightrule=0pt, #1}


%references
\usepackage[backend=biber, style=acmnumeric,sorting=none]{biblatex}
\let\citename\relax
\addbibresource{tcheng.bib}
\renewcommand{\bibfont}{\Small}
\usepackage{url}
\setcounter{biburllcpenalty}{7000}
\setcounter{biburlucpenalty}{8000}

% %  Comment this out to fix all citation
% \renewcommand{\cite}[1]{[??]}
% \renewcommand{\textcite}[1]{Mark et al.}

\newcommand{\bracketellipsis}{[\ldots]\xspace}

% other required packages
\RequirePackage{rotating}
\usepackage[ruled]{algorithm2e} % For algorithms
\renewcommand{\algorithmcfname}{ALGORITHM}
\SetAlFnt{\small}
\SetAlCapFnt{\small}
\SetAlCapNameFnt{\small}
\SetAlCapHSkip{0pt}
\IncMargin{-\parindent}

\renewenvironment{displayquote}
  {\list{}{\small\leftmargin=1em\rightmargin=1em}\item\relax\itshape\color{darkgray}}
  {\endlist}

\newcommand{\smallquote}[2]{
    {\color{darkgray}\texttt{#1}~\faCommentsO~\textit{#2}}
}
\newcommand{\quoteby}[1]{
    {\color{darkgray}\faCommentsO~\texttt{#1}}
}

%% author color
%% color: http://latexcolor.com/
\definecolor{lapislazuli}{rgb}{0.15, 0.38, 0.61}
\newcommand{\hs}[1]{{\color{red}{HS: #1}}}
\newcommand{\tc}[1]{{\color{green}{TC: #1}}}
\newcommand{\lwt}[1]{{\color{olive}{LWT: #1}}}
\newcommand{\kk}[1]{{\color{violet}{KK: #1}}}
\newcommand{\vk}[1]{{\color{blue}{VK: #1}}}
\newcommand{\yhc}[1]{{\color{orange}{YHC: #1}}}
\newcommand{\change}[1]{{\color{black}{#1}}}

%% Rights management information.  This information is sent to you
%% when you complete the rights form.  These commands have SAMPLE
%% values in them; it is your responsibility as an author to replace
%% the commands and values with those provided to you when you
%% complete the rights form.
% \setcopyright{acmcopyright}
% \copyrightyear{2021}
% \acmYear{2021}
% \acmDOI{10.1145/1122445.1122456}

\setcopyright{acmlicensed}
\acmJournal{PACMHCI}
% \acmYear{2021} \acmVolume{5} \acmNumber{CSCW1} \acmArticle{182} \acmMonth{4} \acmPrice{15.00}\acmDOI{10.1145/3449281}

% For Articles 180-195:
% \received{October 2020}
% \received[revised]{January 2021}
% \received[accepted]{January 2021}


% %% These commands are for a PROCEEDINGS abstract or paper.
% \acmConference[CSCW '21]{CSCW '21: The 24rd ACM Conference on Computer-Supported Cooperative Work and Social Computing}{Nov 03 -- 07, 2021}{Toronto, Canada}
% \acmBooktitle{CSCW '21: The 24rd ACM Conference on Computer-Supported Cooperative Work and Social Computing,
% Nov 03 -- 07, 2021, Toronto, Canada}
% \acmPrice{15.00}
% \acmISBN{978-1-4503-9999-9/18/06}


%%
%% Submission ID.
%% Use this when submitting an article to a sponsored event. You'll
%% receive a unique submission ID from the organizers
%% of the event, and this ID should be used as the parameter to this command.
% \acmSubmissionID{V5cscw182}

%%
%% The majority of ACM publications use numbered citations and
%% references.  The command \citestyle{authoryear} switches to the
%% "author year" style.
%%
%% If you are preparing content for an event
%% sponsored by ACM SIGGRAPH, you must use the "author year" style of
%% citations and references.
%% Uncommenting
%% the next command will enable that style.
%%\citestyle{acmauthoryear}

%%
%% end of the preamble, start of the body of the document source.
\begin{document}

%%
%% The "title" command has an optional parameter,
%% allowing the author to define a "short title" to be used in page headers.
% \title[QV vs Likert]{``\textellipsis I can show what I really like.'': 
% Comparing Quadratic Voting with Likert Surveys at aligning respondents' preferences}

\title{Organize, Then Vote: Exploring Cognitive Load in Quadratic Survey Interfaces}

%%
%% The "author" command and its associated commands are used to define
%% the authors and their affiliations.
%% Of note is the shared affiliation of the first two authors, and the
%% "authornote" and "authornotemark" commands
%% used to denote shared contribution to the research.


%% Author list
\author{Ti-Chung Cheng}
\email{tcheng10@illinois.edu}
\orcid{0000-0001-7647-338X}
\affiliation{%
  \institution{University of Illinois Urbana-Champaign}
  \city{Urbana}
  \state{Illinois}
  \country{USA}
}
\author{Yutong Zhang}
\email{yutongz7@illinois.edu}
\authornotemark[1]
\affiliation{%
  \institution{University of Illinois at Urbana-Champaign}
  \city{Urbana}
  \state{Illinois}
  \country{USA}
}
\author{Yi-Hung Chou}
\email{hank0982@uci.edu}
\authornote{Both authors contributed equally to this research.}
\affiliation{%
  \institution{University of California, Irvine}
  \city{Irvine}
  \country{USA}
}
\author{Vinay Koshy}
\orcid{0000-0002-1410-3911}
\affiliation{\institution{Computer Science \\ University of Illinois at Urbana Champaign}
\city{Urbana}
\state{Illinois}
\country{USA}}
\email{vkoshy2@illinois.edu}
\author{Tiffany Wenting Li}
\email{wenting7@illinois.edu}
\affiliation{%
  \institution{University of Illinois Urbana-Champaign}
  \city{Urbana}
  \state{Illinois}
  \country{USA}
}
\author{Karrie Karahalios}
\email{kkarahal@illinois.edu}
\affiliation{%
  \institution{University of Illinois Urbana-Champaign}
  \city{Urbana}
  \state{Illinois}
  \country{USA}
}
\author{Hari Sundaram}
\email{hs1@illinois.edu}
\affiliation{%
  \institution{University of Illinois at Urbana-Champaign}
  \city{Urbana}
  \state{Illinois}
  \country{USA}
}

% %%
% %% By default, the full list of authors will be used in the page
% %% headers. Often, this list is too long, and will overlap
% %% other information printed in the page headers. This command allows
% %% the author to define a more concise list
% %% of authors' names for this purpose.
\renewcommand{\shortauthors}{Ti-Chung Cheng et al.}

%%
%% The abstract is a short summary of the work to be presented in the
%% article.
% For Specific Application, we'd need to suggest it's for survey takers or something similar.
\begin{abstract}
Quadratic Surveys (QS) elicit more accurate individual preferences than traditional surveys, such as Likert-scale surveys. However, the cognitive load associated with QS has hindered its adoption in digital surveys for collective decision-making. We introduce a two-phase ``organize-then-vote'' QS interface based on decision-making and preference construction theories designed to lessen the cognitive load. Since interface design significantly impacts survey results and accuracy, our design scaffolds survey takers' decision-making while managing the cognitive load imposed by QS. In a 2x2 between-subject in-lab study on public resource allotment, we compared our interface with a traditional text interface across QS with $6$ (short) and $24$ (long) options. Our interface reduced satisficing behaviors arising from cognitive overload in long QS conditions. Participants using our interface in the long QSs shifted their cognitive effort from mechanical operations to constructing more comprehensive preferences. This research clarifies how human-centered design improves preference elicitation tools for collective decision-making.
\end{abstract}







%A two-phase interface scaffolds individual decision making process in long QS, shifting cognitive effort from operating the survey tool to constructing more comprehensive preferences.



%%
%% The code below is generated by the tool at http://dl.acm.org/ccs.cfm.
%% Please copy and paste the code instead of the example below.
%%

\begin{CCSXML}
    <ccs2012>
       <concept>
           <concept_id>10003120.10003130.10011762</concept_id>
           <concept_desc>Human-centered computing~Empirical studies in collaborative and social computing</concept_desc>
           <concept_significance>500</concept_significance>
           </concept>
       <concept>
           <concept_id>10003120.10003130.10003134</concept_id>
           <concept_desc>Human-centered computing~Collaborative and social computing design and evaluation methods</concept_desc>
           <concept_significance>500</concept_significance>
           </concept>
       <concept>
           <concept_id>10003120.10003121.10003122</concept_id>
           <concept_desc>Human-centered computing~HCI design and evaluation methods</concept_desc>
           <concept_significance>300</concept_significance>
           </concept>
     </ccs2012>
\end{CCSXML}
    
\ccsdesc[500]{Human-centered computing~Empirical studies in collaborative and social computing}
\ccsdesc[500]{Human-centered computing~Collaborative and social computing design and evaluation methods}
\ccsdesc[300]{Human-centered computing~HCI design and evaluation methods}

%% Keywords.
\keywords{Quadratic Voting; Likert scale; Empirical studies; Collective decision-making}

%% Main Text
\maketitle
\section{Introduction}
% par 1: Introduction to the Problem
% Purpose: Define the problem and explain its significance.

%  What is the problem, what is the challenge, and why is it important
% Interfaces are important because they affect data collection. There is no QS interface, and QS is hard, so the problem is how do we design interfaces for QS?

% Version 1
% Designing user interfaces for emerging survey techniques is rare and challenging. Although new techniques offer the potential for more accurate data collection, existing interfaces are often ill-equipped to handle their complexity. We introduce Quadratic Surveys (QS), a surveying technique that applies the principles of Quadratic Voting (QV) to better elicit individual preferences compared to traditional Likert scale surveys~\cite{chengCanShowWhat2021}. Yet, without well-designed interfaces, even the most promising techniques can struggle with user adoption, ultimately threatening survey response quality~\cite{pielotDidYouMisclick2024, kimComparingDataChatbot2019}. The unique challenge of QS lies in its mechanism: respondents allocate a fixed budget of votes, with the cost of casting additional votes on a single option increasing progressively. This cost structure encourages careful trade-offs and promises to improve the accuracy of preference elicitation~\cite{posner2018radical}. At the same time, this mechanism makes responding to QS cognitively challenging. Therefore, this paper seeks to address the key question:~\textit{How can we design interfaces to support participants in completing Quadratic Surveys (QS) more effectively?}

Designing intuitive survey interfaces is crucial for accurately capturing respondents' preferences, which directly impact the quality and reliability of the data collected. Recent Human-Computer Interaction (HCI) studies highlight how certain survey response formats can increase errors~\cite{pielotDidYouMisclick2024, kimComparingDataChatbot2019} and influence survey effectiveness~\cite{ugur2015evaluating}. In this paper, our goal is to introduce an effective interface for a~\textbf{Quadratic Survey (QS)}, a survey tool designed to elicit preferences more accurately than traditional methods~\cite{chengCanShowWhat2021}. Despite the promise of QSs, there has been no research on designing interfaces to support their unique quadratic mechanisms~\cite{grovesOptimalAllocationPublic1977}, where participants must rank and rate items --- a task that poses significant cognitive challenges. To popularize QSs and ensure high-quality data, this paper addresses the question: \textit{How can we design interfaces to support participants in completing Quadratic Surveys (QSs) more effectively?}

\begin{figure}[ht]
    \centering
    \includegraphics[width=1\textwidth]{content/image/detailed.pdf}
    \caption{The Two-Phase Interface: The interface consists of two phases. Survey respondents can navigate between phases using the top right button. In the organization phase, the interface presents one option at a time to the respondents, and they chose one of four positional choices: ``Lean Positive'', ``Lean Neutral'', ``Lean Negative'', or ``Skip''. Skipped options are hidden and can be evaluated later. The chosen options then appear below. Items can be dragged and dropped across categories or returned to the stack. In the voting phase, options are listed in the order of the four categories. When hovering over each option, respondents can select a vote for that option using a dropdown menu. Each dropdown menu contains the cost associated with the vote. A sort button allows ascending sorting within each category. A summary box tracks the remaining credit balance.}
    \label{fig:interactiveInterface}
    \Description{
    This image shows two screen captures: the Organization Phase at the top and the Voting Phase at the bottom. The Organization Phase screen contains a question titled "What societal issues need more support?" with two sections. One section shows a block with descriptions of an option, and to the right of the block are four choices: "Lean Positive," "Lean Neutral," "Lean Negative," or "Skip." The interface also shows a skipped option. Below the block, three columns contain options inside, each showing the option title and a reassign button. In the Voting Phase, the same title and instructions are displayed, but now options are listed by their previously assigned categories (columns). The image shows the mouse hovering over one of the options, revealing a dropdown menu to allocate votes, along with the associated cost in credits. Each category box has a sorting button on the right, allowing users to reorder options within the category. Dots on the left side of the options indicate that drag-and-drop functionality is available for rearranging options. In the lower right corner, a summary box titled "Credit Summary" displays the remaining credit balance for voting. A button in the top left corner allows users to return to the previous Organization Phase.
    }

\end{figure}

% Zoom into two challenges this paper tackles -- zooming in mental demand and cognitive challenges due to the QS mechanism
We envision an effective interface that navigates participants through the complex mechanism and preference construction process\change{, tailored to a QS.} A QS improves accuracy in individual preference elicitation compared to traditional methods like Likert scales by requiring participants to make trade-offs using a fixed budget of credits, where purchasing $k$ votes for an option in QS costs $k^2$ credits~\cite{quarfoot2017quadratic,chengCanShowWhat2021}. This quadratic cost structure forces respondents to carefully evaluate their preferences, balancing the strength of their support or opposition against the limited budget. \change{However, the process of making these thoughtful trade-offs introduces challenges. As individual preferences are often constructed when presented with the options~\cite{lichtensteinConstructionPreference2006}, the act of weighing costs, evaluating options, and constructing rankings increases cognitive load.} Moreover, QSs, often referred to as Quadratic Voting (QV) in voting scenarios, can involve hundreds of options~\cite{rogersColoradoTriedNew2019, teamTaiwanDigitalMinister}, increasing the risk of cognitive overload and~\change{the taking of mental shortcuts~\cite{simonBehavioralModelRational1955, payneAdaptiveStrategySelection1988, tverskyJudgmentsRepresentativeness}}.
% add and possible breakdown interfaces for mental and interfaces to scaffold complex mechanisms

% ================================ %
% par 2: Approaches to Address the Challenges
% Purpose: Describe the existing approaches related to the problem.
% Key Questions:
%  - What are some broad approaches to addressing these challenges? -- there are none.
%  - Do not go into detail about related work but give an idea of the major themes in related work.
%       - No prior research on QS, but there are existing interfaces -> auto calculation as commonality
%       - prior work on interface for reducing cognitive load, preference construction, and voting

To date, existing quadratic mechanism-powered applications simply present options, allow vote adjustments and automatically calculate votes, costs, and budget usage. Such designs focused heavily on the mechanics operating the tool, rather than supporting possible challenges these application users faced. Survey interface literature, while addressing decision-making and usability, focuses on traditional surveys that do not share the unique option-to-option trade-offs that a QS introduces~\cite{engstrom2020politics, weijtersEffectRatingScale2010, kierujVariationsResponseStyle2010, toepoelSmileysStarsHearts2019, farzandAestheticsEvaluatingResponse2024, pielotDidYouMisclick2024}.~\change{Prior research in HCI and beyond explored techniques to manage cognitive load~\cite{paula2023, oviatt2006human, toepoelSmileysStarsHearts2019, softwareBrad2021, reis2012towards} and scaffold challenging tasks~\cite{task2014, moderate2021, ibili2019effect, amyChatSensing2018} showing promise in supporting preference construction with a QS. Thus, this study aims to bridge this gap.}

% ================================ %
% par 3: Your Proposal
% Purpose: Present your main ideas and proposed solution.
% Key Question:
%  - What are you proposing? Provide a sketch of the major ideas.

\change{We propose a novel interactive two-phase ``organize-then-vote'' QS interface (referred to as the two-phase interface for short, Figure~\ref{fig:interactiveInterface}) after multiple iterations. It aims to facilitate preference construction and reduce cognitive load when making trade-offs through three key elements.} First, the interface scaffolds the preference construction process by having participants initially categorize the survey options into ``Lean Positive,'' ``Lean Neutral,'' or ``Lean Negative.'' This serves as a cognitive warm-up, easing participants into the more complex QS voting task. Second, the interface arranges the options according to these categorizations, providing a structured visual layout. Third, participants can refine the positions of these options using drag-and-drop functionality, giving them greater control and agency in the preference-construction process. %These design features are aligned with preference construction theory and build upon prior research in interface design to reduce cognitive load and enhance user engagement.

To explore how these interface elements mitigate the cognitive load and support preference construction in Quadratic Surveys, we pose the following research questions:
\begin{itemize}
    \item RQ1. How does the number of options in Quadratic Surveys impact respondents' cognitive load?
    \item RQ2a. How does the two-phase interface impact respondents' cognitive load compared to a single-phase text interface?
    \item RQ2b. What are the similarities and differences in sources of cognitive load across the two interfaces?
    \item RQ3. What are the differences in Quadratic Survey respondents' behaviors when coping with long lists of options across the two-phase interface and the single-phase text interface?
\end{itemize}

% ================================ %
% par 4: Main Findings
% Purpose: Summarize the key findings from your work.
% Key Question:
%  - What are the main findings?
We invited 41 participants to a lab study comparing our two-phase interface with a baseline to understand how different interface designs and option lengths ($6$ options or $24$ options) impact cognitive load. 

\change{Self-reported cognitive load using the NASA Task Load Index (NASA-TLX) and semi-structured interviews identified common challenges in Quadratic Surveys (QS), such as preference construction and budget management, while highlighting differences between text and two-phase interfaces. The two-phase interface fostered more strategic engagement with survey options, encouraging consideration of broader impacts in the long QS, reducing time pressure in the short QS, and eliciting greater affirmative satisfaction (e.g., "feeling good"). Quantitative results support these observations: participants in the two-phase interface—particularly in long surveys—traversed the list less frequently but maintained the same number of edits while spending more time per option. This suggests that reduced traversal did not diminish engagement. Together, these findings highlight the organizing phase's role in fostering deeper engagement with survey options.}

% Qualitative findings, measured using the NASA Task Load Index~(NASA-TLX) and semi-structured interviews, revealed that participants using the two-phase interface experienced cognitive demand more from strategic, holistic thinking compared to personal relevance and operational tasks, particularly in longer surveys. Quantitative results showed that, although participants spent more time per option, they made faster decisions during the voting phase, suggesting a more efficient distribution of cognitive effort. We concluded that the two-phase interface mitigated cognitive overload in long QS surveys and shifted mental load toward more strategic thinking, reducing reliance on mental shortcuts like satisficing~\cite{simonBehavioralModelRational1955}.

% ================================ %
% par 5: Main Contributions
% Purpose: Identify and explain the primary contributions of your work.
% Key Structure:
%  1. Line 1: Identify your contribution—conceptualization, framework, interface, algorithm, etc.
%  2. Line 2: Contrast your contribution with prior work.
%  3. Line 3: Explain how you accomplished your contribution.
%  4. Line 4: Emphasize the impact of the contribution—why should anyone care?

\paragraph{Contributions}
We contribute to the HCI community by proposing the first interface specifically designed for QS and QV-like applications, aimed at reducing cognitive challenges and scaffolding preference construction through a two-phase interface with direct manipulation. Before our work, no research had explored QS interfaces, particularly for long QS prone to cognitive overload. Few studies in HCI address interfaces for surveys and questionnaires.~\change{Our study demonstrated how user interfaces can facilitate preference construction in situ and promote deeper engagement with survey options through interface elements. Additionally, this paper offers the first in-depth qualitative analysis of user experiences among Quadratic Mechanism applications, identifying usability challenges and key factors contributing to cognitive load.} The impact of our contribution extends beyond QS, offering design implications for other preference-elicitation tools in~\change{multi-option scenarios}. By making QS easier to use and more accurate, our design also encourages wider adoption among researchers and practitioners.~\change{Finally, our work lays the groundwork for future quadratic mechanisms interface design to better facilitate individuals in communicating their preferences.}

% ================================ %
% Removed text
% Surveys are a ubiquitous tool for collective decision-making, providing decision-makers with aggregated opinions that directly shape the outcomes for those surveyed. For example, states utilize referendums to form policy decisions, organizations like the Pew Research Center survey public perspectives on societal challenges in the United States, and city councils hold forums to gather community concerns.
% and private sectors~\cite{Gov4gitDecentralizedPlatform2023}.
% xiaoTellMeYourself2020, 

% ================================ %
\section{Related Work}
\label{sec:relatedWorks}
This research lies at the intersection of three core areas: quadratic surveys, survey and voting interface design, and choice overload along with cognitive challenges. In this section, we review the related works in each of these areas.

\subsection{Quadratic Survey and the Quadratic Mechanism}
We introduce the term \textbf{Quadratic Survey (QS)} to describe surveys that utilize the quadratic mechanism to collect individual attitudes. The~\textbf{quadratic mechanism} is a theoretical framework designed to encourage the truthful revelation of individual preferences through a quadratic cost function~\cite{grovesOptimalAllocationPublic1977}. This framework gained popularity through~\textbf{Quadratic Voting (QV)}, also known as plural voting, which uses a quadratic cost function in a voting framework to facilitate collective decision-making~\cite{lalley2016quadratic}.%~\textcite{quarfoot2017quadratic} demonstrated that QV effectively gauges public opinion and mitigates the tyranny of the majority in traditional voting systems. Furthermore, QV is not subject to Arrow's impossibility theorem, which states that no voting system can perfectly aggregate individual preferences without trade-offs~\cite{morreau2014arrow}, because it does not require aggregating rankings.  

To illustrate how QS works, we formally define the mechanism: each survey respondent is allocated a fixed budget, denoted by $B$, to distribute among various options. Participants can cast $n$ votes for or against option~$k$. The cost~$c_k$ for each option $k$ is derived as:

\[c_k = n_k^2 \quad \text{where}\quad n_k \in \mathbb{Z}\]

The total cost of all votes must not exceed the participant's budget:

\[\sum_k c_k \leq B\]

Survey results are determined by summing the total votes for each option:

\[ \text{Total Votes for Option } k = \sum_{i=1}^{S} n_{i,k} \]

where $S$ represents the total number of participants, and~$n_{i,k}$ is the number of votes cast by participant~$i$ for option~$k$. Each additional vote for each option increases the marginal cost linearly, encouraging participants to vote proportionally to their level of concern for an issue~\cite{posner2018radical}.

QS adapts these strengths of the quadratic mechanism in \textit{voting} to encourage truthful expression of preferences in \textit{surveys}. Unlike traditional surveys that elicit either rankings~\textit{or} ratings, QS allows for~\textit{both}, enabling participants to cast multiple votes for or against options, incurring a quadratic cost.~\textcite{chengCanShowWhat2021} showed that this mechanism aligns individual preferences with behaviors more accurately than Likert Scale surveys, particularly in resource-constrained scenarios like prioritizing user feedback on user experiences.

In recent years, empirical studies on QV have expanded into various domains~\cite{naylor2017first, cavailleWhoCaresMeasuring}. Applications based on the quadratic mechanism have also grown, including Quadratic Funding, which redistributes funds based on outcomes from consensus made using the quadratic mechanism~\cite{buterinFlexibleDesignFunding2019a, freitasQuadraticFundingIncomplete2024}. Recent work by \textcite{southPluralManagement2024} applies the quadratic mechanism to networked authority management, later used in Gov4git~\cite{Gov4gitDecentralizedPlatform2023}. Despite the increasing breadth and depth of applications utilizing the quadratic mechanism, little attention has been paid to user experience and interface design, which support individuals in expressing their preference intensity. Our work aims to address this by designing interfaces supporting quadratic mechanisms.

\subsection{Design Implications for Surveys, Questionnaires, and Voting Systems}
The relative lack of research in quadratic mechanism and QS interface design is concerning, as prior work on survey and questionnaire interfaces has demonstrated substantial impacts on responses and user experiences, even with seemingly minor design decisions.

Research in the marketing and research communities focusing on survey and questionnaire design, usability, and interactions examines the influence of presentation styles and `response format.'~\textcite{weijtersExtremityHorizontalVertical2021} demonstrated that horizontal distances between options are more influential than vertical distances, with the latter recommended for reduced bias. Slider bars, which operate on a drag-and-drop principle, show lower mean scores and higher nonresponse rates compared to buttons, indicating they are more prone to bias and difficult to use. In contrast, visual analog scales that operate on a point-and-click principle perform better~\cite{toepoelSlidersVisualAnalogue2018}. These prior works highlighted outcomes that are influenced by the different designs.

Voting interfaces, like surveys and questionnaires, elicit individual choices, but often with consequential impacts. A well-known example is the butterfly ballot, whose atypical ballot design may have influenced the outcome of the 2000 U.S. Presidential Election.~\cite{wandButterflyDidIt2001} Researchers like~\textcite{engstrom2020politics},~\textcite{chisnellDemocracyDesignProblem2016}, and organizations like the Center for Civic Design, which publishes reports like ``Designing Usable Ballots''~\cite{DesigningUsableBallots2015}, emphasize how ballot design influences democratic processes.

Existing research surfaced how various voting interface designs shifted voter decisions, influenced human errors, or improved usability through technologies. For instance, states without straight-party voting exhibited higher rates of split-ticket voting~\cite{engstrom2020politics}, and Australian ballots, which list candidates by office without party labels, often give incumbents an advantage. Poor designs, like the butterfly ballot, have led to voter errors due to confusion over punch holes, and splitting contestants across columns increases the likelihood of overvoting~\cite{quesenberyOpinionGoodDesign2020}. \textcite{everettElectronicVotingMachines2008} further explored how digital voting interfaces improve usability over physical voting behaviors, such as lever voting. Other projects like the Caltech-MIT Voting Technology Project have sparked research to address accessibility challenges, resulting in innovations like EZ Ballot~\cite{leeUniversalDesignBallot2016}, Anywhere Ballot~\cite{summers2014making}, and Prime III~\cite{dawkinsPrimeIIIInnovative2009}. In addition, \textcite{gilbertAnomalyDetectionElectronic2013} investigated optimal touchpoints on voting interfaces, and \textcite{conradElectronicVotingEliminates2009} examined zoomable voting interfaces for improved user interactability.

The design of Voting systems and response formats significantly influence respondent behavior, decision accuracy, and cognitive load. Research like~\textcite{galesicDropoutsWebEffects2006} showed that the burden on survey respondents increases dropouts. An effective design would enhance usability and reduce cognitive challenges faced by survey respondents, especially in complex response mechanisms like QS.

\subsection{Cognitive Challenges and Choice Overload}
Despite insights from studies on quadratic mechanisms, voting, and surveying techniques, the challenge of respondents making difficult decisions using quadratic mechanisms remains unexplored in the literature.~\textcite{lichtensteinConstructionPreference2006} identified three key elements that make decisions difficult. These elements include making decisions in unfamiliar contexts, being forced to make tradeoffs due to conflicting choices, and quantifying the value of one's opinions. QS fits all three elements: participants may encounter unfamiliar options set by the decision maker, are constrained by budgets that require tradeoffs, and cast final votes as numerical values. Thus, we believe QS introduces high cognitive load.

Cognitive overload can adversely affect performance, leading individuals to rely on heuristics rather than deliberate, logical decision-making~\cite{daniel2017thinking}. When presented with excessive information, such as too many options, individuals 'satisfice', settling for a 'good enough' solution rather than an optimal one~\cite{simonBehavioralModelRational1955, payneAdaptiveStrategySelection1988, tverskyJudgmentsRepresentativeness}. Subsequently, too many options can overwhelm individuals, resulting in decision paralysis, demotivation, and dissatisfaction~\cite{iyengarWhenChoiceDemotivating2000}.

Additionally,~\textcite{alwinMeasurementValuesSurveys1985} highlighted that the use of ranking techniques in surveys can be time-consuming and potentially more costly to administer. These challenges are compounded when ranking numerous items, requiring substantial cognitive sophistication and concentration from survey respondents \cite{featherMeasurementValuesEffects1973}.

Notable applications of Quadratic Voting include the $2019$ Colorado House, which considered $107$ bills~\cite{coyNewWayVoting2019}, and the $2019$ Taiwan Presidential Hackathon, which featured $136$ proposals~\cite{QuadraticVotingFrontend2022}; both used a single QV question with hundreds of options. Psychological and behavioral research highlights the importance of understanding how individuals navigate and benefit from new interfaces under long-list QS conditions. These empirical applications of QV suggest QS's potential to elicit individual preferences, emphasizing the need to study cognitive load and interface design.

%As \textcite{chengCanShowWhat2021} noted, it is essential to better understand how the number of options influences the usability of QS and to design interfaces that effectively support survey respondents.
% there is limited research on interfaces for Constant Sum surveys~\cite{hauserIntensityMeasuresConsumer1980a}, a mechanism similar to QS that aims to elicit both ranking and rating preferences from individuals.
%The closest work discussing interfaces for QV is an arXiv paper~\cite{} that transformed the knapsack voting platform developed by \textcite{goelKnapsackVotingVoting}
% While both fields have deep insights into understanding design's influence on attitude elicitation, QS's unique capability of supporting both ranking and rating~\cite{chengCanShowWhat2021} makes designing an interface important and challenging. Subsequently, this research aims to understand how this interface influences an individual's QS response behavior. Requiring the distribution of budgets following the quadratic mechanism introduces new and complex decisions. 
% Empirical studies and applications of the quadratic mechanism and quadratic voting have increased in the past few years. Several studies have explored the empirical use cases for QV, including \textcite{quarfoot2017quadratic}'s study on 4,500 participants' attitudes across ten public policies, highlighting differences between QV and Likert scale survey results. \textcite{chengCanShowWhat2021} applied quadratic surveys in Human-Computer Interaction (HCI) and subsequently showed QV's effectiveness in reflecting true preferences in monetary decision tasks. \textcite{naylor2017first} used QV in educational research to gauge student opinions on factors affecting university success, and \textcite{cavailleWhoCaresMeasuring} examined QV in polarized choice scenarios.
\section{Quadratic Survey Interface Design}
\label{sec:interfaceDesign}
In this section, we present the QS interface. \change{Using components from existing QV interfaces described in Section~\ref{sec:relatedWorks} and insights from prior literature, we iterated through paper prototypes and three design pre-tests, detailed in Appendix~\ref{apdx:design}.} In our initial paper prototyping iterations, participants struggled to~\textit{rank} relative preferences among options and~\textit{rate} the degree of trade-offs between them. In this study, we focus on addressing the former challenge, which pertains to preference construction.

\subsection{`Organize-then-Vote': The Two-Phase Interface}
\label{sec:finalInterfaceDesign}

\subsubsection{Justifying a two-phase approach}
The main objective of the two-phase interface is to facilitate preference construction and reduce cognitive load. As shown in Figure~\ref{fig:interactiveInterface}, the interface consists of two steps: an organization phase and a voting phase. In both phases, survey respondents can drag and drop options across the presented list.

\paragraph{A two-phase approach}
Preferences are shaped through a series of decision-making processes~\cite{lichtensteinConstructionPreference2006}. Two major decision-making theories~\change{inspired} this two-step interaction interface design:~\textcite{montgomeryDecisionRulesSearch1983}'s Search for a Dominance Structure Theory (Dominance Theory) and~\textcite{svensonDifferentiationConsolidationTheory1992}'s Differentiation and Consolidation Theory (Diff-Con Theory). The former suggested that decision-makers prioritize creating dominant choices to minimize cognitive effort by focusing on evidently superior options~\cite{montgomeryDecisionRulesSearch1983}. The latter described a two-phase process where decisions are formed by initially~\textit{differentiating} among alternatives and then~\textit{consolidating} these distinctions to form a stable preference~\cite{svensonDifferentiationConsolidationTheory1992}.~\change{During our pre-tests, participants did not appreciate ranking all options prior to voting. Both theories helped explained that decisions are made through eliminating alternatives rather than generating a complete list of ranked choices.} Hence, the two-phase design --- organize-then-vote --- aimed to facilitate this cognitive journey explicitly. The first phase focused on differentiating and identifying dominant options, enabling survey respondents to preliminarily categorize and prioritize their choices. The second phase presented these categorized options in a comparable manner, with drag-and-drop functionality, enhancing one's ability to consolidate preferences. This structured approach aimed to construct a clear decision-making procedure that reduced cognitive load and enhanced clarity and confidence in the decisions made.

\paragraph{Phase 1: Organization Phase}
The goal of the organization phase was to support participants in identifying clearly superior options or partitioning choices into distinguishable groups. In this section, we first describe how the interaction works, then we detail the reasons for the implemented design decisions.

The organizing interface, depicted on the top half of Figure~\ref{fig:interactiveInterface}, sequentially presents each survey option. Participants select a response among three ordinal categories -- ``Lean Positive'', ``Lean Negative'', or ``Lean Neutral''. Once selected, the system moves that option to the respective category. Participants can skip the option if they do not want to indicate a preference. Options within the groups are draggable and rearrangeable to other groups should the participants wish.

To support preference formation, respondents are shown one option at a time, allowing them to either recall a prior judgment or construct a new one based on the presented choices~\cite{strackThinkingJudgingCommunicating1987}. Limiting the information presented this way also helps reduce cognitive load by preventing overload from too many options~\cite{swellerCognitiveLoadTheory2011}. This incremental process ensures that participants form opinions on individual options.

The three possible options --- Lean Positive, Lean Neutral, and Lean Negative --- aim to scaffold participants in constructing their own choice architecture~\cite{munscherReviewTaxonomyChoice2016, thalerNudgeImprovingDecisions2008a}, which strategically segments options into diverse and alternative choice presentations while avoiding biases from defaults. We believed that these three categories were sufficient for participants to segment the options. We do not limit the number of options one can place in each category to prioritize user agency, allowing participants full control over how they organize their preferences~\cite{norman2013design}. Immediate feedback displays the placement of options and allows participants to rearrange them via drag-and-drop, adhering to key interface design principles~\cite{norman2013design}. At the same time, it allows finer-grain control for individuals to surface dominating options and create differentiating groups of options.

\paragraph{Phase 2: Interactive Voting Phase}

The objective of the voting phase is to facilitate the consolidation of differentiated options through interactive elements while reinforcing the differentiation across options constructed by participants in the previous phase. This facilitation is achieved by retaining the drag-and-drop functionality for direct manipulation of position and enabling sorting within each category.

Options are displayed as they are categorized within each category from the previous step and in the following section --- Lean Positive, Lean Neutral, Lean Negative, and Skipped or Undecided --- as detailed on the bottom half of Figure~\ref{fig:interactiveInterface}. The Skipped or Undecided category contains options left in the organization queue, possibly because survey respondents have a pre-existing preference or chose not to organize their thoughts further. The original order within these categories is preserved to maintain and reinforce the differentiated options. This ordering sequence mitigated early prototype concerns where uncategorized options were left at the top of the voting interface confusing survey respondents. Respondents have the flexibility to return to the organization interface at any point during the survey to revise their choices.

In the voting interface, options are draggable, allowing participants to modify or reinforce their preference decisions as needed. Each category features a sort-by-vote function for reordering within the group, which, although it doesn’t affect the final outcome, supports information organization and consolidation. Both features aim to group similar options automatically and emphasize proximity, reducing cognitive load by following the proximity compatibility principle to enhance decision-making~\cite{wickens1990proximity}.

While multiple interaction mechanisms exist, drag-and-drop has been extensively explored in rank-based surveys. For instance,~\textcite{krosnick2018measurement} demonstrated that replacing drag-and-drop with traditional number-filling rank-based questions improved participants' satisfaction with little trade-off in their time. Similarly,~\textcite{timbrook2013comparison} found that integrating drag-and-drop into the ranking process, despite potentially reducing outcome stability, was justified by the increased satisfaction and ease of use reported by respondents. The trade-off was deemed worthwhile as QS did not use the final position of options as part of the outcome if it significantly enhanced user satisfaction and usability~\cite{rintoulVisualAnimatedResponse}. Together, these design decisions led to our belief that a two-phase interface with direct interface manipulation could reduce the cognitive load for survey respondents to form preference decisions when completing QS.

In addition, we made three aesthetic design decisions~\change{considering existing QV-based interfaces}. First, we removed visual elements like icons, emojis, progress bars, and vote visualizations, as prior research indicated that emojis could influence survey interpretations and reduce user satisfaction~\cite{herringGenderAgeInfluences2020, toepoelSmileysStarsHearts2019}. While effective visualizations can aid decision-making, this study does not aim to address that question. Second, the final interface has all options presented on the screen at the same time, intentionally. Unlike all the prototypes and existing interfaces, prior literature emphasized the importance of placing all the options on the same digital ballot screen to avoid losing votes~\cite{CenterCivicDesign}. This echoes the proverb ``out of sight, out of mind,'' where individuals might be biased toward options that are shown to them, and additional effort is required for individuals to retrieve specific information if options are hidden. Last, we decided to use a dropdown positioned to the right of each survey option for ease of access to the budget summary when determining the votes. The layout of the votes and cost was inspired by online shopping cart checkout interfaces where quantities are supplied next to the itemized costs followed by the total checkout amount. After testing two alternative~(Figure~\ref{fig:btn_design}) input methods—click-based buttons,~\change{which participants dislike making multiple clicks}, and a wheel-based design, which offered intuitive control but was unfamiliar to some participants—we opted for a more accessible dropdown menu for vote selection.

\begin{figure}[ht!]
    \centering
    \includegraphics[width=0.8\textwidth]{content/image/prototypes/btn_design.png}
    \caption{Alternative vote control. The click-based design (upper) mirrors traditional vote control used in other QV interfaces, where each click controls one vote. The wheel-based design (the latter two) allows control through both clicks and mouse wheel rotation.}
    \Description{Three voting control interfaces are displayed. Each row represents a different interface. The first row shows a traditional click-based voting interface with options to decrease, increase, or maintain a rating of +3. The second and third rows show a wheel-based voting interface with mouse wheel functionality. In these, the middle row indicates a current rating of +3, with +2 and +4 ratings also visible. The cost for each option is listed on the right, ranging from 4 to 16. The last row mirrors the previous one with a rating of +3 and a cost of 9.}

    \label{fig:btn_design}
\end{figure}

\begin{figure}[ht]
    \centering
    \includegraphics[width=\textwidth]{content/image/detailed_text.pdf}
    \caption{The text-based interface: This interface is based on the two-phase version but does not include the organization phase and lacks the drag-and-drop functionality. Options are randomly positioned.}
    \Description{An image of a voting interface asking users to select societal issues needing support. The title reads, "What societal issues need more support?" with a brief explanatory paragraph underneath. Below, a list of six options is displayed, including "Youth Education Programs and Services," "Advocacy and Education," "Zoos and Aquariums," "Community Foundations," "Environmental Protection and Conservation," and "International Peace, Security, and Affairs." Each option has a description, a current vote count, and a dollar amount. The right side of the image shows an expanded dropdown menu for one of the options with selectable voting choices, such as "1 upvote" and "2 upvotes." A separate box labeled "Credit Summary" shows the remaining credit of 9 and a "Submit" button below it.}
    \label{fig:textInterface}
\end{figure}

\subsection{Baseline Interface: Single-Phase Text Interface} ~\change{We implemented the single-phase text interface (referred to as text interface for short, Figure~\ref{fig:textInterface}) as our control condition to compare how the organizational components influenced participants' cognitive load and behavior. The text-based interface, like all existing interfaces, contains a list of static elements, a summary box, and a vote control. We followed the same design considerations, removing visual elements, presenting all options in the same screen, and using the dropdown for vote control, following the two-phase interface interface to provide a more direct comparison. We position the question prompt at the top followed by a randomly ordered option list to prevent ordering bias~\cite{ferberOrderBiasMail1952, couperWebSurveyDesign2001} below. Individual option costs and the remaining credits' summary box are presented to the right of the screen given our interface layout.}

Both experimental interfaces were developed with a ReactJS frontend and a NextJS backend powered by MongoDB. We open-source both interfaces.\footnote{link-to-github}


% In our first prototyped tool, we aim to help survey respondents rank options to establish relative preferences before voting. As shown in Figure~\ref{fig:qv_rank}, our prototype allows respondents to move options before finalizing their votes. During our pretest, we found that respondents rarely moved the options and some questioned the need for a full ranking since it did not affect the QS submission. Many did not realize the options were draggable until we pointed it out. The main insight from this prototype is that creating a full rank is~\textit{not} essential for establishing~\textit{relative} preferences, leading us to consider selecting a subset of options instead of requiring a full rank among all options.

% First, we surveyed the current implementation of QV interfaces to understand the development of such tools. We presented a selection in Figure~\ref{fig:qv_interface_external}. All five interfaces retained and presented the following components:
% \begin{itemize}
%     \item Option list: A list of options contesting for votes.
%     \item Vote Controls: Two buttons to increase and decrease votes associated with each option.
%     \item Individual vote tally: A representation of votes associated with an option.
%     \item Summary: A summary that automatically calculated the cost across options and the remaining budget.
% \end{itemize}
% Now we present the final interactive interface and describe how it operates. In this subsection, we provide supporting evidence from prior literature that we previously omitted. These pieces of literature were omitted for clarity and focus in the previous subsection but will be reintroduced here. 
% We constructed a text-based interface that included all five components but removed the use of emojis (i.e., thumbs up and thumbs down present in Figure~\ref{fig:wedesignInterface}), progress bars, and other visualizations in the summary section (i.e., progress bars in Figure~\ref{fig:wedesignInterface} and~\ref{fig:chengInterface} or blocks presented in Figure~\ref{fig:rxcvotingInterface}), and the visual cues for individual vote counts (i.e., the colored counts and icons present in Figure~\ref{fig:gov4gitInterface} and~\ref{fig:chengInterface}).

% During this process, we noticed several issues. First, many survey respondents placed most options into the 'option you care about' category, defeating the design's purpose. Second, there were no indicators distinguishing between the selected and remaining options. Respondents did not notice their selections were kept at the top in the voting stage and were unsure why Step 1 was necessary if all options were shown again. This informed two takeaways: selecting options to vote on is too coarse to construct relative preferences, and there needs to be a clearer distinction and connection between the two phases.

% Feedback indicated that survey respondents are comfortable with this two-phase organize-then-vote design. Several user experience issues emerged, but they were addressable without significantly modifying this interaction structure. These issues include: First, dragging and dropping all options into different categories is cumbersome and can mislead respondents into thinking this is a ranking process, which is not the goal. Second, the position of unorganized options at the top of the voting list is counterintuitive. Third, the voting controls are disconnected from the option summaries, dividing attention between the left and right sides of the screen.

% These design decisions led to the interface shown in Figure~\ref{fig:textInterface}. 


% \subsubsection{Paper prototype: visualizing trade-offs}
% The original paper prototype aimed to help visualize survey respondents' tradeoffs among options. 
% The original paper prototype aimed to utilize visual representations to highlight the constrained availability of credit and to explain the costs and trade-offs associated with selecting each available option.
% Early on, we did not know which components made QS more difficult than other survey techniques. We began by surveying existing interfaces (Figure~\ref{fig:qv_interface_external} other than Figure~\ref{fig:gov4gitInterface} which did not exist near the writing of this paper). All four interfaces consist of these common components:
% As we were unsure what made QS more complex than other survey techniques, our investigation began with the existing interface (Figure~\ref{fig:qv_interface_external}, except Figure~\ref{fig:gov4gitInterface} which did not exist at that time. All four interfaces consist of these common components:
% \begin{itemize}
%     \item Option list: A list of options contesting for votes.
%     \item Vote Controls: Buttons to increase and decrease votes associated with each option.
%     \item Individual vote tally: A representation of votes associated with an option.
%     % \item Summary: A summary that automatically calculates the cost across options and the remaining budget.
%     \item Summary: An auto-generated summary of costs and remaining budget.
% \end{itemize}

% To brainstorm ways to help survey respondents manage trade-offs across options, we decomposed these options and explored several innovative layouts. Initially, we thought trade-offs were the core cause of cognitive load. In this paper, we show two versions of the paper prototypes in Figure~\ref{fig:qv_paper}. In both figures, costs are represented by blocks, similar to Figure~\ref{fig:rxcvotingInterface}. We imagine the survey respondents to drag and position options in the space provided unstructurally (Figure~\ref{fig:horizontal_paper}) or structurally (Fig~\ref{fig:vertical_paper}). Similar to the seminal debate on direct manipulation vs. interface agents~\cite{shneidermanDirectManipulationVs1997}, the research team was unsure how much control survey respondents should have over the positioning of the options to aid the decision-making process that considers trade-offs. Different from prior interfaces, we used placements of the interface to denote positive or negative number of votes. After several pretests, we learned that the main process participants aim to do throughout the survey is establishing~\textit{relative} preferences across the options, rather than thinking so much about trade-offs. 
% To further explore the features that contribute most to the complexity of QS, we developed two prototypes shown in Figure~\ref{fig:qv_paper}. Similar to Figure~\ref{fig:rxcvotingInterface}, these prototypes use blocks to represent costs, arranged either unstructurally (Figure~\ref{fig:horizontal_paper}) or structurally (Fig~\ref{fig:vertical_paper}), facilitating the visualization of the trade-offs. Unlike previous interfaces, we utilized the placement of the interface to denote positive or negative vote counts. Several protests indicated that participants primarily focus on establishing relative preferences among options rather than trade-offs. Therefore, in this study, we focused on enhancing designs that facilitate the establishment of relative preferences.
% the two main features of QS we identified are the relative preference through a combined presentation of rankings and ratings, and the option selection trade-offs due to total credit limits. 
% The initial prototyping involves collecting the interface designs for existing quadratic mechanism-based software. Iterative pretests informed each subsequent design. We present these iterations, which aim to enhance user experience in the preference construction process in the following sections.

% In the previous subsection, we highlighted critical prototype iterations that informed the final two-phase interactive process that defines the user journey. 
% We now present the final two-phase interface, its operations, and the supporting literature for comprehensive understanding.
% Then, We also discuss the aesthetic design choices that emerged throughout the iterations.

\section{Experiment Design}
\label{sec:experiment}
Based on the design decisions, we developed a QS interface using a React.js frontend and a Next.js backend powered on MongoDB. Both services were open-sourced~\footnote{link-to-github}.

We recruited participants from a midwestern college town using online ads, digital bulletins, social media posts, physical flyers, and online newsletters. The study's researcher prioritized the non-student population to maximize participant diversity. When recruiting participants, we did not reveal that the goal of this study was to measure their cognitive load and study their behaviors, rather a study that elicited community members' attitudes on societal issues. The reason we withheld such information was to prevent response biases. This study was reviewed and approved by the college Institutional Review Board.

\begin{figure}[ht]
    \centering
    \includegraphics[width=1\textwidth]{content/image/study_flow.pdf}
    \caption{Study protocol}
    \label{fig:studyProtocol}
\end{figure}

Figure~\ref{fig:studyProtocol} shows a visual representation of the study protocol. Study participants were invited to the lab to participate in this study. The reason we made this experiment design decision was to minimize the influence of external factors that could affect the measurement of cognitive load. External factors, more prevalent in remote experiments or those conducted via platforms like MTurk, included potential multitasking or interruptions by others. An in-lab study also allowed participants to operate across a consistent device that researchers had full control over. More specifically, the experiment involved participants operating on a 32-inch vertical monitor. This setup assured study participants, despite any condition in the study, could see all options on a QS, minimizing hidden information from an individual's decision-making process.

After consenting to the study, participants were invited to the study and they watched a pre-recorded video explaining the Quadratic mechanism and how QS operates. This video did not include any hints of either interface and how to operate the interface. Participants were then asked to complete a short quiz. The purpose of the quiz was to ensure that all participants fully understood how QS works. Participants were not screened out if they failed the quiz but were asked to rewatch the video or ask the researcher until they were able to select the correct answer. The device that the participant worked on was screen captured throughout the study.

The researcher then primed the participant that the purpose of this study was to assist local community organizers in understanding community members' preferences on a wide variety of societal issues so they could potentially distribute limited resources better. Participants were randomly placed into one of the four groups:

\begin{itemize}
    \item 6 options with a text-based interface
    \item 6 options with an interactive interface
    \item 24 options with a text-based interface
    \item 24 options with an interactive interface
\end{itemize}

Participants began completing the survey independently, without the researcher's presence. Upon completion, they contacted the researcher, who then requested they complete the NASA-TLX to assess cognitive load. This was followed by a short semi-structured interview to gain insights into the participants' experiences. This interview was audio recorded. The session concluded with a debriefing and a \$15 cash compensation for their participation. The debriefing explained to the participant that not disclosing the purpose of the survey was to measure cognitive load and interface design and allowed for participants to ask any questions.

% Finally, participants complete the situational motivation scale (SIMS) to gauge motivation and a demographic survey.

The study was designed as a between-subject study for two reasons. First, we aimed to minimize the study fatigue that might occur given the complexity of responding to a QS. To complete a QS survey, participants could take up to 20 minutes. Thus, it was difficult to conduct back-to-back experiments that measure cognitive load. We chose not to ask participants to revisit the lab with several days in between, to reduce dropout rates and prevent demotivating participants from attending the in-person experiment, which might occur in a within-subject study design. Second, we aimed to reduce the learning effect that is difficult to remove, especially concerning operating the interface and making decisions on the survey. Recall that preferences are constructed, we wanted to ensure that participants were not influenced by their previous preferences which could influence their perceived cognitive load.

In an ideal world, understanding participants' cognitive load across multiple options would require enumerating all possible numbers of options and eliciting the ``breaking point'' where the participant experiences cognitive overload. Unfortunately, this was not feasible. Iterating through all possible numbers of options was very costly, both in time and resources. Therefore, we referred to prior literature to inform our choice of 6 and 24 options, representing a short and long list of options. To decide the number for the short list, survey methods such as constant sum surveys and Analytic Hierarchy Process (AHP) recommended options fewer than ten and seven, respectively~\cite{moroneyQuestionnaireDesignHow2019, saatyGroupDecisionMaking2013, saatyPrinciplesAnalyticHierarchy1987}. However, we were not aware of any specific works that justified these numbers. \textcite{saatyPrinciplesAnalyticHierarchy1987} associated this value with both the cognitive processing capacity of $7\pm2$~\cite{millerMagicalNumberSeven1956} and a theoretical proof using the consistency ratio of a pairwise comparison metric~\cite{saaty2003magic}. This informed our decision to contain a pair of dependent variables above and below seven options. We turned to experiments designed to study choice overload. A meta-analysis by~\textcite{chernevChoiceOverloadConceptual2015} surveyed 99 choice overload experiments (N = 7202) and summarized that 6 and 24 are the modal values for short and long lists when testing choice overload. These two values were likely rooted in the original choice overload experiment by~\textcite{iyengarWhenChoiceDemotivating2000}. The value six is often used in experiments to understand the effect of choice provision. The value 24 is the maximum number of ecologically valid jams produced by the jam company in the original study. We decided to follow suit with these two values, satisfying the previous decision to choose two values less than and greater than seven.

Next, we describe the context of the survey that participants completed. Participants were asked to complete a societal issue survey. We followed suit as described by~\textcite{chengCanShowWhat2021}, believing that surveying societal issues is a good topic as it is relevant to every citizen and it is easy to convey that there are limited resources in the public sector to be prioritized across different sectors and areas. Participants across all four groups were presented with options randomly drawn from 26 societal issues. These issues were generated from the categories used by Charity Navigator~\cite{CharityNavigatorAnimals2023}, a non-profit organization that evaluates over 20 thousand charities in the United States. The full list of these societal issues is provided in Appendix B.

% Last, we describe the two quantitative measurements taken during the study: cognitive load and motivation. 

Last, we describe the quantitative measurements taken during the study: cognitive load. At the time of this study, several methods existed to measure cognitive load, including performance measures, psychophysiological measures, subjective measures, and analytical measures~\cite{gaoMentalWorkloadMeasurement2013}. Given the nature of QS, a task requiring a long period, adopting performance measures like secondary-task measures in our experiment proved challenging due to the difficulty of designing a secondary task. The secondary task had to use the same cognitive resources as the primary tasks, and the cognitive resource for completing the survey would vary among participants. Similarly, psychophysiological measures such as pupil size~\cite{palinkoEstimatingCognitiveLoad2010} and ECG~\cite{haapalainenPsychophysiologicalMeasuresAssessing2010} could be highly sensitive to external environments and costly to obtain. Consequently, we relied primarily on subjective measures via self-report surveys and analytical measures like time and clicks collected via the interface. We adopted a paper-based weighted NASA Task Load Index (NASA TLX), a multidimensional scoring procedure using the weighted average of six subscale scores to represent overall workload. Weighted NASA-TLX used a priori workload definitions from subjects to weight and average subscale ratings, requiring subjects to evaluate each weight's contribution to the workload of a specific task~\cite{hart1988development, hartNasaTaskLoadIndex2006, cain2007review}. This approach reduced between-rater variability, indicating differences in workload definitions among raters within a task and variations in workload sources between tasks~\cite{cain2007review}. Despite criticisms regarding its validity and vulnerability, NASA-TLX was commonly used due to its low cost and ease of administration~\cite{gaoMentalWorkloadMeasurement2013}. It had been tested on various experimental and lab tasks, and workload scores derived from these tests showed significantly less variability among evaluators than one-dimensional workload scores~\cite{rubioEvaluationSubjectiveMental2004}. Thus, we chose NASA-TLX to measure cognitive load in our study.

% Tabling SIMS for now. It was not used in the analysis. In addition to NASA-TLX, we administered a situational motivation scale (SIMS) to measure participants' motivation (required citation). We posited that motivation would influence mental demand (required citation). SIMS, chosen for its widespread use, helps understand one's intrinsic motivation, extrinsic motivation, identified regulation, and external regulation, and was originally designed to measure self-determination. Both instruments were administered using pen-and-paper.


\input{content/3.1_result_cognitive_load.tex}
\section{Clickstream data: Interface reduces edit distance in long surveys}
\label[SSE]{dist}

Following our findings on cognitive load, we analyze voting behaviors to identify differences in how participants cope with survey lengths, how interfaces influence their behavior, and why the long text interface might exhibit lower cognitive load. All data are publicly available\footnote{link-to-github} to ensure transparency and support further research. This measure reveals trends in participants' navigation and engagement with survey options. We examine three dimensions of this measure: edit distance per option, edit distance per action, and cumulative edit distance throughout the survey.

\begin{figure}[ht]
    \centering
    \includegraphics[width=0.5\textwidth]{content/image/distance/distance_diff_by_version.pdf}
    \caption{Edit Distance Per Option: We sum the total number of edit distances for each option, with the figure using the radius to indicate how often a specific edit distance occurred within an experimental condition. Interpretation: Participants in the two-phase interface completed their votes for more options with fewer edit distances, whereas the Long Text interface shows a long tail of options requiring a wider range of edit distances.}
    \label{fig:dist_per_option}
\end{figure}

\textbf{Edit distance per option:} We sum up all the distances a participant moves while adjusting values for a single option. Each of these totals is referred to as the edit distance per option. Figure~\ref{fig:dist_per_option} illustrates differences across the four experimental conditions, with the long text interface showing the largest variance in the distance traveled and the highest mean. We implement a hierarchical Bayesian framework to model edit distance differences across experimental conditions. The observed distance differences are modeled using an exponential distribution, where the scale parameter is linked to survey length (treated as an ordinal variable), interface type (treated as a categorical variable), interaction effects between length and interface, and controlling for individual user variability. The linear predictor includes a global intercept and slope for length, random effects for each interface condition with an LKJ prior that captures the correlations among interface categories, and user-specific random effects to account for individual heterogeneity. Detailed mathematical formulations of the model are provided in Appendix~\ref{sec:apdx:model_distance_option}. 

\begin{figure}[ht]
    \centering
    \includegraphics[width=0.9\textwidth]{content/image/distance/distance_diff_per_option_effect_size_by_version.pdf}
    \caption{The figure shows the contrast distributions of the mean edit distance per option between pairwise experimental conditions, with the first row representing absolute differences and the second row depicting effect sizes. The main finding is that participants in the long text estimated more edit distance per option compared to those in the short text and the long two-phase condition. Notably, the long two-phase interface required estimated only slightly more edit distances despite the longer survey length.}
    \label{fig:dist_per_option_bayesian}
\end{figure}


Figure~\ref{fig:dist_per_option_bayesian} illustrates the pairwise posterior distributions for differences in edit distances across experimental conditions. For example, the difference in edit distances between the short and long static interfaces has a mode of 9.1, with a 94\% highest density interval (HDI) of [6, 13]. This indicates that participants in the long text interface move approximately 9.1 steps more than those in the short text interface, with a high degree of confidence. The effect size is large (mode = 5.1, 94\% HDI = [3.3, 7.1]), suggesting a statistically significant difference, which is expected due to the greater number of options in the long text interface.

Similarly, participants using the two-phase interface make approximately 8.9 fewer steps per option (mode = 8.9, 94\% HDI = [6.4, 12]) compared to those in the long text interface, with a large effect size (mode = 5.7, 94\% HDI = [4.2, 7.9]). Comparatively, the increase in edit distances between the short and long two-phase interfaces is substantially smaller (mode = 1.7, 94\% HDI = [-0.01, 3.1]) compared to their static counterparts discussed above. The comparison between the short text and short two-phase interfaces shows weak evidence for a difference, with a mode of 1.3 and a 94\% HDI of [-0.78, 3.8]. While the interval includes zero, the posterior distribution slightly favors (with 89.3\% probability) the two-phase interface requiring fewer steps. Results from this model suggest that the organization phase in the two-phase interface reduces participants' edit distance per option on average, especially for the long QS.

\begin{figure}[h]
    \centering
    \includegraphics[width=0.8\textwidth]{content/image/distance/edit_distance_per_action_by_version.pdf}
    \caption{Edit Distance Per Action: This plot shows the frequency of specific edit distances at each step across the text interface and two-phase interface. Interpretation: Participants in the long two-phase interface tend to make adjustments closer to their previous actions, resulting in visually less variance in edit distances throughout the entire survey.}
    \label{fig:step-over-distance}
\end{figure}

\textbf{Edit distance per action:} Building on the statistical disparities observed in the previous analysis and the unique patterns exhibited by long text interface participants, we present analyses focusing on edit distance per action and cumulative edit distance throughout the survey between the long text and long two-phase interfaces. Edit distance per action measures how far participants move during each adjustment while completing the survey. Figure~\ref{fig:step-over-distance} illustrates how, at each step, the number of participants moving a given distance (represented by the size of the dots) varies across experimental conditions. Visually, participants move less on average per option within the two-phase interface, with lower variance at smaller scales. This indicates that participants are making local edits, meaning their adjustments tend to occur near their previous edits in terms of edit distance. This also highlights that the organization phase effectively adjusts option positions for easier access, despite participants still having the freedom to move across the interface as all options are presented to them.

In contrast to earlier analyses, we use a hierarchical Bayesian model (detailed in Appendix~\ref{sec:apdx:model_distance_variance}) to jointly estimate the mean and variance of edit distances across experimental conditions. The model assumes that edit distances are continuous and follow a Normal likelihood. This approach accounts for both central tendencies and variability, using separate predictors for the mean and variance. The model includes hierarchical effects for survey length, interface type, interactions between length and interface, and user-level random effects. Non-centered parametrization is used for survey length and interface type to improve convergence, while interaction effects are modeled with an LKJ prior to capture the correlations between factors. User-level random effects reflect individual differences in behavior, incorporating variability into the model.

\begin{figure}[h]
    \centering
    \includegraphics[width=0.8\textwidth]{content/image/distance/distance_diff_per_step_effect_size_by_version.pdf}
    \caption{The figure shows the contrast distributions of the mean edit distance per step between the two-phase interface and text interface for different survey lengths. The left two subplots represent absolute differences, while the right two depict effect sizes. The main finding is that participants in the long text condition exhibited greater variance in edit distance per step compared to those in the long two-phase interface. Similarly, the short text condition showed higher differences, although these were not statistically significant in Bayesian terms.}
    \label{fig:step-over-distance_bayesian}
\end{figure}

Figure~\ref{fig:step-over-distance_bayesian} illustrates the posterior variance distributions, confirming our hypothesis. Participants in the long text interface exhibit greater variance in movement, frequently navigating across the interface, compared to those in the long two-phase interface. This is evidenced by a variance difference mode of 76 (95\% HDI = [59, 99]) and a large effect size (mode = 7.1, 95\% HDI = [5.5, 9.2]).

\begin{figure}[h]
    \centering
    \begin{minipage}[t]{0.48\textwidth}
        \centering
        \includegraphics[width=\textwidth]{content/image/distance/cumulative_edit_distance_per_option_long_qs_v3v4.pdf}
        \caption{This plot shows how the cumulative edit distances gained over the course of the survey between long text and long interactive groups. Interpretation: Participants in the long two-phase interface tend to make smaller, more incremental adjustments, resulting in a visually flatter slope compared to the text interface.}
        \label{fig:cumulative-distance}
    \end{minipage}
    \hfill
    \begin{minipage}[t]{0.48\textwidth}
        \centering
        \includegraphics[width=\textwidth]{content/image/distance/slope_diff_and_effect_size.pdf}
        \caption{The figure shows the contrast distributions of slope differences in cumulative edit distance between the two-phase interface and text interface for long QS. The left subplots show absolute differences, while the right depict effect sizes. Main Finding: Participants in the long text interface exhibited a steeper slope, indicating a faster increase in cumulative edit distance compared to the long two-phase interface.}
        \label{fig:slope-diff-effect}
    \end{minipage}
\end{figure}


\textbf{Cumulative edit distance for a participant:} This reduction in per action distance due to the two-phase interface's effect on edit distance adds up, as Figure~\ref{fig:cumulative-distance} shows the cumulative edit distance over time. Some long text participants traverse double the amount of distance to complete the task compared to the long two-phase participants. We model this growth rate using a hierarchical Bayesian regression model (Detailed in Appendix~\ref{sec:apdx:model_cum_distance}), with cumulative distance as the predictive variable. The experimental variables include interface type as a categorical variable, individual users modeled with random effects, and steps taken as a continuous variable. The model incorporates a shared global intercept, version-specific intercepts and slopes with partial pooling to balance data across conditions, and user-specific random effects to capture variability. A truncated normal likelihood constrains cumulative distances to positive values and varies these distances across steps for each participant while masking incomplete data.

Figure~\ref{fig:slope-diff-effect} shows that the slope for the long text interface is approximately 4.7, meaning each step by the text interface would add 4.7 edit distance (94\% HDI = [4.2, 5.4]), compared to the long two-phase interface, which shows a statistically significant difference with a mode of 1.4 (94\% HDI = [1.3, 1.7]). These results explain that the variance in edit distance per action and the increase in per option edit distance are consistent across participants between the two groups, showing that the organization phase allows participants to focus on adjusting options within proximity without having to navigate the interface to locate and make adjustments during the voting phase.

\textbf{Evidence from qualitative analysis:} Recall the differences in sources of cognitive load between the two experimental conditions: while two-phase interface participants make adjustments with nearby options, they experience cognitive demand from preference construction due to broader considerations involving more options and higher-order values. Similarly, the qualitative results highlight that long text interface participants construct narrower preferences, yet their edit distance indicates that their movements cover more options.

Notably, fewer participants (60\%, N=6) report precise resource allocation in the long two-phase interface compared to 90\% in the long text interface. These results make it evident that two-phase interface participants are more focused on deliberating preferences than simply completing the survey. Furthermore, the ability to make localized adjustments while considering broader decisions suggests that participants construct preliminary preferences during the grouping phase, allowing them to focus on deciding their votes.

These results provide evidence that the initial pass through the survey items, combined with the organizational phase, helps participants construct preliminary preferences, thereby reducing the need for large traversals between options. This could exemplify that participants in the long text interface are more concerned about operating to 'complete' the task (i.e., looking for an option to adjust votes) rather than continuing to stay engaged with the survey options and the preference construction task, particularly in the long survey.



% Physical demand refers to the physical effort required to complete a task, such as physical exertion or movement. The two-phase interface experienced higher physical demand from increased mouse usage.

% \begin{displayquote}  
% Because with this many (options), especially when I'm thinking \ldots\ Ok, where was (the option) \ldots\ Where was (the option) you know? Oh, that's right. Maybe I could give another upvote to the, you know, whatever~\bracketellipsis \hfill\quoteby{S028 (LT)}  
% \end{displayquote}  
\section{Clickstream data: Interface participants' time spent}
In addition to distance, we analyze the time participants spend per option. We aggregate the total time participants spend per option using the QS system log. For participants in the two-phase interface conditions, this includes both organization and voting times for that option. The results are visualized in Figure~\ref{fig:total_time}.


\begin{figure}[h]
    \centering
    % First subfigure
    \begin{subfigure}[b]{0.52\textwidth}
        \centering
        \includegraphics[width=0.95\textwidth, trim=0 10 0 10, clip]{content/image/results/total_time_per_option.pdf}
        \captionsetup{width=\textwidth, justification=justified} % Adjust the width to match the image width
        \caption{Total Time per option: We identified that the two-phase interface skewed slightly higher than the text interface, as expected. This discrepancy can be attributed to the extra organization step required in the two-phase interface, leading to a slightly longer overall completion time per option.}
        \Description{Violin plot showing total time spent per option in seconds across four interface versions: Short Text, Short 2-Phase, Long Text, and Long 2-Phase. The y-axis ranges from 0 to 60 seconds. Each violin plot has scattered dots representing individual data points. The shape of the Short Text plot is widest between 10 and 20 seconds, tapering at the top and bottom. The Short 2-Phase plot is the narrowest, with most dots concentrated between 10 and 20 seconds. The Long Text plot is narrow and widest near the bottom, between 5 and 15 seconds. The Long 2-Phase plot is widest near the top, between 20 and 40 seconds.}
        \label{fig:total_time}
    \end{subfigure}
    \hfill
    % Second subfigure
    \begin{subfigure}[b]{0.38\textwidth}
        \centering
        \includegraphics[width=0.95\textwidth, trim=0 13 0 13, clip]{content/image/cog/Temporal_scores.pdf}
        \captionsetup{width=\textwidth, justification=justified} % Adjust the width to match the image width
        \caption{Temporal Demand Raw Score: The short text interface results in the highest temporal demand, while the long text interface is the lowest. Two-phase interfaces show moderate temporal demand, suggesting that interactive elements allowed participants to pace themselves better.}
        \label{fig:temporal_cog_score}
    \end{subfigure}
\end{figure}

Overall, participants spend slightly more time per option in the two-phase interface than in the text interface. To quantify these observations, we model the time data using independent beta distributions within a Bayesian framework, assuming independence across experimental conditions. For example, participants using the long two-phase interface spend significantly more time per option than those using the long text interface (medium to large effect size, XX, d=XX), with an even more pronounced difference between the short two-phase and short text interfaces (medium effect size, XX, d=XX). These findings suggest that the interactive two-phase interface encourages longer deliberation, particularly for longer lists of options. Participants in both interfaces tend to spend more time per option with high probability. Details of the modeling procedure and priors are provided in Appendix XX.

Some literature points to increased time leading to time fatigue~\cite{}, which can impair decision-making. Other decision science literature suggests that longer decision times can indicate deeper cognitive processing~\cite{payneAdaptiveDecisionMaker1993}. Our qualitative analysis points to the latter. 

Other than the difference in operational thinking and strategic consideration discussed in Section~\ref{sec:demand}, we find that 37.5\% of participants (N=15) who attribute time to \textit{Decision Making} as a source of temporal demand frame such demand differently. We label a participant as \textit{affirmative} if they describe the pressure to make decisions as a source of temporal demand. For example, \smallquote{S022}{So it didn't take too much time, but obviously there were a lot of things to consider, so there was some temporal demand.} is an affirmative statement. Conversely, we label a participant as \textit{negative} if they express concern about the time and effort they have already invested. For example, \smallquote{S024}{maybe I should just hurry up and make a decision.} is a negative statement.

50\% of participants (N=5) in the long two-phase group describe the pressure to make decisions affirmatively and none negatively. This suggests that their pressure stems from having too many remaining decisions to make, rather than from the time already invested. This is reflected in their higher average time spent per option and overall time spent ($\mu=716.86$ seconds, $\sigma=164.04$ seconds) completing the QS survey compared to the long text group ($\mu=449.64$ seconds, $\sigma=206.97$ seconds). We interpret this as evidence that participants are thoughtfully engaged in constructing their preferences and choose to invest additional time, rather than being driven by decision-related pressures or experiencing a sense of urgency.

Conversely, in the short text group, 50\% of participants (N=5) express concern about the time and effort they have already invested~(\smallquote{S024}{maybe I should just hurry up and make a decision.}) and none frame it affirmatively. Descriptively, participants in the short text group spend comparatively less time than those in the long QS (short text: $\mu=139.83$ seconds, $\sigma=76.43$ seconds; short two-phase: $\mu=178.78$ seconds, $\sigma=61.07$ seconds). This suggests that participants in the short text group expect themselves to complete the task sooner than they actually do. 

Surprisingly, participants in the long text interface exhibit a temporal demand lower than the short text and long 2-phase participants(Figure~\ref{fig:temporal_cog_score}, quantifiable results in Appendix XX), despite spending more time per option and traversing the longest distance (Section~\ref{sec:dist}). Only 30\% of participants (N=3) mention the time spent making a decision as a source of temporal demand. One possible explanation is that some participants are satisficing, which we will discuss further in Section~\ref{ref:secsatisfice}.  

In summary, we interpret the result that participants in the two-phase interface spend more time per option as a sign of deeper cognitive processing. This is further supported by examining participants' nuanced voting behaviors under budget constraint conditions for the long QS, which we omit for brevity. Notably, two-phase interface participants make more small vote adjustments (i.e., adding or removing at most 2 votes on an option) when they have fewer remaining credits, further supporting our claim that they experience deeper engagement with preference construction, which we elaborate on further in Appendix~\ref{apdx:budget_voting_behaviors}.


\section{Discussion and Future Works}
\label{sec:discussion}

In this section, we interpret the results related to cognitive load and survey respondent behaviors, emphasizing why the interactive interface did not uniformly reduce cognitive load in the long text interface while providing practical recommendations for practitioners deploying QS.

Our discussion centers on three key topics: elements of the two-phase interface that support preference construction, design recommendations for practitioners, and future challenges. Ultimately, we conclude that the two-phase interface has differential effects on the short and long surveys. While trends suggest a reduction in cognitive load with the two-phase interface compared to the text interface, we observe evidence of deeper engagement with options and enhanced preference construction, particularly in the long survey condition.

\subsection{Result Interpretation}
\subsubsection{Deeper engagements through preference construction in two-phase interfaces}

Our main findings indicate that the survey results, qualitative data, and observed behavioral differences reveal shifts in the types of cognitive load experienced by participants, especially for those completing the long survey. Cognitive load theory~\cite{swellerCognitiveLoadTheory2011}, when applied to the context of QS, identifies the three components of cognitive load: intrinsic load (the cognitive demand required to understand questions and response options), germane load (associated with deeper processing and evaluation of preferences), and extraneous load (stemming from navigating and operating the survey interface).

Participants are randomly assigned to experimental conditions, with both survey lengths containing options randomly drawn from a common pool to control intrinsic load within the same group.  

In the short survey condition, participants engage with all options simultaneously. The two-phase interface reduces some extraneous load associated with navigating the interface during voting, though it requires participants to complete the grouping phase. Despite this additional task, participants across both interface types report minimal or no physical demand. The two-phase interface likely facilitates easier engagement with preference construction due to its lower-trended cognitive load, as reflected in the increased likelihood of perceived lower cognitive load.  

In the long survey condition, participants cannot engage with all options simultaneously, resulting in a higher intrinsic load at the start of the survey. The organization phase in the two-phase interface shapes participant behavior during the voting phase. While it streamlines the process of locating options, as exemplified by the reduction of edit distance, this benefit may be offset by the additional physical effort required to complete the grouping phase, as reflected by the slightly increased physical demand.  

However, qualitative data regarding the voting task suggest that participants maintain their ability to invoke deeper engagement with options. Quantitative data reveal that participants make no fewer overall edits, with a bimodal distribution suggesting continued revision even at low budgets. Additionally, participants strategically consider broader options as they deliberate on nearby ones. These findings indicate a cognitive shift toward germane load, particularly during the voting phase.

In contrast, participants in the long text interface experience higher extraneous load, evident in shallower reflection and shorter overall voting times, despite exhibiting a greater overall edit distance. While some might argue that the additional grouping phase offers participants more opportunities to familiarize themselves with the options, the long edit distance suggests that participants in the text interface traverse the list frequently, providing ample opportunity to adjust their preferences. Qualitative data indicate that 70\% of long text participants (N=7) scan the list while voting, with edit distance data reflecting multiple passes across the list.

The deliberate one-option-at-a-time presentation during the voting task in the two-phase interface reduces reliance on defaults and encourages deeper reflection. This is best-illustrated by~\texttt{S013}, who emphasizes how the organization phase supports their preference construction:  
\begin{displayquote}  
\bracketellipsis it (organization phase) gives you time to just focus on that single thing and rank it based on how you feel at that moment. \hfill\quoteby{S013 (SI)}  
\end{displayquote}  

Thus, based on this evidence, we argue that a text-based interface is not an optimal solution for long QS where deeper engagement and preference construction are desired. A two-phase interface enables participants to effectively exercise germane load, fostering deeper engagement with the content.


% Conversely, participants using the text interface, like \texttt{S003}, described how default placements influenced their decisions:

% \begin{displayquote}
%     Honestly, if medical research~\bracketellipsis was the first one I saw, I think it would automatically give it a lot more. \hfill\quoteby{S003 (ST)}
% \end{displayquote}

% possiblity from the overall reduction of extraneous load. This additional benifit from added time and slightly higher cognitive load, at the minimum, prevented cognitive overload, , fostering reflection and deliberation which is relatively important when completing surveys.

% Based on the current evidence,  The two-phase interface, by supporting reflection and minimizing reliance on defaults, demonstrates the potential for more effective design strategies in similar contexts.

\subsubsection{Plausible satisficing behaviors in long QS}
In addition, the observed lower overall cognitive load in the long text interface may partly reflect~\textit{satisficing behaviors}. Satisficing refers to participants settling for \textit{good enough} rather than \textit{optimal} decisions~\cite{gigerenzerReasoningFastFrugal1996} when unable to process all available information. Interviews reveal that 40\% of participants (N=4) in this condition describe using satisficing strategies, while none from the long two-phase interface report such behaviors.These strategies are exemplified by participants prioritizing minimal effort over thorough evaluation, as illustrated by:

\begin{displayquote}
    ~\bracketellipsis you thought of enough things, you know, and so it wasn't the most effort I could put in because again, that would have been diminishing returns. I tried to think of enough things~\bracketellipsis and then move on.~\bracketellipsis 
    
    I felt like that (the response) was satisfied, but not perfect. Cause perfect is not a reality. \hfill\quoteby{S036 (ST)}
\end{displayquote}

This quote illustrates satisficing decision-making, where participants settle for suboptimal choices. Additional participants describe similar strategies when deciding on votes:  

\begin{displayquote}
    ~\bracketellipsis Because that was what was left. [Laughter] I probably wouldn't use that on <optionA> instead of <optionB>.~\bracketellipsis \hfill\quoteby{S015 (LT)}

    I tried to use them~\bracketellipsis it went negative, and then I just settled for just \$6 remaining. ~\bracketellipsis I don't think it's perfect. But I think I'm satisfied. Yeah, I'm satisfied.  \hfill\quoteby{S033 (LT)}

    ~\bracketellipsis when I had first started like looking at the first few, I was just doing it kinda like willy nilly, I'm not really paying that much attention to necessarily how many credits I had, or how many categories there were. \hfill\quoteby{S041 (LT)}
\end{displayquote}

These quotes highlight how participants in the long text interface adjust to external constraints rather than carefully weighing internal preferences. This behavior suggests that cognitive overload may lead participants to adopt less effortful strategies. However, further research is needed to fully understand the prevalence and impact of satisficing in long QS surveys.  

\textbf{In summary}, the two-phase interface likely reduces extraneous load, particularly in the long survey condition, facilitating a cognitive \textit{shift} toward deeper reflection and more deliberate decision-making. While the extent to which long QS surveys induce cognitive overload or satisficing remains unclear, the interactive interface shows promise in promoting deeper engagement with options and supporting comprehensive preference construction. The following section explores the specific elements that guide participants toward these outcomes.

% ============================== %
\subsection{Construction of Preference on Quadratic Survey}

Completing QS involves a series of difficult decision tasks~\textcite{lichtensteinConstructionPreference2006}. ~\textcite{svensonDifferentiationConsolidationTheory1992}'s differentiation and consolidation theory help explain how participants process these decisions. The decision process begins with differentiation, where participants identify differences and eliminate less favorable options, followed by consolidation, which strengthens their commitment to selected choices. This theory aligns with how the two-phase interface helps participants decompose options into categories, effectively reducing decision complexity.

Participants start by constructing preferences in situ, especially regarding options they have not previously considered:
\begin{displayquote}
    \bracketellipsis`Oh, there are other aspects that I never care about.' And actually~\ldots some people care <an option>. Sure. Why? Why (should) I spend money on that? \\\hfill\quoteby{S037 (LI)}
\end{displayquote}

Those using the text interface, lacking the interactive tools, find it challenging to facilitate differentiation, as~\texttt{S025} notes:

\begin{displayquote}
    I would like to be able to like, click and drag the categories themselves so I could maybe reorder them to like my priorities.~\bracketellipsis make myself categories and subcategories out of this list~\ldots If I could organize it. \hfill\quoteby{S025 (LT)}
\end{displayquote}

In contrast, the two-phase interface allows participants to express at least one dimension of differentiation more easily. The drag-and-drop feature helps blend this differentiation into the consolidation phase. Not only do participants drag-and-drop options post-voting to reflect and assure a correct vote allocation, but it also enables participants, like~\texttt{S039}, to make fine-grain comparisons between options:  

\begin{displayquote}  
    I think the system was actually really helpful because I could just drag them.~\bracketellipsis I can really compare them, I can drag this one up here, and then compare it to the top one~\bracketellipsis \hfill\quoteby{S039 (SI)}  
\end{displayquote}  

The bi-modal behavior observed in the long interactive interface participants aligns with the differentiation and consolidation framework, as described in the results. Participants in the two-phase interface begin differentiating options earlier, allowing them to later adjust fine-grain votes. The faster and smaller vote updates indicate participants are consolidating. The less prominent bi-modal behavior from the long text interface participants implies that the interface guides this decision framework, as participant~\texttt{037} explains:

\begin{displayquote}
    I only start from the positive one~\bracketellipsis I finish everything~\ldots and then I move to the second part (the neutral box).~\bracketellipsis I want to focus on these and make sure that resources are at least they get the attention they want. And if there's surplus and they can move to the second part. \hfill\quoteby{S037 (LI)}
\end{displayquote}

In addition, the three key elements of the organization phase—presenting options one at a time, grouping them into categories, and enabling drag-and-drop—work together to structure participant preferences. These elements align with cognitive strategies like~\textit{\smash{problem decomposition}}~\cite{simonSciencesArtificial1996} and~\textit{\smash{dimension reduction}}, which reduce cognitive overload. Bounded rationality highlights how cognitive limitations lead to sub-optimal decision-making due to the inability to process all available information~\cite{simonBehavioralModelRational1955}. It illuminates the importance of decision-making support interfaces rather than serving as a critique of human behaviors. One participant explains how the organization phase breaks down complex decisions into manageable steps:  

\begin{displayquote}  
\bracketellipsis being able to have a preliminary categorization of all the topics. First, it introduced me to all the topics,~\bracketellipsis to think about and process~\bracketellipsis being able to digest all the information prior to actually allocating the budget or completing the quadratic survey. \hfill\quoteby{S009 (LI)}  
\end{displayquote}  

Participants using the two-phase interface, especially in the long version, organize options along dimensions such as topics (e.g., health vs. humanitarian) and preferences (positive vs. negative) before voting. Others express that the upfront introduction of all options and the ability to rank and group them help manage their cognitive load effectively. In contrast, almost half of the participants using the long text interface, like~\texttt{S028}, express a desire for features that could help reduce the decision space when responding to the QS, further supporting the importance of these organizational design elements:

\begin{displayquote}  
Because with this many (options), especially when I'm thinking \ldots\ Ok, where was (the option) \ldots\ Where was (the option) you know? Oh, that's right. Maybe I could give another upvote to the, you know, whatever~\bracketellipsis \hfill\quoteby{S028 (LT)}  
\end{displayquote}  

This quote reflects participants' need to manually track and revisit options, which occupies extraneous load, without a more structured interface.  

These evidence explain how the organization phase and the drag-and-drop features support differentiation and consolidation, and scaffold a decision-making framework that enables deeper engagement.  

\textbf{In summary}, participants construct their preferences as they complete QS. We observe behaviors and qualitative insights that align with the differentiation and consolidation theory in decision-making. Our interface scaffolds many of the differentiation stages through pre-voting organization and some consolidation phases through drag-and-drop, explaining how the two-phase approach supports preference construction to yield deeper engagement with QS options.  

% ========================= %
\subsection{Future Work: Opportunities for better budget management}
Budget management is a recurring theme in participant interviews. While they appreciated the automatic calculation feature in modern QV interfaces, we identified three challenges for future QS interfaces: ~\textit{cognitive load},~\textit{the cold-start problem}, and~\textit{navigating between budget, votes, and outcome}.

\subsubsection{Automatic calculation is critical}
Over one-third of participants ($N=14$) from all four experiment conditions emphasized the importance of automated calculation for deriving costs and tracking expenditures. For example:

\begin{displayquote}
I thought I have \bracketellipsis (to) do all the numbers or calculations myself \bracketellipsis The credit summary section was really wonderful in doing all the calculations on that end. \hfill\quoteby{S005 (LT)}
\end{displayquote}

The quotes marked the importance that QS must be facilitated by computer-supported interfaces.

\subsubsection{The coldstart problem}
We notice from the study that one of the biggest challenges for participants is deciding 'how many votes' to start with. This challenge pertains to the initial vote, not the relative vote. Some participants began by equally distributing their credits to all options and then made adjustments. Others established $1$, $2$, and $3$ votes as starting points. A small handful surprisingly used the tutorial's example of 4 upvotes as their anchor.

This arbitrary voting decision echoes discussions in prior literature about the existence of an absolute value for individuals. Coherent arbitrariness~\cite{arielyCoherentArbitrarinessStable2003}, similar to the anchoring effect in marketing, refers to participants' willingness to allocate votes, which can be influenced by an arbitrary value. However, the ordinal utility remains intact among the set of preferences.

\subsubsection{Navigating Between Budget, Votes, and Actual Impact}
The third challenge is participants' confusion between budget, votes, and outcomes, despite understanding their definitions. One participant stated:

\begin{displayquote}

~\bracketellipsis get rid of the upvote column or just get rid of the word upvote and just really focus on the money column. Listen. You're an organization or your participant. You have X amount of dollars you need to. You can only distribute X amount of dollars to these causes. So you have to figure out which ones get the most, which ones don't get as much.~\bracketellipsis 

Interviewer: ~\bracketellipsis Do you feel that the more votes you're giving to a cause you're actually spending more on it?

Yeah. \hfill\quoteby{S003 (ST)}
\end{displayquote}

Participants like \texttt{S003} bypassed the quadratic formulation, directly translating votes to resource allocation. While this does not invalidate the power of the quadratic mechanism, it causes frustration and friction for participants to construct a clear picture of how to make voting decisions. Future interfaces should better communicate these relationships to facilitate respondents' trade-offs.

\textbf{In summary}, while the interface supports budget management through automated cost calculation, participants still face cognitive load from managing the budget. The cold-start problem and the confusion between budget, votes, and actual impact are open questions for future research. These challenges highlight the need for better budget management support to complete the QS interface.

\subsection{Quadratic Survey Usage, Design Recommendations and Future Work}
With a deeper understanding of how survey respondents interact with QS and the sources of cognitive load, we recognize that while this current interface may not significantly reduce cognitive load, it represents a crucial step toward constructing better interfaces to support individuals responding to QS. In this subsection, we outline usage and design recommendations applicable to all applications using the quadratic mechanism and highlight directions for future work.

\subsubsection{Usage Recommendation: QS for Critical Evaluations}
Our study highlighted the complex cognitive challenges and in-depth consideration required when ranking and rating options using QS, even in a short survey. Similar to survey respondents needing to make trade-offs across options, researchers and agencies seeking additional insights and alignment with respondent preferences must ensure that survey respondents have the cognitive capacity to complete such surveys rigorously. QS should be designed for critical evaluations, such as investment decisions, or situations where participants have ample time to think and process the survey. Pactioners should also caution the use of long QS. If long QS is not avoidable, considering allowing participants to deliberate on each option prior to deploying QS without the organizing phase. For instance, revealing the options ahead of time can aid in preference construction.

\subsubsection{Design Recommendations}
\paragraph{Use Organization Phases for Quadratic Mechanism Applications}
Our study demonstrated that preference construction can shift from operational to strategic and higher-level causes. An additional organizational phase with direct manipulation capability allows survey respondents to engage in higher-level critical thinking. We believe this approach should extend beyond QS to other ranking-based surveying tools, such as rank-choice voting and constant sum surveys. Further research should examine how implementing such functionality alters survey respondents' mental models.

\paragraph{Facilitate Differentiation through Categorization, Not Ranking}
Participants in our study were less inclined to provide a full rank unless necessary. The final 'rank' of option preferences often emerged as a byproduct of their vote allocation, constructed in situ. Therefore, for survey designs to be effective in constructing preferences, it is more important to facilitate differentiation than to focus on direct manipulation solely for fine-tuning. Emphasizing categorization can better support participants in articulating their preferences.

\subsubsection{Future Work: Support for Absolute Credit Decision}
Deciding the absolute amount of credits in QS is highly demanding. Designing interfaces and interactions that address the cold start challenge and help participants decide the absolute vote value while considering ways to limit direct influences remains an open question. Future research should explore innovative solutions to support participants in making these complex decisions effectively.

By implementing these recommendations and pursuing future research directions, we can improve the usability and effectiveness of QS and other quadratic mechanism-powered applications, ultimately aiding respondents in making more informed and accurate decisions.

\section{Limitations}
\label{sec:limitations}
Evaluating the QS interface is challenging due to its novelty. During the study, we identified several limitations that require further research.

\paragraph{Understanding results influence on decision-makers}
Further research is required to understand how the QS interface impacts decision-makers and broader societal resource distributions. Since QS is still in its early stages, we prioritize its widespread adoption and usage before attempting a comprehensive assessment of its influence on decision-making. Future studies will examine how decision-makers interpret and use QS data, as well as its broader implications for societal decisions.

\paragraph{Individual differences in cognitive capacity}
Variations in individual cognitive capacity influenced participants' cognitive scores. For example, participants with more experience in decision-making might be able to manage multiple options more effectively. A within-subject study could clarify cognitive load shifts, but deconstructing established preferences and altering options further complicates this. Thus, we opted for this in-depth, between-subject study, although the small sample size may introduce noise that distorts the actual cognitive load. Future research should quantify the impact of different QS interfaces. In addition, participants completed this study in a controlled lab environment with options displayed on a large screen. Future work should also explore how individuals respond to QS on smaller devices in a less controlled environment.

\paragraph{Limited experience with QS}
Participants had no prior experience with the QS interface. Following a tutorial and quiz, participants proceeded to complete tasks using the QS interface. While participants understood the QS mechanics, familiarity with the interface still influences strategies and cognitive load. As quadratic mechanisms become more prevalent, future research can compare novices and experts.

\paragraph{Duration between clicks to represent decision-making}
Click duration may include time spent considering other options, so it must be treated as an approximate measure of decision-making time. For instance, deciding between two options may take longer for the first option and less time for the second. Despite its limitations, this approach provides valuable insights into decision-making within our experimental constraints.

% 

% Recall that this survey aims to assist community organizers in distributing resources to a societal cause. This participant decided to `skip' over the quadratic formulation and the concept that their votes are governed by the quadratic formulation, drawing a direct translation between votes and the resources to which community organizers ought to contribute. 
% \begin{displayquote} I guess to see what my ranking looks like~\ldots and see if I could give more money or not. \hfill\quoteby{S021 (LI)} \end{displayquote}
% \begin{displayquote}
% If I had to choose a number like that in the beginning. That would have been really bad, but positive, neutral, negative. That was good enough. \hfill\quoteby{S026 (LI)}
% \end{displayquote}

% \begin{displayquote}
% I think \ldots\ ranking at the beginning one's impression towards these issues helps to like determine how many votes should be put towards them.  \hfill\quoteby{S002 (SI)}
% \end{displayquote}

% \begin{displayquote}
% If anything, I think I would like to be able to like, click and drag the categories themselves so I could maybe reorder them to like my priorities. \hfill\quoteby{S025 (LT)}
% \end{displayquote}

%  of well-organized interfaces in managing cognitive load.

% Participants ($N=4$, $2$ using the long two-phase interface) mentioned that organization support helped them to allot the intensity of votes by helping them focus and prioritize options through ranking. This exercise allows them to follow a clear decision-making process that avoids confusion.
% It is important to note that bounded rationality does not critique or exploit biases, but emphasizes the importance of designing interfaces that prevent decisions which diverge from one's true preferences. For example,~\underline{\smash{problem decomposition}}~\cite{simonSciencesArtificial1996} and~\underline{\smash{dimension reduction}} are strategic approaches to managing cognitive overload. Several participants would create a two-axis grouping, regardless of their experiment group. Participants clustered topics (e.g., health vs. humanitarian) and preferences (positive vs. negative). The difference between conditions was whether these groupings were representable on the interface.
% In addition, results indicate long text interface participants were satisfied due to cognitive overload from having too many options. They have to read more text, allocate more credits, and consider more options. Section~\ref{sec:cog_result} and Section~\ref{sec:behave_result} show how counterintuitive that this group had fewer participants experiencing high cognitive load compared to the short text interface. This group also experienced the least temporal demand (Sec.~\ref{sec:temporal}) while showing no difference in time spent per option compared to the text interface (Figure~\ref{fig:vote_time}). Participants in the long text interface also expressed the least frustration with operational tasks (Sec~\ref{sec:frustration}). 
% % These participants engaged with higher-level strategic challenges, in contrast to the more operational tasks emphasized in the text interface. 
% \paragraph{Familarity to the options}
% 1. primed on the local community, 
% 2. limited experience with qs
% We also acknowledge the possibility that the elicited values are pure noise and do not reflect the actual cognitive load. This could be due to the small sample size, the nature of the task, or the participants' understanding of the cognitive load scale. While this true for small sample sizes, we believe that the qualitative insights from the interviews provide a more nuanced understanding of the cognitive load sources. We detail limitations in Section~\ref{sec:limitations}.
% Maybe large scale AB testing and within subject testing in periodic collective decsion making enviornments.
% 3. time associated with the option.

% \subsection{The Quadratic Mechanism is Challenging}
% % We know QV is accurate and that QM allows specific expression of preferences
% % However, QM is diffucult to manage, internally construction of preference is diffuclt but so is the QM.

% % we tried to scaffold the construction of preference in interface design, for which we did help participants get to the exact values faster, but identifying and managing the construction is not something organization interfaces can fully support
% Most challenges participants faced come from the task itself: deciding the number of votes/credits to allot. I created the following hierarchical theme
% CI_3: deciding number of votes and credits (N=9/40, v1:1; v2:1; v3:4; v4:3)
% We can see participants in the long version group struggles more with this challenge. (2/20 vs. 7/20). So what exactly contributes to this decision process? We broke it down to the following themes:

% % Challenge lies in the mechanism itself
% CI_1: working with the QS mechanism (N=6, v1:5; v3:1)
% distinguishing between credits and votes (v3 participant)
% quadratic mechanism (all the rest)
% The first finding is that non of the interactive interface groups (v2 and v4) expressed feeling challenged due to the QS mechanism. The second finding is that the majority of this challenge comes from the short-list group. I think an explanation to this is clearer when we put up the second theme:

% CI_2: use up the remaining credits (N=4, v1:3, v2:1)
% The participants is struggling to express specific level of preference with limited credits. 
% Revisit one of the quotes from CI_1:
% “I wish I could just put the $2 towards the museum, or something like that.” (S036,v1)
% “It would be nice if I can use that one credit if there is an option, because the way it is done is in quadratic...I don't know why that is there...but if there is an option to not have it, and just [inaudible], that would be awesome.” (S012, v1)
% In other words, the expressiveness is constraint by the limited credit, amplified by the quadratic nature, forcing participants to forgo unused credits. This is also likey tied to prospect theory, that we will discuss later.
% The interface in the second group could have eliminated this because some options were eliminated, or that some ranking were established, prior to the voting process.


% \subsection{Construction of Preference}
%  \subsection{Design Implications}
% % Your content for the subsection on design implications goes here

% IN_T1: Dropdown (N=6/40)
% This is a common issue that participants dislike, across all versions.
% 3 from v1, the rest of the version each has 1

% IP_T3: Seeing all options – a sign in making decisions
% Participants like the ability to see all options on one screen (N=8/40)
% Comparing Long (N=5) and Short (N=3)
% The interactive interface requires a stronger need to see all options, as I hypothesis that this is because the need to interact and see the hierarchical groupings (Text: 2; Interactive: 6)

%% on positioning shift and the power of priming
%  use the performance quotes to highlight how participants are thinking in the shoes of decision-maker
% look for literature
% survey designed for decison makers to aid decisions

% Participants either felt positive or no issues using all four interfaces (N=33/40).


% \subsection{Limitations and Future Work}
% % Your content for the subsection on limitations and future work goes here


% We first show that participants constructed their preferences in situ. While some participants had existing preferences (e.g., environmental issues are important), they needed to reconsider aspects of the options or map them to their beliefs.

% \begin{displayquote}

% ~\bracketellipsis the other part of the mental demanding was probably trying to associate with (what) I'm concerned in soci(ety)~\bracketellipsis is that question able to deal with my social concerns like, for example, climate change~\bracketellipsis How does that fit in?

% \noindent \hfill -- S006, long interactive interface
% \end{displayquote}


% Behavior analysis in section~\ref{res:act} of participants using the long text and interactive interfaces revealed that they made small adjustments on the votes, clustered toward budget depiction with lesser time spent. These fine-grain adjustments indicated that participants are making less ad-hoc decisions; rather, they are deciding how to better utilize the remainder of the budget when the budget runs low. We identified a bi-modal interaction pattern.

% indicated that participants are making less ad-hoc decisions; rather, they are deciding how to better utilize the remainder of the budget when the budget runs low. We identified a bi-modal interaction pattern.

% \

% Conversely, in the text interface, one participant proactively mentioned a request to add click-and-drag functionalities to the interface. The participant described such function to group by topic categorization and also priority placement through direct manipulation.


% Throughout the preference construction journey, we confirm that the two-stage interactive interface and the direct manipulation through drag-and-drop facilitated participants in constructing and reflecting on their preferences, adhering to preference construction theory.

% Additionally, several participants mentioned how the direct manipulation functionality, allowing individuals to drag and drop options for repositioning, supports their reflective thinking during preference construction. One participant noted:
% \begin{displayquote}
% So I tried to make a ranking \bracketellipsis and by creating this ranking, by dragging the related issues \ldots\ I don’t know \ldots\ that helped me organize my ideas.
% \noindent \hfill -- S021, long interactive interface.
% \end{displayquote}

% into these categories, making completing the entire QS a series of difficult decisions.
% Literature from~\textcite{lichtensteinConstructionPreference2006} identifies three types of difficult decision-making scenarios: when one's preferences are not clearly defined, necessitating trade-offs, or quantifying opinions.  
% Since the interface supported some participants in managing their limited cognitive ability to make decisions, as shown in the previous subsection, we argue that the interactive interface \textit{shifted} the cognitive focus onto contributing to more in-depth preference construction and fine-tuning, even if it did not significantly reduce the cognitive load. Here we provide more evidence.

% Literature from~\textcite{lichtensteinConstructionPreference2006} identifies three types of difficult decision-making scenarios: when one's preferences are not clearly defined, necessitating trade-offs, or quantifying opinions. 


% Two participants highlighted the importance of automated calculation regarding the cost for each vote.
% Twelve participants highlighted the summarization box and the automated summation of the current credit spent, allowing them to focus on managing their next voting decision and expressing their preferences.

% \begin{displayquote}
% I like that I don't have to make the calculation of the dollars that it does it automatically. So if I had to do it myself it would be more tedious. And so I think that that effort and frustration and mental demand would be much higher. So I appreciate that that calculation occurs automatically and very easily.
% \noindent \hfill -- S017, short interactive interface.
% \end{displayquote}

% This is less significant in the short QS likely due to the reduced complexity~\footnote{We show in Appendix~\ref{sec:appendix_short_breakdown} that short interfaces exhibits the same bimodal behaviors but less obvious.}.

\section{Conclusion}
Surveys enable decision-makers to aggregate crowd opinions. In this study, we use QS to elicit individual responses in the context of social resource allotment. After multiple design iterations, we propose an interactive interface for QS. We then examined its influence on individuals' cognitive load and behaviors when faced with societal issues of varying lengths. In a 2x2 between-subject study, we had participants experience either a long or short QS using a text-based or interactive interface. NASA-TLX questionnaires and interviews revealed that participants using the interactive interface for a long QS demonstrated a more comprehensive and critical evaluation of societal issues, despite not experiencing a lower cognitive load. Participants using the long text interface experienced cognitive overload, which led to satisficing behaviors or mental shortcuts. Analyzing click-stream data, we identified that participants made fine-grain iterations using the long interactive interface when credits were low. We demonstrate that a two-phase, organize-then-vote interface can scaffold the complex decision-making process, helping individuals express their opinions for collective societal decisions. Through the iterative design process and detailed interviews, we identified future directions and design recommendations for collective decision-making applications using the quadratic mechanism.
\begin{acks}
We thank the voluntary participants who participated in the pretest, pilot, and the study. We thank all members of the Social Spaces Group and the Crowd Dynamics Lab for their support and early feedback. Additional thanks to Yi-Ting Kuo, Hsin-Ni Yu, Yun-Shan Sam Yang, Katherine Chou, and the anonymous reviewers who provided valuable feedback to this work. This work was partially supported by Just Infrastructures at University of Illinois at Urbana-Champaign.
\end{acks}






\printbibliography


%TC:ignore
% \appendix
% Interface design appendix
\section{Voting Interface Breakdown}\label{apdx:relatedVoting}
Compared to digital survey interfaces, there exists a rich literature on voting interfaces, which we argue is a special type of survey interface. We categorize these related works into three main categories detailed below:

\paragraph{Designs that shifted voter decisions: } For example, states without straight-party ticket voting~(where voters can select all candidates from one party through a single choice) exhibited higher rates of split-ticket voting~\cite{engstrom2020politics}. Another example from the Australian ballot showing incumbency advantages is where candidates are listed by the office they are running for, with no party labels or boxes.
\paragraph{Designs that influenced errors: } Butterfly ballots increased voter errors because voters could not correctly identify the punch hole on the ballot. Splitting contestants across columns increases the chance for voters to overvote~\cite{quesenberyOpinionGoodDesign2020}. On the other hand, \textcite{everettElectronicVotingMachines2008} showed the use of incorporating physical voting behaviors, like lever voting, into graphical user interfaces.

\paragraph{Designs that incorporated technologies: } Other projects like the Caltech-MIT Voting Technology Project have sparked research to address accessibility challenges, resulting in innovations like EZ Ballot~\cite{leeUniversalDesignBallot2016}, Anywhere Ballot~\cite{summers2014making}, and Prime III~\cite{dawkinsPrimeIIIInnovative2009}. In addition, \textcite{gilbertAnomalyDetectionElectronic2013} investigated optimal touchpoints on voting interfaces, and \textcite{conradElectronicVotingEliminates2009} examined zoomable voting interfaces.

\section{Voting Interfaces and Response Format}

Research in the marketing and research communities focusing on survey and questionnaire design, usability, and interactions examines the influence of presentation styles and `response format.'~\textcite{weijtersExtremityHorizontalVertical2021} demonstrated that horizontal distances between options are more influential than vertical distances, with the latter recommended for reduced bias. Slider bars, which operate on a drag-and-drop principle, show lower mean scores and higher nonresponse rates compared to buttons, indicating they are more prone to bias and difficult to use. In contrast, visual analog scales that operate on a point-and-click principle perform better~\cite{toepoelSlidersVisualAnalogue2018}. These studies show how even small design changes can have a large impact on usability, highlighting the importance of designing interfaces that prioritize human-centered interaction rather than focusing solely on functionality.

Voting interfaces are a specialized type of survey interface that not only elicit individual choices but often have consequential impacts. For example, the butterfly ballot, an atypical design, may have influenced the outcome of the 2000 U.S. Presidential Election~\cite{wandButterflyDidIt2001}. Research has shown that ballot interfaces can significantly influence democratic processes~\cite{engstrom2020politics, chisnellDemocracyDesignProblem2016, DesigningUsableBallots2015}. Several studies also highlighted how voting interface designs shift voter decisions~\cite{engstrom2020politics}, reduce usability errors~\cite{quesenberyOpinionGoodDesign2020, everettElectronicVotingMachines2008}, or improve interaction~\cite{leeUniversalDesignBallot2016, summers2014making, dawkinsPrimeIIIInnovative2009, gilbertAnomalyDetectionElectronic2013, conradElectronicVotingEliminates2009}. We provide more details to these voting interfaces in the Appendix~\ref{apdx:relatedVoting}.

From the QV implementations, response format literature, and voting interfaces, we identified how interfaces significantly influence respondent behavior, decision accuracy, and cognitive load. While these systems are functional, they lack the human-centered design needed to reduce cognitive load and make them truly usable, rather than simply operable. These burdens are especially problematic for complex systems like QS, where high cognitive demands may deter researchers and users alike. Developing effective, human-centered interfaces for QS could enhance usability, reduce cognitive overload, and increase adoption in both research and practical applications.




%TC:endignore

\end{document}
\endinput

