% \documentclass[format=acmsmall, natbib=false, review=false, authordraft=false, anonymous=true, screen=true]{acmart}
\documentclass[acmsmall, natbib=false, anonymous=true, screen]{acmart}

\usepackage{enumitem}
\usepackage{graphicx}  % another package that works for figures
\usepackage{booktabs} % For formal tables
\usepackage{cleveref} % for better references
\usepackage{caption,subcaption}
\usepackage[english]{babel}% Recommended
\usepackage{csquotes}% Recommended
\usepackage{siunitx}
\usepackage{tabularx}
\usepackage{xcolor}
\usepackage{rotating}
\usepackage{pdflscape}
\usepackage{afterpage}
\usepackage{hyperref}
% \usepackage{rerunfilecheck}
% \usepackage{epstopdf}
\usepackage{lscape}
\usepackage{wrapfig}
\usepackage{sidecap}
\usepackage{rotating}

\usepackage{mdframed}
\usepackage{fontawesome} % for the icon
\usepackage{amsthm}
\usepackage{tcolorbox}

\newtcolorbox{tldrbox}[1][]{colback=gray!10, colframe=darkgray, left=5pt, right=5pt, top=3pt, bottom=3pt, arc=0pt, outer arc=0pt, toprule=0pt, bottomrule=0pt, leftrule=1.5pt, rightrule=0pt, #1}


%references
\usepackage[backend=biber, style=acmnumeric,sorting=none]{biblatex}
\let\citename\relax
\addbibresource{tcheng.bib}
\renewcommand{\bibfont}{\Small}
\usepackage{url}
\setcounter{biburllcpenalty}{7000}
\setcounter{biburlucpenalty}{8000}

%  Comment this out to fix all citation
\renewcommand{\cite}[1]{[??]}
\renewcommand{\textcite}[1]{Mark et al.}

\newcommand{\bracketellipsis}{[\ldots]\xspace}

% other required packages
\RequirePackage{rotating}
\usepackage[ruled]{algorithm2e} % For algorithms
\renewcommand{\algorithmcfname}{ALGORITHM}
\SetAlFnt{\small}
\SetAlCapFnt{\small}
\SetAlCapNameFnt{\small}
\SetAlCapHSkip{0pt}
\IncMargin{-\parindent}

\renewenvironment{displayquote}
  {\list{}{\small\leftmargin=1em\rightmargin=1em}\item\relax\itshape\color{darkgray}}
  {\endlist}

\newcommand{\smallquote}[2]{
    {\color{darkgray}\texttt{#1}~\faCommentsO~\textit{#2}}
}
\newcommand{\quoteby}[1]{
    {\color{darkgray}\faCommentsO~\texttt{#1}}
}

%% author color
%% color: http://latexcolor.com/
\definecolor{lapislazuli}{rgb}{0.15, 0.38, 0.61}
\newcommand{\hs}[1]{{\color{red}{HS: #1}}}
\newcommand{\tc}[1]{{\color{green}{TC: #1}}}
\newcommand{\lwt}[1]{{\color{olive}{LWT: #1}}}
\newcommand{\kk}[1]{{\color{violet}{KK: #1}}}
\newcommand{\vk}[1]{{\color{blue}{VK: #1}}}
\newcommand{\yhc}[1]{{\color{orange}{YHC: #1}}}
\newcommand{\change}[1]{{#1}}

%% Rights management information.  This information is sent to you
%% when you complete the rights form.  These commands have SAMPLE
%% values in them; it is your responsibility as an author to replace
%% the commands and values with those provided to you when you
%% complete the rights form.
% \setcopyright{acmcopyright}
% \copyrightyear{2021}
% \acmYear{2021}
% \acmDOI{10.1145/1122445.1122456}

\setcopyright{acmlicensed}
\acmJournal{PACMHCI}
% \acmYear{2021} \acmVolume{5} \acmNumber{CSCW1} \acmArticle{182} \acmMonth{4} \acmPrice{15.00}\acmDOI{10.1145/3449281}

% For Articles 180-195:
% \received{October 2020}
% \received[revised]{January 2021}
% \received[accepted]{January 2021}


% %% These commands are for a PROCEEDINGS abstract or paper.
% \acmConference[CSCW '21]{CSCW '21: The 24rd ACM Conference on Computer-Supported Cooperative Work and Social Computing}{Nov 03 -- 07, 2021}{Toronto, Canada}
% \acmBooktitle{CSCW '21: The 24rd ACM Conference on Computer-Supported Cooperative Work and Social Computing,
% Nov 03 -- 07, 2021, Toronto, Canada}
% \acmPrice{15.00}
% \acmISBN{978-1-4503-9999-9/18/06}


%%
%% Submission ID.
%% Use this when submitting an article to a sponsored event. You'll
%% receive a unique submission ID from the organizers
%% of the event, and this ID should be used as the parameter to this command.
% \acmSubmissionID{V5cscw182}

%%
%% The majority of ACM publications use numbered citations and
%% references.  The command \citestyle{authoryear} switches to the
%% "author year" style.
%%
%% If you are preparing content for an event
%% sponsored by ACM SIGGRAPH, you must use the "author year" style of
%% citations and references.
%% Uncommenting
%% the next command will enable that style.
%%\citestyle{acmauthoryear}

%%
%% end of the preamble, start of the body of the document source.
\begin{document}

%%
%% The "title" command has an optional parameter,
%% allowing the author to define a "short title" to be used in page headers.
% \title[QV vs Likert]{``\textellipsis I can show what I really like.'': 
% Comparing Quadratic Voting with Likert Surveys at aligning respondents' preferences}

\title{Quadratic Survey Interface}

%%
%% The "author" command and its associated commands are used to define
%% the authors and their affiliations.
%% Of note is the shared affiliation of the first two authors, and the
%% "authornote" and "authornotemark" commands
%% used to denote shared contribution to the research.


%% Author list
\author{Ti-Chung Cheng}
\email{tcheng10@illinois.edu}
\orcid{0000-0001-7647-338X}
\affiliation{%
  \institution{University of Illinois Urbana-Champaign}
  \city{Urbana}
  \state{Illinois}
  \country{USA}
}
\author{Yutong Zhang}
\email{yutongz7@illinois.edu}
\authornotemark[1]
\affiliation{%
  \institution{University of Illinois at Urbana-Champaign}
  \city{Urbana}
  \state{Illinois}
  \country{USA}
}
\author{Yi-Hung Chou}
\email{hank0982@uci.edu}
\authornote{Both authors contributed equally to this research.}
\affiliation{%
  \institution{University of California, Irvine}
  \city{Irvine}
  \country{USA}
}
\author{Vinay Koshy}
\orcid{0000-0002-1410-3911}
\affiliation{\institution{Computer Science \\ University of Illinois at Urbana Champaign}
\city{Urbana}
\state{Illinois}
\country{USA}}
\email{vkoshy2@illinois.edu}
\author{Tiffany Wenting Li}
\email{wenting7@illinois.edu}
\affiliation{%
  \institution{University of Illinois Urbana-Champaign}
  \city{Urbana}
  \state{Illinois}
  \country{USA}
}
\author{Karrie Karahalios}
\email{kkarahal@illinois.edu}
\affiliation{%
  \institution{University of Illinois Urbana-Champaign}
  \city{Urbana}
  \state{Illinois}
  \country{USA}
}
\author{Hari Sundaram}
\email{hs1@illinois.edu}
\affiliation{%
  \institution{University of Illinois at Urbana-Champaign}
  \city{Urbana}
  \state{Illinois}
  \country{USA}
}

% %%
% %% By default, the full list of authors will be used in the page
% %% headers. Often, this list is too long, and will overlap
% %% other information printed in the page headers. This command allows
% %% the author to define a more concise list
% %% of authors' names for this purpose.
\renewcommand{\shortauthors}{Ti-Chung Cheng et al.}

%%
%% The abstract is a short summary of the work to be presented in the
%% article.
\begin{abstract}
This study introduces a new two-phase interface for Quadratic Surveys (QS) designed to scaffold users' decision-making process in order to better manage the cognitive load associated with QS's complexity. QS's complexity has greatly hindered its widespread adoption in surveying the public's opinions on societal issues for collective decision-making, despite its ability to elicit more accurate individual preferences compared to traditional surveys like the Likert-scale surveys as prior CSCW research has shown. Prior work shows that survey interface design significantly influences results and accuracy. To realize QS's full potential, we iteratively designed a two-phase ``organize-then-vote'' two-phase interface for QS based on decision-making and preference construction theories. Through a 2x2 between-subject in-lab study in a public resource allotment decision context, we compared this novel interface with a traditional text interface across two QSs with different lengths, one short (6 options) and one long (24 options). We found that our two-phase interface reduced participants' satisficing behaviors caused by cognitive overload in long QS conditions. In addition, we noted a shift in participants' cognitive effort from operating QS to constructing more comprehensive preferences when participants used our two-phase interface for long QS. This research contributes to CSCW by demonstrating how human-centered digital interface design can enhance the effectiveness of preference elicitation tools in collective decision-making on societal issues.
\end{abstract}

%A two-phase interface scaffolds individual decision making process in long QS, shifting cognitive effort from operating the survey tool to constructing more comprehensive preferences.



%%
%% The code below is generated by the tool at http://dl.acm.org/ccs.cfm.
%% Please copy and paste the code instead of the example below.
%%

\begin{CCSXML}
    <ccs2012>
       <concept>
           <concept_id>10003120.10003130.10011762</concept_id>
           <concept_desc>Human-centered computing~Empirical studies in collaborative and social computing</concept_desc>
           <concept_significance>500</concept_significance>
           </concept>
       <concept>
           <concept_id>10003120.10003130.10003134</concept_id>
           <concept_desc>Human-centered computing~Collaborative and social computing design and evaluation methods</concept_desc>
           <concept_significance>500</concept_significance>
           </concept>
       <concept>
           <concept_id>10003120.10003121.10003122</concept_id>
           <concept_desc>Human-centered computing~HCI design and evaluation methods</concept_desc>
           <concept_significance>300</concept_significance>
           </concept>
     </ccs2012>
\end{CCSXML}
    
\ccsdesc[500]{Human-centered computing~Empirical studies in collaborative and social computing}
\ccsdesc[500]{Human-centered computing~Collaborative and social computing design and evaluation methods}
\ccsdesc[300]{Human-centered computing~HCI design and evaluation methods}

%% Keywords.
\keywords{Quadratic Voting; Likert scale; Empirical studies; Collective decision-making}

%% Main Text
\maketitle
\section{Introduction}
% par 1: Introduction to the Problem
% Purpose: Define the problem and explain its significance.

%  What is the problem, what is the challenge, and why is it important
% Interfaces are important because they affect data collection. There is no QS interface, and QS is hard, so the problem is how do we design interfaces for QS?

% Version 1
% Designing user interfaces for emerging survey techniques is rare and challenging. Although new techniques offer the potential for more accurate data collection, existing interfaces are often ill-equipped to handle their complexity. We introduce Quadratic Surveys (QS), a surveying technique that applies the principles of Quadratic Voting (QV) to better elicit individual preferences compared to traditional Likert scale surveys~\cite{chengCanShowWhat2021}. Yet, without well-designed interfaces, even the most promising techniques can struggle with user adoption, ultimately threatening survey response quality~\cite{pielotDidYouMisclick2024, kimComparingDataChatbot2019}. The unique challenge of QS lies in its mechanism: respondents allocate a fixed budget of votes, with the cost of casting additional votes on a single option increasing progressively. This cost structure encourages careful trade-offs and promises to improve the accuracy of preference elicitation~\cite{posner2018radical}. At the same time, this mechanism makes responding to QS cognitively challenging. Therefore, this paper seeks to address the key question:~\textit{How can we design interfaces to support participants in completing Quadratic Surveys (QS) more effectively?}

Designing intuitive survey interfaces is crucial for accurately capturing respondents' preferences, which directly impact the quality and reliability of the data collected. Recent Human-Computer Interaction (HCI) studies highlight how certain survey response formats can increase errors~\cite{pielotDidYouMisclick2024, kimComparingDataChatbot2019} and influence survey effectiveness~\cite{ugur2015evaluating}. In this paper, our goal is to introduce an effective interface for a~\textbf{Quadratic Survey (QS)}, a survey tool designed to elicit preferences more accurately than traditional methods~\cite{chengCanShowWhat2021}. Despite the promise of QSs, there has been no research on designing interfaces to support their unique quadratic mechanisms~\cite{grovesOptimalAllocationPublic1977}, where participants must rank and rate items --- a task that poses significant cognitive challenges. To popularize QSs and ensure high-quality data, this paper addresses the question: \textit{How can we design interfaces to support participants in completing Quadratic Surveys (QSs) more effectively?}

\begin{figure}[ht]
    \centering
    \includegraphics[width=1\textwidth]{content/image/detailed.pdf}
    \caption{The Two-Phase Interface: The interface consists of two phases. Survey respondents can navigate between phases using the top right button. In the organization phase, the interface presents one option at a time to the respondents, and they chose one of four positional choices: ``Lean Positive'', ``Lean Neutral'', ``Lean Negative'', or ``Skip''. Skipped options are hidden and can be evaluated later. The chosen options then appear below. Items can be dragged and dropped across categories or returned to the stack. In the voting phase, options are listed in the order of the four categories. When hovering over each option, respondents can select a vote for that option using a dropdown menu. Each dropdown menu contains the cost associated with the vote. A sort button allows ascending sorting within each category. A summary box tracks the remaining credit balance.}
    \label{fig:interactiveInterface}
    \Description{
    This image shows two screen captures: the Organization Phase at the top and the Voting Phase at the bottom. The Organization Phase screen contains a question titled "What societal issues need more support?" with two sections. One section shows a block with descriptions of an option, and to the right of the block are four choices: "Lean Positive," "Lean Neutral," "Lean Negative," or "Skip." The interface also shows a skipped option. Below the block, three columns contain options inside, each showing the option title and a reassign button. In the Voting Phase, the same title and instructions are displayed, but now options are listed by their previously assigned categories (columns). The image shows the mouse hovering over one of the options, revealing a dropdown menu to allocate votes, along with the associated cost in credits. Each category box has a sorting button on the right, allowing users to reorder options within the category. Dots on the left side of the options indicate that drag-and-drop functionality is available for rearranging options. In the lower right corner, a summary box titled "Credit Summary" displays the remaining credit balance for voting. A button in the top left corner allows users to return to the previous Organization Phase.
    }

\end{figure}

% Zoom into two challenges this paper tackles -- zooming in mental demand and cognitive challenges due to the QS mechanism
We envision an effective interface that navigates participants through the complex mechanism and preference construction process\change{, tailored to a QS.} A QS improves accuracy in individual preference elicitation compared to traditional methods like Likert scales by requiring participants to make trade-offs using a fixed budget of credits, where purchasing $k$ votes for an option in QS costs $k^2$ credits~\cite{quarfoot2017quadratic,chengCanShowWhat2021}. This quadratic cost structure forces respondents to carefully evaluate their preferences, balancing the strength of their support or opposition against the limited budget. \change{However, the process of making these thoughtful trade-offs introduces challenges. As individual preferences are often constructed when presented with the options~\cite{lichtensteinConstructionPreference2006}, the act of weighing costs, evaluating options, and constructing rankings increases cognitive load.} Moreover, QSs, often referred to as Quadratic Voting (QV) in voting scenarios, can involve hundreds of options~\cite{rogersColoradoTriedNew2019, teamTaiwanDigitalMinister}, increasing the risk of cognitive overload and~\change{the taking of mental shortcuts~\cite{simonBehavioralModelRational1955, payneAdaptiveStrategySelection1988, tverskyJudgmentsRepresentativeness}}.
% add and possible breakdown interfaces for mental and interfaces to scaffold complex mechanisms

% ================================ %
% par 2: Approaches to Address the Challenges
% Purpose: Describe the existing approaches related to the problem.
% Key Questions:
%  - What are some broad approaches to addressing these challenges? -- there are none.
%  - Do not go into detail about related work but give an idea of the major themes in related work.
%       - No prior research on QS, but there are existing interfaces -> auto calculation as commonality
%       - prior work on interface for reducing cognitive load, preference construction, and voting

To date, existing quadratic mechanism-powered applications simply present options, allow vote adjustments and automatically calculate votes, costs, and budget usage. Such designs focused heavily on the mechanics operating the tool, rather than supporting possible challenges these application users faced. Survey interface literature, while addressing decision-making and usability, focuses on traditional surveys that do not share the unique option-to-option trade-offs that a QS introduces~\cite{engstrom2020politics, weijtersEffectRatingScale2010, kierujVariationsResponseStyle2010, toepoelSmileysStarsHearts2019, farzandAestheticsEvaluatingResponse2024, pielotDidYouMisclick2024}.~\change{Prior research in HCI and beyond explored techniques to manage cognitive load~\cite{paula2023, oviatt2006human, toepoelSmileysStarsHearts2019, softwareBrad2021, reis2012towards} and scaffold challenging tasks~\cite{task2014, moderate2021, ibili2019effect, amyChatSensing2018} showing promise in supporting preference construction with a QS. Thus, this study aims to bridge this gap.}

% ================================ %
% par 3: Your Proposal
% Purpose: Present your main ideas and proposed solution.
% Key Question:
%  - What are you proposing? Provide a sketch of the major ideas.

\change{We propose a novel interactive two-phase ``organize-then-vote'' QS interface (referred to as the two-phase interface for short, Figure~\ref{fig:interactiveInterface}) after multiple iterations. It aims to facilitate preference construction and reduce cognitive load when making trade-offs through three key elements.} First, the interface scaffolds the preference construction process by having participants initially categorize the survey options into ``Lean Positive,'' ``Lean Neutral,'' or ``Lean Negative.'' This serves as a cognitive warm-up, easing participants into the more complex QS voting task. Second, the interface arranges the options according to these categorizations, providing a structured visual layout. Third, participants can refine the positions of these options using drag-and-drop functionality, giving them greater control and agency in the preference-construction process. %These design features are aligned with preference construction theory and build upon prior research in interface design to reduce cognitive load and enhance user engagement.

To explore how these interface elements mitigate the cognitive load and support preference construction in Quadratic Surveys, we pose the following research questions:
\begin{itemize}
    \item RQ1. How does the number of options in Quadratic Surveys impact respondents' cognitive load?
    \item RQ2a. How does the two-phase interface impact respondents' cognitive load compared to a single-phase text interface?
    \item RQ2b. What are the similarities and differences in sources of cognitive load across the two interfaces?
    \item RQ3. What are the differences in Quadratic Survey respondents' behaviors when coping with long lists of options across the two-phase interface and the single-phase text interface?
\end{itemize}

% ================================ %
% par 4: Main Findings
% Purpose: Summarize the key findings from your work.
% Key Question:
%  - What are the main findings?
We invited 41 participants to a lab study comparing our two-phase interface with a baseline to understand how different interface designs and option lengths ($6$ options or $24$ options) impact cognitive load. 

\change{Self-reported cognitive load using the NASA Task Load Index (NASA-TLX) and semi-structured interviews identified common challenges in Quadratic Surveys (QS), such as preference construction and budget management, while highlighting differences between text and two-phase interfaces. The two-phase interface fostered more strategic engagement with survey options, encouraging consideration of broader impacts in the long QS, reducing time pressure in the short QS, and eliciting greater affirmative satisfaction (e.g., "feeling good"). Quantitative results support these observations: participants in the two-phase interface—particularly in long surveys—traversed the list less frequently but maintained the same number of edits while spending more time per option. This suggests that reduced traversal did not diminish engagement. Together, these findings highlight the organizing phase's role in fostering deeper engagement with survey options.}

% Qualitative findings, measured using the NASA Task Load Index~(NASA-TLX) and semi-structured interviews, revealed that participants using the two-phase interface experienced cognitive demand more from strategic, holistic thinking compared to personal relevance and operational tasks, particularly in longer surveys. Quantitative results showed that, although participants spent more time per option, they made faster decisions during the voting phase, suggesting a more efficient distribution of cognitive effort. We concluded that the two-phase interface mitigated cognitive overload in long QS surveys and shifted mental load toward more strategic thinking, reducing reliance on mental shortcuts like satisficing~\cite{simonBehavioralModelRational1955}.

% ================================ %
% par 5: Main Contributions
% Purpose: Identify and explain the primary contributions of your work.
% Key Structure:
%  1. Line 1: Identify your contribution—conceptualization, framework, interface, algorithm, etc.
%  2. Line 2: Contrast your contribution with prior work.
%  3. Line 3: Explain how you accomplished your contribution.
%  4. Line 4: Emphasize the impact of the contribution—why should anyone care?

\paragraph{Contributions}
We contribute to the HCI community by proposing the first interface specifically designed for QS and QV-like applications, aimed at reducing cognitive challenges and scaffolding preference construction through a two-phase interface with direct manipulation. Before our work, no research had explored QS interfaces, particularly for long QS prone to cognitive overload. Few studies in HCI address interfaces for surveys and questionnaires.~\change{Our study demonstrated how user interfaces can facilitate preference construction in situ and promote deeper engagement with survey options through interface elements. Additionally, this paper offers the first in-depth qualitative analysis of user experiences among Quadratic Mechanism applications, identifying usability challenges and key factors contributing to cognitive load.} The impact of our contribution extends beyond QS, offering design implications for other preference-elicitation tools in~\change{multi-option scenarios}. By making QS easier to use and more accurate, our design also encourages wider adoption among researchers and practitioners.~\change{Finally, our work lays the groundwork for future quadratic mechanisms interface design to better facilitate individuals in communicating their preferences.}

% ================================ %
% Removed text
% Surveys are a ubiquitous tool for collective decision-making, providing decision-makers with aggregated opinions that directly shape the outcomes for those surveyed. For example, states utilize referendums to form policy decisions, organizations like the Pew Research Center survey public perspectives on societal challenges in the United States, and city councils hold forums to gather community concerns.
% and private sectors~\cite{Gov4gitDecentralizedPlatform2023}.
% xiaoTellMeYourself2020, 

% ================================ %
\section{Related Work}
\label{sec:relatedWorks}
This research lies at the intersection of three core areas: quadratic surveys, survey and voting interface design, and choice overload along with cognitive challenges. In this section, we review the related works in each of these areas.

\subsection{Quadratic Survey and the Quadratic Mechanism}
We introduce the term \textbf{Quadratic Survey (QS)} to describe surveys that utilize the quadratic mechanism to collect individual attitudes. The~\textbf{quadratic mechanism} is a theoretical framework designed to encourage the truthful revelation of individual preferences through a quadratic cost function~\cite{grovesOptimalAllocationPublic1977}. This framework gained popularity through~\textbf{Quadratic Voting (QV)}, also known as plural voting, which uses a quadratic cost function in a voting framework to facilitate collective decision-making~\cite{lalley2016quadratic}.%~\textcite{quarfoot2017quadratic} demonstrated that QV effectively gauges public opinion and mitigates the tyranny of the majority in traditional voting systems. Furthermore, QV is not subject to Arrow's impossibility theorem, which states that no voting system can perfectly aggregate individual preferences without trade-offs~\cite{morreau2014arrow}, because it does not require aggregating rankings.  

To illustrate how QS works, we formally define the mechanism: each survey respondent is allocated a fixed budget, denoted by $B$, to distribute among various options. Participants can cast $n$ votes for or against option~$k$. The cost~$c_k$ for each option $k$ is derived as:

\[c_k = n_k^2 \quad \text{where}\quad n_k \in \mathbb{Z}\]

The total cost of all votes must not exceed the participant's budget:

\[\sum_k c_k \leq B\]

Survey results are determined by summing the total votes for each option:

\[ \text{Total Votes for Option } k = \sum_{i=1}^{S} n_{i,k} \]

where $S$ represents the total number of participants, and~$n_{i,k}$ is the number of votes cast by participant~$i$ for option~$k$. Each additional vote for each option increases the marginal cost linearly, encouraging participants to vote proportionally to their level of concern for an issue~\cite{posner2018radical}.

QS adapts these strengths of the quadratic mechanism in \textit{voting} to encourage truthful expression of preferences in \textit{surveys}. Unlike traditional surveys that elicit either rankings~\textit{or} ratings, QS allows for~\textit{both}, enabling participants to cast multiple votes for or against options, incurring a quadratic cost.~\textcite{chengCanShowWhat2021} showed that this mechanism aligns individual preferences with behaviors more accurately than Likert Scale surveys, particularly in resource-constrained scenarios like prioritizing user feedback on user experiences.

In recent years, empirical studies on QV have expanded into various domains~\cite{naylor2017first, cavailleWhoCaresMeasuring}. Applications based on the quadratic mechanism have also grown, including Quadratic Funding, which redistributes funds based on outcomes from consensus made using the quadratic mechanism~\cite{buterinFlexibleDesignFunding2019a, freitasQuadraticFundingIncomplete2024}. Recent work by \textcite{southPluralManagement2024} applies the quadratic mechanism to networked authority management, later used in Gov4git~\cite{Gov4gitDecentralizedPlatform2023}. Despite the increasing breadth and depth of applications utilizing the quadratic mechanism, little attention has been paid to user experience and interface design, which support individuals in expressing their preference intensity. Our work aims to address this by designing interfaces supporting quadratic mechanisms.

\subsection{Design Implications for Surveys, Questionnaires, and Voting Systems}
The relative lack of research in quadratic mechanism and QS interface design is concerning, as prior work on survey and questionnaire interfaces has demonstrated substantial impacts on responses and user experiences, even with seemingly minor design decisions.

Research in the marketing and research communities focusing on survey and questionnaire design, usability, and interactions examines the influence of presentation styles and `response format.'~\textcite{weijtersExtremityHorizontalVertical2021} demonstrated that horizontal distances between options are more influential than vertical distances, with the latter recommended for reduced bias. Slider bars, which operate on a drag-and-drop principle, show lower mean scores and higher nonresponse rates compared to buttons, indicating they are more prone to bias and difficult to use. In contrast, visual analog scales that operate on a point-and-click principle perform better~\cite{toepoelSlidersVisualAnalogue2018}. These prior works highlighted outcomes that are influenced by the different designs.

Voting interfaces, like surveys and questionnaires, elicit individual choices, but often with consequential impacts. A well-known example is the butterfly ballot, whose atypical ballot design may have influenced the outcome of the 2000 U.S. Presidential Election.~\cite{wandButterflyDidIt2001} Researchers like~\textcite{engstrom2020politics},~\textcite{chisnellDemocracyDesignProblem2016}, and organizations like the Center for Civic Design, which publishes reports like ``Designing Usable Ballots''~\cite{DesigningUsableBallots2015}, emphasize how ballot design influences democratic processes.

Existing research surfaced how various voting interface designs shifted voter decisions, influenced human errors, or improved usability through technologies. For instance, states without straight-party voting exhibited higher rates of split-ticket voting~\cite{engstrom2020politics}, and Australian ballots, which list candidates by office without party labels, often give incumbents an advantage. Poor designs, like the butterfly ballot, have led to voter errors due to confusion over punch holes, and splitting contestants across columns increases the likelihood of overvoting~\cite{quesenberyOpinionGoodDesign2020}. \textcite{everettElectronicVotingMachines2008} further explored how digital voting interfaces improve usability over physical voting behaviors, such as lever voting. Other projects like the Caltech-MIT Voting Technology Project have sparked research to address accessibility challenges, resulting in innovations like EZ Ballot~\cite{leeUniversalDesignBallot2016}, Anywhere Ballot~\cite{summers2014making}, and Prime III~\cite{dawkinsPrimeIIIInnovative2009}. In addition, \textcite{gilbertAnomalyDetectionElectronic2013} investigated optimal touchpoints on voting interfaces, and \textcite{conradElectronicVotingEliminates2009} examined zoomable voting interfaces for improved user interactability.

The design of Voting systems and response formats significantly influence respondent behavior, decision accuracy, and cognitive load. Research like~\textcite{galesicDropoutsWebEffects2006} showed that the burden on survey respondents increases dropouts. An effective design would enhance usability and reduce cognitive challenges faced by survey respondents, especially in complex response mechanisms like QS.

\subsection{Cognitive Challenges and Choice Overload}
Despite insights from studies on quadratic mechanisms, voting, and surveying techniques, the challenge of respondents making difficult decisions using quadratic mechanisms remains unexplored in the literature.~\textcite{lichtensteinConstructionPreference2006} identified three key elements that make decisions difficult. These elements include making decisions in unfamiliar contexts, being forced to make tradeoffs due to conflicting choices, and quantifying the value of one's opinions. QS fits all three elements: participants may encounter unfamiliar options set by the decision maker, are constrained by budgets that require tradeoffs, and cast final votes as numerical values. Thus, we believe QS introduces high cognitive load.

Cognitive overload can adversely affect performance, leading individuals to rely on heuristics rather than deliberate, logical decision-making~\cite{daniel2017thinking}. When presented with excessive information, such as too many options, individuals 'satisfice', settling for a 'good enough' solution rather than an optimal one~\cite{simonBehavioralModelRational1955, payneAdaptiveStrategySelection1988, tverskyJudgmentsRepresentativeness}. Subsequently, too many options can overwhelm individuals, resulting in decision paralysis, demotivation, and dissatisfaction~\cite{iyengarWhenChoiceDemotivating2000}.

Additionally,~\textcite{alwinMeasurementValuesSurveys1985} highlighted that the use of ranking techniques in surveys can be time-consuming and potentially more costly to administer. These challenges are compounded when ranking numerous items, requiring substantial cognitive sophistication and concentration from survey respondents \cite{featherMeasurementValuesEffects1973}.

Notable applications of Quadratic Voting include the $2019$ Colorado House, which considered $107$ bills~\cite{coyNewWayVoting2019}, and the $2019$ Taiwan Presidential Hackathon, which featured $136$ proposals~\cite{QuadraticVotingFrontend2022}; both used a single QV question with hundreds of options. Psychological and behavioral research highlights the importance of understanding how individuals navigate and benefit from new interfaces under long-list QS conditions. These empirical applications of QV suggest QS's potential to elicit individual preferences, emphasizing the need to study cognitive load and interface design.

%As \textcite{chengCanShowWhat2021} noted, it is essential to better understand how the number of options influences the usability of QS and to design interfaces that effectively support survey respondents.
% there is limited research on interfaces for Constant Sum surveys~\cite{hauserIntensityMeasuresConsumer1980a}, a mechanism similar to QS that aims to elicit both ranking and rating preferences from individuals.
%The closest work discussing interfaces for QV is an arXiv paper~\cite{} that transformed the knapsack voting platform developed by \textcite{goelKnapsackVotingVoting}
% While both fields have deep insights into understanding design's influence on attitude elicitation, QS's unique capability of supporting both ranking and rating~\cite{chengCanShowWhat2021} makes designing an interface important and challenging. Subsequently, this research aims to understand how this interface influences an individual's QS response behavior. Requiring the distribution of budgets following the quadratic mechanism introduces new and complex decisions. 
% Empirical studies and applications of the quadratic mechanism and quadratic voting have increased in the past few years. Several studies have explored the empirical use cases for QV, including \textcite{quarfoot2017quadratic}'s study on 4,500 participants' attitudes across ten public policies, highlighting differences between QV and Likert scale survey results. \textcite{chengCanShowWhat2021} applied quadratic surveys in Human-Computer Interaction (HCI) and subsequently showed QV's effectiveness in reflecting true preferences in monetary decision tasks. \textcite{naylor2017first} used QV in educational research to gauge student opinions on factors affecting university success, and \textcite{cavailleWhoCaresMeasuring} examined QV in polarized choice scenarios.
\section{Quadratic Survey Interface Design}
\label{sec:interfaceDesign}
In this section, we present the QS interface. \change{Using components from existing QV interfaces described in Section~\ref{sec:relatedWorks} and insights from prior literature, we iterated through paper prototypes and three design pre-tests, detailed in Appendix~\ref{apdx:design}.} In our initial paper prototyping iterations, participants struggled to~\textit{rank} relative preferences among options and~\textit{rate} the degree of trade-offs between them. In this study, we focus on addressing the former challenge, which pertains to preference construction.

\subsection{`Organize-then-Vote': The Two-Phase Interface}
\label{sec:finalInterfaceDesign}

\subsubsection{Justifying a two-phase approach}
The main objective of the two-phase interface is to facilitate preference construction and reduce cognitive load. As shown in Figure~\ref{fig:interactiveInterface}, the interface consists of two steps: an organization phase and a voting phase. In both phases, survey respondents can drag and drop options across the presented list.

\paragraph{A two-phase approach}
Preferences are shaped through a series of decision-making processes~\cite{lichtensteinConstructionPreference2006}. Two major decision-making theories~\change{inspired} this two-step interaction interface design:~\textcite{montgomeryDecisionRulesSearch1983}'s Search for a Dominance Structure Theory (Dominance Theory) and~\textcite{svensonDifferentiationConsolidationTheory1992}'s Differentiation and Consolidation Theory (Diff-Con Theory). The former suggested that decision-makers prioritize creating dominant choices to minimize cognitive effort by focusing on evidently superior options~\cite{montgomeryDecisionRulesSearch1983}. The latter described a two-phase process where decisions are formed by initially~\textit{differentiating} among alternatives and then~\textit{consolidating} these distinctions to form a stable preference~\cite{svensonDifferentiationConsolidationTheory1992}.~\change{During our pre-tests, participants did not appreciate ranking all options prior to voting. Both theories helped explained that decisions are made through eliminating alternatives rather than generating a complete list of ranked choices.} Hence, the two-phase design --- organize-then-vote --- aimed to facilitate this cognitive journey explicitly. The first phase focused on differentiating and identifying dominant options, enabling survey respondents to preliminarily categorize and prioritize their choices. The second phase presented these categorized options in a comparable manner, with drag-and-drop functionality, enhancing one's ability to consolidate preferences. This structured approach aimed to construct a clear decision-making procedure that reduced cognitive load and enhanced clarity and confidence in the decisions made.

\paragraph{Phase 1: Organization Phase}
The goal of the organization phase was to support participants in identifying clearly superior options or partitioning choices into distinguishable groups. In this section, we first describe how the interaction works, then we detail the reasons for the implemented design decisions.

The organizing interface, depicted on the top half of Figure~\ref{fig:interactiveInterface}, sequentially presents each survey option. Participants select a response among three ordinal categories -- ``Lean Positive'', ``Lean Negative'', or ``Lean Neutral''. Once selected, the system moves that option to the respective category. Participants can skip the option if they do not want to indicate a preference. Options within the groups are draggable and rearrangeable to other groups should the participants wish.

To support preference formation, respondents are shown one option at a time, allowing them to either recall a prior judgment or construct a new one based on the presented choices~\cite{strackThinkingJudgingCommunicating1987}. Limiting the information presented this way also helps reduce cognitive load by preventing overload from too many options~\cite{swellerCognitiveLoadTheory2011}. This incremental process ensures that participants form opinions on individual options.

The three possible options --- Lean Positive, Lean Neutral, and Lean Negative --- aim to scaffold participants in constructing their own choice architecture~\cite{munscherReviewTaxonomyChoice2016, thalerNudgeImprovingDecisions2008a}, which strategically segments options into diverse and alternative choice presentations while avoiding biases from defaults. We believed that these three categories were sufficient for participants to segment the options. We do not limit the number of options one can place in each category to prioritize user agency, allowing participants full control over how they organize their preferences~\cite{norman2013design}. Immediate feedback displays the placement of options and allows participants to rearrange them via drag-and-drop, adhering to key interface design principles~\cite{norman2013design}. At the same time, it allows finer-grain control for individuals to surface dominating options and create differentiating groups of options.

\paragraph{Phase 2: Interactive Voting Phase}

The objective of the voting phase is to facilitate the consolidation of differentiated options through interactive elements while reinforcing the differentiation across options constructed by participants in the previous phase. This facilitation is achieved by retaining the drag-and-drop functionality for direct manipulation of position and enabling sorting within each category.

Options are displayed as they are categorized within each category from the previous step and in the following section --- Lean Positive, Lean Neutral, Lean Negative, and Skipped or Undecided --- as detailed on the bottom half of Figure~\ref{fig:interactiveInterface}. The Skipped or Undecided category contains options left in the organization queue, possibly because survey respondents have a pre-existing preference or chose not to organize their thoughts further. The original order within these categories is preserved to maintain and reinforce the differentiated options. This ordering sequence mitigated early prototype concerns where uncategorized options were left at the top of the voting interface confusing survey respondents. Respondents have the flexibility to return to the organization interface at any point during the survey to revise their choices.

In the voting interface, options are draggable, allowing participants to modify or reinforce their preference decisions as needed. Each category features a sort-by-vote function for reordering within the group, which, although it doesn’t affect the final outcome, supports information organization and consolidation. Both features aim to group similar options automatically and emphasize proximity, reducing cognitive load by following the proximity compatibility principle to enhance decision-making~\cite{wickens1990proximity}.

While multiple interaction mechanisms exist, drag-and-drop has been extensively explored in rank-based surveys. For instance,~\textcite{krosnick2018measurement} demonstrated that replacing drag-and-drop with traditional number-filling rank-based questions improved participants' satisfaction with little trade-off in their time. Similarly,~\textcite{timbrook2013comparison} found that integrating drag-and-drop into the ranking process, despite potentially reducing outcome stability, was justified by the increased satisfaction and ease of use reported by respondents. The trade-off was deemed worthwhile as QS did not use the final position of options as part of the outcome if it significantly enhanced user satisfaction and usability~\cite{rintoulVisualAnimatedResponse}. Together, these design decisions led to our belief that a two-phase interface with direct interface manipulation could reduce the cognitive load for survey respondents to form preference decisions when completing QS.

In addition, we made three aesthetic design decisions~\change{considering existing QV-based interfaces}. First, we removed visual elements like icons, emojis, progress bars, and vote visualizations, as prior research indicated that emojis could influence survey interpretations and reduce user satisfaction~\cite{herringGenderAgeInfluences2020, toepoelSmileysStarsHearts2019}. While effective visualizations can aid decision-making, this study does not aim to address that question. Second, the final interface has all options presented on the screen at the same time, intentionally. Unlike all the prototypes and existing interfaces, prior literature emphasized the importance of placing all the options on the same digital ballot screen to avoid losing votes~\cite{CenterCivicDesign}. This echoes the proverb ``out of sight, out of mind,'' where individuals might be biased toward options that are shown to them, and additional effort is required for individuals to retrieve specific information if options are hidden. Last, we decided to use a dropdown positioned to the right of each survey option for ease of access to the budget summary when determining the votes. The layout of the votes and cost was inspired by online shopping cart checkout interfaces where quantities are supplied next to the itemized costs followed by the total checkout amount. After testing two alternative~(Figure~\ref{fig:btn_design}) input methods—click-based buttons,~\change{which participants dislike making multiple clicks}, and a wheel-based design, which offered intuitive control but was unfamiliar to some participants—we opted for a more accessible dropdown menu for vote selection.

\begin{figure}[ht!]
    \centering
    \includegraphics[width=0.8\textwidth]{content/image/prototypes/btn_design.png}
    \caption{Alternative vote control. The click-based design (upper) mirrors traditional vote control used in other QV interfaces, where each click controls one vote. The wheel-based design (the latter two) allows control through both clicks and mouse wheel rotation.}
    \Description{Three voting control interfaces are displayed. Each row represents a different interface. The first row shows a traditional click-based voting interface with options to decrease, increase, or maintain a rating of +3. The second and third rows show a wheel-based voting interface with mouse wheel functionality. In these, the middle row indicates a current rating of +3, with +2 and +4 ratings also visible. The cost for each option is listed on the right, ranging from 4 to 16. The last row mirrors the previous one with a rating of +3 and a cost of 9.}

    \label{fig:btn_design}
\end{figure}

\begin{figure}[ht]
    \centering
    \includegraphics[width=\textwidth]{content/image/detailed_text.pdf}
    \caption{The text-based interface: This interface is based on the two-phase version but does not include the organization phase and lacks the drag-and-drop functionality. Options are randomly positioned.}
    \Description{An image of a voting interface asking users to select societal issues needing support. The title reads, "What societal issues need more support?" with a brief explanatory paragraph underneath. Below, a list of six options is displayed, including "Youth Education Programs and Services," "Advocacy and Education," "Zoos and Aquariums," "Community Foundations," "Environmental Protection and Conservation," and "International Peace, Security, and Affairs." Each option has a description, a current vote count, and a dollar amount. The right side of the image shows an expanded dropdown menu for one of the options with selectable voting choices, such as "1 upvote" and "2 upvotes." A separate box labeled "Credit Summary" shows the remaining credit of 9 and a "Submit" button below it.}
    \label{fig:textInterface}
\end{figure}

\subsection{Baseline Interface: Single-Phase Text Interface} ~\change{We implemented the single-phase text interface (referred to as text interface for short, Figure~\ref{fig:textInterface}) as our control condition to compare how the organizational components influenced participants' cognitive load and behavior. The text-based interface, like all existing interfaces, contains a list of static elements, a summary box, and a vote control. We followed the same design considerations, removing visual elements, presenting all options in the same screen, and using the dropdown for vote control, following the two-phase interface interface to provide a more direct comparison. We position the question prompt at the top followed by a randomly ordered option list to prevent ordering bias~\cite{ferberOrderBiasMail1952, couperWebSurveyDesign2001} below. Individual option costs and the remaining credits' summary box are presented to the right of the screen given our interface layout.}

Both experimental interfaces were developed with a ReactJS frontend and a NextJS backend powered by MongoDB. We open-source both interfaces.\footnote{link-to-github}


% In our first prototyped tool, we aim to help survey respondents rank options to establish relative preferences before voting. As shown in Figure~\ref{fig:qv_rank}, our prototype allows respondents to move options before finalizing their votes. During our pretest, we found that respondents rarely moved the options and some questioned the need for a full ranking since it did not affect the QS submission. Many did not realize the options were draggable until we pointed it out. The main insight from this prototype is that creating a full rank is~\textit{not} essential for establishing~\textit{relative} preferences, leading us to consider selecting a subset of options instead of requiring a full rank among all options.

% First, we surveyed the current implementation of QV interfaces to understand the development of such tools. We presented a selection in Figure~\ref{fig:qv_interface_external}. All five interfaces retained and presented the following components:
% \begin{itemize}
%     \item Option list: A list of options contesting for votes.
%     \item Vote Controls: Two buttons to increase and decrease votes associated with each option.
%     \item Individual vote tally: A representation of votes associated with an option.
%     \item Summary: A summary that automatically calculated the cost across options and the remaining budget.
% \end{itemize}
% Now we present the final interactive interface and describe how it operates. In this subsection, we provide supporting evidence from prior literature that we previously omitted. These pieces of literature were omitted for clarity and focus in the previous subsection but will be reintroduced here. 
% We constructed a text-based interface that included all five components but removed the use of emojis (i.e., thumbs up and thumbs down present in Figure~\ref{fig:wedesignInterface}), progress bars, and other visualizations in the summary section (i.e., progress bars in Figure~\ref{fig:wedesignInterface} and~\ref{fig:chengInterface} or blocks presented in Figure~\ref{fig:rxcvotingInterface}), and the visual cues for individual vote counts (i.e., the colored counts and icons present in Figure~\ref{fig:gov4gitInterface} and~\ref{fig:chengInterface}).

% During this process, we noticed several issues. First, many survey respondents placed most options into the 'option you care about' category, defeating the design's purpose. Second, there were no indicators distinguishing between the selected and remaining options. Respondents did not notice their selections were kept at the top in the voting stage and were unsure why Step 1 was necessary if all options were shown again. This informed two takeaways: selecting options to vote on is too coarse to construct relative preferences, and there needs to be a clearer distinction and connection between the two phases.

% Feedback indicated that survey respondents are comfortable with this two-phase organize-then-vote design. Several user experience issues emerged, but they were addressable without significantly modifying this interaction structure. These issues include: First, dragging and dropping all options into different categories is cumbersome and can mislead respondents into thinking this is a ranking process, which is not the goal. Second, the position of unorganized options at the top of the voting list is counterintuitive. Third, the voting controls are disconnected from the option summaries, dividing attention between the left and right sides of the screen.

% These design decisions led to the interface shown in Figure~\ref{fig:textInterface}. 


% \subsubsection{Paper prototype: visualizing trade-offs}
% The original paper prototype aimed to help visualize survey respondents' tradeoffs among options. 
% The original paper prototype aimed to utilize visual representations to highlight the constrained availability of credit and to explain the costs and trade-offs associated with selecting each available option.
% Early on, we did not know which components made QS more difficult than other survey techniques. We began by surveying existing interfaces (Figure~\ref{fig:qv_interface_external} other than Figure~\ref{fig:gov4gitInterface} which did not exist near the writing of this paper). All four interfaces consist of these common components:
% As we were unsure what made QS more complex than other survey techniques, our investigation began with the existing interface (Figure~\ref{fig:qv_interface_external}, except Figure~\ref{fig:gov4gitInterface} which did not exist at that time. All four interfaces consist of these common components:
% \begin{itemize}
%     \item Option list: A list of options contesting for votes.
%     \item Vote Controls: Buttons to increase and decrease votes associated with each option.
%     \item Individual vote tally: A representation of votes associated with an option.
%     % \item Summary: A summary that automatically calculates the cost across options and the remaining budget.
%     \item Summary: An auto-generated summary of costs and remaining budget.
% \end{itemize}

% To brainstorm ways to help survey respondents manage trade-offs across options, we decomposed these options and explored several innovative layouts. Initially, we thought trade-offs were the core cause of cognitive load. In this paper, we show two versions of the paper prototypes in Figure~\ref{fig:qv_paper}. In both figures, costs are represented by blocks, similar to Figure~\ref{fig:rxcvotingInterface}. We imagine the survey respondents to drag and position options in the space provided unstructurally (Figure~\ref{fig:horizontal_paper}) or structurally (Fig~\ref{fig:vertical_paper}). Similar to the seminal debate on direct manipulation vs. interface agents~\cite{shneidermanDirectManipulationVs1997}, the research team was unsure how much control survey respondents should have over the positioning of the options to aid the decision-making process that considers trade-offs. Different from prior interfaces, we used placements of the interface to denote positive or negative number of votes. After several pretests, we learned that the main process participants aim to do throughout the survey is establishing~\textit{relative} preferences across the options, rather than thinking so much about trade-offs. 
% To further explore the features that contribute most to the complexity of QS, we developed two prototypes shown in Figure~\ref{fig:qv_paper}. Similar to Figure~\ref{fig:rxcvotingInterface}, these prototypes use blocks to represent costs, arranged either unstructurally (Figure~\ref{fig:horizontal_paper}) or structurally (Fig~\ref{fig:vertical_paper}), facilitating the visualization of the trade-offs. Unlike previous interfaces, we utilized the placement of the interface to denote positive or negative vote counts. Several protests indicated that participants primarily focus on establishing relative preferences among options rather than trade-offs. Therefore, in this study, we focused on enhancing designs that facilitate the establishment of relative preferences.
% the two main features of QS we identified are the relative preference through a combined presentation of rankings and ratings, and the option selection trade-offs due to total credit limits. 
% The initial prototyping involves collecting the interface designs for existing quadratic mechanism-based software. Iterative pretests informed each subsequent design. We present these iterations, which aim to enhance user experience in the preference construction process in the following sections.

% In the previous subsection, we highlighted critical prototype iterations that informed the final two-phase interactive process that defines the user journey. 
% We now present the final two-phase interface, its operations, and the supporting literature for comprehensive understanding.
% Then, We also discuss the aesthetic design choices that emerged throughout the iterations.

\section{Experiment Design}
\label{sec:experiment}
Based on the design decisions, we developed a QS interface using a React.js frontend and a Next.js backend powered on MongoDB. Both services were open-sourced~\footnote{link-to-github}.

We recruited participants from a midwestern college town using online ads, digital bulletins, social media posts, physical flyers, and online newsletters. The study's researcher prioritized the non-student population to maximize participant diversity. When recruiting participants, we did not reveal that the goal of this study was to measure their cognitive load and study their behaviors, rather a study that elicited community members' attitudes on societal issues. The reason we withheld such information was to prevent response biases. This study was reviewed and approved by the college Institutional Review Board.

\begin{figure}[ht]
    \centering
    \includegraphics[width=1\textwidth]{content/image/study_flow.pdf}
    \caption{Study protocol}
    \label{fig:studyProtocol}
\end{figure}

Figure~\ref{fig:studyProtocol} shows a visual representation of the study protocol. Study participants were invited to the lab to participate in this study. The reason we made this experiment design decision was to minimize the influence of external factors that could affect the measurement of cognitive load. External factors, more prevalent in remote experiments or those conducted via platforms like MTurk, included potential multitasking or interruptions by others. An in-lab study also allowed participants to operate across a consistent device that researchers had full control over. More specifically, the experiment involved participants operating on a 32-inch vertical monitor. This setup assured study participants, despite any condition in the study, could see all options on a QS, minimizing hidden information from an individual's decision-making process.

After consenting to the study, participants were invited to the study and they watched a pre-recorded video explaining the Quadratic mechanism and how QS operates. This video did not include any hints of either interface and how to operate the interface. Participants were then asked to complete a short quiz. The purpose of the quiz was to ensure that all participants fully understood how QS works. Participants were not screened out if they failed the quiz but were asked to rewatch the video or ask the researcher until they were able to select the correct answer. The device that the participant worked on was screen captured throughout the study.

The researcher then primed the participant that the purpose of this study was to assist local community organizers in understanding community members' preferences on a wide variety of societal issues so they could potentially distribute limited resources better. Participants were randomly placed into one of the four groups:

\begin{itemize}
    \item 6 options with a text-based interface
    \item 6 options with an interactive interface
    \item 24 options with a text-based interface
    \item 24 options with an interactive interface
\end{itemize}

Participants began completing the survey independently, without the researcher's presence. Upon completion, they contacted the researcher, who then requested they complete the NASA-TLX to assess cognitive load. This was followed by a short semi-structured interview to gain insights into the participants' experiences. This interview was audio recorded. The session concluded with a debriefing and a \$15 cash compensation for their participation. The debriefing explained to the participant that not disclosing the purpose of the survey was to measure cognitive load and interface design and allowed for participants to ask any questions.

% Finally, participants complete the situational motivation scale (SIMS) to gauge motivation and a demographic survey.

The study was designed as a between-subject study for two reasons. First, we aimed to minimize the study fatigue that might occur given the complexity of responding to a QS. To complete a QS survey, participants could take up to 20 minutes. Thus, it was difficult to conduct back-to-back experiments that measure cognitive load. We chose not to ask participants to revisit the lab with several days in between, to reduce dropout rates and prevent demotivating participants from attending the in-person experiment, which might occur in a within-subject study design. Second, we aimed to reduce the learning effect that is difficult to remove, especially concerning operating the interface and making decisions on the survey. Recall that preferences are constructed, we wanted to ensure that participants were not influenced by their previous preferences which could influence their perceived cognitive load.

In an ideal world, understanding participants' cognitive load across multiple options would require enumerating all possible numbers of options and eliciting the ``breaking point'' where the participant experiences cognitive overload. Unfortunately, this was not feasible. Iterating through all possible numbers of options was very costly, both in time and resources. Therefore, we referred to prior literature to inform our choice of 6 and 24 options, representing a short and long list of options. To decide the number for the short list, survey methods such as constant sum surveys and Analytic Hierarchy Process (AHP) recommended options fewer than ten and seven, respectively~\cite{moroneyQuestionnaireDesignHow2019, saatyGroupDecisionMaking2013, saatyPrinciplesAnalyticHierarchy1987}. However, we were not aware of any specific works that justified these numbers. \textcite{saatyPrinciplesAnalyticHierarchy1987} associated this value with both the cognitive processing capacity of $7\pm2$~\cite{millerMagicalNumberSeven1956} and a theoretical proof using the consistency ratio of a pairwise comparison metric~\cite{saaty2003magic}. This informed our decision to contain a pair of dependent variables above and below seven options. We turned to experiments designed to study choice overload. A meta-analysis by~\textcite{chernevChoiceOverloadConceptual2015} surveyed 99 choice overload experiments (N = 7202) and summarized that 6 and 24 are the modal values for short and long lists when testing choice overload. These two values were likely rooted in the original choice overload experiment by~\textcite{iyengarWhenChoiceDemotivating2000}. The value six is often used in experiments to understand the effect of choice provision. The value 24 is the maximum number of ecologically valid jams produced by the jam company in the original study. We decided to follow suit with these two values, satisfying the previous decision to choose two values less than and greater than seven.

Next, we describe the context of the survey that participants completed. Participants were asked to complete a societal issue survey. We followed suit as described by~\textcite{chengCanShowWhat2021}, believing that surveying societal issues is a good topic as it is relevant to every citizen and it is easy to convey that there are limited resources in the public sector to be prioritized across different sectors and areas. Participants across all four groups were presented with options randomly drawn from 26 societal issues. These issues were generated from the categories used by Charity Navigator~\cite{CharityNavigatorAnimals2023}, a non-profit organization that evaluates over 20 thousand charities in the United States. The full list of these societal issues is provided in Appendix B.

% Last, we describe the two quantitative measurements taken during the study: cognitive load and motivation. 

Last, we describe the quantitative measurements taken during the study: cognitive load. At the time of this study, several methods existed to measure cognitive load, including performance measures, psychophysiological measures, subjective measures, and analytical measures~\cite{gaoMentalWorkloadMeasurement2013}. Given the nature of QS, a task requiring a long period, adopting performance measures like secondary-task measures in our experiment proved challenging due to the difficulty of designing a secondary task. The secondary task had to use the same cognitive resources as the primary tasks, and the cognitive resource for completing the survey would vary among participants. Similarly, psychophysiological measures such as pupil size~\cite{palinkoEstimatingCognitiveLoad2010} and ECG~\cite{haapalainenPsychophysiologicalMeasuresAssessing2010} could be highly sensitive to external environments and costly to obtain. Consequently, we relied primarily on subjective measures via self-report surveys and analytical measures like time and clicks collected via the interface. We adopted a paper-based weighted NASA Task Load Index (NASA TLX), a multidimensional scoring procedure using the weighted average of six subscale scores to represent overall workload. Weighted NASA-TLX used a priori workload definitions from subjects to weight and average subscale ratings, requiring subjects to evaluate each weight's contribution to the workload of a specific task~\cite{hart1988development, hartNasaTaskLoadIndex2006, cain2007review}. This approach reduced between-rater variability, indicating differences in workload definitions among raters within a task and variations in workload sources between tasks~\cite{cain2007review}. Despite criticisms regarding its validity and vulnerability, NASA-TLX was commonly used due to its low cost and ease of administration~\cite{gaoMentalWorkloadMeasurement2013}. It had been tested on various experimental and lab tasks, and workload scores derived from these tests showed significantly less variability among evaluators than one-dimensional workload scores~\cite{rubioEvaluationSubjectiveMental2004}. Thus, we chose NASA-TLX to measure cognitive load in our study.

% Tabling SIMS for now. It was not used in the analysis. In addition to NASA-TLX, we administered a situational motivation scale (SIMS) to measure participants' motivation (required citation). We posited that motivation would influence mental demand (required citation). SIMS, chosen for its widespread use, helps understand one's intrinsic motivation, extrinsic motivation, identified regulation, and external regulation, and was originally designed to measure self-determination. Both instruments were administered using pen-and-paper.

\begin{figure}[h]
    \centering
    % Top figure
    \begin{subfigure}[b]{\textwidth}
        \centering
        \includegraphics[width=\textwidth]{content/image/demo/demo_age_group_vertical.pdf}
        \caption{Age distribution}
        \label{fig:demoAge}
    \end{subfigure}
    
    \vspace{0.5cm} % Add some vertical space between the rows

    % Bottom figures
    \begin{subfigure}[b]{0.45\textwidth}
        \centering
        \includegraphics[width=\textwidth]{content/image/demo/demo_gender.pdf}
        \caption{Gender distribution}
        \label{fig:demoGender}
    \end{subfigure}
    \hfill
    \begin{subfigure}[b]{0.45\textwidth}
        \centering
        \includegraphics[width=\textwidth]{content/image/demo/demo_ethnicity.pdf}
        \caption{Ethnicity distribution}
        \label{fig:demoEthnicity}
    \end{subfigure}
    
    \caption{Demographic distributions: Age, Gender, and Ethnicity}
    \label{fig:Demographics}
\end{figure}

% maybe more the figure to the appendix?
\section{Cognitive Load and Sources across Experiment Conditions}
\label{sec:cog_result}
In this section, we present the cognitive load across experiment groups and the sources contributing to each cognitive load dimension. Given the limited number of participants, we focus on descriptive statistics and qualitative assessments of cognitive load. Quantitative data includes metrics from the survey tasks, while qualitative insights come from post-survey interviews transcribed and analyzed by the first author.

The first author conducted an inductive thematic analysis process~\cite{olsonWaysKnowingHCI2014}. They coded snippets from each transcript based on specific research questions and topics of interest for the qualitative analysis. Similar codes were merged within each research question or topic to form relevant themes. When differences were hypothesized, they applied a deductive coding process to text snippets related to a specific research question or topic of interest.

The results for this section are organized as follows: We start with participant demographics and then provide an overview of our cognitive load findings. We then examine the six dimensions used in the NASA-TLX survey: mental demand, physical demand, temporal demand, performance, effort, and frustration.

\subsection{Demographics}
We recruited a total of $41$ participants, allocating ten to each experiment condition. Due to data quality concerns, we excluded one participant's data\footnote{The participant expressed the experiment as a fake setup that they do not need to complete it seriously.}. The mean age of the participants was $34.63$ years old, with a detailed age distribution presented alongside the county population distribution in Figure~\ref{fig:demoAge}. This comparison reveals that our sample closely matches the county's demographic profile, albeit with a slightly higher representation of younger adults, particularly in the 35-45 age range. As shown in Figure~\ref{fig:demoGender}, the majority of participants skewed toward females.

Regarding ethnicity, $51.2\%$ of the participants identified as White, $26.8\%$ as Asian, $7.3\%$ as African American, and $4.9\%$ as Hispanic. Additionally, $9.8\%$ of participants reported mixed ethnicity.

\subsection{Overall Cognitive Load}
\label{sec:cog}
\begin{figure}[ht]
    \centering
    \begin{subfigure}[b]{0.45\textwidth}
        \centering
        \includegraphics[width=\textwidth]{content/image/results/nasatlx_final_value.pdf}
        \caption{NASA-TLX Weight Score Distribution}
        \label{fig:nasatlx-final1}
    \end{subfigure}
    \hfill
    \begin{subfigure}[b]{0.47\textwidth}
        \centering
        \includegraphics[width=\textwidth]{content/image/results/nasatlx_cog_value_interpreted.pdf}
        \caption{NASA-TLX Cognitive Interpretation}
        \label{fig:nasatlx-final2}
    \end{subfigure}
    \caption{This figure shows the box plot results for weighted NASA-TLX scores across experiment groups and participant counts based on individual score interpretations. In~\ref{fig:nasatlx-final1}, we observe a downward trend in cognitive load for the short QS, while the long QS shows an upward trend. Interestingly, there is a counterintuitive downward trend between short and long text interfaces. In~\ref{fig:nasatlx-final2}, these trends are clearer when NASA-TLX scores are grouped into five tiers.}
    \label{fig:nasatlx-final}
\end{figure}

To answer~\textbf{RQ1} and~\textbf{RQ2a}, we derive the weighted NASA-TLX scores across the four experiment conditions. We show these results in Figure~\ref{fig:nasatlx-final}. Weighted NASA-TLX uses a continuous 0-100 score, with higher values indicating greater cognitive load. We use predefined mappings of NASA-TLX values to cognitive levels: low, medium, somewhat high, high, and very high, as listed by~\textcite{hart1988development}. We show value interpretations in Figure~\ref{fig:nasatlx-final2}. We found that:

\begin{itemize}
    \item Short text interface: The median cognitive load was $39.00$, with a mean of $43.23$ and a standard deviation of $17.65$. $8$ participants reported somewhat high or above, with $4$ reporting high cognitive load.
    \item Short interactive interface: The median cognitive load was $29.85$, with a mean of $35.36$ and a standard deviation of $18.17$. $5$ participants reported somewhat high or above, with $3$ reporting high cognitive load.
    \item Long text interface: The median cognitive load was $33.85$, with a mean of $34.60$ and a standard deviation of $17.69$. $5$ participants reported somewhat high or above, with $2$ reporting high cognitive load.
    \item Long interactive interface: The median cognitive load was $42.70$, with a mean of $42.02$ and a standard deviation of $18.48$. $6$ participants reported somewhat high or above, with $3$ reporting high cognitive load.
\end{itemize}

These results partially answer our first two research questions. First, across the short survey, the interactive interface decreased cognitive load compared to the text interface. This is evident from the median cognitive load decrease from $39.00$ to $29.85$, with more participants reporting lower cognitive load using the interactive interface. The short text interface had the most participants ($N=8$) rating their cognitive load as somewhat high or above. The other three conditions had similar distributions, with about half experiencing medium and half somewhat high or high loads.

Second, contrary to our expectations, the long text interface had a lower cognitive load than the long interactive interface. The cognitive load for the long text interface was even lower than that for the short text interface, with a median cognitive load of $33.85$ compared to $39.00$ in the short text interface. This is counterintuitive, as prior literature suggests that more options can heighten cognitive load~\cite{swellerCognitiveLoadTheory2011}.

By deduction, if the interactive interface increased cognitive load in the long survey, we might expect a similar increase in the short interactive interface compared to the short text interface. However, we observed a lower cognitive load in the short interactive interface. This discrepancy suggests two plausible explanations:

\paragraph{Cognitive Overload Prevention by Interactive Interface} The long survey caused cognitive overload, but the interactive components may have prevented participants from taking mental shortcuts, which would typically reduce measured cognitive load~\cite{daniel2017thinking, simonBehavioralModelRational1955, payneAdaptiveStrategySelection1988, tverskyJudgmentsRepresentativeness}. This prevention could result in a higher cognitive load in the long interactive interface compared to the long text interface. In other words, the interactive interface may have shifted participants' cognitive load from some dimensions to others, maintaining their overall cognitive load at a higher level but not overloaded. If this is true, we expect to see differences among the qualitative explanations of sources, specifically differences in the perceived causes of cognitive load. We will explore this in the next subsections (Subsections~\ref{sec:mental}-\ref{sec:fustration}).

\paragraph{A Pure Increase of Cognitive Load Due to Interactivity} It is also possible that the long survey introduced cognitive overload, and the interactive interface did not influence participants' preference construction but only increased cognitive load due to the added interactivity. In other words, participants are asked to perform additional operations with interactive elements that contribute to a higher cognitive load without providing sufficient cognitive benefits. If this is true, we should expect behavior data to show similar voting patterns across conditions, as the added interactions primarily focused on the pre-organization of the options rather than influencing the decision-making process itself. We will explore this in the section~\ref{sec:behave_result}.

We also acknowledge the possibility that the elicited values are pure noise and do not reflect the actual cognitive load. This could be due to the small sample size, the nature of the task, or the participants' understanding of the cognitive load scale. While this might be true for small sample sizes, we believe that the qualitative insights from the interviews provide a more nuanced understanding of the cognitive load sources. We detail these limitations in Section~\ref{sec:limitations}.

% ============================================= %
\begin{table}[h]
    \caption{This table lists all the causes participants mentioned as contributing to their Mental Demand. The shaded cells represent the percentage of participants citing each source of mental demand, allowing for comparison within columns. The abbreviations are: ST (Short Text Interface), SI (Short Interactive Interface), LT (Long Text Interface), and LI (Long Interactive Interface). Short and Long refer to the sum across both interfaces; Text and Inter refer to the sum across both survey lengths. We include Sparklines for comparisons across these experiment groups.}
    \label{tbl:mental}
    \includegraphics[width=\linewidth]{content/image/cog/mental_table.png}
\end{table}
\subsection{Source of Mental Demand}
\label{sec:mental}

\vspace{5pt}
\begin{tldrbox}
    \faInfoCircle~\xspace\textbf{Takeaway:} Participants across all groups highlighted~\textbf{budget management} and~\textbf{preference construction} as their primary sources of mental demand. We observed two key differences: First, slightly more participants using the text interface reported mental demand from precisely determining the number of votes for options compared to the interactive interface. Second, when it comes to long QS, participants using the long interactive interface considered broader societal impacts and evaluated options holistically, while those in the long text interface focused on personal relevance and individual issues. These differences indicate that the interactive interface encouraged deeper thinking, shifting the source of mental demand. % We find evidence that the interactive interface prevented cognitive overload.
\end{tldrbox}

Mental demand refers to the extent of mental and perceptual activity required. Interview results showed the top two causes that increased participants' mental demand were~\textit{Budget management} and~\textit{Preference construction}. Table~\ref{tbl:mental} listed all causes. We discuss in detail below.

Fourteen participants expressed demand from budgeting within limited credit (N=4), tracking remaining credits (N=10), and maximizing credit use (N=8). For example:

\subsubsection{Mental Demand Source: Budget management} $14$ participants expressed demand from budgeting within limited credit ($N=4$), tracking remaining credits ($N=10$), and maximizing credit use ($N=8$). For example:

\begin{displayquote}
How many I got left that~\ldots\ that I haven't voted on yet, and seeing if I and looking at the remaining credits, I'm trying to mentally divide that up before I start allocating upvotes and downvotes.

\small{\noindent \hfill -- S006, long interactive interface, budget within limited credit}
\end{displayquote}

\begin{displayquote}
And then I just wanted to make sure that I used all the credit that I had available to me, and also knowing that in order to like show your support for certain societal issues you had to like that was giving a tangential take away from other societal issues that you could support as well.
    
\noindent \hfill -- S032, short text interface, track remaining credits.
\end{displayquote}

In the first quote, the participant struggles with not running out of credit while allocating credits. The second quote highlights the challenge of maximizing spend while ensuring sufficient differentiation. Both relate to effective budget management.

We further categorized budget management causes as operational (completing an operation, e.g., using the last credit) or strategic (achieving a higher goal, e.g., spreading credits across options). Strategic planning does not refer to gaming out others or 'winning' a game but rather to high-level thinking processes that consider strategies and plans to tackle a challenge, compared to operational tasks such as adjusting a specific vote value. Most long survey participants reported operational causes, indicating they didn't face enough mental demand to consider additional strategies.

\subsubsection{Mental Demand Source: Preference construction}
Almost all participants ($N=39$) reported increased mental demand from preference construction. We broken it down into three sources: determining relative preference ($N=16$), option prioritization ($N=17$), and precise resource allocation ($N=30$), For example, participants would focus on internal evaluation and construct comparisons among different options:

\begin{displayquote}
Figuring out my priorities, and how much I prioritize option 1 over option 2. What is the difference between those 2 on my priority list?

\hfill -- S002, short interactive interface, determining relative preference
\end{displayquote}

% I think the whole time I was trying to balance, I think II think I partly was discovering my what's the word I want to use bias isn't quite right. My priorities (S031, I)

Participants would locate higher priority options through trade-off decision making or map existing internal preferences into a subset of options:

\begin{displayquote}
I knew which ones that I wanted to dedicate the most to, and I knew which one I wanted to dedicate the least to. But it was that middle area that was kind of a grey area.
    
\noindent \hfill -- S008, short interactive interface, option prioritization
\end{displayquote}

% I knew which ones that I wanted to dedicate the most to, and I knew which one I wanted to dedicate the least to. But it was that middle area that was kind of a grey area.
% \noindent \hfill -- S024, short text interface, option prioritization

Finally, participants tried to allocate specific values or make specific 
adjustments to represent their preferences. 

\begin{displayquote}
I'm not sure how to put into words~\ldots like having to pick how many upvotes would go to each one
    
\noindent \hfill -- S023, long text interface, precice resource allocation
\end{displayquote}

Almost all participants mentioned preference construction as a source of mental demand, supporting the theory that preference construction is a difficult and mentally demanding task. Notably, more participants using the text interface reported mental demand from precise resource allocation compared to the interactive interface ($18$ vs. $12$). We conjecture that the interactive interface helped participants make more informed decisions, reducing their mental demands in this area.


\subsubsection{Mental Demand Source: Other Sources}
We identified four additional sources causing participants' mental demands: \textit{experiment setup}, \textit{number of options}, \textit{QS mechanism}, and \textit{external factors}. 

$24$ participants mentioned the experiment setup mainly related to understanding and recalling their experience with the options. $6$ participants, all from the long QS, found the number of options added mental demand. $4$ participants cited working with getting familiar with the QS mechanism as a source of mental demand. These are sources related to the study design. $12$ participants mentioned external factors, such as considering the consequences of their results or the challenges decision-makers face. $4$ participants reported an increase and another four a reduction in mental demand due to the interface design. $8$ participants expressed mental demand from justifying their choices and reflecting on their responses, questioning whether their votes truly reflected their preferences or if the amount of credit spent was justified.

\subsubsection{Takeaway: A different scope of preference construction and budget management approach among long QS groups}
\label{sec:mental_takeaway}

While these sources are common across all experiment groups, we highlight a notable difference when we focus on participants across interfaces completing the long QS. \textbf{They focus on a different scope of preference construction}. More specifically, participants ($N=8$) in the long text interface tend to experience mental demand from preference construction by thinking about issues more narrowly and focusing on personal relevance. Conversely, participants ($N=9$) in the long interactive interface experience higher mental demand from considering the broader societal impact and evaluating options more comprehensively. Only four participants in the long text interface expressed a comprehensive view, and three participants in the long interactive interface expressed a narrow and personal view.

% \begin{displayquote}
% \bracketellipsis also seeing the long list and obviously having to pick between quite a few things that I do feel very strongly about and having to figure out which ones do, I feel more strongly about than others.
    
% \noindent \hfill -- S023, long text interface
% \end{displayquote}

\begin{displayquote}
Trying to figure out what upvotes I should give it you know~\ldots compared to~\ldots I even kind of went back compared to the other topics: <topic one> compared to <topic two>, and even with like <topic three>, I kind of went back and forth between those two. \bracketellipsis So it was very mental tasking for me.

\noindent \hfill -- S015, long text interface
\end{displayquote}

% \begin{displayquote}
% \bracketellipsis really having to think, especially with so many different societal issues. How do I personally prioritize them? And to what extent do I prioritize them?
    
% \noindent \hfill -- S009, long interactive interface
% \end{displayquote}

\begin{displayquote}
\bracketellipsis really going through the rest of the categories and deciding okay, which are the pressing issues of our time and which are the pressing issues for this particular society that that I live in. \bracketellipsis You know these causes need a lot more funding, and and others can probably still have some sort of an impact, even with less resources.

\noindent \hfill -- S019, long interactive interface
\end{displayquote}

In the first quote, participants felt mental demand focusing on three options, trying to recall specific characteristics to differentiate them. In the second quote, participants considered the societal impact of options, aiming to maximize their effect. This difference highlighted our belief that the organization phase prompted participants to consider a broader range of factors in their decisions. Across both interfaces, participants in the long survey tended towards operational mental demands related to budget management. We argue the interactive interface prevented participants from using heuristics to narrow their choices, which only appeared when final votes were determined, an area with less support from the interface. We conclude the interactive interface scaffolded the decision-making process, preventing cognitive overload, and thus shifted mental demand sources. This shift explains why we did not notice significant differences in mental demand raw values (Figure~\ref{fig:mental_cog_score}) across the four experiment groups.

% In addition, we also find that long text interface participants focused on more operational behaviors such as:

% \begin{displayquote}
% So I wanted to be fair.~\bracketellipsis I actually took my calculator out and said~\bracketellipsis  how much would it be if I equally distributed it and then how do I do that? Do I wanna do it all equally or not?

% \noindent \hfill -- S020, long text interface
% \end{displayquote}

% compared to more procedures involving more strategic planning such as:

% \begin{displayquote}
% I wanted to make sure I wanted to give some credit to everything~\bracketellipsis I'm trying to make sure that I had without doing a lot of~\ldots I guess redos is trying to kind of get it right the first time on how I weight things.

% \noindent \hfill -- S032, long interactive interface
% \end{displayquote}



% IN_T4: Wanting more information on the options (N=6/40)
% 5. While the numbers seem small, non of this request came from v3. This could explain that participants are already overloaded from the existing the task.

\begin{figure}[h]
    \begin{minipage}[t]{0.45\textwidth}
        \centering
        \includegraphics[width=\textwidth, trim=0 13 0 13, clip]{content/image/cog/Mental_scores.pdf}
        \captionsetup{width=0.9\textwidth, justification=justified}
        \caption{Mental Demand Raw Score: Across all four experiment groups, participants' reported mental demand is spread across a wide range with many participants experiencing high mental demand.}
        \label{fig:mental_cog_score}
    \end{minipage}
    \hfill
    \begin{minipage}[t]{0.45\textwidth}
        \centering
        \includegraphics[width=\textwidth, trim=0 13 0 13, clip]{content/image/cog/Physical_scores.pdf}
        \captionsetup{width=0.9\textwidth, justification=justified}
        \caption{Physical Demand Raw Score: Participants other than the long interactive interface reported minimal physical demand. The long interactive interface had the highest physical demand, likely due to increased mouse clicks and extended time spent looking at the vertical screen.}
        \label{fig:physical_cog_score}
    \end{minipage}
\end{figure}

% ============================================= %
\begin{table}[h]
    \caption{Physical Demand Causes: Most participants expressed little or no physical demand. Results reflected that participants in the long interactive interface required more actions, hence the higher mention of mouse usage as a source.}
    \label{tbl:physical}
    \includegraphics[width=\linewidth]{content/image/cog/physical_table.png}
\end{table}

\subsection{Source of Physical Demand} 
\label{sec:physical}

\vspace{5pt}
\begin{tldrbox}
    \faInfoCircle~\xspace\textbf{Takeaway:} Participants across all groups highlighted \textit{reading}, \textit{using the mouse}, and \textit{navigating a vertical screen} as cause of physical demand. Most participants experienced little or minimal physical demand. Interactive interface users experienced higher physical demand due to increased mouse usage.
\end{tldrbox}

Physical demand refers to the physical effort required to complete a task, such as physical exertion or movement. Since this study involves participants sitting in front of a computer screen completing a survey, most participants reported minimal physical demand($N=32$). We nonetheless report the sources of this minimal demand, which include reading text on the screen ($N=4$), using the mouse ($N=16$), and moving their head to navigate the vertical screen ($N=5$). Participants emphasized that these demands were minimal, which is reflected in the low values reported in the NASA-TLX physical demand scores (Figure~\ref{fig:physical_cog_score})

Notably, $11$ out of $20$ participants who used the interactive interface mentioned physical demand from using the mouse, reflecting their increased interaction with the interface. Table~\ref{tbl:physical} shows the distribution of participants across different sources of physical demand. This is further supported by the raw NASA-TLX physical demand scores, which show a significant visual difference between short and long interactive interfaces as well as between text and interactive interfaces in long surveys.

% ============================================= %
\begin{table}[h]
    \caption{Temporal Demand Sources: Decision-making and Operational Tasks are the main causes. Participants framed their decision-making sources differently.}
    \label{tbl:temporal}
    \includegraphics[width=\linewidth]{content/image/cog/temporal_table.png}
\end{table}

\subsection{Source of Temporal Demand} 
\label{sec:temporal}
\vspace{5pt}

\begin{tldrbox}
    \faInfoCircle~\xspace\textbf{Takeaway:} Participants faced increased temporal demand from: \textit{Budget}, \textit{Decision Complexity}, and \textit{Operational Tasks}. Notably, the interactive interface managed temporal demand more effectively, allowing participants to pace themselves and avoid misperceiving task difficulty.
\end{tldrbox}

Temporal demand measures the time pressure participants feel during a task. Lower demand means a more leisurely pace. The main sources of increased temporal demand are (Table~\ref{tbl:temporal}) of increased temporal demand are:~\textit{Budget}, ~\textit{Decision Complexity}, and ~\textit{Operational Tasks}. 

\subsubsection{Temporal Demand Source: Budget}
Budget emerged as a theme across all conditions, with four participants feeling rushed as their credits decreased, translating the increasing marginal cost of votes into higher temporal demand. As one participant said:

\begin{displayquote}
When the money was decreasing, as I was casting more upvotes or downvotes so as the money decreases I felt kind of rushed.
            
\noindent \hfill -- S034, long interactive interface
\end{displayquote}


\begin{wrapfigure}{r}{0.45\textwidth} % Adjust the width as needed
    \centering
    \includegraphics[width=0.45\textwidth, trim=0 13 0 13, clip]{content/image/cog/Temporal_scores.pdf}
    \captionsetup{width=0.40\textwidth, justification=justified} % Adjust the width to match the image width
    \caption{Temporal Demand Raw Score: The short text interface results in the highest temporal demand, while the long text interface has the lowest. Both interactive interfaces, particularly the short interactive, show moderate temporal demand, suggesting that interactive elements help manage time pressure more effectively, allowing participants to pace themselves better and engage more deeply in the tasks.}
    \label{fig:temporal_cog_score}
\end{wrapfigure}

\subsubsection{Temporal Demand Source: Decision Complexity through different lens}
Decision Complexity refers to when participants felt that there are many decisions to make. These causes appear in two forms -- affirmative and negative. Affirmative perception refers to participants explicitly expressing that there are many decisions~\textit{to make}, while negative perception refers to participants describing concerns regarding the time and effort~\textit{already invested} in the survey.

\begin{displayquote}
\bracketellipsis when you are being presented the ideas that they're that they are being put together, and you need to allocate the resources~\ldots Say, you know, this one is more important than that one~\ldots that's the part when it gets tricky, so that you spend more time here. 
    
\noindent \hfill -- S037, long interactive interface
\end{displayquote}

\begin{displayquote}
\bracketellipsis so at first it was like, `Okay, this is fine.' But then on the end, I was like, maybe I should just hurry up and make a decision. So it's like at first it would been here, but then I kinda moved up near the end when I was hanging a waffling between my upvotes.
\noindent \hfill -- S024, short text interface
\end{displayquote}

The first quote illustrates temporal demand from the decisions participants need to make. The second highlights demand from the time already devoted. In the short text interface, half of the participants expressed negative perceptions of temporal demand, whereas in the long interactive interface, half had affirmative perceptions.

\subsubsection{Temporal Demand Source: Operational Tasks}
Operational tasks involve actions like updating votes and completing the survey. For instance, one participant aimed to operate swiftly:

\begin{displayquote}
I wanna get through things in an efficient manner which doesn't necessarily mean I rush it. But it does mean that I do things expeditiously. Especially. I'd like to think I'm somewhat computer-savvy. And so to be able to move through this quickly and efficiently. I do take pride in, but it's all personal stuff. It's not nothing outwardly influencing me. 
        
\noindent \hfill -- S032, short text interface
\end{displayquote}

\begin{displayquote}
I want the credit done but I don't want to be overthinking.
            
\noindent \hfill -- S013, short text interface
\end{displayquote}

The first quote refers to the participant's aim to operate swiftly on the interface, not specifically related to decision making. Similarly, the latter focuses on using the credit to complete a specific goal. We found that temporal demand is higher for the short survey experiment group. Over half of the participants from the short interface wanted to complete the task swiftly and quickly, compared to $5$ participants from the long QS group.

\subsubsection{Takeaway: Temporal demand managed through interactive interface}
The raw NASA-TLX values in Fig~\ref{fig:temporal_cog_score} visually indicate two important points. First, temporal demand trended lower for the interactive interface in the short QS condition, while it trended higher for the long QS condition. Second, the long text interface exhibited the lowest temporal demand, which is counterintuitive since participants in this condition made no fewer decisions and operations compared to the short text group. According to our interpretation of mental demand results in Section~\ref{sec:mental_takeaway}, participants likely did not experience temporal demand because they applied heuristics to reduce the number of decisions, thereby lowering their cognitive load and decision-making instances.

Additionally, participants in the long interactive condition reported that the numerous required operations created temporal demand, preventing them from taking mental shortcuts and shifting their cognitive load to different dimensions.

Furthermore, participants in the short text QS expressed high temporal demand and perceived it negatively, likely misperceiving task difficulty. Conversely, even though the short interactive interface required more decisions, participants reported less temporal demand from decision-making, resulting in a lower overall score. This suggests that the interactive interface slowed them down without increasing temporal demand, allowing them to pace themselves and engage in more in-depth thinking, thereby preventing a misperception of task difficulty.

These observations across experimental conditions support the plausible explanation that the interactive interface mitigated cognitive overload, as evidenced by the different sources of temporal demand.

%  TODO: move to discussion?
% It is also worth noting that three participants from the 20 who responded to the long survey mentioned that the vertical screen's ability to see all options facilitated direct comparisons and transparency about the entirety of the task, which reduced the temporal demand.

% \begin{displayquote}
% (Seeing) all at once I can see how many there are, so it's kind of like I can kind of tell when I will be done.

% \noindent \hfill -- S041, long text interface
% \end{displayquote}

% ============================================= %
\begin{table}[h]
    \caption{Performance Causes: Most causes are shared across experiment conditions. We provided qualitative interpretations of their own perfornace assessments.}
    \label{tbl:physical}
    \includegraphics[width=\linewidth]{content/image/cog/perf_table.png}
\end{table}

\subsection{Source of Performance}
\label{sec:performance}
\vspace{5pt}

\begin{tldrbox}
    \faInfoCircle~\xspace\textbf{Takeaway:} Participants experienced performance demands due to \textit{Operational Actions} and \textit{Social Responsibility}. Despite similar performance scores across groups, more participants using the interactive interface felt more positive about their performance.
\end{tldrbox}

Performance refers to a person's perception of their success in completing a task. Lower values mean good performance; higher values mean poor performance. We found minimal qualitative differences between experiment groups regarding influence sources. We identified two performance demands from the interviews: \textit{Operational Actions} and \textit{Social Responsibility}. Despite most participants reporting positively on their performance, nuances exist in how different groups interpret their performance.

\subsubsection{Performance Source: Operational Actions}
Operational actions, like the theme presented in temporal demand, refer to specific, executable procedures participants perform in the survey. All experiment groups share these sources. Six participants felt pressured to spend all their credits or stay within budget. Five participants worried choice of votes didn't reflect their true preferences. Additionally, six participants experienced performance demand due to limited time, energy, and resources, which ties into other specific cognitive demands like decision fatigue and time pressure. Here we show two examples:

\begin{wrapfigure}{r}{0.45\textwidth} % Adjust the width as needed
    \centering
    \includegraphics[width=0.45\textwidth, trim=0 13 0 13, clip]{content/image/cog/Performance_scores.pdf}
    \captionsetup{width=0.40\textwidth, justification=justified} % Adjust the width to match the image width
    \caption{Performance Demand Raw Score: Participants showed indifferent performance raw scores across experiment conditions, all trending toward satisfactory.}
    \label{fig:performance_cog_score}
\end{wrapfigure}

\begin{displayquote}
I don't think I did it perfectly, because I didn't have 0 remaining credits.
    
\noindent \hfill -- S024, short text interface, budget management
\end{displayquote}

\begin{displayquote}
I'm concerned that it's not as reflective of my views as I wanted to be like, or I was concerned about it.~\bracketellipsis I was concerned that maybe it didn't.

\noindent \hfill -- S041, long text interface, preference reflectiveness
\end{displayquote}


\subsubsection{Performance Source: Social Responsibility}
Social Responsibility is a noteworthy performance demand, categorized into \textit{decision-maker responsibility} (N=8) and \textit{uncertainty of the outcome} (N=3). The former refers to individuals feeling guilty because they couldn’t avoid specific tradeoffs or wanted to be fair. This theme resembles `External Demand' in Mental Demand. For example,

\begin{displayquote}
I don't want people to think that I just like don't care about <ethnicity> people at all. I also don't think like government funding should go towards like religious organizations. You know what I mean. So I don't want somebody to think that like, I just don't care about <ethnicity> people.
    
\noindent \hfill -- S041, long text interface, decision-maker responsibility
\end{displayquote}

In this quote, the participant put themselves inside the shoes of a member of the government, rather than a citizen expressing their own attitudes. This shift in roles introduced the performance demand and demonstrated that QS mirrors the decision-maker's dilemma in individual survey responses. This characteristic also applies to the latter, further highlighting participants' attempts to foresee an outcome:

\begin{displayquote}
If I was actually running a government funding~\bracketellipsis I don't know how this (the survey results) might actually affect people. Some of these things might be unpopular or bad, or have outcomes that I didn't forsee.
    
\noindent \hfill -- S027, short interactive interface, uncertainity of the outcome
\end{displayquote}

Social responsibility also spans experiment groups. Raw NASA-TLX scores (Figure~\ref{fig:performance_cog_score}) show participants had indistinguishable performance scores. This aligns with the interview results where most participants felt positive about their final submission. We also analyzed participants' satisfactory statements regarding performance.

\subsubsection{Takeaway: Half of the participants using the interactive interface~\textit{Felt good}}
We identified three types of satisfactory statements regarding self-reported performance:
\begin{itemize}
    \item \textit{Did their best} refers to statements where a participant stated they exhausted their maximum effort to complete the task.
    \item \textit{Feel good} efers to statements where a participant expressed positive emotions or satisfaction about their performance or the outcome.
    \item \textit{Good enough} refers to statements where a participant acknowledged that their performance or the outcome was acceptable or satisfactory, but not necessarily perfect or the best possible.
\end{itemize}

Approximately the same number of participants in each of the four experiment groups expressed \textit{Good enough}. Meanwhile, participants using the interactive interface across short and long groups had almost double the number of participants ($N=11$) who expressed \textit{Feel good} compared to the text interface ($N=6$).The text interface had slightly more participants ($N=5$) who expressed \textit{Did their best} compared to the interactive interface ($N=3$).

These results highlight key takeaways: First, participants from all experiment groups expressed satisficing behaviors (\textit{Good enough}) at similar frequencies. Second, participants using the interactive interface were generally more positive about their experience and the outcomes.

% TODO: Need to check the reflective thinking part a bit. I **think** there are differences but it is unclear, need to go back to raw code.


% ============================================= %
\begin{table}[h]
    \caption{Mental Demand Table, needs to be updated with some new terms definitions for some of the columns.}
    \label{tbl:physical}
    \includegraphics[width=\linewidth]{content/image/cog/effort_table.png}
\end{table}

\subsection{Source of Effort}
\label{sec:effort}

\vspace{5pt}
\begin{tldrbox}
    \faInfoCircle~\xspace\textbf{Takeaway:} Participants' effort demands stem from \textit{Operational Tasks} and \textit{Strategic Planning}. Operational tasks include navigating interfaces and managing budgets, with text interface users reporting more effort in these areas. Strategic planning involves both personal and global considerations, with interactive interface users focusing more on broader, communal values. Overall, participants using the interactive interface reported more effort from strategic planning, while those using the text interface reported more effort from operational tasks.
\end{tldrbox}
Effort refers to the amount of work required to achieve a level of performance. It includes the intensity of both mental and physical resources expended during the task.

Similar to our analysis for mental demand, we code the source of effort into to major categories:~\textit{Operational Tasks} and~\textit{Strategic Planning}.

\subsubsection{Operational Tasks} Similiar to performance, we focus on operational tasks that contributed to effort with a narrow focus including: navigating the interface, managing the budget at an operational scale (i.e., making sure not to run out of budget, making specific updates between two options), or translating an opinion to a quantifiable adjustment on the survey. These narrower low level operations involves taking effort to making updates or actions related to the interface itself. We show two examples associated with different aspects of operational tasks that influence precieved effort:

\begin{displayquote}
And then I wanted to bump up (an option) maybe to 4 or <option> to 5 and realize I couldn't. My point (number of votes) had to like back down a little bit~\ldots So that would be effort came in of how do I want to really rearrange this to make it (the budget spending) maximize?

\noindent \hfill -- S029, short text interface
\end{displayquote}

\begin{displayquote}
So it was like it was very~\ldots I have to put a lot of effort in terms of you know~\ldots think about each dimension that if I give one credit to <option name> whether it will affect my credits on <another option name>.

\noindent \hfill -- S005, long text interface
\end{displayquote}

Notably, $14$ of the $20$ participants using the text interface expressed overwhelminly mentioned sources related to such sources, compared to less than half of the participants ($N=7$) from the interactive interface, with the lowest mention by the long interactive interface group ($N=2$). We review the other category before making interpretations.

\subsubsection{Strategic Planning} Opposite to operational tasks, strategic planning follow definitions established for mental demand which involces higher level strategies to complete the survey. We further derive two distinct types of planning:~\textit{personal} and~\textit{global}.~\textit{Personal strategic planning} refers to taking effort to translate preferences onto the survey without considering governing values or broader beliefs. For example, this participant expressed effort from retrieving past experiences to inform their decisions:

\begin{displayquote}
~\bracketellipsis having that prior experience and being able to quickly link it to a tangible thing that I've experienced in my personal life.

\noindent \hfill -- S032, short text interface
\end{displayquote}

\begin{displayquote}
And really the bulk of the effort was how to rank order these (options) and allocate the resources behind the upvotes so that I can accurately depict what I want~\ldots say, a committee to focus on and allocate actual fungible resources, too. 

\noindent \hfill -- S019, long interactive interface
\end{displayquote}

While the difference in the number of citations to personal strategic planning are less pronounced across groups, the interactive interface (N=13) still scores slightly higher counts compared to the text interface (N=9).~\textit{Global strategic planning}, on the other hand, involves participants formulating strategies to align with broader, communal values. This includes ensuring fairness, considering the impact of different options on the entire community, and evaluating the complex relationships between various options. For example:

\begin{displayquote}
I think, imagining the trying to imagine every outcome trying to to imagine what what else would be encompassed, encompassed by each category.

\noindent \hfill -- S027, short interactive interface
\end{displayquote}

\begin{displayquote}
Hey, even though I don't really like this idea. But what if they're important? They sort of kind of deserve some attention~\ldots that's why I think I have the effort here.

\noindent \hfill -- S037, long interactive interface
\end{displayquote}
    
Both examples shows considerations beyond personal experiences, considering outcomes or social values. We notice that nearly twice as many participants (N=7) in the interactive interface expressed effort from global strategic efforts compared to the text interface (N=4). Altogether, we observe more participants using the interactive interface (N=17) reported sources of strategic effort compared to those using text-based interfaces (N=11). 

\begin{wrapfigure}{r}{0.45\textwidth} % Adjust the width as needed
    \centering
    \includegraphics[width=0.45\textwidth, trim=0 13 0 13, clip]{content/image/cog/Effort_scores.pdf}
    \captionsetup{width=0.40\textwidth, justification=justified} % Adjust the width to match the image width
    \caption{Effort Raw Score: Text placed here}
    \label{fig:effort_cog_score}
\end{wrapfigure}

Qualitative analysis in this subsection added clear evidence that the source of cognitive demand for effort differs between text and interactive interfaces, similiar to mental demand. Participants using the interactive interface focus less on operational tasks and more on strategic planning, specifcially global strategic planning, where they think about options holistically and beyond the option itself. This is in contrast to participants using the text interface, who focus more on operational tasks and a narrower strategic planning scope. The raw NASA-TLX effort scores (Figure~\ref{fig:effort_cog_score}) can then be explained that even though reported values are similar across the four experiment groups, the sources of effort differ between text and interactive interfaces.

% ============================================= %
\begin{table}[h]
    \caption{Mental Demand Table, needs to be updated with some new terms definitions for some of the columns.}
    \label{tbl:fustration}
    \includegraphics[width=\linewidth]{content/image/cog/fustration_table.png}
\end{table}
\subsection{Source of Physical Demand} 
\label{sec:fustration}

\vspace{5pt}
\begin{tldrbox}
    \faInfoCircle~\xspace\textbf{Takeaway:} Participants experienced frustration from two main sources: \textit{Operational Actions} and \textit{Strategic Planning}. Operational actions included managing budgets, deciding final values for options, and understanding content, with text interface users reporting more frustration in these areas. Strategic planning frustration stemmed from conflicts between personal and societal preferences and making trade-offs among options. Overall, while operational frustrations were common across all groups, participants in the long text interface reported slightly less frustration, likely due to fewer operational challenges.
\end{tldrbox}

Fustration is the last dimension of NASA-TLX. It refers to the extent to which the participant is annoyed, irritated, or discouraged during the task.

Following the previous analysis, we categorize the sources of frustration into three major themes:~\textit{Operational Actions} and~\textit{Strategic Planning}

\subsubsection{Operational Actions} Similar to the previous definitions, $15$ participants highlighted this source for frustration. Six participants expressed frustration regarding credit management (i.e., overspending budget); four participants mentioned had trouble deciding the final value for the options; three participants are frustrated because they need to make multiple decisions; five participants were frustrated with the quadratic mechanism; four participants are frustrated trying to understand the content of the option or how the option connects to them. For example, 

\begin{displayquote}
I was slightly frustrated when doing the task, probably because there was a budget that we kind of had to stick with it.

\noindent \hfill -- S001, long text interface, quadratic mechanism
\end{displayquote}

\begin{displayquote}
i think just frustration~\bracketellipsis because when i was making like the decisions on how many upvotes I could put in each section, I was running out of credits.

\noindent \hfill -- S013, short interactive interface, budget management
\end{displayquote}

These demonstrated participants frustration because they are hindered by not being able to complete specific operational actions or constraints presented by QS. What is noteable is that all experiment groups had almost half of the participants express operational frustration compared to only two participants from the long text interface group. It is not clear why they did not encounter similiar frustration.

\subsubsection{Strategic Planning} For frustration, we further derived strategic planning into two types: ~\textit{lower-level} and~\textit{higher-level}. For the former, Four participants expressed conflict between their personal preferences and what they believe would be other people's preferences. Eight participants experienced conflict between making tradeoffs among a few options. For example:

\begin{displayquote}
Because I know that's important to other people. But it just doesn't to me.
    
\noindent \hfill -- S010, short interactive interface
\end{displayquote}

\begin{displayquote}
I would have loved to have given more to other groups~\ldots and I felt stressed like~\bracketellipsis well~\ldots it's a group that you know is still~\ldots you know~\ldots important~\bracketellipsis
\noindent \hfill -- S020, long text interface
\end{displayquote}

These quotes showed participants trying to ahere to lower-level strategies such as considering personal prefernces or making trade-offs within a smaller scope. Compare to~\textit{higher-level strategic planning}, where six participants expressed conflicts that touch on the broader society and their core values of looking at the broader scope. Eight participants felt frustrated because they were forced to make trade-offs among \textit{all} options instead of a few. For example,

\begin{displayquote}
I had to consider how I feel towards that~\ldots how religious media broadcasting is being used in like today's society~\ldots today's political environment. So yeah~\ldots you really have to consider what is important to you. 
\noindent \hfill -- S020, long text interface, value conflicts
\end{displayquote}

\begin{displayquote}
I think the frustration is~\ldots I wish that we could help all of these causes, but you know it's just like anything else. You can't do everything and when it's not~\ldots  I feel like it's hard to quantify how much some of these things should be supported versus others. So when you're talking about upvotes and things that's challenging to me, it's frustrating.
\noindent \hfill -- S026, long interactive interface, considering all options
\end{displayquote}

\begin{wrapfigure}{r}{0.45\textwidth} % Adjust the width as needed
    \centering
    \includegraphics[width=0.45\textwidth, trim=0 13 0 13, clip]{content/image/cog/Frustration_scores.pdf}
    \captionsetup{width=0.40\textwidth, justification=justified} % Adjust the width to match the image width
    \caption{Fustration Raw Score: Participants other than the long interactive interface reported minimal physical demand. The long interactive interface had the highest physical demand, likely due to increased mouse clicks and extended time spent looking at the vertical screen.}
    \label{fig:fustration_cog_score}
\end{wrapfigure}

Fustration that stemmed from strategy planning are spread across all experimental conditions. Reflecting on the raw NASA-TLX scores (Figure~\ref{fig:fustration_cog_score}), We only see a slight difference in less frustration from the long text interface participants compared with the rest of the participants, likey due to the less frustration from operational tasks. Thus, we interpret that frustration comes more from individual's ability to discern and make decisions and not necessarily tied to specific methods in the construction of preference.

\subsection{Summary}
To recap, the analysis identified the different sources of cognitive load experienced by participants. More specifically, it highlighted differences across experimental conditions. Interactive interfaces, especially long ones, drive participants to adopt a holistic view and encourage higher-level deliberation, indicated by increased mental demand and effort. Conversely, participants perceived more operational demand when completing specific tasks using the text interface. Mental demand, effort, and temporal demand highlighted the urgency participants felt to complete tasks swiftly. This distinction doesn't mean one group of participants excludes the other group's demands, but it highlights that the main source of demand shifts with different interfaces.




% \begin{figure}[ht]
%     \centering
%     \includegraphics[width=\textwidth]{content/image/cog/nasatlx_final_value_with_CI.pdf}
%     \caption{NASA-TLX Results}
%     \label{fig:nasatlx-with-ci}
% \end{figure}
\section{Behavior Results}
\label{sec:behave_result}
To answer RQ3, we investigate time-to-action and remaining credit differences across experiment conditions. Time-to-action is a widely used metric in decision sciences to understand individual behaviors. For example,~\textcite{payneAdaptiveDecisionMaker1993} theorized that longer decision time represents more complex and deeper cognitive processing. Additionally, resource allocation strongly influences decision making. \textcite{chengCanShowWhat2021} showed that the number of given credits influences the validity of QV. Decision science studies like \textcite{Shah2015a} and \cite{debruijnPovertyEconomicDecision2022} showed how scarcity influences decisions, increases risk aversion, and adds cognitive load. In this section, we use these two measures as proxies to understand participant behaviors. We publicly shared all analyzed participant interaction data\footnote{link-to-github} to support transparency and facilitate further research.

\newsavebox{\savefig}

\begin{figure}[htbp]
    \centering
    \savebox{\savefig}{
        \begin{minipage}{0.78\pdfpageheight}
            \begin{subfigure}[b]{0.26\pdfpageheight}
                \centering
                \includegraphics[width=\textwidth]{content/image/results/total_time_per_option.pdf}
                \caption{Total Time per option}
                \label{fig:total_time}
            \end{subfigure}
            % \hfill
            \begin{subfigure}[b]{0.26\pdfpageheight}
                \centering
                \includegraphics[width=\textwidth]{content/image/results/org_time_per_option.pdf}
                \caption{Organization Time per option}
                \label{fig:org_time}
            \end{subfigure}
            % \hfill
            \begin{subfigure}[b]{0.26\pdfpageheight}
                \centering
                \includegraphics[width=\textwidth]{content/image/results/voting_time_per_option.pdf}
                \caption{Voting Time per option}
                \label{fig:vote_time}
            \end{subfigure}
        \end{minipage}
    }
    \rotatebox{90}{%
        \begin{minipage}{\wd\savefig}
            \usebox{\savefig}
            \caption{Swimlane Diagram}
        \end{minipage}
    }
    \label{fig:time_per_option_full}
\end{figure}


\subsection{Time Spent per Options}
\label{sec:time_per_option}
Our first analysis focuses on understanding how much time participants spent per option across different stages and experiment conditions. Based on the QS system log, we can derive the following detailed logs of participant actions:~\textit{the option}  involved in the interaction,~\textit{the type of interaction} (such as updating a certain number of votes), and~\textit{the time} between this interaction and the previous one.

We aggregate all the time spent on each option as the total time spent for that option. Organization time includes the time participants spent placing options into preference categories and the drag-and-drop time associated with each option during the organization phase. Voting time strictly refers to the time participants took to update vote values for each option. To minimize noise, we intentionally drop all the time participants spent on the first option in the organization phase or voting phase. The goal is to exclude time spent on reading the prompt, forming their preference, or understanding the interface.

Figure~\ref{fig:time_per_option_full} each dot represents one option for one participant. Figure~\ref{fig:total_time} shows total time, figure~\ref{fig:org_time} shows organization time, and figure~\ref{fig:vote_time} shows voting time. The violin plot shows the distribution of the dots and the three horizontal lines represent the median, 25th percentile, and 75th percentile of the time spent for that interface. We limited the y-axis to 1 minute to improve visualization clarity.

Participants spent more time on the interactive interface than the text interface in both short and long surveys. A non-parametric Mann-Whitney U test confirmed this observation. For the long QS, the Mann-Whitney U test results showed a significant difference between the text interface and the interactive interface ($U=15536$, $p<0.0000001$). The effect size was small (Rank-biserial: $-0.304$, Cohen's d: $-0.030$) and the power of the test was $0.061$. For the short QS, the Mann-Whitney U test results showed a significant difference between the text interface and the interactive interface ($U=573$, $p=0.01$). The effect size was small (Rank-biserial: $-0.37$, Cohen's d: $-0.082$) and the power of the test was $0.066$. These results indicate that participants spent slightly less time on the text interface. This is expected as organizing options in the interactive interface takes more time. We break down the total time spent into organization time and voting time in Figure~\ref{fig:org_time} and Figure~\ref{fig:vote_time}.

We observed minimal difference in organization time (Figure~\ref{fig:org_time}) between short and long interface. The interface was designed with this in mind given that options are shown one at a time, and participants can drag and drop them into the preference categories when needed. Examining the voting time (Figure~\ref{fig:vote_time}), there is no significant difference in voting time between the text and interactive interfaces in the short survey ($p>0.4$, Power=$0.051$). However, in the long QS, there is a statistically significant difference ($U=24053$, $p<0.005$) in voting time between the text and interactive interfaces. The effect size was small (Rank-biserial: $0.167$, Cohen's d: $0.017$) and the power of the test was $0.053$. This indicates that participants spent slightly less time on the interactive interface in the long survey. This supports our hypothesis that the two-step design in the interactive interface facilitates more efficient decision-making, especially in longer surveys.

\subsection{Budget and Voting Behaviors}
To further analyze participant behaviors, we break down the aggregated time from the previous analysis and examine fine-grain interactions. Specifically, we examine if there are differences among behavior across interfaces. As we outlined, credit scarcity might influence decision making.  Figure~\ref{fig:voting_all} plots the time of voting actions over the remainder of the participant's budget across the text and interactive interface across all four groups. Each bar shows the number of actions accumulated across participants at specific percentages of remaining credits. A KDE plot is provided to better visualize the trends. We chose not to follow ~\textcite{quarfoot2017quadratic} focusing on the number of accumulated votes over an individual's time given that each individual's total time spent differ across experiment conditions.

Comparing experiment groups, we see fewer differences in the short QS but different interaction distributions between the two interfaces in the long QS. Given the significant differences in voting time between the text and interactive interfaces for the long QS, we focus on deciphering the voting action changes between these two conditions in this subsection.

\begin{figure}[ht]
    \centering
    \includegraphics[width=\textwidth]{content/image/results/clickstream_action_distribution.pdf}
    \caption{voting actions across all options (needs to update chart text, remove normalization, and change the dot colors.)}
    \label{fig:voting_all}
\end{figure}

In Figure~\ref{fig:voting_all}, we see two distinct patterns between the short survey and the long survey in terms of participant behaviors. In long surveys, participants exhibited more actions both when the budget was abundant and when it began to run out. This pattern was more pronounced with the long interactive interface. We further separated the behaviors where participants made large or small changes to the options, specifically for the long version. In Figure~\ref{fig:voting_v3_v4}, we define an adjustment of four or more votes as large, which we plotted in the first row of the figure. Adjustments of two or fewer votes are considered small, which is $10\%$ of the possible values one can choose among the maximum of 21 votes.

\begin{figure}[ht]
    \centering
    \includegraphics[width=\textwidth]{content/image/results/combined_density_plots.pdf}
    \caption{Breakdown of voting actions (needs to update chart text)}
    \label{fig:voting_v3_v4}
\end{figure}

Instead of showing the number of actions, we plotted all actions against the time it took to make them. Revisiting the KDE curve in the second row in Figure~\ref{fig:voting_all} and the curve of the second row in Figure~\ref{fig:voting_v3_v4} which represents the small vote adjustment across both interfaces, we see a stronger bimodal action distribution. In fact, the bimodal distribution is more pronounced in the interactive interface. This suggests that participants make small adjustments both at the beginning and towards the end of the QS. However, the interactive interface shows more frequent and faster edits towards the end. Visually, dots are more clustered in the long interactive interface for small vote adjustments compared to the long text interface. The Mann-Whitney U Test on the time spent on small vote adjustments showed significant differences ($U=13037$, $p<0.001$), with a small effect size (Rank-biserial: $0.227$, Cohen's d: $0.195$) and a power of $0.381$. Based on the KDE plots in the first row of Figure~\ref{fig:voting_v3_v4}, participants also made more large vote adjustments early on that spread more equally compared to the text interface. This indicates that participants had a clearer idea of how to distribute their credits across the options.

In interviews, five participants highlighted the importance of the interface's flexibility and their use of an incremental, iterative approach. All these participants used the interactive interface. While this doesn't mean participants using the text interface didn't take an iterative approach, it highlights that the interactive interface encouraged iterative and incremental updates. As one participant pointed out:

\begin{displayquote}
I like the fact that it remembers everything that you know. If if you make a mistake, that you don't lose all the work that you've already done. so I think that's very important is that it's an iterative process.

\noindent \hfill -- S019, long interactive interface.
\end{displayquote}

~\textbf{In summary, the interactive interface allowed participants to better structure their preferences and make faster iterative adjustments, as designed based on the differentiation and consolidation theory.} In other words, we show~\textit{different} voting behaviors presented across long text and interactive interface, hence~\textbf{rejecting} the alternative plausible explaination on the hightened cognitive load in the long interactive interface due to a pure increase of cognitive load Due to interactivity, hence further concluding that the interactive interface prevents satisficing from cognitive overload in long QS.

% This suggests that participants in the interactive interface are more likely to make larger adjustments to their votes. This is consistent with the observation that participants in the interactive interface spent more time on voting actions in the long survey. We also see a cluster of voting actions in the bottom left corner of the interactive interface for small vote adjustments. This suggests that participants in the interactive interface are more likely to make small adjustments when their budget is running low. This is consistent with the observation that participants in the interactive interface spent less time on voting actions in the long survey.
% First, surface the bimodal action distribution in both plots, with a even stronger signal for long interactive interface participants. Second, the plot demonstrated a clear cluster of voting actions in the bottom left corner of the interactive interface for small vote adjustments. In other words, participants made much smaller but more rapid adjustments when their budgets were running low. Second, larger adjustments are made when the participants have more options comparing the two plots on the first row. We interpret this behavior as participants in the interactive interface have constructed a clearer image of option preferences and, hence, have the ability to take larger strides in allotting their budget and deciding the number of votes at the beginning of the survey. Toward the end, participants using the interactive interface are then making fine-tuned adjustments to ensure that their preferences are reflected in their submissions.

% add qualitative support


% Ti-Chung Cheng: So what elements of the software interface do you dislike, or like the most, if any, when expressing your preferences on responding to societal issues?
% S009: Hmm! What I like the most actually, probably the sorting function. I think that it really helped me organize my thoughts really, clearly, in terms of what I would dislike the most. really, not that much. I would say. yeah, also, like how we could categorize it, like even within the voting stage rather than just at the categorization stage. 

% Ti-Chung Cheng: Can you tell me a little bit more about the screen? How did the vertical screen help you?
% S037: I think because it helps the layout of, because it's like a long 3 bar. So it's easier for you to to drag and drop, and you can actually sort it, judging by the votes. But I do not do that. But I think this layout is could be helpful in that aspect.



\section{Discussion and Future Works}
\label{sec:discussion}

In this section, we interpret the results related to cognitive load and survey respondent behaviors, emphasizing why the interactive interface did not uniformly reduce cognitive load in the long text interface while providing practical recommendations for practitioners deploying QS.

Our discussion centers on three key topics: elements of the two-phase interface that support preference construction, design recommendations for practitioners, and future challenges. Ultimately, we conclude that the two-phase interface has differential effects on the short and long surveys. While trends suggest a reduction in cognitive load with the two-phase interface compared to the text interface, we observe evidence of deeper engagement with options and enhanced preference construction, particularly in the long survey condition.

\subsection{Result Interpretation}
\subsubsection{Deeper engagements through preference construction in two-phase interfaces}

Our main findings indicate that the survey results, qualitative data, and observed behavioral differences reveal shifts in the types of cognitive load experienced by participants, especially for those completing the long survey. Cognitive load theory~\cite{swellerCognitiveLoadTheory2011}, when applied to the context of QS, identifies the three components of cognitive load: intrinsic load (the cognitive demand required to understand questions and response options), germane load (associated with deeper processing and evaluation of preferences), and extraneous load (stemming from navigating and operating the survey interface).

Participants are randomly assigned to experimental conditions, with both survey lengths containing options randomly drawn from a common pool to control intrinsic load within the same group.  

In the short survey condition, participants engage with all options simultaneously. The two-phase interface reduces some extraneous load associated with navigating the interface during voting, though it requires participants to complete the grouping phase. Despite this additional task, participants across both interface types report minimal or no physical demand. The two-phase interface likely facilitates easier engagement with preference construction due to its lower-trended cognitive load, as reflected in the increased likelihood of perceived lower cognitive load.  

In the long survey condition, participants cannot engage with all options simultaneously, resulting in a higher intrinsic load at the start of the survey. The organization phase in the two-phase interface shapes participant behavior during the voting phase. While it streamlines the process of locating options, as exemplified by the reduction of edit distance, this benefit may be offset by the additional physical effort required to complete the grouping phase, as reflected by the slightly increased physical demand.  

However, qualitative data regarding the voting task suggest that participants maintain their ability to invoke deeper engagement with options. Quantitative data reveal that participants make no fewer overall edits, with a bimodal distribution suggesting continued revision even at low budgets. Additionally, participants strategically consider broader options as they deliberate on nearby ones. These findings indicate a cognitive shift toward germane load, particularly during the voting phase.

In contrast, participants in the long text interface experience higher extraneous load, evident in shallower reflection and shorter overall voting times, despite exhibiting a greater overall edit distance. While some might argue that the additional grouping phase offers participants more opportunities to familiarize themselves with the options, the long edit distance suggests that participants in the text interface traverse the list frequently, providing ample opportunity to adjust their preferences. Qualitative data indicate that 70\% of long text participants (N=7) scan the list while voting, with edit distance data reflecting multiple passes across the list.

The deliberate one-option-at-a-time presentation during the voting task in the two-phase interface reduces reliance on defaults and encourages deeper reflection. This is best-illustrated by~\texttt{S013}, who emphasizes how the organization phase supports their preference construction:  
\begin{displayquote}  
\bracketellipsis it (organization phase) gives you time to just focus on that single thing and rank it based on how you feel at that moment. \hfill\quoteby{S013 (SI)}  
\end{displayquote}  

Thus, based on this evidence, we argue that a text-based interface is not an optimal solution for long QS where deeper engagement and preference construction are desired. A two-phase interface enables participants to effectively exercise germane load, fostering deeper engagement with the content.


% Conversely, participants using the text interface, like \texttt{S003}, described how default placements influenced their decisions:

% \begin{displayquote}
%     Honestly, if medical research~\bracketellipsis was the first one I saw, I think it would automatically give it a lot more. \hfill\quoteby{S003 (ST)}
% \end{displayquote}

% possiblity from the overall reduction of extraneous load. This additional benifit from added time and slightly higher cognitive load, at the minimum, prevented cognitive overload, , fostering reflection and deliberation which is relatively important when completing surveys.

% Based on the current evidence,  The two-phase interface, by supporting reflection and minimizing reliance on defaults, demonstrates the potential for more effective design strategies in similar contexts.

\subsubsection{Plausible satisficing behaviors in long QS}
In addition, the observed lower overall cognitive load in the long text interface may partly reflect~\textit{satisficing behaviors}. Satisficing refers to participants settling for \textit{good enough} rather than \textit{optimal} decisions~\cite{gigerenzerReasoningFastFrugal1996} when unable to process all available information. Interviews reveal that 40\% of participants (N=4) in this condition describe using satisficing strategies, while none from the long two-phase interface report such behaviors.These strategies are exemplified by participants prioritizing minimal effort over thorough evaluation, as illustrated by:

\begin{displayquote}
    ~\bracketellipsis you thought of enough things, you know, and so it wasn't the most effort I could put in because again, that would have been diminishing returns. I tried to think of enough things~\bracketellipsis and then move on.~\bracketellipsis 
    
    I felt like that (the response) was satisfied, but not perfect. Cause perfect is not a reality. \hfill\quoteby{S036 (ST)}
\end{displayquote}

This quote illustrates satisficing decision-making, where participants settle for suboptimal choices. Additional participants describe similar strategies when deciding on votes:  

\begin{displayquote}
    ~\bracketellipsis Because that was what was left. [Laughter] I probably wouldn't use that on <optionA> instead of <optionB>.~\bracketellipsis \hfill\quoteby{S015 (LT)}

    I tried to use them~\bracketellipsis it went negative, and then I just settled for just \$6 remaining. ~\bracketellipsis I don't think it's perfect. But I think I'm satisfied. Yeah, I'm satisfied.  \hfill\quoteby{S033 (LT)}

    ~\bracketellipsis when I had first started like looking at the first few, I was just doing it kinda like willy nilly, I'm not really paying that much attention to necessarily how many credits I had, or how many categories there were. \hfill\quoteby{S041 (LT)}
\end{displayquote}

These quotes highlight how participants in the long text interface adjust to external constraints rather than carefully weighing internal preferences. This behavior suggests that cognitive overload may lead participants to adopt less effortful strategies. However, further research is needed to fully understand the prevalence and impact of satisficing in long QS surveys.  

\textbf{In summary}, the two-phase interface likely reduces extraneous load, particularly in the long survey condition, facilitating a cognitive \textit{shift} toward deeper reflection and more deliberate decision-making. While the extent to which long QS surveys induce cognitive overload or satisficing remains unclear, the interactive interface shows promise in promoting deeper engagement with options and supporting comprehensive preference construction. The following section explores the specific elements that guide participants toward these outcomes.

% ============================== %
\subsection{Construction of Preference on Quadratic Survey}

Completing QS involves a series of difficult decision tasks~\textcite{lichtensteinConstructionPreference2006}. ~\textcite{svensonDifferentiationConsolidationTheory1992}'s differentiation and consolidation theory help explain how participants process these decisions. The decision process begins with differentiation, where participants identify differences and eliminate less favorable options, followed by consolidation, which strengthens their commitment to selected choices. This theory aligns with how the two-phase interface helps participants decompose options into categories, effectively reducing decision complexity.

Participants start by constructing preferences in situ, especially regarding options they have not previously considered:
\begin{displayquote}
    \bracketellipsis`Oh, there are other aspects that I never care about.' And actually~\ldots some people care <an option>. Sure. Why? Why (should) I spend money on that? \\\hfill\quoteby{S037 (LI)}
\end{displayquote}

Those using the text interface, lacking the interactive tools, find it challenging to facilitate differentiation, as~\texttt{S025} notes:

\begin{displayquote}
    I would like to be able to like, click and drag the categories themselves so I could maybe reorder them to like my priorities.~\bracketellipsis make myself categories and subcategories out of this list~\ldots If I could organize it. \hfill\quoteby{S025 (LT)}
\end{displayquote}

In contrast, the two-phase interface allows participants to express at least one dimension of differentiation more easily. The drag-and-drop feature helps blend this differentiation into the consolidation phase. Not only do participants drag-and-drop options post-voting to reflect and assure a correct vote allocation, but it also enables participants, like~\texttt{S039}, to make fine-grain comparisons between options:  

\begin{displayquote}  
    I think the system was actually really helpful because I could just drag them.~\bracketellipsis I can really compare them, I can drag this one up here, and then compare it to the top one~\bracketellipsis \hfill\quoteby{S039 (SI)}  
\end{displayquote}  

The bi-modal behavior observed in the long interactive interface participants aligns with the differentiation and consolidation framework, as described in the results. Participants in the two-phase interface begin differentiating options earlier, allowing them to later adjust fine-grain votes. The faster and smaller vote updates indicate participants are consolidating. The less prominent bi-modal behavior from the long text interface participants implies that the interface guides this decision framework, as participant~\texttt{037} explains:

\begin{displayquote}
    I only start from the positive one~\bracketellipsis I finish everything~\ldots and then I move to the second part (the neutral box).~\bracketellipsis I want to focus on these and make sure that resources are at least they get the attention they want. And if there's surplus and they can move to the second part. \hfill\quoteby{S037 (LI)}
\end{displayquote}

In addition, the three key elements of the organization phase—presenting options one at a time, grouping them into categories, and enabling drag-and-drop—work together to structure participant preferences. These elements align with cognitive strategies like~\textit{\smash{problem decomposition}}~\cite{simonSciencesArtificial1996} and~\textit{\smash{dimension reduction}}, which reduce cognitive overload. Bounded rationality highlights how cognitive limitations lead to sub-optimal decision-making due to the inability to process all available information~\cite{simonBehavioralModelRational1955}. It illuminates the importance of decision-making support interfaces rather than serving as a critique of human behaviors. One participant explains how the organization phase breaks down complex decisions into manageable steps:  

\begin{displayquote}  
\bracketellipsis being able to have a preliminary categorization of all the topics. First, it introduced me to all the topics,~\bracketellipsis to think about and process~\bracketellipsis being able to digest all the information prior to actually allocating the budget or completing the quadratic survey. \hfill\quoteby{S009 (LI)}  
\end{displayquote}  

Participants using the two-phase interface, especially in the long version, organize options along dimensions such as topics (e.g., health vs. humanitarian) and preferences (positive vs. negative) before voting. Others express that the upfront introduction of all options and the ability to rank and group them help manage their cognitive load effectively. In contrast, almost half of the participants using the long text interface, like~\texttt{S028}, express a desire for features that could help reduce the decision space when responding to the QS, further supporting the importance of these organizational design elements:

\begin{displayquote}  
Because with this many (options), especially when I'm thinking \ldots\ Ok, where was (the option) \ldots\ Where was (the option) you know? Oh, that's right. Maybe I could give another upvote to the, you know, whatever~\bracketellipsis \hfill\quoteby{S028 (LT)}  
\end{displayquote}  

This quote reflects participants' need to manually track and revisit options, which occupies extraneous load, without a more structured interface.  

These evidence explain how the organization phase and the drag-and-drop features support differentiation and consolidation, and scaffold a decision-making framework that enables deeper engagement.  

\textbf{In summary}, participants construct their preferences as they complete QS. We observe behaviors and qualitative insights that align with the differentiation and consolidation theory in decision-making. Our interface scaffolds many of the differentiation stages through pre-voting organization and some consolidation phases through drag-and-drop, explaining how the two-phase approach supports preference construction to yield deeper engagement with QS options.  

% ========================= %
\subsection{Future Work: Opportunities for better budget management}
Budget management is a recurring theme in participant interviews. While they appreciated the automatic calculation feature in modern QV interfaces, we identified three challenges for future QS interfaces: ~\textit{cognitive load},~\textit{the cold-start problem}, and~\textit{navigating between budget, votes, and outcome}.

\subsubsection{Automatic calculation is critical}
Over one-third of participants ($N=14$) from all four experiment conditions emphasized the importance of automated calculation for deriving costs and tracking expenditures. For example:

\begin{displayquote}
I thought I have \bracketellipsis (to) do all the numbers or calculations myself \bracketellipsis The credit summary section was really wonderful in doing all the calculations on that end. \hfill\quoteby{S005 (LT)}
\end{displayquote}

The quotes marked the importance that QS must be facilitated by computer-supported interfaces.

\subsubsection{The coldstart problem}
We notice from the study that one of the biggest challenges for participants is deciding 'how many votes' to start with. This challenge pertains to the initial vote, not the relative vote. Some participants began by equally distributing their credits to all options and then made adjustments. Others established $1$, $2$, and $3$ votes as starting points. A small handful surprisingly used the tutorial's example of 4 upvotes as their anchor.

This arbitrary voting decision echoes discussions in prior literature about the existence of an absolute value for individuals. Coherent arbitrariness~\cite{arielyCoherentArbitrarinessStable2003}, similar to the anchoring effect in marketing, refers to participants' willingness to allocate votes, which can be influenced by an arbitrary value. However, the ordinal utility remains intact among the set of preferences.

\subsubsection{Navigating Between Budget, Votes, and Actual Impact}
The third challenge is participants' confusion between budget, votes, and outcomes, despite understanding their definitions. One participant stated:

\begin{displayquote}

~\bracketellipsis get rid of the upvote column or just get rid of the word upvote and just really focus on the money column. Listen. You're an organization or your participant. You have X amount of dollars you need to. You can only distribute X amount of dollars to these causes. So you have to figure out which ones get the most, which ones don't get as much.~\bracketellipsis 

Interviewer: ~\bracketellipsis Do you feel that the more votes you're giving to a cause you're actually spending more on it?

Yeah. \hfill\quoteby{S003 (ST)}
\end{displayquote}

Participants like \texttt{S003} bypassed the quadratic formulation, directly translating votes to resource allocation. While this does not invalidate the power of the quadratic mechanism, it causes frustration and friction for participants to construct a clear picture of how to make voting decisions. Future interfaces should better communicate these relationships to facilitate respondents' trade-offs.

\textbf{In summary}, while the interface supports budget management through automated cost calculation, participants still face cognitive load from managing the budget. The cold-start problem and the confusion between budget, votes, and actual impact are open questions for future research. These challenges highlight the need for better budget management support to complete the QS interface.

\subsection{Quadratic Survey Usage, Design Recommendations and Future Work}
With a deeper understanding of how survey respondents interact with QS and the sources of cognitive load, we recognize that while this current interface may not significantly reduce cognitive load, it represents a crucial step toward constructing better interfaces to support individuals responding to QS. In this subsection, we outline usage and design recommendations applicable to all applications using the quadratic mechanism and highlight directions for future work.

\subsubsection{Usage Recommendation: QS for Critical Evaluations}
Our study highlighted the complex cognitive challenges and in-depth consideration required when ranking and rating options using QS, even in a short survey. Similar to survey respondents needing to make trade-offs across options, researchers and agencies seeking additional insights and alignment with respondent preferences must ensure that survey respondents have the cognitive capacity to complete such surveys rigorously. QS should be designed for critical evaluations, such as investment decisions, or situations where participants have ample time to think and process the survey. Pactioners should also caution the use of long QS. If long QS is not avoidable, considering allowing participants to deliberate on each option prior to deploying QS without the organizing phase. For instance, revealing the options ahead of time can aid in preference construction.

\subsubsection{Design Recommendations}
\paragraph{Use Organization Phases for Quadratic Mechanism Applications}
Our study demonstrated that preference construction can shift from operational to strategic and higher-level causes. An additional organizational phase with direct manipulation capability allows survey respondents to engage in higher-level critical thinking. We believe this approach should extend beyond QS to other ranking-based surveying tools, such as rank-choice voting and constant sum surveys. Further research should examine how implementing such functionality alters survey respondents' mental models.

\paragraph{Facilitate Differentiation through Categorization, Not Ranking}
Participants in our study were less inclined to provide a full rank unless necessary. The final 'rank' of option preferences often emerged as a byproduct of their vote allocation, constructed in situ. Therefore, for survey designs to be effective in constructing preferences, it is more important to facilitate differentiation than to focus on direct manipulation solely for fine-tuning. Emphasizing categorization can better support participants in articulating their preferences.

\subsubsection{Future Work: Support for Absolute Credit Decision}
Deciding the absolute amount of credits in QS is highly demanding. Designing interfaces and interactions that address the cold start challenge and help participants decide the absolute vote value while considering ways to limit direct influences remains an open question. Future research should explore innovative solutions to support participants in making these complex decisions effectively.

By implementing these recommendations and pursuing future research directions, we can improve the usability and effectiveness of QS and other quadratic mechanism-powered applications, ultimately aiding respondents in making more informed and accurate decisions.

\section{Limitations}
\label{sec:limitations}
Evaluating the QS interface is challenging due to its novelty. During the study, we identified several limitations that require further research.

\paragraph{Understanding results influence on decision-makers}
Further research is required to understand how the QS interface impacts decision-makers and broader societal resource distributions. Since QS is still in its early stages, we prioritize its widespread adoption and usage before attempting a comprehensive assessment of its influence on decision-making. Future studies will examine how decision-makers interpret and use QS data, as well as its broader implications for societal decisions.

\paragraph{Individual differences in cognitive capacity}
Variations in individual cognitive capacity influenced participants' cognitive scores. For example, participants with more experience in decision-making might be able to manage multiple options more effectively. A within-subject study could clarify cognitive load shifts, but deconstructing established preferences and altering options further complicates this. Thus, we opted for this in-depth, between-subject study, although the small sample size may introduce noise that distorts the actual cognitive load. Future research should quantify the impact of different QS interfaces. In addition, participants completed this study in a controlled lab environment with options displayed on a large screen. Future work should also explore how individuals respond to QS on smaller devices in a less controlled environment.

\paragraph{Limited experience with QS}
Participants had no prior experience with the QS interface. Following a tutorial and quiz, participants proceeded to complete tasks using the QS interface. While participants understood the QS mechanics, familiarity with the interface still influences strategies and cognitive load. As quadratic mechanisms become more prevalent, future research can compare novices and experts.

\paragraph{Duration between clicks to represent decision-making}
Click duration may include time spent considering other options, so it must be treated as an approximate measure of decision-making time. For instance, deciding between two options may take longer for the first option and less time for the second. Despite its limitations, this approach provides valuable insights into decision-making within our experimental constraints.

% 

% Recall that this survey aims to assist community organizers in distributing resources to a societal cause. This participant decided to `skip' over the quadratic formulation and the concept that their votes are governed by the quadratic formulation, drawing a direct translation between votes and the resources to which community organizers ought to contribute. 
% \begin{displayquote} I guess to see what my ranking looks like~\ldots and see if I could give more money or not. \hfill\quoteby{S021 (LI)} \end{displayquote}
% \begin{displayquote}
% If I had to choose a number like that in the beginning. That would have been really bad, but positive, neutral, negative. That was good enough. \hfill\quoteby{S026 (LI)}
% \end{displayquote}

% \begin{displayquote}
% I think \ldots\ ranking at the beginning one's impression towards these issues helps to like determine how many votes should be put towards them.  \hfill\quoteby{S002 (SI)}
% \end{displayquote}

% \begin{displayquote}
% If anything, I think I would like to be able to like, click and drag the categories themselves so I could maybe reorder them to like my priorities. \hfill\quoteby{S025 (LT)}
% \end{displayquote}

%  of well-organized interfaces in managing cognitive load.

% Participants ($N=4$, $2$ using the long two-phase interface) mentioned that organization support helped them to allot the intensity of votes by helping them focus and prioritize options through ranking. This exercise allows them to follow a clear decision-making process that avoids confusion.
% It is important to note that bounded rationality does not critique or exploit biases, but emphasizes the importance of designing interfaces that prevent decisions which diverge from one's true preferences. For example,~\underline{\smash{problem decomposition}}~\cite{simonSciencesArtificial1996} and~\underline{\smash{dimension reduction}} are strategic approaches to managing cognitive overload. Several participants would create a two-axis grouping, regardless of their experiment group. Participants clustered topics (e.g., health vs. humanitarian) and preferences (positive vs. negative). The difference between conditions was whether these groupings were representable on the interface.
% In addition, results indicate long text interface participants were satisfied due to cognitive overload from having too many options. They have to read more text, allocate more credits, and consider more options. Section~\ref{sec:cog_result} and Section~\ref{sec:behave_result} show how counterintuitive that this group had fewer participants experiencing high cognitive load compared to the short text interface. This group also experienced the least temporal demand (Sec.~\ref{sec:temporal}) while showing no difference in time spent per option compared to the text interface (Figure~\ref{fig:vote_time}). Participants in the long text interface also expressed the least frustration with operational tasks (Sec~\ref{sec:frustration}). 
% % These participants engaged with higher-level strategic challenges, in contrast to the more operational tasks emphasized in the text interface. 
% \paragraph{Familarity to the options}
% 1. primed on the local community, 
% 2. limited experience with qs
% We also acknowledge the possibility that the elicited values are pure noise and do not reflect the actual cognitive load. This could be due to the small sample size, the nature of the task, or the participants' understanding of the cognitive load scale. While this true for small sample sizes, we believe that the qualitative insights from the interviews provide a more nuanced understanding of the cognitive load sources. We detail limitations in Section~\ref{sec:limitations}.
% Maybe large scale AB testing and within subject testing in periodic collective decsion making enviornments.
% 3. time associated with the option.

% \subsection{The Quadratic Mechanism is Challenging}
% % We know QV is accurate and that QM allows specific expression of preferences
% % However, QM is diffucult to manage, internally construction of preference is diffuclt but so is the QM.

% % we tried to scaffold the construction of preference in interface design, for which we did help participants get to the exact values faster, but identifying and managing the construction is not something organization interfaces can fully support
% Most challenges participants faced come from the task itself: deciding the number of votes/credits to allot. I created the following hierarchical theme
% CI_3: deciding number of votes and credits (N=9/40, v1:1; v2:1; v3:4; v4:3)
% We can see participants in the long version group struggles more with this challenge. (2/20 vs. 7/20). So what exactly contributes to this decision process? We broke it down to the following themes:

% % Challenge lies in the mechanism itself
% CI_1: working with the QS mechanism (N=6, v1:5; v3:1)
% distinguishing between credits and votes (v3 participant)
% quadratic mechanism (all the rest)
% The first finding is that non of the interactive interface groups (v2 and v4) expressed feeling challenged due to the QS mechanism. The second finding is that the majority of this challenge comes from the short-list group. I think an explanation to this is clearer when we put up the second theme:

% CI_2: use up the remaining credits (N=4, v1:3, v2:1)
% The participants is struggling to express specific level of preference with limited credits. 
% Revisit one of the quotes from CI_1:
% “I wish I could just put the $2 towards the museum, or something like that.” (S036,v1)
% “It would be nice if I can use that one credit if there is an option, because the way it is done is in quadratic...I don't know why that is there...but if there is an option to not have it, and just [inaudible], that would be awesome.” (S012, v1)
% In other words, the expressiveness is constraint by the limited credit, amplified by the quadratic nature, forcing participants to forgo unused credits. This is also likey tied to prospect theory, that we will discuss later.
% The interface in the second group could have eliminated this because some options were eliminated, or that some ranking were established, prior to the voting process.


% \subsection{Construction of Preference}
%  \subsection{Design Implications}
% % Your content for the subsection on design implications goes here

% IN_T1: Dropdown (N=6/40)
% This is a common issue that participants dislike, across all versions.
% 3 from v1, the rest of the version each has 1

% IP_T3: Seeing all options – a sign in making decisions
% Participants like the ability to see all options on one screen (N=8/40)
% Comparing Long (N=5) and Short (N=3)
% The interactive interface requires a stronger need to see all options, as I hypothesis that this is because the need to interact and see the hierarchical groupings (Text: 2; Interactive: 6)

%% on positioning shift and the power of priming
%  use the performance quotes to highlight how participants are thinking in the shoes of decision-maker
% look for literature
% survey designed for decison makers to aid decisions

% Participants either felt positive or no issues using all four interfaces (N=33/40).


% \subsection{Limitations and Future Work}
% % Your content for the subsection on limitations and future work goes here


% We first show that participants constructed their preferences in situ. While some participants had existing preferences (e.g., environmental issues are important), they needed to reconsider aspects of the options or map them to their beliefs.

% \begin{displayquote}

% ~\bracketellipsis the other part of the mental demanding was probably trying to associate with (what) I'm concerned in soci(ety)~\bracketellipsis is that question able to deal with my social concerns like, for example, climate change~\bracketellipsis How does that fit in?

% \noindent \hfill -- S006, long interactive interface
% \end{displayquote}


% Behavior analysis in section~\ref{res:act} of participants using the long text and interactive interfaces revealed that they made small adjustments on the votes, clustered toward budget depiction with lesser time spent. These fine-grain adjustments indicated that participants are making less ad-hoc decisions; rather, they are deciding how to better utilize the remainder of the budget when the budget runs low. We identified a bi-modal interaction pattern.

% indicated that participants are making less ad-hoc decisions; rather, they are deciding how to better utilize the remainder of the budget when the budget runs low. We identified a bi-modal interaction pattern.

% \

% Conversely, in the text interface, one participant proactively mentioned a request to add click-and-drag functionalities to the interface. The participant described such function to group by topic categorization and also priority placement through direct manipulation.


% Throughout the preference construction journey, we confirm that the two-stage interactive interface and the direct manipulation through drag-and-drop facilitated participants in constructing and reflecting on their preferences, adhering to preference construction theory.

% Additionally, several participants mentioned how the direct manipulation functionality, allowing individuals to drag and drop options for repositioning, supports their reflective thinking during preference construction. One participant noted:
% \begin{displayquote}
% So I tried to make a ranking \bracketellipsis and by creating this ranking, by dragging the related issues \ldots\ I don’t know \ldots\ that helped me organize my ideas.
% \noindent \hfill -- S021, long interactive interface.
% \end{displayquote}

% into these categories, making completing the entire QS a series of difficult decisions.
% Literature from~\textcite{lichtensteinConstructionPreference2006} identifies three types of difficult decision-making scenarios: when one's preferences are not clearly defined, necessitating trade-offs, or quantifying opinions.  
% Since the interface supported some participants in managing their limited cognitive ability to make decisions, as shown in the previous subsection, we argue that the interactive interface \textit{shifted} the cognitive focus onto contributing to more in-depth preference construction and fine-tuning, even if it did not significantly reduce the cognitive load. Here we provide more evidence.

% Literature from~\textcite{lichtensteinConstructionPreference2006} identifies three types of difficult decision-making scenarios: when one's preferences are not clearly defined, necessitating trade-offs, or quantifying opinions. 


% Two participants highlighted the importance of automated calculation regarding the cost for each vote.
% Twelve participants highlighted the summarization box and the automated summation of the current credit spent, allowing them to focus on managing their next voting decision and expressing their preferences.

% \begin{displayquote}
% I like that I don't have to make the calculation of the dollars that it does it automatically. So if I had to do it myself it would be more tedious. And so I think that that effort and frustration and mental demand would be much higher. So I appreciate that that calculation occurs automatically and very easily.
% \noindent \hfill -- S017, short interactive interface.
% \end{displayquote}

% This is less significant in the short QS likely due to the reduced complexity~\footnote{We show in Appendix~\ref{sec:appendix_short_breakdown} that short interfaces exhibits the same bimodal behaviors but less obvious.}.
\section{Conclusion}
Surveys enable decision-makers to aggregate crowd opinions. In this study, we use QS to elicit individual responses in the context of social resource allotment. After multiple design iterations, we propose an interactive interface for QS. We then examined its influence on individuals' cognitive load and behaviors when faced with societal issues of varying lengths. In a 2x2 between-subject study, we had participants experience either a long or short QS using a text-based or interactive interface. NASA-TLX questionnaires and interviews revealed that participants using the interactive interface for a long QS demonstrated a more comprehensive and critical evaluation of societal issues, despite not experiencing a lower cognitive load. Participants using the long text interface experienced cognitive overload, which led to satisficing behaviors or mental shortcuts. Analyzing click-stream data, we identified that participants made fine-grain iterations using the long interactive interface when credits were low. We demonstrate that a two-phase, organize-then-vote interface can scaffold the complex decision-making process, helping individuals express their opinions for collective societal decisions. Through the iterative design process and detailed interviews, we identified future directions and design recommendations for collective decision-making applications using the quadratic mechanism.
% \begin{acks}
We thank the voluntary participants who participated in the pretest, pilot, and the study. We thank all members of the Social Spaces Group and the Crowd Dynamics Lab for their support and early feedback. Additional thanks to Yi-Ting Kuo, Hsin-Ni Yu, Yun-Shan Sam Yang, Katherine Chou, and the anonymous reviewers who provided valuable feedback to this work. This work was partially supported by Just Infrastructures at University of Illinois at Urbana-Champaign.
\end{acks}






% \printbibliography

\appendix
% Interface design appendix
\section{Voting Interface Breakdown}\label{apdx:relatedVoting}
Compared to digital survey interfaces, there exists a rich literature on voting interfaces, which we argue is a special type of survey interface. We categorize these related works into three main categories detailed below:

\paragraph{Designs that shifted voter decisions: } For example, states without straight-party ticket voting~(where voters can select all candidates from one party through a single choice) exhibited higher rates of split-ticket voting~\cite{engstrom2020politics}. Another example from the Australian ballot showing incumbency advantages is where candidates are listed by the office they are running for, with no party labels or boxes.
\paragraph{Designs that influenced errors: } Butterfly ballots increased voter errors because voters could not correctly identify the punch hole on the ballot. Splitting contestants across columns increases the chance for voters to overvote~\cite{quesenberyOpinionGoodDesign2020}. On the other hand, \textcite{everettElectronicVotingMachines2008} showed the use of incorporating physical voting behaviors, like lever voting, into graphical user interfaces.

\paragraph{Designs that incorporated technologies: } Other projects like the Caltech-MIT Voting Technology Project have sparked research to address accessibility challenges, resulting in innovations like EZ Ballot~\cite{leeUniversalDesignBallot2016}, Anywhere Ballot~\cite{summers2014making}, and Prime III~\cite{dawkinsPrimeIIIInnovative2009}. In addition, \textcite{gilbertAnomalyDetectionElectronic2013} investigated optimal touchpoints on voting interfaces, and \textcite{conradElectronicVotingEliminates2009} examined zoomable voting interfaces.

\section{Voting Interfaces and Response Format}

Research in the marketing and research communities focusing on survey and questionnaire design, usability, and interactions examines the influence of presentation styles and `response format.'~\textcite{weijtersExtremityHorizontalVertical2021} demonstrated that horizontal distances between options are more influential than vertical distances, with the latter recommended for reduced bias. Slider bars, which operate on a drag-and-drop principle, show lower mean scores and higher nonresponse rates compared to buttons, indicating they are more prone to bias and difficult to use. In contrast, visual analog scales that operate on a point-and-click principle perform better~\cite{toepoelSlidersVisualAnalogue2018}. These studies show how even small design changes can have a large impact on usability, highlighting the importance of designing interfaces that prioritize human-centered interaction rather than focusing solely on functionality.

Voting interfaces are a specialized type of survey interface that not only elicit individual choices but often have consequential impacts. For example, the butterfly ballot, an atypical design, may have influenced the outcome of the 2000 U.S. Presidential Election~\cite{wandButterflyDidIt2001}. Research has shown that ballot interfaces can significantly influence democratic processes~\cite{engstrom2020politics, chisnellDemocracyDesignProblem2016, DesigningUsableBallots2015}. Several studies also highlighted how voting interface designs shift voter decisions~\cite{engstrom2020politics}, reduce usability errors~\cite{quesenberyOpinionGoodDesign2020, everettElectronicVotingMachines2008}, or improve interaction~\cite{leeUniversalDesignBallot2016, summers2014making, dawkinsPrimeIIIInnovative2009, gilbertAnomalyDetectionElectronic2013, conradElectronicVotingEliminates2009}. We provide more details to these voting interfaces in the Appendix~\ref{apdx:relatedVoting}.

From the QV implementations, response format literature, and voting interfaces, we identified how interfaces significantly influence respondent behavior, decision accuracy, and cognitive load. While these systems are functional, they lack the human-centered design needed to reduce cognitive load and make them truly usable, rather than simply operable. These burdens are especially problematic for complex systems like QS, where high cognitive demands may deter researchers and users alike. Developing effective, human-centered interfaces for QS could enhance usability, reduce cognitive overload, and increase adoption in both research and practical applications.





\end{document}
\endinput

